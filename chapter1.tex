% !TeX root = main.tex
\chapter{拓扑空间与度量空间}
     \pagenumbering{arabic}
\section{基本概念}

     \begin{Definition}[拓扑空间]\label{def:拓扑空间}
          设 $ E $ 是一个集合, 称 $ E $ 的子集族 $ \tau $ 是一个\textbf{拓扑}\index{T!拓扑}, 若 $ \tau $ 满足: 
          \begin{enumerate}[($ \mathrm{O}_1 $), itemindent=2.5\parindent]
                \item $ E, \varnothing\in \tau $;
               \item\label{item:O2} $ \tau $ 中任意多元素的并仍是 $ \tau $ 中元素 (任意并);
               \item\label{item:O3} $ \tau $ 中有限多元素的交仍是 $ \tau $ 中的元素(有限交) . 
          \end{enumerate}
          并称 $ (E, \tau) $ 为一个\textbf{拓扑空间}\index{T!拓扑空间}, $ \tau $ 中的元素称为\textbf{开集}\index{K!开集}, 在不引起歧义的情况下, 可以简称 $ E $ 是一个拓扑空间. 
     \end{Definition}

     \begin{Remark}\label{rmk:平凡离散}
          对集合 $ E $ , 称 $ \tau=\{ E, \varnothing \} $ 为\textbf{平凡拓扑}\index{P!平凡拓扑}, 称 $ \tau=2^{E} $ 为\textbf{离散拓扑} \index{L!离散拓扑}. 
     \end{Remark}

     \begin{Example}
          $ E=\R $, 取 $ \tau=\Big\{ \bigcup\limits_{j\in\Z}(x_{j}: x_{j+1}), (x_{j})_{j\in\Z}\subset \R\cup\{ \pm\infty \} \Big\} $是一个拓扑, 这一拓扑称为 $ E $ 的\textbf{自然拓扑}\index{Z!自然拓扑}. 
     \end{Example}

     \begin{Definition}[度量空间]\label{def:度量空间}
          设 $ E $ 是非空集合, $ d: E\times E\to \R $ 满足 $ \forall x, y\in E $
          \begin{enumerate}[(1)]
               \item 非负性: $ d(x, y)\geqslant 0 $;
               \item 正定性: $ d(x, y)=0\Longleftrightarrow x=y $ ;
               \item 对称性: $ d(x, y)=d(y, x) $ ;
               \item 三角不等式: $ d(x, z)\leqslant d(x, y)+d(y, z) $ 
          \end{enumerate}
          则称 $ (E, d) $ 是一个\textbf{度量空间}\index{D!度量空间}, 并称 $ d $ 是 $ E $ 上的\textbf{度量}\index{D!度量}
     \end{Definition}

     \begin{Remark}\label{rmk:度量不唯一}
          度量并不唯一, 例如对 $ (E, d) $ 定义度量
          \begin{equation*}
               \begin{aligned}
                    r(x, y) = \min\{ d(x, y), 1 \}, \\
                    d'(x, y) = rd(x, y), r\in\R_{+},
               \end{aligned}
          \end{equation*}
          就是两个不同的度量. 
     \end{Remark}
     
     度量空间也是拓扑空间, 记 $ B(x, r)=\{ y: d(y, x)<r \} $ 可将 $ \tau $ 中的元素定义为: 
     \[
          U\in E, U\in \tau\Leftrightarrow \forall x\in U\exists r>0(B(x, r)\subset U)
     \] 
     该拓扑 $ \tau $ 称为由度量 $ d $ \textbf{诱导}的拓扑. 

     \begin{Example} 
          在实 Euclid 空间 $ \R^{n} $ 上赋以度量
          \begin{equation}
               d(x, y)=\bigg( \sum_{i=1}^{n}(x_{i}-y_{i})^{2} \bigg)^{1/2}\tag{Euclid距离}
          \end{equation}
          以 $ C[a, b] $ 以在 $ [a, b] $ 上的连续函数全体, 在其上赋以度量
          \begin{equation}
               d(x, y) = \max_{t\in [a, b]} |x(t)-y(t)|\tag{一致距离}
          \end{equation}
     \end{Example}

     \begin{Definition}[闭集]\label{def:闭集}
          设 $ (E, d) $ 是一个拓扑空间, $ A\subset E $ , 若 $ A^{c} $ 是开集, 则称 $ A $ 是 $ E $ 上(关于 $ \tau $ )的\textbf{闭集}.\index{B!闭集}
     \end{Definition}
     \begin{Example}
          设 $ (E, d) $ 是一个度量空间, 则闭球 $ \baro{B}(x, r)=\{ y: d(y, x)\leqslant r \} $ 是闭集. 
     \end{Example} 
     \begin{Proposition}\label{prop:闭集的性质}
          闭集具有以下性质:
          \begin{enumerate}[(1)]
               \item 全空间 $ E $ 和空集 $ \varnothing $ 是闭集;
               \item 任意多个闭集的交仍是闭集(任意交);
               \item 有限多个闭集的并仍是闭集(有限并).
          \end{enumerate}
     \end{Proposition}

     \subsection{邻域和邻域基}

     \begin{Definition}[邻域基]\label{def:邻域基}
          设 $ (E, \tau) $ 是一个拓扑空间, $ \alpha\in E $, 
          \begin{enumerate}[(1)]
               \item 若子集 $ V\subset E $ 满足
               \[
                    \exists U\in \tau(x\in U\land U\subset V)
               \]
               则称 $ V $ 是 $ x $ 的一个\textbf{邻域}\index{L!邻域}, 并记 $ \CN(x) $ 为全体 $ x $ 的邻域的集合, 称为 $ x $ 的\textbf{邻域系};
               \item 若 $ \CN(x) $ 的子集族 $ \CB(x) $ 满足
               \[
                    \forall V\in \CN(x)\exists U\in \CB(x)(U\subset V)
               \]  
               则称 $ \CB(x) $ 是 $ x $的\textbf{邻域基}\index{L!邻域基}. 
          \end{enumerate}
     \end{Definition}
     \begin{Remark}
          $ \CN(x) $是 $ x $  的邻域基; 包含 $ x $ 的所有开集的集合是 $ x $ 的邻域基. 
     \end{Remark}
     \begin{Example}
          设 $ (E, d) $ 是一个度量空间, $ x\in E $, 则 
          \[
               \CB(x)=\left\{ B(x,\frac{1}{n}): n\geqslant 1 \right\} 
          \]
          是 $ x $ 的邻域基. (可用来描述极限)
     \end{Example}
     \begin{Definition}[粘着集]
          设 $ (E, \tau) $ 是一个拓扑空间, $ A\subset E $ 
          \begin{enumerate}[(1)]
               \item 设 $ x\in E $ , 若 $ \forall V\in \CN(x)((V\cap A)\sm \{ x \}\neq\varnothing) $, 则称 $ x $ 是 $ A $ 的\textbf{凝聚点}\index{N!凝聚点};
               \item 若 $ x\in A $ 或 $ A $  的凝聚点, 则称 $ x $ 是 $ A $ 的\textbf{粘着点}, 记 $ A $ 的粘着点的全体为 $ \baro{A} $ , 称为 $ A $ 的\textbf{粘着集}\index{N!粘着集}, 也即
               \[
                    x\in \baro{A}\Leftrightarrow \forall V\in \CN(x)(V\cap A\neq\varnothing).    
               \]
          \end{enumerate}
     \end{Definition}
     \begin{Remark}
          $ E $ 中包含 $ A $ 的最小闭集称为 $ A $ 的\textbf{闭包}\index{B!闭包}, 可知 $ A $ 的闭包与粘着集是相同的, 它们从不同的角度描述了相同的集合. 
     \end{Remark}
     \begin{Definition}[稠密]\label{def:稠密}
          设 $ A\subset E $ , 若 $ \baro{A}=E $ , 则称 $ A $ 在 $ E $ 中\textbf{稠密}\index{C!稠密}, 或称 $ A $ 是 $ E $ 的一个\textbf{稠子集}
     \end{Definition}
     \begin{Example}
          有利数集 $ \Q $ 与无理数集 $ \mathbb{J} $ 都在 \R 中稠密. 
     \end{Example}
     \begin{Definition}[内部]\label{def:内部}
          设 $ A\subset E $, 若 $ A\in\CN(x) $ , 则称 $ x $ 是 $ A $ 的\textbf{内点},  $ A $ 的内点的全体称为 $ A $ 的\textbf{内部}\index{N!内部}, 记作 $ \mathring{A} $ 或 $ \mathrm{Int}(A) $.  
     \end{Definition}
     \begin{Definition}[边界]\label{def:边界}
          设 $ A\subset E $, 称 $ \partial A = \baro{A}\sm\mathring{A} $ 是 $ A $ 的\textbf{边界}\index{B!边界}.
     \end{Definition}
     \begin{Example}
          设 $ E=\R $ , 取 $ A = (2, 3) $, 则 $ \partial A = \{ 2, 3 \} $ ; 取 $ A = [2, 3) $ 则 $ \partial A = \{ 2, 3 \} $. 
          
          设 $ E = \R^{2} $, 取 $ A = (2, 3)\times \{0\} $ , 则 $ \partial A = [2, 3]\times \{0\} $ , 此时 $ A\subset \partial A $  ; 取 $ \mathbb{D}=B(0, 1) $ , 则 $ \partial \mathbb{D}=\mathbb{S}^{1} $ (单位圆周). 
     \end{Example}
     \begin{Theorem}\label{thm:闭包和内部的性质}
          设 $ E $ 是一个拓扑空间, $ A, B\subset E $ , 则:
          \begin{enumerate}[(1)]
               \item $ \baro{\baro{A}}=\baro{A} $, $ \mathring{\mathring{A}}=\mathring{A} $; (幂等性)
               \item $ E\sm\mathring{A}=\overline{E\sm A} $;
               \item \label{item:闭包内部}$ \overline {A\cup B}=\baro{A}\cup \baro{B} $; $ \mathring{\widehat{A\cap B}}=\mathring{A}\cap\mathring{B} $.   
          \end{enumerate}
     \end{Theorem}
     \begin{Remark}
          考虑定理~\ref{thm:闭包和内部的性质}~的~\ref{item:闭包内部}~,注意 $ \overline {A\cap B}=\baro{A}\cap \baro{B} $ 和 $ \mathring{\widehat{A\cup B}}=\mathring{A}\cup\mathring{B} $ 不一定成立. 比如取 $ A=\Q, B=\mathbb{J} $, 则此时 $ \overline{A\cap B}=\baro{\varnothing}=\varnothing $  , 而 $ \baro{A}\cap \baro{B}=\R\cap\R=\R\neq\overline {A\cup B} $ ; 同时 $ \mathring{\widehat{A\cap B}}=\mathring{\R}=\R $, 但是 $ \mathring{A}\cup\mathring{B}=\varnothing\cup\varnothing=\varnothing\neq \mathring{\widehat{A\cap B}} $  
     \end{Remark}
     \begin{Definition}[拓扑比较]\label{def:拓扑比较}
          设 $ \tau, \tau' $ 是 $ E $ 的拓扑, 若 $ \tau'\subset\tau $, 则称 $ \tau $ 是 $ \tau' $ 的\textbf{强拓扑}\index{Q!强拓扑}, 也即: $ \tau' $--开集一定是 $ \tau $--开集.   
     \end{Definition}
     \begin{Example}
          $ E=\R $ 时, 平凡拓扑 $ \subset $ 自然拓扑 $ \subset $ 离散拓扑. 
     \end{Example}
     \begin{Definition}[拓扑子空间]\label{def:拓扑子空间}
          设 $ (E, \tau) $ 是一个拓扑空间, $ F\subset E $ , 在 $ F $ 上定义
          \[
               \tau_{F}=\{ U\cap F: U\in\tau \} , 
          \]
          则 $ \tau_{F} $ 是一个拓扑, 称为由 $ \tau $  诱导的拓扑, 并称 $ (F, \tau_{F}) $ 是 $ (E, \tau) $ 的\textbf{拓扑子空间}\index{T!拓扑子空间}. 
     \end{Definition}
     \begin{Example}
          设 $ E=\R $ , 取 $ F=(2, 3) $ , 则 $ \tau_{F} $ 中元素是 \R 中开集; 
          
          取 $ F=[2, 3) $ , 则 $ \forall2<x<3, [2,x) $ 是 $ \tau_{F} $ 中元素, 但不是 \R 中开集. 
     \end{Example}
     \begin{Example}
          设 $ (E,d) $ 是度量空间, $ F\subset E $, 取 $ \delta=d|_{F\times F} $ 则 $ (F, \delta) $ 是度量空间, 且当 $ E,  F $ 分别赋以 $ d, \delta $ 诱导的拓扑时,  $ F $ 是 $ E $ 的拓扑子空间. 
     \end{Example}
     \begin{Proposition}
          设 $ F\subset E, A\subset F $ 则 $ A $ 是 $ F $ 中的闭集 $ \Longleftrightarrow $ 存在闭集 $ B\subset E $ 使得 $ A=B\cap F $ . 
     \end{Proposition}
     \begin{Proof}
          由闭集性质可以知道
          \[
               \begin{aligned}
                    A \text{\,是\,} F \text{\,中闭集\,} & \Longleftrightarrow F\sm A \text{\,是\,} F \text{\,中开集\,} ;\\
                    & \Longleftrightarrow \exists D\in \tau(F\sm A=D\cap F);\\
                    & \Longleftrightarrow A=D^{c}\cap F
               \end{aligned}     
          \]
          其中$ D^{c} $是闭集. 	 \qed
     \end{Proof}

     \subsection{分离空间}

     \begin{Definition}[Hausdorff空间]\label{def:Hausdorff空间}
          设 $ (E, \tau) $ 是拓扑空间, 若 
          \[
               \forall x, y\in E(x\neq y)\exists U\in \CN(x)\exists V\in\CN(x) (U\cap V)=\varnothing
          \]
          则称 $ (E, \tau) $ 是\textbf{Hausdorff空间}\index{H!Hausdorff空间}或\textbf{分离空间}. 
     \end{Definition}
     \begin{Example}
          所有离散拓扑空间都是 Hausdorff空间, 所有度量空间都是 Hausdorff空间, 元素个数多于一个的平凡拓扑空间不是 Hausdorff空间(因为所有元素都只有一个邻域, 即 $ E $ ). 
     \end{Example}
     \begin{Proposition}\label{prop:Hausdorff空间的相关命题1}
          由 Hausdorff空间的定义可以知道
          \begin{enumerate}[(1)]
               \item  $ E $ 是 Hausdorff空间 $ \Longleftrightarrow $ $ \forall x\in E $, 其所有闭邻域的交为 $\{ x \}$
               \item $ E $ 是 Hausdorff空间 $ \Longrightarrow $ $\bigcap \CN(x)=\{x\}$;
               \item $ E $ 是 Hausdorff空间 $ \Longrightarrow $ $ \forall F\subset E $, $ F $ 也是Hausdorff空间. 
          \end{enumerate}
     \end{Proposition}
     
\section{完备性}
     \subsection{序列的极限}
     \begin{Definition}[极限]\label{def:极限}
          设 $ (E, \tau) $ 是拓扑空间, $ (x_{n})_{n\geqslant 1} $ 是 $ E $ 中的序列,  $ x\in E $, 若
          \[
               \forall V\in\CN(x)\exists n_{0}\in\N(n\geqslant n_{0}\Rightarrow x_{n}\in V)
          \]
          则称 $ (x_{n})_{n\geqslant1} $ \textbf{收敛}于 $ x $, 并称 $ x $ 是序列 $ (x_{n})_{n\geqslant1} $ 的极限, 记作
          \[
               \tau-\lim_{n\to \infty}x_{n}=x. 
          \] 
          当不引起歧义的时候, 前面的 $ \tau- $ 可以省略. 
     \end{Definition}
     \begin{Remark}
          对定义\ref{def:极限}进行几点说明:
          \begin{enumerate}[(1)]
               \item 上述定义中的 $ \CN(x) $ 可换成 $ \CB{x} $ ;
               \item \R 中序列的极限至多只有一个, 例如 $\lim\limits_{n\to\infty}\frac{1}{n}=0 $, $ \lim\limits_{n\to\infty} n=\infty$(无极限), 取 $ (x_{n})_{n\geqslant1}=\left\{ 1, -1, 1,  -1\ldots \right\} $ 也无极限;
               \item\label{item:Hausdorff空间上极限唯一性} 一般地, Hausdorff空间中的序列至多只有一个极限, 而平凡拓扑空间中的任意序列收敛到任意元素;
               \item 对度量空间 $ (E, d) $ , 有
               \[
                    \lim_{n\to \infty}x_{n}=x \Leftrightarrow \lim_{n\to\infty}d(x_{n}, x)=0, 
               \]
               取邻域基 $ \CB(x)=\left\{ B(x, \frac{1}{n}): n\geqslant1 \right\} $即可.  
          \end{enumerate}
     \end{Remark}
     \begin{Proposition}
          设 $ E $ 是度量空间, 则 $ A\subset E $是闭集的充要条件是对任意 $ A $中序列  $ (x_{n})_{n\geqslant1} $ 当 $ \lim\limits_{n\to\infty}x_{n}=x_{0} $ 时, 都有$ x_{0}\in A $.  
     \end{Proposition}
     \begin{Proof}
          \textsl{必要性}. 若 $ x_{0}\notin A $, 则 $ \exists r>0 (B(x, r)\cap A=\varnothing) $, 而由极限的定义知 
          \[
               \exists n_{0}\in\N(n\geqslant n_{0}\Rightarrow x_{n}\in B(x, r))\Rightarrow x_{n}\notin A. 
          \]
          与 $ (x_{n})_{n\geqslant1}\subset A $ 矛盾. 

          \textsl{充分性}. 只需证 $ A^{c} $ 是开集, 用反证法, 若 $ A^{c} $ 不是开集, 则有
          \[
               \exists x\in A^{c}\,\forall r>0(B(x, r)\cap A\neq\varnothing),
          \]
          取 $ r=1/n, n=1, 2\ldots $ , 则有 $ x_{n}\in B(x, 1/n)\cap A $ . 由构造可知 $ (x_{n})_{n\geqslant1}\subset A $且 $ x_{n}\to x\,(n\to\infty) $  , 又由已知条件知 $ x\in A $, 矛盾. 故 $ A $ 为闭集. \qed 
     \end{Proof}
     \begin{Definition}[序列的粘着值]\label{def:序列的粘着值}
          设 $ (x_{n})_{n\geqslant1} $ 是 $ E $ 中的序列,  $ x\in E $, 若
          \[
               \forall V\in\CN(x)\,\forall n_{0}\in\N\,\exists n\geqslant n_{0}(x_{n}\in V),
          \]
          则称 $ x $ 是序列 $ (x_{n})_{n\geqslant1} $ 的\textbf{粘着值}\index{N!粘着值}.  
     \end{Definition}
     \begin{Example}
          \R 中序列 $ (1, -1, 1, -1\ldots) $ 的粘着值为 $ 1, -1 $.  
     \end{Example}
     \begin{Remark}
          序列的粘着值与集合的粘着点没有关系, 如 \R 上的序列 $ (x_{n})_{n\geqslant1}=(1, 1-\sqrt{2}, {1}/{2}, {1}/{2}-\sqrt{2}, {1}/{3}, {1}/{3}-\sqrt{2}, \ldots, {1}/{n}, {1}/{n}-\sqrt{2}, \ldots) $, 该序列的粘着值 $ 0, \sqrt{2} $ 均不在序列中, 而集合 $ A=\{ x_{n} \}_{n\geqslant1} $ 的粘着集为 $ A\cup\{ 0,-\sqrt{2} \} $. 

          再取 $ (x_{n})_{n\geqslant1}=(1, 1, \ldots) $ 则该序列的粘着值为$1$,  但 $ 1 $ 不是 $ A=\{ 1 \} $ 的凝聚点.  
     \end{Remark}
     \subsection{Cauchy列与完备性}
     \begin{Definition}[Cauchy列]\label{def:Cauchy列}
          设 $ (E, d) $是度量空间, $ (x_{n})_{n\geqslant1} $ 是 $ E $  中序列, 若
          \[
               \forall \varepsilon>0\,\exists n_{0}\in\N\,(n,m\geqslant n_{0}\Rightarrow d(x_{m},x_{n}<\varepsilon),
          \]
          则称 $ (x_{n})_{n\geqslant1} $ 是一个\textbf{Cauchy列}\index{C!Cauchy列}
     \end{Definition}
     \begin{Example}
          在 \R 中,  $ (1/n)_{n\geqslant1} $ 是Cauchy列, 收敛到0; $ ((1+{1}/{n})^{n})_{n\geqslant1} $ 是Cauchy列, 收敛到\me;

          而在 \Q 中, $ (1/n)_{n\geqslant1} $ 是Cauchy列, 收敛到0; $ ((1+{1}/{n})^{n})_{n\geqslant1} $ 是Cauchy列, 不收敛.
     \end{Example}
     \begin{Proposition}\label{prop:Cauchy列的性质}
          Cauchy列有以下性质:
          \begin{enumerate}[(1)]
               \item 若 $ (x_{n})_{n\geqslant1} $ 是收敛列, 则 $ (x_{n})_{n\geqslant1} $ 是Cauchy列;
               \item 若 $ (x_{n})_{n\geqslant1} $ 是Cauchy列且有收敛子列, 则 $ (x_{n})_{n\geqslant1} $ 是收敛列;
               \item 若 $ (x_{n})_{n\geqslant1} $ 是Cauchy列, 则 $ (x_{n})_{n\geqslant1} $ 有界. (即 $ \exists x\in E\,\exists r>0\,((x_{n})_{n\geqslant1} \subset B(x, r)) $ )
          \end{enumerate}
     \end{Proposition}
     \begin{Definition}[完备]\label{def:完备}
          设 $ (E, d) $ 是度量空间, 若 $ (E, d) $ 中任一 Cauchy列收敛, 则称 $ E $ 是\textbf{完备度量空间}或 $ E $ 是\textbf{完备}\index{W!完备}的, 此时也说 $ d $ 是完备的. 
     \end{Definition}
     \begin{Example}
          实 Euclid 空间 $ \R^{n} $, 复 Euclid 空间 $ \C^{n} $ 依 Euclid度量是完备的; \Q, \J 依Euclid度量不完备. 
     \end{Example}
     \begin{Example}
          在空间 $ C[a, b] $ 上赋以度量
          \[
               d(x, y)=\max_{0\leqslant t\leqslant1}\abs{x(t)-y(t)}
          \]
          时, 度量空间 $ (C[a, b], d) $ 是完备的, 其中 $ C[a, b] $ 是定义在 $ [0, 1] $ 上的连续函数全体
     \end{Example}
     \begin{Proof}
          取  $ (x_{n})_{n\geqslant1} $ 是Cauchy列, 即
          \[
               \forall\varepsilon>0\,\exists n_{0}\in\N\,(n,m\geqslant n_{0}\Rightarrow d(x_{n},x_{m})<\varepsilon)
          \]
          即 $ \forall n, m\geqslant n_{0}\,\forall a\leqslant t\leqslant b\,(\abs{x_{n}(t)-x_{m}(t)}<\varepsilon) $ , 则有 $ (x_{n}(t))_{n\geqslant1} $ 是 $ \K $ 上的 Cauchy列\,(其中$\K=\R \text{\,或\,}\C$), 从而是收敛列, 并记
          \[
               x(t)=\lim_{n\to\infty}x_{n}(t), 
          \]
          则 $ \max\limits_{a\leqslant x\leqslant b}\abs{x(t)-x_{n}(t)}\to 0\,(n\to\infty) $, 从而 $ (x_{n})_{n\geqslant1} $  一致收敛到 $ x $ , 由  $ x_{n}\in C[a, b] $ 知 $ x\in C[a, b] $ , 即 $ (C[a, b], d) $ 完备. \qed
     \end{Proof}
     \begin{Remark}
          对 $ C[0, 1] $ , 若赋以度量 $ d(x, y)=\dint_{0}^{1}\abs{x(t)-y(t)}\diff x $, 则 $ (C[0, 1], d) $ 不完备, 反例如下:

          取 $ C[0, 1] $ 中的函数列 $ (x_{n}(t))_{n\geqslant1}=(t^{n})_{x\geqslant1} $, 易证  $ (x_{n})_{n\geqslant1} $ 是 Cauchy列, 但若记其极限函数是 $ x(t) $ 则有
          \[
               x(t)=\begin{cases}
                    0, & 0<x<1; \\
                    1, & x=1. 
               \end{cases}
          \]
          从而在空间给定的情形下, 完备性依赖与距离, 因此\textbf{完备不是一个拓扑概念}
     \end{Remark}
     \begin{Theorem}
          度量空间 $ (E, d) $ 完备的充分必要条件是对任意单调递减的非空子集列 $ (A_{n})_{n\geqslant1} $ 若 $ \lim\limits_{n\to\infty}\diam(A_{n})=0 $ , 则 $ \bigcap\limits_{n\geqslant1}A_{n} $ 是单点集. 其中 $ \diam(A)=\sup\limits_{x,y\in A}d(x, y) $ 是 $ A $  的\textbf{直径}. 
     \end{Theorem}
     \begin{Proof}
          \textsl{必要性} . 因为 $ \lim\limits_{n\to\infty}\diam(A_{n})=0 $ 则
          \[
               \forall \varepsilon>0\,\exists n_{0}\in\N\,(n\geqslant N\Rightarrow \diam(A_{n})<\varepsilon). 
          \]
          当 $ m>n>N $ 时, 取 $ x_{n}\in A_{n} $ , $ x_{m}\in A_{m} $ 有
          \[
               d(x_{m}, x_{n})<\diam(A_{n})<\varepsilon. 
          \]
          可以知道如此构造出的 $ (x_{n})_{n\geqslant1} $ 为Cauchy列, 又由 $ (E, d) $ 的完备性知 $ (x_{n})_{n\geqslant1} $ 收敛, 不妨设 $\lim\limits_{n\to\infty}x_{n}=x$ , 由 $ A_{n} $ 非空可知 $ x\in A_{n} $ , 故 $ x\in\bigcap\limits_{n\geqslant1}A_{n} $ . 

          再设 $ y\in\bigcap\limits_{n\geqslant1} A_{n} $ , 因 $ d(x_{n}, y)<\diam(A_{n})<\varepsilon $ 知 $ (x_{n})_{n\geqslant1} $ 收敛于 $ y $ , 由 $ E $ 是度量空间可知 $ E $ 是 Hausdorff空间, 从而$ x=y $, 故 $ \bigcap\limits_{n\geqslant1} A_{n} $ 为单点集. 

          \textsl{充分性} . 取 $ (x_{n})_{n\geqslant1} $ 是 $ E $ 中的 Cauchy列, 取 $ A_{n}=\left\{ x_{m}:m>n \right\} $, 则 $ (A_{n})_{n\geqslant1}) $ 是一个单调递减的非空闭子集列, 下面检验 $ \lim\limits_{n\to\infty}\diam(A_{n})=0 $ . 

          由 $ (x_{n})_{n\geqslant1} $ 是 Cauchy列可知
          \[
               \forall \varepsilon>0\, \exists n_{0}\in\N\,(m, n>n_{0}\Rightarrow d(x_{m}, x_{n})<\varepsilon),
          \]
          则 $ \forall x, y\in A_{n} $ , 取 $ x_{n}, x_{m}\in A_{n} $ 使得 $ d(x, x_{n}), d(x_{m})<\varepsilon $ 则
          \[
               d(x, y)\leqslant d(x, x_{n})+d(x_{n}, x_{m})+d(x_{m}, y) < 3\varepsilon         
          \]
          从而 $ \diam(A_{n})<3\varepsilon $, 也即 $\lim\limits_{n\to\infty} \diam(A_{n})=0 $. 

          由条件知 $ \bigcap\limits_{n\geqslant1}A_{n} $ 是单点集, 不妨记作 $ \{ x_{0} \} $ , 则 $ \lim\limits_{n\to\infty}x_{n}=x_{0} $, 从而 $ (E, d) $  是完备的. \qed
     \end{Proof}
     \begin{Theorem}
          设 $ (E, d) $ 是度量空间, $ A\subset E $. 
          \begin{enumerate}[(1)]
               \item 若 $ (A, d) $ 完备, 则 $ A $ 是 $ (E, d) $ 中的闭集;
               \item 若 $ (E, d) $ 完备且 $ A $ 是闭集, 则 $ (A, d) $ 完备. 
          \end{enumerate} 
     \end{Theorem}
     \begin{Proof}
          (1) 设 $ x $ 是 $ A $ 的凝聚点, 则存在 $ (x_{n})_{n\geqslant1}\subset A $, 使得 $ \lim\limits_{n\to\infty}x_{n}=x $ , 则 $ (x_{n})_{n\geqslant1} $ 是 $ A $ 中的 Cauchy列. 由 $ (A, d) $完备性可知 $ (x_{n})_{n\geqslant1} $ 收敛, 从而有 $ x\in A $ , 此时 $ A=\baro{A} $, 因此 $ A $ 是 $ E $ 中闭集. 

          (2) 任取 $ (x_{n})_{n\geqslant1} $ 是 $ A $ 中的Cauchy列, 则 $ (x_{n})_{n\geqslant1} $ 也是 $ E $ 中的Cauchy列, 而 $ (E, d) $ 完备, 故 $ (x_{n})_{n\geqslant1} $ 收敛, 记 $ \lim\limits x_{n}=x $, 由 $ A $ 是闭集知 $ x\in A $, 故 $ (A, d) $ 完备. \qed
     \end{Proof}
\section{连续映射与不动点定理}
     \subsection{连续与一致连续}
     \begin{Definition}[连续]\label{def:连续}
          设 $ E $, $ F $ 为拓扑空间,  $ f:E\to F, x\in E $, 若
          \[
               \forall V\in\CN(f(x))\,\exists U\in\CN(x)\,(f(U)\subset V),
          \]
          则称 $ f $ 在 $ x $ 处\textbf{连续}\index{L!连续}, 或等价地
          \[
               \forall V\in\CN(f(x))\,(f^{-1}(V)\in\CN(x))
          \]
          若 $ f $ 在 $ E $ 的每一点都连续, 则称 $ f $ 在 $ E $ 上连续. 
     \end{Definition}
     \begin{Remark}
          若 $ E $ , $ F $ 为度量空间, 若任取序列 $ (x_{n})_{n\geqslant1} $ 收敛到 $ x $ , 都有 $ \lim\limits_{n\to\infty}f(x_{n})=f(x) $, 则称 $ f $ 在 $ x $ 处连续. 
     \end{Remark}
     \begin{Proposition}\label{prop:连续映射的性质}
          连续映射具有以下性质:
          \begin{enumerate}[(1)]
               \item $ f $ 在 $ E $ 上连续 $ \Longleftrightarrow $ 对 $ F $ 上的开集 $ V $ , 有 $ f^{-1}(V) $ 是 $ E $ 中开集;
               \item $ f $ 在 $ E $ 上连续 $ \Longleftrightarrow $ 对 $ F $ 上的闭集 $ B $ , 有 $ f^{-1}(B) $ 是 $ E $ 中闭集;
               \item 连续映射的复合是连续映射;
               \item $ \tau $, $ \tau' $ 是 $ E $ 中拓扑,  $ \tau $ 是 $ \tau' $ 的强拓扑 $ \Longleftrightarrow $ $ \mathrm{id}_{E}:(E, \tau)\to(E, \tau') $ 连续. 
          \end{enumerate}
     \end{Proposition}
     \begin{Definition}[开映射, 同胚]\label{def:开映射, 同胚}
          设 $ E $, $ F $ 是拓扑空间,  $ f:E\to F $ 
          \begin{enumerate}[(1)]
               \item 若任取开集 $ U\subset E $ , 有 $ f(U) $ 是 $ F $ 中开集, 则称 $ F $ 是一个\textbf{开映射}\index{K!开映射};
               \item 若 $ f $ 是双射, 且 $ f\text{\,与\,}f^{-1} $ 连续, 则称 $ f $ 是\textbf{同胚映射}\index{T!同胚}, 若 $ E $, $ F $ 间有同胚映射, 则称 $ E\text{\,与\,}F $ \textbf{同胚}.  
          \end{enumerate}
     \end{Definition}
     \begin{Definition}[一致连续]\label{def:一致连续}
          设 $ (E, d) $ 与 $ (F, \delta) $ 是度量空间, 若 $ f:E\to F $ 满足
          \[
               \forall\varepsilon>0\,\exists\eta>0\,(d(x, y)<\eta\Rightarrow\delta(f(x), f(y))<\varepsilon),
          \]
          则称 $ f $ 是\textbf{一致连续映射}\index{Y!一致连续映射}. 
     \end{Definition}

     \begin{Remark}
          一致连续映射一定是连续映射, 它将 Cauchy列映成 Cauchy列.

          $ f(x)=x $ 在\R 上一致连续,  $ f(x)=x^{2} $ 在任何\R 的有限区间上一致连续, 而在无限区间上不一致连续.  
     \end{Remark}

     \begin{Theorem}[一致连续映射的扩展]\label{thm:一致连续映射的扩展}
          设 $ (E, d)\text{\,与\,}(F, \delta) $ 是度量空间,  $ (F, \delta) $ 完备,  $ E_{0}\text{\,是\,}E $ 的稠子集, 若 $ f:E_{0}\to F $ 一致连续, 则 $ f $ 可唯一地扩展成 $ (E, d)\text{\,到\,}(F, \delta) $ 的一致连续映射 $ \tilde{f}:E\to F $.  
     \end{Theorem}
     \begin{Proof}
          \textbf{(Step 1)} 构造 $ \tilde{f} $. 
          
          由 $ E_{0}\text{\,在\,}E $ 中稠密可知 
          \[
               \forall x\in E\,\exists (x_{n})_{n\geqslant1}\subset E_{0}(\lim_{n\to\infty}x_{n}=x), 
          \] 
          定义 $ \tilde{f}(x)=\lim\limits_{n\to\infty}f(x_{n}) $ , 需证明 $ (f(x_{n}))_{n\geqslant1} $ 也是 Cauchy列, 且 $ \tilde{f}(x) $ 不依赖 $ (x_{n})_{n\geqslant1} $ 的选取. 
          \begin{enumerate}[1\degree]
               \item 因为 $ (x_{n})_{n\geqslant1} $ 是 Cauchy列, $ f $ 一致连续, 知 $ (f(x_{n}))_{n\geqslant1} $ 也是 Cauchy列. 因为 $ (F, \delta) $ 完备, 故存在 $ \tilde{f}(x) $ 使得 $ \lim\limits_{n\to\infty}f(x_{n})=\tilde{f}(x) $.
               \item 设 $ (x'_{n})_{n\geqslant1}\subset E_{0} $, 且 $ \lim\limits_{n\to\infty}x'_{n}=x $ , 记 $ y'=\lim\limits_{n\to\infty}f(x'_{n}) $ , 令 $ y=\tilde{f}(x) $ 则
               \[
                    \begin{aligned}
                         \delta(y, y') & \leqslant \delta(y, f(x_{n}))+\delta(f(x_{n}), f(x'_{n}))+\delta(f(x'_{n}), y)\\
                         & \to 0\,(n\to\infty)
                    \end{aligned}
               \]
               即 $ y=y' $ , 从而 $ \tilde{f} $ 在 $ E $ 上是良定的. 
          \end{enumerate} 

          \textbf{(Step 2)} 证明 $ \tilde{f} $ 是一致连续映射. 

          对任意 $ \varepsilon>0 $ , 因为 $ f $ 一致连续, 则 $ d(x, y)<\eta/3 $ 时, $ \delta(f(x), f(y))<\varepsilon $, 而 
          \[
               \forall x', y'\in E\,\exists  (x_{n})_{n\geqslant1}\subset E_{0} \,\exists  (y_{n})_{n\geqslant1}\subset E_{0} \,(\lim_{n\to\infty}x_{n}=x'\land \lim_{n\to\infty}y_{n}=y' ), 
          \]  
          也即 
          \[
               \exists n_{0}\in\N\,\forall n\geqslant n_{0}\,(d(x_{n}, x')<\frac{\eta}{3}\land d(y_{n}, y')<\frac{\eta}{3}), 
          \] 
          则当 $ d(x', y')<\eta/3 $ 时, 有
          \[
               \begin{aligned}
                    d(x_{n}, y_{n}) & \leqslant d(x_{n}, x')+d(x', y')+d(y', y_{n})\\
                    & < \frac{\eta}{3}+\frac{\eta}{3}+\frac{\eta}{3}=\eta.
               \end{aligned}
          \]
          又因为 $ f $ 一致连续, 故 $ \delta(f(x_{n}), f(y_{n}))<\varepsilon $, 而又有
          \[
               \tilde{f}(x')=\lim_{n\to\infty}f(x_{n}), \tilde{f}(y')=\lim_{n\to\infty}f(y_{n})
          \]
          即
          \[
               \exists n_{1}\in\N\,\forall n\geqslant n_{1}\,(\delta(\tilde{f}(x_{n}), \tilde{f}(x'))<\varepsilon\land \delta(\tilde{f}(y_{n}), \tilde{f}(y'))<\varepsilon)
          \]
          则
          \[
               \begin{aligned}
                    \delta(\tilde{f}(x'), \tilde{f}(y')) & \leqslant \delta(\tilde{f}(x_{n}), \tilde{f}(x'))+\delta(\tilde{f}(y_{n}), \tilde{f}(x_{n}))+\delta(\tilde{f}(y_{n}), \tilde{f}(y'))\\
                    & <\varepsilon+\varepsilon+\varepsilon=3\varepsilon. 
               \end{aligned}
          \]
          从而 $ \tilde{f} $ 是一致连续的. 

          \textbf{(Step 3)} 证明 $ \tilde{f} $ 的唯一性. 

          
          设 $ {f}' $ 是 $ f $ 的另一个扩张, 由 $ E_{0} $ 稠密, 知
          \[
               \forall x\in E\,\exists (x_{n})_{n\geqslant1} \subset E_{0}\,(\lim_{n\to\infty}x_{n}=x),
          \]
          则有
          \[
               f'(x)=\lim_{n\to\infty}f'(x_{n})=\lim_{n\to\infty}f(x_{n})=\tilde{f}(x), 
          \]
          从而 $ f'=\tilde{f} $. \qed
     \end{Proof}

     \subsection{压缩映照原理}

     \begin{Definition}[H\"older映射]\label{def:H\"older映射}
          设 $ (E, d) $ 与 $ (F, \delta) $ 是度量空间,  $ f:E\to F $, $ 0<\alpha<1 $ , 
          \begin{enumerate}[(1)]
               \item 若
               \[
                    \exists \lambda>0\,\forall x, y\in E\,(\delta(f(x), f(y))\leqslant\lambda d(x, y)^{\alpha}),
               \]
               则称 $ f $ 是阶数为 $ \alpha $ 的\textbf{H\"older映射}\index{H!H\"older映射};
               \item 若 $ f $ 为阶数为 $ 1 $ 的 H\"older映射, 则称 $ f $ 是\textbf{Lipschitz映射}\index{L!Lipschitz映射}, 并称使得不等式成立的最小常数 $ \lambda $是 $ f $ 的\textbf{Lipschitz常数}.
               \item 若 $ f $ 的Lipschitz常数 $ \lambda<1 $ , 则称 $ f $ 是\textbf{压缩映射.}
          \end{enumerate} 
     \end{Definition}

     \begin{Remark}
          由定义知压缩映射 $ \subset $ Lipschitz映射 $ \subset $ H\"older映射 $ \subset $ 一致连续映射.
     \end{Remark}

     \begin{Example}
          $ f(x)=\sin x $ 是 Lipschitz映射, 且其 Lipschitz常数 $ \lambda=1 $ ;

          $ f(x)=\abs{x} $ 也是 Lipschitz映射, 其 Lipschitz常数 $ \lambda=1 $ ;
          
          $ f(x)=\sqrt{x} $ 不是 Lipschitz映射, 但它是阶为 $ 1/2 $ 的 H\"older映射, 因为 $ f'(x)=\frac{1}{2\sqrt{x}}\to\infty\,(x\to 0) $ , 但
          \[
               \abs{\sqrt{x}-\sqrt{y}}=\frac{\abs{x-y}}{\abs{\sqrt{x}+\sqrt{y}}}=\frac{\sqrt{x-y}}{\sqrt{x}+\sqrt{y}}\cdot\abs{x-y}^{1/2}\leqslant\abs{x-y}^{1/2}.
          \]
     \end{Example}

     \begin{Theorem}[压缩映照原理]\label{thm:压缩映照原理}
          设 $ (E, d) $ 是完备度量空间,  $ f:E\to E $ 是压缩映射, 则 $ f $ 有唯一不动点, 即 $ \exists!x\,(f(x)=x) $ .
     \end{Theorem}

     \begin{Proof}
          任取 $ x_{1}\in E $ , 并归纳地定义 $ x_{n+1}=f(x_{n}) $ , 则 $ (x_{n})_{n\geqslant1} $ 是 $ E $ 中序列, 往证 $ (x_{n})_{n\geqslant1} $ 是 Cauchy列, 令 $ \lambda $ 是 $ f $ 的Lipschitz常数, 则 $ \lambda<1 $ , 此时
          \[
               \begin{aligned}
                    d(x_{n+1}, x_{n}) & = d(f(x_{n}), f(x_{n-1})) \leqslant \lambda d(x_{n}, x_{n-1}) \\
                    & =\lambda d(f(x_{n-1}), f(x_{n-2}))\leqslant \lambda^{2} d(x_{n-1}, x_{n-2}) \\
                    & \leqslant \lambda^{n} d(x_{2}, x_{1}).
               \end{aligned}
          \]
          则 $ \forall p\in \N $ , 有
          \[
               d(x_{n+p}, x_{n})\leqslant\sum_{i=n+1}^{n+p}d(x_{i}, x_{i-1})\leqslant\sum_{i=n+1}^{n+p}\lambda^{i-2}d(x_{2}, x_{1})=\frac{\lambda^{n-1}}{1-\lambda}d(x_{2}, x_{1})\to 0\,(n\to\infty)
          \]
          从而 $ (x_{n})_{n\geqslant1} $ 是 Cauchy列, 因为 $ E $ 完备, 则存在 $ x\in E $ , 使得 $ \lim\limits_{n\to \infty}x_{n}=x $ , 此时
          \[
               f(x)=f(\lim_{n\to\infty}x_{n})=\lim_{n\to\infty}f(x_{n})=\lim_{n\to\infty}x_{n+1}=x,
          \]
          即 $ x $ 是 $ f $ 的不动点.

          若 $ y\in E $ 也是 $ f $ 的不动点, 则由
          \[
               d(x, y)=d(f(x), f(y))\leqslant\lambda d(x, y)\,(\lambda<1)
          \]
          知 $ d(x, y)=0 $ , 即 $ x=y $, 故不动点唯一.\qed 
     \end{Proof}



\section{度量空间的完备化}
	\begin{Definition}[等距同构]\label{def:等距同构}
           设 $ (E, d) $ 与 $ (F, \delta) $ 是度量空间, 设 $ f:(E, d)\to(F, \delta) $, 若
           \[
                \forall x, y\in E\,\big(d(x, y)=\delta(f(x), f(y))\big),
           \] 
           则称 $ f $ 是 $ (E, d) $ 到 $ (F, \delta) $ 的\textbf{等距映射}; 若 $ f $ 是双射, 则称 $ f $ 是 $ (E, d) $ 到 $ (F, \delta) $ 的\textbf{等距同构映射}\index{D!等距同构映射}.
     \end{Definition}

     \begin{Example}
          设 $ E:[0,\infty), d(x, y)=\abs{x-y} $, $ F=[0, \infty), \delta(x, y)=\abs{x^{2}-y^{2}} $, 则 $ f:E\to F, x\mapsto \sqrt{x} $ 是等距同构. 
     \end{Example}

     下面证明本节的主要定理

     \begin{Theorem}[度量空间的完备化]
           设 $ (E, d) $ 是度量空间, 则存在等距意义下唯一的完备度量空间 $ (\widehat{E}, \hat{d}) $ 满足
           
           (i) $ E\subset \widehat{E} $;\hspace{6em}(ii) $ \hat{d}|_{E}=d $;\hspace{6em}  (iii)$ E $ 在 $ \widehat{E} $ 中稠密. 

           \noindent 并称 $ (\widehat{E}, \hat{d}) $ 是 $ (E, d) $ 的\textbf{完备化空间}.
     \end{Theorem}
     \begin{Proof}

          \textbf{(Step 1)} 构造空间 $ (\widehat{E}, \hat{d}) $ .

          取 $ \widetilde{E} $ 是 $ E $ 中 Cauchy列的全体, 并在 $ \widetilde{E} $ 上取等价关系
          \[
               (x_{n})_{n\geqslant1}\sim (y_{n})_{n\geqslant1}\Longleftrightarrow \lim_{n\to\infty}d(x_{n},y_{n})=0,
          \]
          并取 $ \widehat{E}=\widetilde{E}/\sim $ . 因为 $ \forall a\in E $ , 存在常数列  $ (a)_{n\geqslant1} $, 使得 $ \lim\limits_{n\to\infty}a=a $, 记其等价类为 $ \hat{a} $ , 则存在映射 $ \iota :E\to\widehat{E}, a\mapsto \hat{a} $ 是一个单射, 由此知 $ E\subset\widehat{E} $. 

          定义 $ \widehat{E} $ 上的度量 $ \hat{d} $ , 设 $ (x_{n})_{n\geqslant1} $ 与 $ (y_{n})_{n\geqslant1} $ 分别是 $ \hat{x} $, $ \hat{y} $ 的代表元, 则
          \[
               \hat{d}(\hat{x}, \hat{y})=\lim_{n\to\infty}d(x_{n},y_{n}).
          \]
          因 $ \hat{d} $ 的非负性, 正定性, 对称性与三角不等式由 $ d $ 的相应性质可证, 故只需证明 $ \lim\limits_{n\to\infty}d(x_{n}, y_{n}) $ 存在且不依赖代表元的选取.

          \begin{enumerate}[(1)]

               \item 由 $ (x_{n})_{n\geqslant1} $ , $ (y_{n})_{n\geqslant1} $ 都是 Cauchy列, 当 $ m, n\to\infty $ 时
               \[
                    \begin{aligned}
                         \abs{d(x_{n}, y_{n})-d(x_{m}, y_{m})} & \leqslant \abs{d(x_{n}, y_{n})-d(x_{m}, y_{n})+ d(x_{m}, y_{n})-d(x_{m}, y_{m})}\\
                         & \leqslant \abs{d(x_{n}, y_{n})-d(x_{m}, y_{n})}+\abs{d(x_{m}, y_{n})-d(x_{m}, y_{m})}\\
                         & \leqslant d(x_{n}, x_{m})+d(y_{n}, y_{m})\to 0
                    \end{aligned}
               \]
               则 $ d(x_{n}, y_{n})_{n\geqslant1} $ 是实的Cauchy列,  故 $ \lim\limits_{n\to\infty}d(x_{n}, y_{n}) $ 存在;

               \item 再设 $ (x'_{n})_{n\geqslant1} $ , $ (y'_{n})_{n\geqslant1} $ 也是 $ \hat{x} $ , $ \hat{y} $ 的代表元, 则 $ n\to\infty $ 时
               \[
                    \abs{d(x'_{n}, y'_{n})-d(x_{n}, y_{n})}\leqslant d(x'_{n}, x_{n})+d(y'_{n}, y_{n})\to0.
               \]
               从而 $ \hat{d} $ 是一个度量, 且 $ \forall a, b\in E $ , 有 $ \hat{d}(a, b)=d(a, b) $ , 从而 $ \hat{d}|_{E}=d $ .
          \end{enumerate}
          接下来证明 $ E $ 在 $ \widehat{E} $ 中稠密. 任取 $ \hat{x}\in\widehat{E} $ , 设 $ (x_{n})_{n\geqslant1} $ 是它的一个代表元, 对每一个 $ n\in\N $ , 定义 $ \baro{x}_{n} $ 表示常数列 $ (x_{n})_{m\geqslant1} $(可看成与 $ x_{n}\in E $ 为同一元素) , 那么
           \[
               \lim_{n\to\infty}\hat{d}(\hat{x}, \baro{x}_{n})=\lim_{n\to\infty}\lim_{m\to\infty}d (x_{m}, d_{n})\to 0
          \]
          从而 $ E $ 在 $ \widehat{E} $ 中稠密. 

          \textbf{(Step 2)} 证明 $ (\widehat{E}, \hat{d}) $ 完备.

          取 $ (\hat{x}_{(n)})_{n\geqslant1} $ 为 $ \widehat{E} $ 中的Cauchy列, 由 $ E $ 在 $ \widehat{E} $ 中稠密可知
          \[
               \forall n\in \N\,\exists x_{n}\in E\,\left(\hat{d}(\hat{x}_{(n)}, \baro{x}_{n})<\frac{1}{n}\right)
          \]
          于是可得
          \[
               \begin{aligned}
                    d(x_{n}, x_{m}) & =\hat{d}(\baro{x}_{n}, \baro{x}_{m}) \leqslant \hat{d}(\baro{x}_{n}, \hat{x}_{(n)})+\hat{d}(\hat{x}_{(m)}, \hat{x}_{(n)})+\hat{d}(\baro{x}_{m}, \hat{x}_{(m)})\\
                    & < \frac{1}{n}+\hat{d}(\hat{x}_{(m)}, \hat{x}_{(n)})+\frac{1}{m}\to0\,(m, n\to\infty)
               \end{aligned}
          \]
          即可知 $ (x_{n})_{n\geqslant1} $ 是 Cauchy列, 则有 $ (x_{n})_{n\geqslant1}\in\widetilde{E} $ , 设 $ (x_{n})_{n\geqslant1} $ 在 $ \widehat{E} $ 中的代表元为 $ \hat{x} $ , 则有
          \[
               \hat{d}(\hat{x}_{(n)}, \hat{x})\leqslant\hat{d}(\hat{x}_{(n)}, \baro{x}_{n})+\hat{d}(\hat{x}, \baro{x}_{n})<\frac{1}{n}+\hat{d}(\hat{x}, \baro{x}_{n})\to 0\,(n\to\infty),
          \]
          因此, $ \lim\limits_{n\to\infty}\hat{x}_{(n)}=\hat{x} $ , 所以 $ (\widehat{E}, \hat{d}) $ 是一个完备的度量空间.

          \textbf{(Step 3)} 最后证明这种完备化在等距意义下一唯一的.

          我们可以把 $ (\widehat{E}, \hat{d}) $ 满足的条件归纳为
          \begin{enumerate}[(1)]
               \item $ \iota:E\to\widehat{E} $ 是单射, 可记为 $ E\hookrightarrow \widehat{E} $;
               \item $ \hat{d}|_{\iota(E)}=d $ : 任取 $ a, b\in E $ , 有 $ \hat{d}(\iota(a), \iota(b))=d(a, b) $ ;
               \item $ \iota(E) $ 在 $ \widehat{E} $ 中稠密 .
          \end{enumerate}
          也就是说 $ E $ 与 $ \widehat{E} $ 上的稠密子集 $ \iota(E) $ 等距同构.

          设 $ (E', d') $ 是另一个满足如上条件的度量空间, 并记 $ E $ 到 $ E' $ 的等距映射为 $ \iota' $ , 定义映射
          \[
               \begin{aligned}
                    f: \iota(E)  \to E'\\
                    \hat{a} \mapsto \iota'(a)
               \end{aligned}
          \]
          由 (1), (2) 可知 $ f $ 是等距的的, 则它是一致连续的. 又由 (3) 知 $ \iota(E) $ 在 $ \widehat{E} $ 中稠密, 以及 $ (E', d') $ 完备, 根据定理\ref{thm:一致连续映射的扩展}可知 $ f $ 可唯一地扩展成 $ \widehat{E} $ 到 $ E' $ 的一致连续映射 $ \tilde{f} $. 对任意 $ \hat{x}, \hat{y}\in \widehat{E} $ , 其代表元分别为 $ (x_{n})_{n\geqslant1} $与 $ (y_{n})_{n\geqslant1} $, 则有
          \[
               \begin{aligned}
                    d'(\tilde{f}(\hat{x}), \tilde{f}(\hat{y})) & = d'(\lim_{n\to\infty}f(x_{n}), \lim_{n\to\infty}f(y_{n}))\\
                    & = \lim_{n\to\infty} d'(f(x_{n}), f(y_{n}))\\
                    & = \lim_{n\to\infty} d(x_{n}, y_{n})=\hat{d}(\hat{x}, \hat{y}).
               \end{aligned}
          \]
          故 $ \tilde{f}:\widehat{E}\to E' $ 也是等距映射, 因此也是单射. 由 $ \iota'(E) $ 在 $ E' $ 中稠密可知 $ \tilde{f} $ 是满射, 故 $ \tilde{f} $ 是双射. 所以在等距同构意义下, $ E $ 的完备化度量空间是唯一的. \qed
     \end{Proof}

     \begin{Remark}
          若 $ (E, d) $ 完备, 则 $ \widehat{E}=E $ , 这说明完备化只需做一次, 且完备化空间是包含 $ E $ 的最小完备化空间.
     \end{Remark}

\section{紧性}
     \subsection{紧性, 紧空间}
     \begin{Definition}[紧性]\label{def:紧性}
           设 $ E $ 是拓扑空间,
           \begin{enumerate}[(1)]
                \item 若 $ E $ 上的开集族 $ (O_{i})_{i\in \alpha} $ 满足 $ E=\bigcup_{i\in\alpha}O_{i} $ , 则称 $ (O_{i})_{i\in\alpha} $ 是 $ E $ 的一个\textbf{开覆盖}.
                \item 若对 $ E $ 的任意开覆盖都有有限子覆盖, 则称 $ E $ 是\textbf{紧}的\index{J!紧}.
           \end{enumerate}
     \end{Definition}

     \begin{Proposition}\label{prop:紧性的另一刻画}
           $ E $ 紧的充分必要条件是任意闭集族 $ (F_{i})_{i\in\alpha}\subset E $, 若 $ \forall \beta\in\fin\alpha $ 都有 $ \bigcap_{i\in\beta}F_{i}\neq \varnothing $ , 则有 $ \bigcap_{i\in\alpha}F_{i}\neq\varnothing $. 
     \end{Proposition}

     \begin{Proof}
          \textsl{必要性}. 用反证法, 假设 $ \bigcap_{i\in\alpha}F_{i}=\varnothing $ , 即 $ \bigcup_{i\in\alpha}F_{i}^{c}=E $ , 即 $ (F_{i}^{c})_{i\in\alpha} $ 是 $ E $ 的一个开覆盖, 因为 $ E $ 是紧的,  所以存在有限集 $ \beta\in\fin\alpha $ ,  使得 $ \bigcup_{i\in\beta}F_{i}^{c}=E $ , 即 $ \bigcap_{i\in\beta}F_{i}^{c}=\varnothing $ , 矛盾.

          \textsl{充分性}. 任取 $ (O_{i})_{i\in\alpha} $ 是 $ E $ 的开覆盖, 即 $ \bigcup_{i\in\alpha}O_{i}=E $ , 则 $ \bigcap_{i\in\alpha}O_{i}^{c}=\varnothing $ , 则存在有限集 $ \beta\in\fin \alpha $, 使得 $ \bigcap_{i\in\beta}O_{i}^{c}=\varnothing $ , 即 $ \bigcup_{i\in\beta}O_{i}=E $ , 从而 $ E $ 是紧的.\qed
     \end{Proof}

     \begin{Proposition}
          设 $ E $ 是紧的拓扑空间, $ (F_{n})_{n\geqslant1} $ 是 $ E $ 中单降的非空闭集族, 则 $ \bigcap_{n\geqslant1}F_{n}\neq\varnothing $. 
     \end{Proposition}

     \begin{Proof}
          对任意 $ j_{i}\in\N\,(i=1, 2, 3,\dots k) $ , 都有 $ \bigcap_{i=1}^{k}F_{i}=F_{\max\{ j_{i} \}}\neq\varnothing $ , 故 $ \bigcap_{n\geqslant1}F_{n}\neq\varnothing $ .\qed
     \end{Proof}

     \begin{Theorem}
           设 $ E $ 是拓扑空间,  $ F\subset E $ 是子空间, 则 $ F $ 紧的充分必要条件是若 $ E $ 中的开集族 $ (O_{i})_{i\in\alpha} $ 满足 $ F\subset \bigcup_{i\in\alpha}O_{i} $ , 则存在有限集 $ \beta\in\fin\alpha $ , 使得 $ F\subset\bigcup_{i\in\beta}O_{i} $ (也就是说, $ F $ 任一在 $ E $ 中的开覆盖 $ (O_{i})_{i\in\alpha} $ 都存在有限的子覆盖).
     \end{Theorem}
     
     \begin{Proof}
          设 $ F\subset E $ , $ (O_{i})_{i\in\alpha} $ 是 $ E $ 中的开集族, 且满足 $ F\subset \bigcup_{i\in\alpha}O_{i} $ . 令
          \[
               U_{i}=F\cap O_{i}, i\in\alpha,
          \]
          则 $ (U_{i})_{i\in\alpha} $ 是空间 $ F $ 上的开覆盖, 反之,  $ F $ 的任意开覆盖都可以表示成以上形式.

          \textsl{必要性}. 若 $ F $ 是紧子空间, 那么存在有限集 $ \beta\in\fin\alpha $ , 使得 $ F=\bigcup_{i\in\beta}U_{i} $ , 所以有 $ F\subset\bigcup_{i\in\beta}O_{i} $.

          \textsl{充分性}. 若存在有限集 $ \beta\in\fin\alpha $, 使得 $ F\bigcup_{i\in\beta}O_{i} $ , 则
          \[
               F=F\cap\bigg( \bigcup_{i\in\beta}O_{i} \bigg)=\bigcap_{i\in\beta}U_{i},
          \]
          所以 $ F $ 是紧子空间.\qed
     \end{Proof}

     \begin{Proposition}\label{prop:紧->闭}
          设 $ E $ 是 Hausdorff空间且 $ F\subset E $ 紧, 则 $ F $ 是闭集. 
     \end{Proposition}

     \begin{Proof}
          令 $ x\in F^{c} $ , 由 $ E $ 是 Hausdorff空间, 则对任意 $  y\in F $ , 存在 $ U_{y}\in\CN(x) $, $ V_{y}\in\CN(y) $ , 其中 $ U_{y} $, $ V_{y} $ 是开集, 使得 $ U_{y}\cap V_{y}=\varnothing $ , 则 $\{ V_{y}:y\in F \}$ 是 $ F $ 的开覆盖, 因为 $ F $ 是紧集, 所以存在 $ y_{1}, y_{2}, \dots, y_{n} $, 使得 $ \{ V_{y_{i}}: i=1, 2, \dots, n \} $ 是 $ F $ 的开覆盖, 令 $ U=\bigcap_{i=1}^{n}U_{y_{i}} $ 是开集 (同时也是 $ x $ 的开邻域), 而 $ U\cap V_{y_{i}}=\varnothing\,(i=1, 2, \dots, n) $ , 从而 $ U\cap F=\varnothing $. 因此 $ F^{c} $ 是开集, 即 $ F $ 是闭集.\qed
     \end{Proof}

     \begin{Proposition}
          设 $ E $ 是紧 Hausdorff空间, 则 $ F\subset E $ 紧的充分必要条件是 $ F $ 是闭集. 
     \end{Proposition}
     
     \begin{Proof}
          \textsl{必要性}. 由命题\ref{prop:紧->闭}可得.

          \textsl{充分性}. 设 $ E $ 是紧的, $ F\subset E $ 为闭集, 设 $ (F_{i})_{i\geqslant1} $ 是 $ F $ 的一族闭子集, 且具有有限交性质 (即 $ \forall \beta\in\fin\alpha $, $ \bigcap_{i\in\beta}F_{i}\neq\varnothing $), 则由 $ F $ 是闭集,  $ F_{i} $ 也是闭集. 由命题\ref{prop:紧性的另一刻画}知 $ \bigcap_{i\in\alpha}F_{i}\neq\varnothing $ , 从而 $ F $ 是紧的.\qed
     \end{Proof}

     \begin{Proposition}
          设 $ E $ 是 Hausdorff空间,  $ (K_{i})_{i\in\alpha} $ 是 $ E $ 中紧集族. 若 $ \bigcap_{i\in\alpha}K_{i}=\varnothing $, 则存在有限集 $ \beta\in\fin\alpha $, 使得 $ \bigcap_{i\in\beta}K_{i}=\varnothing $.   
     \end{Proposition}
     
     \begin{Proof}
          在集族 $ (K_{i})_{i\in\alpha} $ 中任取一元素, 记为 $ K_{0} $, 因为 $ \bigcap_{i\in\alpha}K_{i}=\varnothing $, 
          知 $ \bigcup_{i\in\alpha}K_{i}^{c}=E $, 由 $ E $ 是 Hausdorff的, 故 $ K_{i}^{c} $ 是开集, 从而 $ \bigcup_{i\in\alpha}K_{i}^{c} $ 是 $ K_{0} $ 的开覆盖, 由于 $ K_{0} $ 是紧的, 故有有限集 $ \beta\in\fin\alpha $, 使得 $ \bigcup_{i\in\beta}K_{i}^{c}\subset K_{0} $ 则
          \[
               K_{0}\cap \bigg( \bigcup_{i\in\beta}K_{i} \bigg)^{c}=\varnothing \Rightarrow K_{0}\cap\bigcap_{i\in\beta}K_{i}=\varnothing.
          \]
          即存在有限个子集的交为空集.\qed
     \end{Proof}

     \begin{Proposition}
          设 $ E $ 是紧的 Hausdorff空间, 则 $ E $ 中每一点都有紧邻域基.
     \end{Proposition}

     \begin{Proof}
          任取 $ x\in E $, 取 $ V $ 是 $ x $ 的开邻域, 则 $ \baro{V} $ 闭, 因为 $ E $ 是紧的 Hausdorff空间, 从而 $ \baro{V} $ 是 $ x $ 的一个紧邻域. 往证 $ \forall U\in\CN(x) $ 是开邻域, $ \exists V\in\CN(x) $ 是开邻域且 $ \baro{V}\subset U $. 
          
          注意到 $ U $ 是开集, 则 $ U^{c} $ 为闭集, 同时可知 $ U^{c} $ 为紧集. $ \forall y\in U^{c} $, 存在  $ U_{y}, V_{y} $分别为 $ y, x $ 的开邻域, 使得 $ U_{y}\cap V_{y}=\varnothing $. 因为 $ U^{c} $ 紧, 则存在 $ y_{1}, y_{2}, \dots, y_{n} $, 使得 $ U^{c}\subset \bigcup_{i=1}^{n}U_{y_{i}} $, 再令 $ V=\bigcap_{i=1}^{n}V_{y_{j}} $, 则有
          \[
               V\cap\left( \bigcup_{i=1}^{n}U_{y_{i}} \right)=\varnothing\Rightarrow \baro{V}\cap\left( \bigcup_{i=1}^{n}U_{y_{i}} \right)\Rightarrow \baro{V}\subset U.
          \]
          从而 $ x $ 有紧邻域基.\qed
     \end{Proof}

     \begin{Proposition}
          设 $ E $ 是紧集,  $ F $ 是拓扑空间,  $ f:E\to F $ 连续, 则 $ f(E) $ 是 $ F $ 中的紧集(即连续映射将紧集映成紧集)
     \end{Proposition}

     \begin{Proof}
          取 $ (O_{i})_{i\in\alpha} $ 是 $ f(E) $ 的开覆盖, 则 $ (f^{-1}(O_{i}))_{i\in\alpha} $ 是 $ E $ 的开覆盖, 因为 $ E $ 紧, 则存在有限集 $ \beta\in\fin\alpha $, 使得 $ \bigcup_{i\in\beta}f^{-1}(O_{i})=E $, 故 $ \bigcup_{O_{i}}=f(E) $. 即 $ f(E) $ 紧.\qed
     \end{Proof}

     \begin{Corollary}
          设 $ E $ 是紧集, $ f:E\to\R $ 是连续映射, 则 $ f $ 在 $ f $ 在 $ E $ 上有界且取得上确界与下确界.
     \end{Corollary}

     \begin{Proposition}
          设 $ E $ 是紧集,  $ F $ 是Hausdorff空间,  $ f:E\to F $ 是连续的单射, 则有 $ f:E\to f(E) $ 是同胚映射.
     \end{Proposition}

     \begin{Proof}
          只需证明 $ f^{-1}:F\to E $ 是连续映射, 即 $ f $ 把闭集映成闭集. 设 $ A $ 是 $ E $ 中的闭集, 因为 $ E $ 紧, 所以 $ A $ 紧, 因为 $ f $ 是连续映射, 则 $ f(A) $ 紧. 又因为 $ F $ 是 Hausdorff的, 则 $ f(A) $ 是 $ f(E) $ 中闭集, 故 $ f^{-1} $ 是连续映射, 故 $ f:E\to f(E) $ 是同胚映射.\qed
     \end{Proof}

     \begin{Definition}[局部紧]\label{def:局部紧}\index{J!局部紧}
           设 $ E $ 是拓扑空间, 若 $ E $ 中每一点都有一个紧邻域, 则称 $ E $ 是\textbf{局部紧}空间.
     \end{Definition}

     \begin{Example}
          \R 与 $ (0, 1] $ 是局部紧的.
     \end{Example}

     \begin{Remark}
          若 $ E $ 是局部紧的 Hausdorff空间, 则 $ E $ 中每一点都有一个紧邻域基.
     \end{Remark}

     \begin{Theorem}[Urysohn引理]
           设 $ E $ 是局部紧的 Hausdorff空间. 设 $ A $ 和 $ B $ 是$ E $ 中两个不相交的非空闭子集, 并设其中的一个是紧集, 那么存在一个连续的函数 $ f:E\to[0,1] $ 使得 $ f|_{A}=0 $, 且 $ f|_{B}=1 $.
     \end{Theorem}

     \begin{Theorem}
           设 $ (E, d), (F, \delta) $ 是度量空间, 且 $ E $ 紧. 若 $ f:E\to F $ 连续, 则 $ f $ 一致连续.
     \end{Theorem}

     \begin{Proof}
          因为 $ f $ 连续, 则 
          \[
               \forall x\in E\,\forall \varepsilon>0\,\exists\eta_{x}>0\,(y\in B(x, \eta_{x})\Rightarrow \delta(f(x), f(y))<\varepsilon).
          \]
          设 $\left\{ B(x, \eta_{x}/2) : x\in E \right\}$ 是 $ E $ 的一个开覆盖, 因为 $ E $ 是紧集, 则存在 $ x_{1}, x_{2},\dots,x_{n} $ 使得
          \[
               \bigcup_{i=1}^{n} B\left(x_{i}, \frac{\eta_{x_{i}}}{2}\right)=E.
          \]
          令 $ \eta=\min\limits_{i=1, 2, \dots, n}\{ x_{i} \} $, 对 $ \forall y\in E $, 当 $ d(x', y)<\eta $ 时, 存在 $ j\in\N, 1\leqslant j\leqslant n $, 使得 $ x'\in B(x_{j}, \eta_{x_{j}}) $, 则
          \[
               d(x_{j}, y) \leqslant d(y, x')+d(x', x_{j})\leqslant\eta_{x_{j}}/2+\eta<\eta_{x_{j}}.
          \]
          故 $ y\in B(x_{j}, \eta_{x_{j}}) $, 所以有 $ \delta(f(y), f(x_{j}))<\varepsilon $, 则
          \[
               \delta(f(x'), f(y))\leqslant \delta(f(x'), f(x_{j}))+\delta(f(x_{j}), f(y))<2\varepsilon.
          \]
          因此 $ f $ 一致连续.\qed
     \end{Proof}

     \subsection{紧度量空间的刻画}
    
     \begin{Example}
     在实直线$ \mathbb{R} $上, 紧致与有界闭等价, 且有以下的等价表述:
     \begin{enumerate}[(1)]
     \item 无限子集必有凝聚点;
     \item 任一序列必有收敛子列;
     \item 完备且可被有限多个半径为$ \varepsilon $的开球覆盖.
     \end{enumerate}
     \end{Example}
     
     类似地, 度量空间的紧性也可以如此刻画:
     
     \begin{Theorem}[紧等价]\label{thm:紧等价}\index{L!列紧}\index{X!序列紧}\index{Y!预紧}
     下面的命题等价:
     \begin{enumerate}[(1)]
     \item $ (E,d) $是紧空间;
     \item $ (E,d) $是\textbf{列紧}的, 即$ (E,d) $中的无限子集必有凝聚点;
     \item $ (E,d) $是\textbf{序列紧}的, 即$ (E,d) $中任一序列都有收敛子列;
     \item $ (E,d) $是完备的且是\textbf{预紧}的, 即$ \forall\varepsilon>0 $, $ E $可被有限多个以$ \varepsilon $为半径的开球覆盖. (若$ F\subset E $, $ (B(x,\varepsilon))_{x\in F} $覆盖$ E $, 则称$ F $是$ E $的一个$ \varepsilon $-\textbf{网})
     \end{enumerate}
     \end{Theorem}

     \begin{Proof}
     (1) $ \Rightarrow $ (2) : 用反证法, 设存在无限集$ F $无凝聚点, 即
     \[
     \forall x\in E\,\exists\eta_x>0\,(B(x,\eta_x)\cap(F\sm\{x\})=\varnothing),
     \]
     则$ (B(x,\eta_x))_{x\in E} $是$ E $的开覆盖. 因为$ E $是紧的, 存在$ x_1,x_2,\cdots,x_n $使得
     \[
     \bigcup_{j=1}^nB(x_j,\eta_{x_j})=E,
     \]
     而$ B(x_j,\eta_{x_j})\cap(F\sm\{x_j\}=\varnothing $, 从而$ B(x_j,\eta_{x_j}) $至多包含$ F $中的一个点, 则$ E=\bigcup_{j=1}^nB(x_j,\eta_{x_j}) $至多包含$ F $中的有限个点, 这与$ F $是无限集矛盾.
     
     (2) $ \Rightarrow $ (3) : 任取$ (x_n)_{n\geqslant 1} $是$ E $中的序列, 若$ (x_n)_{n\geqslant 1} $中只取有限多个不同值, 则$ (x_n)_{n\geqslant 1} $有收敛子列是显然的. 若$ (x_n)_{n\geqslant 1} $中有无限多个不同值, 则$ A=\{ x_n : n\geqslant 1 \} $是无限集, 由$ E $列紧知$ A $有凝聚点$ x $, 则存在$ (x_{n_k})_{k\geqslant 1}\subset(x_n)_{n\geqslant 1} $使得$ \lim\limits_{k\to\infty}x_{n_{k}}=x $.
     
     (3) $ \Rightarrow $ (4) : 任取$ E $中的Cauchy列$ (x_n)_{n\geqslant 1} $, 由$ E $序列紧可知$ (x_n)_{n\geqslant 1} $存在收敛子列, 从而$ (x_n)_{n\geqslant 1} $是收敛列, 这说明$ (E,d) $是完备的.
     
     用反证法, 反设$ E $不是预紧的, 即存在$ \varepsilon_0>0 $使得$ E $的开覆盖$ (B(x,\varepsilon))_{x\in E} $不存在有限子覆盖. 取$ x_1\in E $, 由$ B(x_1,\varepsilon)\ne E $, 可取$ x_2\in E\sm B(x_1,\varepsilon_0) $. 依此继续下去, 取
     \[
     x_n\in E\sm\left( \bigcup_{i=1}^{n-1}B(x_i,\varepsilon_0) \right)
     \]
     可得一$ E $中的序列$ (x_n)_{n\geqslant 1} $满足
     \[
     \forall j\ne k\,(d(x_j,x_k)\geqslant\varepsilon_0),
     \]
     即$ (x_n)_{n\geqslant 1} $不存在收敛子列, 矛盾.
     
     (4) $ \Rightarrow $ (3) : 因为$ E $是完备的, 只需证明任意序列$ (x_n)_{n\geqslant 1}\subset E $存在Cauchy子序列, 令$ \varepsilon_1=1/2 $, 由预紧性可知存在有限子集$ F\subset E $使得$ E=\bigcup_{x\in E}B(x,\varepsilon_1) $, 则$ (x_n)_{n\geqslant 1} $存在无穷子序列使得$ (x_{1,i})_{i\geqslant 1}\subset B(x,\varepsilon_1) $. 此时
     \[
     \forall j, k\,(d(x_{1,j},x_{1,k})<1).
     \]
     再令$ \varepsilon_2=1/2^2 $, 则$ (x_{1,i})_{i\geqslant 1} $存在无穷子列$ (x_{2,i})_{i\geqslant 1}\subset B(x',\varepsilon_2) $, 由此进行下去得到一串序列
     \[
     (x_n)_{n\geqslant 1}\supset(x_{1,i})_{i\geqslant 1}\supset(x_{2,i})_{i\geqslant 1}\supset\cdots\supset(x_{n,i})_{i\geqslant 1}\supset\cdots
     \]
     且$ d(x_{n,j},x_{n,k})<1/2^{n-1} $. 从而可取$ (x_{i,i})_{i\geqslant 1} $是一Cauchy列.
     
     (3) $ \Rightarrow $ (1) : 由$ E $序列紧可知$ E $预紧, 设$ (O_i)_{i\in\alpha} $是$ E $的开覆盖且不存在有限子覆盖. 由$ E $预紧可知$ \forall n\in\N $, 存在$ \frac{1}{n} $-网$ F_n $使得$ (B(x,1/n))_{x\in F_n} $覆盖$ E $, 则$ \exists y_n\in F_n $使得$ B(y_n,1/n) $不能被有限个$ O_i $覆盖.
     
     考虑这样构造出的序列$ (y_n)_{n\geqslant 1} $, 由$ E $序列紧可知$ (y_n)_{n\geqslant 1} $存在收敛到$ y $的子序列$ (y_{n_k})_{k\geqslant 1} $. 不妨设存在某个$ i_0\in\alpha $使得$ y\in O_{i_0} $, 且存在$ \eta>0 $使得$ B(y,\eta)\subset O_{i_0} $, 则取$ n_k $使得$ \frac{1}{n_k}<\frac{\eta}{2} $且$ d(y_{n_k},y)<\frac{\eta}{2} $, 则$ \forall x\in B(y_{n_k},1/n_k) $有
     \[
     d(x,y)\leqslant d(x,y_{n_k})+d(y_{n_k},y)<\frac{\eta}{2}+\frac{1}{n_k}<\eta.
     \]
     即$ B(y_{n_k},1/n_k)\subset B(y,\eta)\subset O_{i_0} $, 这与$ B(y_{n_k},1/n_k) $不能被有限个$ O_i $覆盖矛盾.\qed
     \end{Proof}
     
     \begin{Definition}[相对紧]\label{def:相对紧}\index{X!相对紧}
     设$ (E,d) $是一个度量空间, $ A\subset E $. 若$ \bar{A} $是紧集, 则称$ A $\textbf{相对紧}.
     \end{Definition}
     
     \begin{Remark}
     有关预紧性的刻画, 由定理\ref{thm:紧等价}中的``(3) $ \Leftrightarrow $ (4)"可知
     \begin{enumerate}[(1)]
     \item $ E $是预紧的$ \Longleftrightarrow $ $ E $的任一序列存在Cauchy子列;
     
     \item $ A $相对紧$ \Longleftrightarrow $ $ A $中无穷序列存在子列收敛到$ E $中的元素;
     
     \item 若$ E $是完备的, 则$ A $相对紧$ \Longleftrightarrow $ $ A $预紧.
     \end{enumerate}
     \end{Remark}
     
\section{乘积拓扑}
	
	\begin{Definition}[乘积拓扑]\label{def:乘积拓扑}\index{C!乘积拓扑}
	设$ (E_i,\tau_i)_{i\in\alpha} $是一族拓扑空间, $ E=\prod\limits_{i\in\alpha}E_i $是$ E_i $的Descartes积, 其任一元素$ x=(x_i)_{i\in\alpha} $, $ x_i\in E_i $.
	\begin{enumerate}[(1)]
	\item 若$ \beta\in\fin\alpha $, $ U_i $是$ (E_i,\tau_i) $的开集, 则称
	\[
	O=\prod_{i\in\beta}U_i\times\prod_{i\in\alpha\sm\beta}E_i
	\]
	是$ E $的\textbf{基础开集}.
	
	\item\label{item:乘积拓扑任意并} 称由基础开集的并构成的集合是$ E $的开集, 所有这样的开集构成的集合称为$ E $上的\textbf{乘积拓扑}.
	\end{enumerate}
	\end{Definition}
	
	\begin{Remark}
	关于乘积拓扑的一些注记:
	\begin{enumerate}[(1)]
	\item 考虑乘积拓扑$ \tau $:
	
	\hspace{2em}(O$ _1 $) $ \varnothing\in\tau,\ E\in\tau $是显然的.
	
	\hspace{2em}(O$ _2 $) 由\ref{item:乘积拓扑任意并}中的任意并性质是显然的.
	
	\hspace{2em}(O$ _3 $) 设$ O=\prod\limits_{i\in\beta}U_i\times\prod\limits_{i\in\alpha\sm\beta}E_i $, $ O'=\prod\limits_{i\in\beta'}U'_i\times\prod\limits_{i\in\alpha\sm\beta'}E_i $, 其中$ \beta,\beta'\in\fin\alpha $, 那么
	\[
	O\cap O'=\prod_{i\in\beta\cap\beta'}(U_i\cap U_i')\times\prod_{i\in\beta\sm\beta'}U_i\times\prod_{i\in\beta'\sm\beta}U_i'\times\prod_{i\in\alpha\sm(\beta\cup\beta')}E_i
	\]
	是开集(这因$ \beta\cup\beta'\in\fin\alpha $), 由数学归纳法可知对有限个开集的交都成立. 从而$ \tau $ well-defined.
	
	\item 若$ \alpha $是无限集, $ O=\prod\limits_{i\in\alpha}U_i $通常不是开集. 若$ \forall i\in\alpha,\ U_i\ne E_i $, 且$ x=(x_i)_{i\in\alpha}\in O $. 假设$ O $是开集, 则存在基础开集$ O' $使得$ x\in O' $且$ O'\subset O $. 但基础开集$ O' $应形如$ \prod\limits_{i\in\beta}U_i\times\prod\limits_{i\in\alpha\sm\beta}E_i\supset O $, 矛盾.
	
	\item $ \R^n $上的自然拓扑与$ \R\times\R\times\cdots\times\R $上的乘积拓扑是一致的.
	
	\item 若$ \forall i\in\alpha $, 有$ F_i\subset E_i $是闭集, 那么$ \prod\limits_{i\in\alpha}F_i $是闭集, 这因
	\[
	\left(\prod_{i\in\alpha}F_i\right)^c=\bigcup_{i\in\alpha}\left(F_i^c\times\prod_{j\ne i}E_j\right).
	\]
	
	\item 设$ \alpha=\alpha_1\cup\alpha_2 $, $ E^{(1)}=\prod\limits_{i\in\alpha_1}E_i $, $ E^{(2)}=\prod\limits_{i\in\alpha_2}E_i $, $ E=\prod\limits_{i\in\alpha}E_i $. 则$ E $上的乘积拓扑与$ E^{(1)}\times E^{(2)} $上的乘积拓扑一致.
	\end{enumerate}
	\end{Remark}
	
	\begin{Theorem}
	设$ (E_i)_{i\in\alpha} $是一族拓扑空间, $ E=\prod_{i\in\alpha}E_i $, 令$ \tau $是$ E $上的乘积拓扑, 定义
	\[
	p_i : E\to E_i\qquad p_i(x)=x_i,
	\]
	其中$ x=(x_i)_{i\in\alpha} $, 则$ \tau $是使得各$ p_i $连续的最弱拓扑, 并且$ p_i $是开映射. 这里$ p_i $称为\textbf{正规投影}.
	\end{Theorem}
	\begin{Proof}
	对$ i\in\alpha $, 任取$ U_i\subset E_i $是开集, 由
	\[
	p_i^{-1}(U_i)=U_i\times\prod_{j\ne i}E_j
	\]
	是开集可知$ p_i $是连续的.
	
	设$ \tau' $是使得$ p_i $连续的拓扑, 则任取$ U_i\subset E_i $是开集, 应有
	\[
	p_i^{-1}(U_i)=U_i\times\prod_{j\ne i}E_j\in\tau',
	\]
	由\ref{item:O3}可知$ \tau $的基础开集是$ \tau' $中的元素, 从而$ \tau\subset\tau' $, 即$ \tau $是使得$ p_i $连续的拓扑中最弱的一个.
	
	最后说明$ p_i $是开映射. 设$ O $是$ E $中的开集, 往证$ p_i(O) $也是开集. 因为
	\[
	\forall \tilde{y}_i\in p_i(O)\,\exists y=(y_j)_{j\in\alpha}\in O(y_i=\tilde{y}_i)
	\]
	因为$ O $是开的, 存在基础开集$ O'=\prod\limits_{j\in\beta}U_j\times\prod\limits_{i\in\alpha\sm\beta}E_j $使得$ y\in O' $且$ O'\subset O $. 从而
	\[
	p_i(O')=\begin{cases}
	U_i & ,i\in\beta\\ E_i &, i\notin\beta
	\end{cases}
	\]
	是开集, 故$ p_i $是开映射.\qed
	\end{Proof}
	
	\begin{Corollary}
	设$ (E_i)_{i\in\alpha} $是一族拓扑空间, $ E=\prod\limits_{i\in\alpha}E_i $是乘积拓扑空间.
	\begin{enumerate}[(1)]
	\item 设$ (x^{(n)})_{n\geqslant 1} $是$ E $中的序列, $ x\in E $, 则$ (x^{(n)})_{n\geqslant 1} $收敛到$ x $的充分必要条件是$\forall i\in\alpha,\  (x_i^{(n)})_{n\geqslant 1} $收敛到$ x_i $.
	
	\item 设$ F $是拓扑空间, 映射$ f : F\to E $连续的充分必要条件是$ \forall i\in\alpha,\ p_i\circ f $是连续的.
	\end{enumerate}
	\end{Corollary}
	
	\begin{Proposition}\label{prop:乘积拓扑空间的继承性质}
	设$ (E_i)_{i\in\alpha} $是一族拓扑空间 $ E=\prod\limits_{i\in\alpha}E_i $是乘积拓扑空间.
	\begin{enumerate}[(1)]
	\item 若$ \forall i\in\alpha $, $ E_i $是Hausdorff空间, 则$ E $也是Hausdorff空间.
	
	\item 若$ \forall i\in\alpha $, $ E_i $是紧空间, 则$ E $也是紧空间.
	
	\item 设$ \alpha=\N^\ast $, 若$ \forall i\in\alpha,\ E_i $是可度量化的, 则$ E $是可度量化的. (可度量化是指存在$ E $上的度量$ d $使得$ d $诱导的拓扑与$ E $上原本定义的拓扑一致)
	\end{enumerate}
	\end{Proposition}
	
	\begin{Remark}
	设$ (E,d_E) $和$ (F,d_F) $都是度量空间, 在$ E\times F $上赋以度量
	\[
	d_{E\times F}\left((x_1,y_1),(x_2,y_2)\right)=\max\{ d_E(x_1,x_2), d_F(y_1,y_2) \},
	\]
	则$ (E,d_E) \times (F,d_F) $的乘积拓扑与$ (E\times F, d_{E\times F}) $上度量$ d_{E\times F} $诱导的拓扑一致.
	\end{Remark}
	
	\begin{Proposition}
	设$ (E,d) $是度量空间, 则$ d : E\times E\to\R $是连续的.
	\end{Proposition}
	\begin{Corollary}
	设$ (E,d) $是度量空间, $ \forall A\subset E $, 映射$ x\mapsto d(x,A) $是连续的.
    \end{Corollary}
	
\section{赋范线性空间}
	
	\subsection{Banach空间}
	
	\begin{Definition}[赋范空间]\label{def:赋范空间}\index{F!赋范空间}
	设$ E $是$ \K $上的向量空间, $ \norm{\cdot} : E\to[0,\infty) $是$ E $上的实值函数, 满足$ \forall x,y\in E,\,\forall\lambda\in\K $有
	
	\begin{enumerate}[(1)]
	\item 正定性: $ \norm{x}=0\Longleftrightarrow x=0 $;
	
	\item 齐次性: $ \norm{\lambda x}=\abs{\lambda}\cdot\norm{x} $;
	
	\item 三角不等式: $ \norm{x+y}\leqslant\norm{x}+\norm{y} $.
	\end{enumerate} 
	则称$ \norm{\cdot} $是$ E $上的\textbf{范数}, 并称$ (E,\norm{\cdot}) $是\textbf{赋范(线性)空间}.
	\end{Definition}
	
	\begin{Example}
	常见空间上的范数:
	
	\begin{enumerate}[(1)]
	\item 在Euclid空间$ \K^n $上, 定义
		\[
		\norm{x}_p=\left(\sum_{i=1}^n\abs{x_i}^p\right)^{1/p}\qquad(1\leqslant p<\infty),
		\]
		当$ p=\infty $时定义为$ \norm{x}_\infty=\max\limits_{1\leqslant i\leqslant n}\abs{x_i} $, 则有$ \norm{x}_p\,(1\leqslant p\leqslant\infty) $是$ \K^n $上的范数, 并称$ p=2 $的情形称为$ \K^n $上的\textbf{Euclid范数}. 另外, 对$ n $阶正定矩阵$ A $, 若定义
		\[
		\norm{x}_A=\norm{Ax}_2,
		\]
		则它也是$ \K^n $上的范数.
	
	\item 在连续函数空间$ C[a,b] $上定义
	\[
	\norm{x}=\sup_{a\leqslant t\leqslant b}\abs{x(t)},
	\]
	则$ \norm\cdot $是$ C[a,b] $上的范数.
	\end{enumerate}
	
	\end{Example}
	
	\begin{Remark}
	赋范空间一定是度量空间, 因度量可以被范数诱导:
	\[
	d(x,y):=\norm{x-y}.
	\]
	\end{Remark}
	
	\begin{Definition}[Banach空间]\label{def:Banach空间}\index{B!Banach空间}
	设$ (E,\norm{\cdot}) $是赋范空间, $ d(x,y)=\norm{x-y} $是范数诱导的距离. 若$ (E,d) $完备, 则称$ (E,\norm{\cdot}) $是\textbf{Banach空间}.
	\end{Definition}
	
	由Banach空间的定义可知任意赋范空间$ E,\norm{\cdot} $的完备化$ (\widehat{E},\norm\cdot) $是Banach空间. 同时因为在赋范空间上有自然的线性运算, 可以在赋范空间上定义级数:
	
	\begin{Definition}[级数]\label{def:级数}\index{J!级数}
	设$ (E,\norm{\cdot}) $是赋范空间, $ \sum\limits_{n\geqslant 1} $是空间中的\textbf{级数}, 并称$ S_n=\sum\limits_{k=1}^nx_k $是级数的\textbf{部分和}.
	
	\begin{enumerate}[(1)]
	\item 若$ (S_n)_{n\geqslant 1} $在$ (E,\norm\cdot) $中依范数收敛到$ S $, 则称级数$ \sum\limits_{n\geqslant 1}x_n $在$ (E,\norm\cdot) $上\textbf{收敛}, 并称$ S $是$ \sum\limits_{n\geqslant 1}x_n $的\textbf{和}, 记作$ S=\sum\limits_{n\geqslant 1}x_n $.
	
	\item 若$ (S_n)_{n\geqslant 1} $是Cauchy列, 则称级数$ \sum\limits_{n\geqslant 1}x_n $是\textbf{Cauchy级数}.
	
	\item 若$ \sum\limits_{n\geqslant 1}\norm{x_n} $收敛, 则称级数$ \sum\limits_{n\geqslant 1}x_n $\textbf{绝对收敛}.
	\end{enumerate}
	\end{Definition}
	
	\begin{Remark}
	考虑赋范空间$ ((0,1),\abs\cdot) $上的级数$ \sum\limits_{n\geqslant 1}2^{-n} $, 则$ \sum\limits_{n\geqslant 1}2^{-n} $绝对收敛但不收敛.
	\end{Remark}
	
	\begin{Theorem}
	赋范空间$ (E,\norm\cdot) $是Banach空间当且仅当$ (E,\norm\cdot) $上绝对收敛的级数一定收敛.
	\end{Theorem}
	\begin{Proof}
	\textsl{必要性.} 设$ \sum\limits_{n\geqslant 1}x_n $绝对收敛, 并记$ S_n $是部分和, 因为$ \forall n,p\in\mathbb{N} $, 有
	\[
	\norm{S_{n+p}-S_n}=\norm{\sum_{k=n+1}^{n+p}x_k}\leqslant\sum_{k=n+1}^{n+p}\norm{x_k}.
	\]
	由$ \sum\limits_{n\geqslant 1}x_n $收敛可知
	\[
	\forall p\in\mathbb{N}\left(\lim_{n\to\infty}\norm{S_{n+p}-S_n}=0\right),
	\]
	从而$ (S_n)_{n\geqslant 1} $是Cauchy列. 由$ (E,\norm\cdot) $是Banach空间知其完备, 从而$ (S_n)_{n\geqslant 1} $收敛, 即级数$ \sum\limits_{n\geqslant 1}x_n $收敛.
	
	\textsl{充分性.} 设$ (x_n)_{n\geqslant 1} $是Cauchy列, 则
	\[
	\exists(x_{n_k})_{k\geqslant 1}\subset(x_n)_{n\geqslant 1}\left(\norm{x_{n_{k+1}}-x_{n_k}}<2^{-k}\right),
	\]
	从而$ \sum\limits_{k\geqslant 1}\norm{x_{n_{k+1}}-x_{n_k}} $收敛, 即$ \sum\limits_{k\geqslant 1}(x_{n_{k+1}}-x_{n_k}) $绝对收敛, 故它收敛, 其部分和$ S_{n_k}=x_{n_{k+1}} $, 从而序列$ (x_{n_k})_{k\geqslant 1} $收敛. 由$ (x_n)_{n\geqslant 1} $是Cauchy列知它收敛, 于是$ (E,\norm\cdot) $完备, 即$ (E,\norm\cdot) $是Banach空间.\qed
	\end{Proof}
	
	\begin{Definition}[范数等价]\index{D!等价}
		设$ \norm{\cdot}_1 $与$ \norm{\cdot}_2 $是线性空间$ E $上的两个范数, 若
		\[
		\exists c_1,c_2>0\,\forall x\in E\,(c_1\norm{x}_1\leqslant\norm{x}_2\leqslant c_2\norm{x}_1),
		\]
		则称$ \norm{\cdot}_1 $与$ \norm{\cdot}_2 $\textbf{等价}.
	\end{Definition}
	
	\begin{Remark}
	记$ \id_E : (E,\norm{\cdot}_1)\to(E\,\norm{\cdot}_2) $是恒等映射, 则由
	\[
	\norm{x-y}_2\leqslant c_2\norm{x-y}_1
	\]
	可知$ \id_E $是Lipschitz的. 再考虑$ \id_E^{-1} : (E,\norm{\cdot}_2)\to(E\,\norm{\cdot}_1) $, 由
	\[
	\norm{x-y}_1\leqslant\frac{1}{c_1}\norm{x-y}_2
	\]
	可知$ \id_E^{-1} $也是Lipschitz的. 从而$ \id_E $与$ \id_E^{-1} $都是一直连续的. 于是$ (E,\norm\cdot_1) $完备当且仅当$ (E,\norm\cdot_2) $完备, $ A\subset E $在$ (E,\norm\cdot_1) $上紧当且仅当$ A $在$ (E,\norm\cdot_2) $上紧.
	\end{Remark}
	
	\begin{Theorem}
	设$ E $是数域$ \K $上的有限维线性空间, 则$ E $上所有范数都等价.
	\end{Theorem}
	\begin{Proof}
	设$ \dim E=n $, 取$ E $中的一组基$ e_1,e_2,\cdots,e_n $, 则
	\[
	\forall x\in E\,\exists[x_1,x_2,\cdots,x_n]^\mathrm{T}\in\K^n\,\left( x=\sum_{i=1}^nx_ie_i \right),
	\]
	那么可以定义
	\[
	\varPhi : \K^n\to E,\qquad \begin{bmatrix}
	x_1\\x_2\\\vdots\\x_n
	\end{bmatrix}\mapsto\sum_{i=1}^nx_ie_i.
	\]
	从而$ \varPhi $是一个线性同构.
	
	令$ \K^n $上的范数$ \norm\cdot_\infty $为
	\[
	\norm{\begin{bmatrix}x_1\\x_2\\\vdots\\x_n\end{bmatrix}}_\infty=\max_{1\leqslant i\leqslant n}\abs{x_i},
	\]
	由此可以诱导$ E $上的范数$ \norm\cdot_\infty $:
	\[
	\norm{x}_\infty=\norm{\varPhi^{-1}(x)}_\infty.
	\]
	再任取$ E $上的另一个范数$ \norm{\cdot} $, 则有
	\[
	\norm{x}=\norm{\sum_{i=1}^nx_ie_i}\leqslant\sum_{i=1}^n\abs{x_i}\norm{e_i}\leqslant\norm{x}_\infty\sum_{i=1}^n\norm{e_i}.
	\]
	令$ c_2=\sum_{i=1}^n\norm{e_i} $, 则有$ \norm{x}\leqslant c_2\norm{x}_\infty $. 再定义映射
	\[
	\varphi : \K^n\to\C,\qquad\begin{bmatrix}x_1\\x_2\\\vdots\\x_n\end{bmatrix}\mapsto\norm{\varPhi\left(\begin{bmatrix}
	x_1\\x_2\\\vdots\\x_n
	\end{bmatrix}\right)}=\norm{x}.
	\]
	易知$ \varphi $是连续的, 则$ \varphi $在$ \K^n $的单位球面$ \mathbb{S}^{n-1} $上有下确界, 记作$ c_1 $, 则$ c_1\geqslant 0 $. 而对$ \forall x\in E $, 因
	\[
	\norm{x}=\varphi\left(\begin{bmatrix}x_1\\x_2\\\vdots\\x_n\end{bmatrix}\right)=\norm{\begin{bmatrix}x_1\\x_2\\\vdots\\x_n\end{bmatrix}}_\infty\varphi\left(\begin{bmatrix}x_1\\x_2\\\vdots\\x_n\end{bmatrix}\cdot\norm{\begin{bmatrix}x_1\\x_2\\\vdots\\x_n\end{bmatrix}}_\infty^{-1}\right)\geqslant c_1\norm{\begin{bmatrix}x_1\\x_2\\\vdots\\x_n\end{bmatrix}}_\infty=c_1\norm{x}_\infty.
	\]
	可知$ \norm{x}\geqslant c_1\norm{x}_\infty $. 则有
	\[
	c_1\norm{x}_\infty\leqslant\norm{x}\leqslant c_2\norm{x}_\infty,
	\]
	从而$ E $上所有范数均等价.\qed
	\end{Proof}
	
	\begin{Remark}\label{rmk:有限维赋范空间等价K^n}
	由上述定理可知任何有限维赋范空间在范数等价的意义下均可看作$ \K^n $, 从而有限维赋范空间都是完备的, 并且有限维赋范空间上的有界闭集都是紧集.
	\end{Remark}
	
	\begin{Theorem}[\textbf{Riesz}]\label{thm:Riesz定理}
	设$ (E,\norm{\cdot}) $是赋范空间, 则$ E $是有限维的当且仅当$ E $的闭单位球是紧的.
	\end{Theorem}
	
	为了证明上面的Riesz定理, 我们先说明下面一个引理. 这一引理的证明技巧性相对较强.
	
	\begin{Lemma}\label{lem:Riesz前引理}
	设$ (E,\norm\cdot) $是赋范空间, $ F $是$ E $的闭线性子空间且$ F\ne E $, 则
	\[
	\forall\varepsilon>0\,\exists e\in E\,(\norm{e}=1\land d(e,F)\geqslant 1-\varepsilon).
	\]
	\end{Lemma}
	\begin{Proof}
	令$ x\in E\sm F $, 取$ d=d(x,F)=\inf\{ \norm{x-y} : y\in F \} $, 则
	\[
	\exists y_0\in F\left(d\leqslant\norm{x-y_0}\leqslant\frac{d}{1-\varepsilon}\right).
	\]
	再令$ e=\frac{x-y_0}{\norm{x-y_0}} $, 那么有$ \norm{e}=1 $, 且$ \forall z\in F $都有
	\[
	\norm{e-z}=\norm{\frac{x-y_0}{\norm{x-y_0}}-z}=\frac{1}{\norm{x-y_0}}\norm{x-y_0-\norm{x-y_0}z}\geqslant\frac{1}{d/(1-\varepsilon)}d=1-\varepsilon.
	\]
	其中由于$ y_0+\norm{x-y_0}z\in F $, 故$ \norm{x-y_0-\norm{x-y_0}z}\geqslant d $. 于是$ d(e,F)\geqslant 1-\varepsilon $.\qed
	\end{Proof}
	
	下面我们可以证明Riesz定理:
	
	\textbf{定理\,\,\ref{thm:Riesz定理}\,\,的证明}\ \ \textsl{必要性}. 由注\ref{rmk:有限维赋范空间等价K^n}可知.
	
	\textsl{充分性}. 用反证法, 假设$ E $是无限维的, 则可以取一单位向量$ x_1 $, 并且记$ F_1=\mathrm{span}\{x_1\}=\{ \lambda_1x_1 : \lambda_1\in\K \} $. 由$ \dim F_1=1<\infty $可知$ F_1 $完备, 从而$ F_1 $是$ E $的闭线性子空间. 故由引理\ref{lem:Riesz前引理}可知
	\[
	\exists x_2\in E\,\left(\norm{x_2}=1\land d(x_2,F_1)\geqslant\frac{1}{2}\right).
	\]
	再取$ F_2=\mathrm{span}\{x_1,x_2\} $, 同理$ F_2 $是$ E $的闭线性子空间, 从而
	\[
	\exists x_3\in E\,\left(\norm{x_3}=1\land d(x_3,F_2)\geqslant\frac{1}{2}\right).
	\]
	依此进行下去可得一序列$ (x_n)_{n\geqslant 1}\subset\{ x : \norm{x}=1 \} $使得
	\[
	\forall n\ne m\,\left(\norm{x_n-x_m}\geqslant\frac{1}{2}\right),
	\]
	即$ (x_n)_{n\geqslant 1} $没有收敛子列. 但由$ \{ x : \norm{x}=1 \} $紧可知它是序列紧的, 即$ (x_n)_{n\geqslant 1} $存在收敛子列, 矛盾.\qed
	
     \subsection{一个Banach空间的例子—— $ L^p $空间}
     
     \begin{Definition}[$ \ell^{p} $空间]\label{def:lp空间}~

          当 $ 0<p<\infty $ 时, 定义
          \[
               \ell^{p}=\bigg\{ x=(x_{n})_{n\geqslant1}: \norm{x}_{p}=\bigg( \sum\limits_{n\geqslant1}\abs{x_{n}}^{p} \bigg)^{1/p}<\infty, x_{n}\in\K \bigg\}
          \]
           当 $ p=\infty $ 时, 定义
           \[
                \ell^{\infty}=\{ x=(x_{n})_{n\geqslant1}:\norm{x}_{\infty}=\sup_{n\geqslant1}\abs{x_{n}}<\infty, x_{n}\in\K \}
           \]
           称其为\textbf{$ \ell^{p} $空间}\index{L!$ \ell^{p} $空间}, 再取 $ c_{0} $ 表示所有极限为 $ 0 $ 的数列构成的集合, 显然有 $ c_{0}\subset\ell^{\infty} $. 
     \end{Definition}
	
	