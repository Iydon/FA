% !TeX root = main.tex
\chapter{拓扑空间与度量空间}
\pagenumbering{arabic}
\section{基本概念}

 \begin{Def}[拓扑空间]\label{def:拓扑空间}
      设 $ E $ 是一个集合, 称 $ E $ 的子集族 $ \tau $ 是一个\textbf{拓扑}\index{T!拓扑}, 若 $ \tau $ 满足: 
     \begin{enumerate}[(1)]
          \item $ E, \varnothing\in \tau $;
          \item $ \tau $ 中任意多元素的并仍是 $ \tau $ 中元素 (任意并);
          \item $ \tau $ 中有限多元素的交仍是 $ \tau $ 中的元素(有限交) . 
     \end{enumerate}
     并称 $ (E, \tau) $ 为一个\textbf{拓扑空间}\index{T!拓扑空间}, $ \tau $ 中的元素称为\textbf{开集}\index{K!开集}, 在不引起歧义的情况下, 可以简称 $ E $ 是一个拓扑空间. 
 \end{Def}
 \begin{Rmk}\label{rmk:平凡离散}
      对集合 $ E $ , 称 $ \tau=\{ E, \varnothing \} $ 为\textbf{平凡拓扑}\index{P!平凡拓扑}, 称 $ \tau=2^{E} $ 为\textbf{离散拓扑} \index{L!离散拓扑}. 
 \end{Rmk}
 \begin{Ex}
      $ E=\R $, 取 $ \tau=\Big\{ \bigcup\limits_{j\in\Z}(x_{j}, x_{j+1}), (x_{j})_{j\in\Z}\subset \R\cup\{ \pm\infty \} \Big\} $是一个拓扑, 这一拓扑称为 $ E $ 的\textbf{自然拓扑}\index{Z!自然拓扑}. 
 \end{Ex}
 \begin{Def}[度量空间]\label{def:度量空间}
      设 $ E $ 是非空集合, $ d: E\times E\to \R $ 满足 $ \forall x, y\in E $
      \begin{enumerate}[(1)]
           \item 非负性: $ d(x, y)\geqslant 0 $;
           \item 正定性: $ d(x, y)=0\Leftrightarrow x=y $ ;
           \item 对称性: $ d(x, y)=d(y, x) $ ;
           \item 三角不等式: $ d(x, z)\leqslant d(x, y)+d(y, z) $ 
      \end{enumerate}
      则称 $ (E, d) $ 是一个\textbf{度量空间}\index{D!度量空间}, 并称 $ d $ 是 $ E $ 上的\textbf{度量}\index{D!度量}
 \end{Def}
 \begin{Rmk}\label{rmk:度量不唯一}
      度量并不唯一, 例如对 $ (E, d) $ 定义度量
      \begin{equation*}
          \begin{aligned}
               r(x, y) = \min\{ d(x, y), 1 \}, \\
               d'(x, y) = rd(x, y), r\in\R_{+},
          \end{aligned}
      \end{equation*}
      就是两个不同的度量. 
 \end{Rmk}
 
 度量空间也是拓扑空间, 记 $ B(x, r)=\{ y: d(y, x)<r \} $ 可将 $ \tau $ 中的元素定义为: 
 \[
      U\in E, U\in \tau\Leftrightarrow \forall x\in U\exists r>0(B(x, r)\subset U)
 \] 
 该拓扑 $ \tau $ 称为由度量 $ d $ \textbf{诱导}的拓扑. 

 \begin{Ex} 
      在实 Euclid 空间 $ \R^{n} $ 上赋以度量
      \begin{equation}
           d(x, y)=\bigg( \sum_{i=1}^{n}(x_{i}-y_{i})^{2} \bigg)^{1/2}\tag{Euclid距离}
      \end{equation}
     以 $ C[a, b] $ 以在 $ [a, b] $ 上的连续函数全体, 在其上赋以度量
     \begin{equation}
          d(x, y) = \max_{t\in [a, b]} |x(t)-y(t)|\tag{一致距离}
     \end{equation}
 \end{Ex}
 \begin{Def}[闭集]\label{def:闭集}
      设 $ (E, d) $ 是一个拓扑空间, $ A\subset E $ , 若 $ A^{c} $ 是开集, 则称 $ A $ 是 $ E $ 上(关于 $ \tau $ )的\textbf{闭集}.\index{B!闭集}
 \end{Def}
 \begin{Ex}
      设 $ (E, d) $ 是一个度量空间, 则闭球 $ \baro{B}(x, r)=\{ y: d(y, x)\leqslant r \} $ 是闭集. 
 \end{Ex} 
 \begin{Prop}\label{prop:闭集的性质}
      闭集具有以下性质:
      \begin{enumerate}[(1)]
           \item 全空间 $ E $ 和空集 $ \varnothing $ 是闭集;
           \item 任意多个闭集的交仍是闭集(任意交);
           \item 有限多个闭集的并仍是闭集(有限并).
      \end{enumerate}
 \end{Prop}

 \subsection{邻域和邻域基}

 \begin{Def}[邻域基]\label{def:邻域基}
       设 $ (E, \tau) $ 是一个拓扑空间, $ \alpha\in E $, 
       \begin{enumerate}[(1)]
            \item 若子集 $ V\subset E $ 满足
            \[
                 \exists U\in \tau(x\in U\wedge U\subset V)
            \]
            则称 $ V $ 是 $ x $ 的一个\textbf{邻域}\index{L!邻域}, 并记 $ \CN(x) $ 为全体 $ x $ 的邻域的集合, 称为 $ x $ 的\textbf{邻域系};
            \item 若 $ \CN(x) $ 的子集族 $ \CB(x) $ 满足
            \[
                 \forall V\in \CN(x)\exists U\in \CB(x)(U\subset V)
            \]  
            则称 $ \CB(x) $ 是 $ x $的\textbf{邻域基}\index{L!邻域基}. 
       \end{enumerate}
 \end{Def}
 \begin{Rmk}
      $ \CN(x) $是 $ x $  的邻域基; 包含 $ x $ 的所有开集的集合是 $ x $ 的邻域基. 
 \end{Rmk}
 \begin{Ex}
      设 $ (E, d) $ 是一个度量空间, $ x\in E $, 则 
      \[
          \CB(x)=\left\{ B(x,\frac{1}{n}): n\geqslant 1 \right\} 
      \]
      是 $ x $ 的邻域基. (可用来描述极限)
 \end{Ex}
 \begin{Def}[粘着集]
     设 $ (E, \tau) $ 是一个拓扑空间, $ A\subset E $ 
     \begin{enumerate}[(1)]
          \item 设 $ x\in E $ , 若 $ \forall V\in \CN(x)((V\cap A)\bs \{ x \}\neq\varnothing) $, 则称 $ x $ 是 $ A $ 的\textbf{凝聚点}\index{N!凝聚点};
          \item 若 $ x\in A $ 或 $ A $  的凝聚点, 则称 $ x $ 是 $ A $ 的\textbf{粘着点}, 记 $ A $ 的粘着点的全体为 $ \baro{A} $ , 称为 $ A $ 的\textbf{粘着集}\index{N!粘着集}, 也即
          \[
               x\in \baro{A}\Leftrightarrow \forall V\in \CN(V\cap A\neq\varnothing).    
          \]
     \end{enumerate}
 \end{Def}
 \begin{Rmk}
      $ E $ 中包含 $ A $ 的最小闭集称为 $ A $ 的\textbf{闭包}\index{B!闭包}, 可知 $ A $ 的闭包与粘着集是相同的, 它们从不同的角度描述了相同的集合. 
 \end{Rmk}
 \begin{Def}[稠密]\label{def:稠密}
       设 $ A\subset E $ , 若 $ \baro{A}=E $ , 则称 $ A $ 在 $ E $ 中\textbf{稠密}\index{C!稠密}, 或称 $ A $ 是 $ E $ 的一个\textbf{稠子集}
 \end{Def}
 \begin{Ex}
      有利数集 $ \Q $ 与无理数集 $ \mathbb{J} $ 都在 \R 中稠密. 
 \end{Ex}
 \begin{Def}[内部]\label{def:内部}
       设 $ A\subset E $, 若 $ A\in\CN(x) $ , 则称 $ x $ 是 $ A $ 的\textbf{内点},  $ A $ 的内点的全体称为 $ A $ 的\textbf{内部}\index{N!内部}, 记作 $ \mathring{A} $ 或 $ \mathrm{Int}(A) $.  
 \end{Def}
 \begin{Def}[边界]\label{def:边界}
       设 $ A\subset E $, 称 $ \partial A = \baro{A}\bs\mathring{A} $ 是 $ A $ 的\textbf{边界}\index{B!边界}.
 \end{Def}
 \begin{Ex}
      设 $ E=\R $ , 取 $ A = (2, 3) $, 则 $ \partial A = \{ 2, 3 \} $ ; 取 $ A = [2, 3) $ 则 $ \partial A = \{ 2, 3 \} $. 
      
      设 $ E = \R^{2} $, 取 $ A = (2, 3)\times \{0\} $ , 则 $ \partial A = [2, 3]\times \{0\} $ , 此时 $ A\subset \partial A $  ; 取 $ \mathbb{D}=B(0, 1) $ , 则 $ \partial \mathbb{D}=\mathbb{S}^{1} $ (单位圆周). 
 \end{Ex}
 \begin{Thm}\label{thm:闭包和内部的性质}
       设 $ E $ 是一个拓扑空间, $ A, B\subset E $ , 则:
       \begin{enumerate}[(1)]
            \item $ \baro{\baro{A}}=\baro{A} $, $ \mathring{\mathring{A}}=\mathring{A} $; (幂等性)
            \item $ E\bs\mathring{A}=\overline{E\bs A} $;
            \item \label{item:闭包内部}$ \overline {A\cup B}=\baro{A}\cup \baro{B} $; $ \mathring{\widehat{A\cap B}}=\mathring{A}\cap\mathring{B} $.   
       \end{enumerate}
 \end{Thm}
 \begin{Rmk}
      考虑定理~\ref{thm:闭包和内部的性质}~的~\ref{item:闭包内部}~,注意 $ \overline {A\cap B}=\baro{A}\cap \baro{B} $ 和 $ \mathring{\widehat{A\cup B}}=\mathring{A}\cup\mathring{B} $ 不一定成立. 比如取 $ A=\Q, B=\mathbb{J} $, 则此时 $ \overline{A\cap B}=\baro{\varnothing}=\varnothing $  , 而 $ \baro{A}\cap \baro{B}=\R\cap\R=\R\neq\overline {A\cup B} $ ; 同时 $ \mathring{\widehat{A\cap B}}=\mathring{\R}=\R $, 但是 $ \mathring{A}\cup\mathring{B}=\varnothing\cup\varnothing=\varnothing\neq \mathring{\widehat{A\cap B}} $  
 \end{Rmk}
 \begin{Def}[拓扑比较]\label{def:拓扑比较}
       设 $ \tau, \tau' $ 是 $ E $ 的拓扑, 若 $ \tau'\subset\tau $, 则称 $ \tau $ 是 $ \tau' $ 的\textbf{强拓扑}\index{Q!强拓扑}, 也即: $ \tau' $--开集一定是 $ \tau $--开集.   
 \end{Def}
 \begin{Ex}
      $ E=\R $ 时, 平凡拓扑 $ \subset $ 自然拓扑 $ \subset $ 离散拓扑. 
 \end{Ex}
 \begin{Def}[拓扑子空间]\label{def:拓扑子空间}
       设 $ (E, \tau) $ 是一个拓扑空间, $ F\subset E $ , 在 $ F $ 上定义
       \[
          \tau_{F}=\{ U\cap F: U\in\tau \} , 
       \]
       则 $ \tau_{F} $ 是一个拓扑, 称为由 $ \tau $  诱导的拓扑, 并称 $ (F, \tau_{F}) $ 是 $ (E, \tau) $ 的\textbf{拓扑子空间}\index{T!拓扑子空间}. 
 \end{Def}
 \begin{Ex}
      设 $ E=\R $ , 取 $ F=(2, 3) $ , 则 $ \tau_{F} $ 中元素是 \R 中开集; 
      
      取 $ F=[2, 3) $ , 则 $ \forall2<x<3, [2,x) $ 是 $ \tau_{F} $ 中元素, 但不是 \R 中开集. 
 \end{Ex}
 \begin{Ex}
      设 $ (E,d) $ 是度量空间, $ F\subset E $, 取 $ \delta=d|_{F\times F} $ 则 $ (F, \delta) $ 是度量空间, 且当 $ E,  F $ 分别赋以 $ d, \delta $ 诱导的拓扑时,  $ F $ 是 $ E $ 的拓扑子空间. 
 \end{Ex}
 \begin{Prop}
      设 $ F\subset E, A\subset F $ 则 $ A $ 是 $ F $ 中的闭集 $ \Leftrightarrow $ 存在闭集 $ B\subset E $ s.t. $ A=B\cap F $ . 
 \end{Prop}
 \begin{Prf}
       由闭集性质可以知道
       \[
          \begin{aligned}
               A \text{\,是\,} F \text{\,中闭集\,} & \Leftrightarrow F\bs A \text{\,是\,} F \text{\,中开集\,} ;\\
               & \Leftrightarrow \exists D\in \tau(F\bs A=D\cap F);\\
               & \Leftrightarrow A=D^{c}\cap F
          \end{aligned}     
       \]
       其中$ D^{c} $是闭集. 	 \qed
 \end{Prf}

\subsection{分离空间}

 \begin{Def}[Hausdorff空间]\label{def:Hausdorff空间}
       设 $ (E, \tau) $ 是拓扑空间, 若 
       \[
          \forall x, y\in E(x\neq y)\exists U\in \CN(x)\exists V\in\CN(x) (U\cap V)=\varnothing
       \]
       则称 $ (E, \tau) $ 是\textbf{Hausdorff空间}\index{H!Hausdorff空间}或\textbf{分离空间}. 
 \end{Def}
 \begin{Ex}
      所有离散拓扑空间都是 Hausdorff空间, 所有度量空间都是 Hausdorff空间, 元素个数多于一个的平凡拓扑空间不是 Hausdorff空间(因为所有元素都只有一个邻域, 即 $ E $ ). 
 \end{Ex}
 \begin{Prop}
      以下4个命题相互等价
      \begin{enumerate}[(1)]
           \item  $ E $ 是 Hausdorff空间;
           \item $ \forall x\in E $, 其所有闭邻域的交为 $\{ x \}$;
           \item $\bigcap \CN(x)=\{x\}$;
           \item $ \forall F\subset E $, $ F $ 也是Hausdorff空间. 
      \end{enumerate}
 \end{Prop}
 
 \section{完备性}
 \subsection{序列的极限}
 \begin{Def}[极限]\label{def:极限}
       设 $ (E, \tau) $ 是拓扑空间, $ (x_{n})_{n\geqslant 1} $ 是 $ E $ 中的序列,  $ x\in E $, 若
       \[
            \forall V\in\CN(x)\exists n_{0}\in\N(n\geqslant n_{0}\Leftarrow x_{n}\in V)
       \]
       则称 $ (x_{n})_{n\geqslant1} $ \textbf{收敛}于 $ x $, 并称 $ x $ 是序列 $ (x_{n})_{n\geqslant1} $ 的极限, 记作
       \[
            \tau-\lim_{n\to \infty}x_{n}=x. 
       \] 
       当不引起歧义的时候, 前面的 $ \tau- $ 可以省略. 
 \end{Def}
\begin{Rmk}
     \begin{enumerate}[(1)]
          \item 上述定义中的 $ \CN(x) $ 可换成 $ \CB(x) $ ;
          \item \R 中序列的极限至多只有一个, 例如 $\lim\limits_{n\to\infty}\frac{1}{n}=0 $, $ \lim\limits_{n\to\infty} =\infty$(无极限), 取 $ (x_{n})_{n\geqslant1}=\left\{ 1, -1, 1,  -1\ldots \right\} $ 也无极限;
          \item 一般地, Hausdorff空间中的序列至多只有一个极限, 而平凡拓扑空间中的任意序列收敛到任意元素;
          \item 对度量空间 $ (E, d) $ , 有
          \[
               \lim_{n\to \infty}x_{n}=x \Leftrightarrow \lim_{n\to\infty}d(x_{n}, x)=0, 
          \]
          取邻域基 $ \CB(x)=\left\{ B(x, \frac{1}{n}): n\geqslant1 \right\} $即可.  
     \end{enumerate}
\end{Rmk}
\begin{Prop}
     设 $ E $ 是度量空间, 则 $ A\subset E $是闭集的充要条件是对任意 $ A $中序列  $ (x_{n})_{n\geqslant1} $ 当 $ \lim\limits_{n\to\infty}x_{n}=x_{0} $ 时, 都有$ x_{0}\in A $.  
\end{Prop}
\begin{Prf}
     \textit{必要性}. 若 $ x_{0}\notin A $, 则 $ \exists r>0 (B(x, r)\cap A=\varnothing) $, 而由极限的定义知 
     \[
          \exists n_{0}\in\N(n\geqslant n_{0}\Rightarrow x_{n}\in B(x, r))\Rightarrow x_{n}\in A. 
     \]
     与 $ (x_{n})_{n\geqslant1}\subset A $ 矛盾. 

     \textit{充分性}. 只需证 $ A^{c} $ 是开集, 用反证法, 若 $ A^{c} $ 是闭集, 则有
     \[
          \exists x\in A^{c}\,\forall r>0(B(x, r)\cap A\neq\varnothing),
     \]
     取 $ r=1/n, n=1, 2\ldots $ , 则有 $ x_{n}\in B(x, 1/n)\cap A $ . 由构造可知 $ (x_{n})_{n\geqslant1}\subset A $且 $ x_{n}\to x\,(n\to\infty) $  , 又由已知条件知 $ x\in A $, 矛盾. 故 $ A $ 为闭集. \qed 
\end{Prf}
\begin{Def}[序列的粘着值]\label{def:序列的粘着值}
      设 $ (x_{n})_{n\geqslant1} $ 是 $ E $ 中的序列,  $ x\in E $, 若
      \[
           \forall V\in\CN(x)\,\forall n_{0}\in\N\,\exists n\geqslant n_{0}(x_{n}\in V),
      \]
      则称 $ x $ 是序列 $ (x_{n})_{n\geqslant1} $ 的\textbf{粘着值}\index{N!粘着值}.  
\end{Def}
\begin{Ex}
     \R 中序列 $ (1, -1, 1, -1\ldots) $ 的粘着值为 $ 1, -1 $.  
\end{Ex}
\begin{Rmk}
     序列的粘着值与集合的粘着点没有关系, 如 \R 上的序列 $ (x_{n})_{n\geqslant1}=(1, 0.8, {1}/{2}, {1}/{2}-0.2, {1}/{3}, {1}/{3}-0.2, \ldots, {1}/{n}, {1}/{n}-0.2, \ldots) $, 该序列的粘着值 $ 0, -0.2 $ 均不在序列中, 而集合 $ A=\{ x_{n} \}_{n\geqslant1} $ 的粘着集为 $ A\cup\{ 0,-0.2 \} $. 

     再取 $ (x_{n})_{n\geqslant1}=(1, 1, \ldots) $ 则该序列的粘着值为$1$,  但 $ 1 $ 不是 $ A=\{ 1 \} $ 的凝聚点.  
\end{Rmk}
\subsection{Cauchy列与完备性}
\begin{Def}[Cauchy列]\label{def:Cauchy列}
      设 $ (E, d) $是度量空间, $ (x_{n})_{n\geqslant1} $ 是 $ E $  中序列, 若
      \[
           \forall \varepsilon>0\,\exists n_{0}\in\N\,(n,m\geqslant n_{0}\Rightarrow d(x_{m},x_{n}<\varepsilon),
      \]
      则称 $ (x_{n})_{n\geqslant1} $ 是一个\textbf{Cauchy列}\index{C!Cauch列}
\end{Def}
\begin{Ex}
     在 \R 中,  $ (1/n)_{n\geqslant1} $ 是Cauchy列, 收敛到0; $ ((1+{1}/{n})^{n})_{n\geqslant1} $ 是Cauchy列, 收敛到\me;

     而在 \Q 中, $ (1/n)_{n\geqslant1} $ 是Cauchy列, 收敛到0; $ ((1+{1}/{n})^{n})_{n\geqslant1} $ 是Cauchy列, 不收敛.
\end{Ex}
\begin{Prop}
     Cauchy列有以下性质:
     \begin{enumerate}[(1)]
          \item 若 $ (x_{n})_{n\geqslant1} $ 是收敛列, 则 $ (x_{n})_{n\geqslant1} $ 是Cauchy列;
          \item 若 $ (x_{n})_{n\geqslant1} $ 是Cauchy列且有收敛子列, 则 $ (x_{n})_{n\geqslant1} $ 是收敛列;
          \item 若 $ (x_{n})_{n\geqslant1} $ 是Cauchy列, 则 $ (x_{n})_{n\geqslant1} $ 有界. (即 $ \exists x\in E\,\exists r>0\,((x_{n})_{n\geqslant1} \subset B(x, r)) $ )
     \end{enumerate}
\end{Prop}
\begin{Def}[完备]\label{def:完备}
      设 $ (E, d) $ 是度量空间, 若 $ (E, d) $ 中任一 Cauchy列收敛, 则称 $ E $ 是\textbf{完备度量空间}或 $ E $ 是\textbf{完备}\index{W!完备}的, 此时也说 $ d $ 是完备的. 
\end{Def}
\begin{Ex}
     实 Euclid 空间 $ \R^{n} $, 复 Euclid 空间 $ \C^{n} $ 依 Euclid度量是完备的; \Q, \J 依Euclid度量不完备. 
\end{Ex}
\begin{Ex}
     在空间 $ C[a, b] $ 上赋以度量
     \[
          d(x, y)=\max_{0\leqslant t\leqslant1}\abs{x(t)-y(t)}
     \]
     时, 度量空间 $ (C[a, b], d) $ 是完备的, 其中 $ C[a, b] $ 是定义在 $ [0, 1] $ 上的连续函数全体
\end{Ex}
\begin{Prf}
     取  $ (x_{n})_{n\geqslant1} $ 是Cauchy列, 即
     \[
          \forall\varepsilon>0\,\exists n_{0}\in\N\,(n,m\geqslant n_{0}\Rightarrow d(x_{n},x_{m})<\varepsilon)
     \]
     即 $ \forall n, m\geqslant n_{0}\,\forall a\leqslant t\leqslant b\,(\abs{x_{n}(t)-x_{m}(t)}<\varepsilon) $ , 则有 $ (x_{n}(t))_{n\geqslant1} $ 是 $ \mathbb{K} $ 上的 Cauchy列\,(其中$\mathbb{K}=\R \text{\,或\,}\C$), 从而是收敛列, 并记
     \[
          x(t)=\lim_{n\to\infty}x_{n}(t), 
     \]
     则 $ \max\limits_{a\leqslant x\leqslant b}\abs{x(t)-x_{n}(t)}\to 0\,(n\to\infty) $, 从而 $ (x_{n})_{n\geqslant1} $  一致收敛到 $ x $ , 由  $ x_{n}\in C[a, b] $ 知 $ x\in C[a, b] $ , 即 $ (C[a, b], d) $ 完备. \qed
\end{Prf}
\begin{Rmk}
     对 $ C[0, 1] $ , 若赋以度量 $ d(x, y)=\dint_{0}^{1}\abs{x(t)-y(t)}\diff x $, 则 $ (C[0, 1], d) $ 不完备, 反例如下:

     取 $ C[0, 1] $ 中的函数列 $ (x_{n}(t))_{n\geqslant1}=(t^{n})_{x\geqslant1} $, 易证  $ (x_{n})_{n\geqslant1} $ 是 Cauchy列, 但若记其极限函数是 $ x(t) $ 则有
     \[
          x(t)=\begin{cases}
               0, & 0<x<1; \\
               1, & x=1. 
          \end{cases}
     \]
     从而在空间给定的情形下, 完备性依赖与距离, 因此\textbf{完备不是一个拓扑概念}
\end{Rmk}
\begin{Thm}
     度量空间 $ (E, d) $ 完备 $ \Leftrightarrow $ 对任意单调递减的非空子集列 $ (A_{n})_{n\geqslant1} $ 若 $ \lim\limits_{n\to\infty}\diam(A_{n})=0 $ , 则 $ \bigcap\limits_{n\geqslant1}A_{n} $ 是单点集. 其中 $ \diam(A)=\sup\limits_{x,y\in A}d(x, y) $ 是 $ A $  的\textbf{直径}. 
\end{Thm}
\begin{Prf}
     \textit{必要性}. 因为 $ \lim\limits_{n\to\infty}\diam(A_{n})=0 $ 则
     \[
          \forall \varepsilon>0\,\exists n_{0}\in\N\,(n\geqslant N\Rightarrow \diam(A_{n})<\varepsilon). 
     \]
     当 $ m>n>N $ 时, 取 $ x_{n}\in A_{n} $ , $ x_{m}\in A_{m} $ 有
     \[
          d(x_{m}, x_{n})<\diam(A_{n})<\varepsilon. 
     \]
     可以知道如此构造出的 $ (x_{n})_{n\geqslant1} $ 为Cauchy列, 又由 $ (E, d) $ 的完备性知 $ (x_{n})_{n\geqslant1} $ 收敛, 不妨设 $\lim\limits_{n\to\infty}x_{n}=x$ , 由 $ A_{n} $ 非空可知 $ x\in A_{n} $ , 故 $ x\in\bigcap\limits_{n\geqslant1}A_{n} $ . 

     再设 $ y\in\bigcap\limits_{n\geqslant1} A_{n} $ , 因 $ d(x_{n}, y)<\diam(A_{n})<\varepsilon $ 知 $ (x_{n})_{n\geqslant1} $ 收敛于 $ y $ , 由 $ E $ 是度量空间可知 $ E $ 是 Hausdorff空间, 从而$ x=y $, 故 $ \bigcap\limits_{n\geqslant1} A_{n} $ 为单点集. 

     \textit{充分性}. 取 $ (x_{n})_{n\geqslant1} $ 是 $ E $ 中的 Cauchy列, 取 $ A_{n}=\left\{ x_{m}:m>n \right\} $, 则 $ (A_{n})_{n\geqslant1}) $ 是一个单调递减的非空闭子集列, 下面检验 $ \lim\limits_{n\to\infty}\diam(A_{n})=0 $ . 

     由 $ (x_{n})_{n\geqslant1} $ 是 Cauchy列可知
     \[
           \forall \varepsilon>0\, \exists n_{0}\in\N\,(m, n>n_{0}\Rightarrow d(x_{m}, x_{n})<\varepsilon),
     \]
     则 $ \forall x, y\in A_{n} $ , 取 $ x_{n}, x_{m}\in A_{n} $ 使得 $ d(x, x_{n}), d(x_{m})<\varepsilon $ 则
     \[
          d(x, y)\leqslant d(x, x_{n})+d(x_{n}, x_{m})+d(x_{m}, y) < 3\varepsilon         
     \]
     从而 $ \diam(A_{n})<3\varepsilon $, 也即 $\lim\limits_{n\to\infty} \diam(A_{n})=0 $. 

     由条件知 $ \bigcap\limits_{n\geqslant1}A_{n} $ 是单点集, 不妨记作 $ \{ x_{0} \} $ , 则 $ \lim\limits_{n\to\infty}x_{n}=x_{0} $, 从而 $ (E, d) $  是完备的. \qed
\end{Prf}
\begin{Thm}
     设 $ (E, d) $ 是度量空间, $ A\subset E $. 
     \begin{enumerate}[(1)]
          \item 若 $ (A, d) $ 完备, 则 $ A $ 是 $ (E, d) $ 中的闭集;
          \item 若 $ (E, d) $ 完备且 $ A $ 是闭集, 则 $ (A, d) $ 完备. 
     \end{enumerate} 
\end{Thm}
\begin{Prf}
     (1) 设 $ x $ 是 $ A $ 的凝聚点, 则存在 $ (x_{n})_{n\geqslant1}\subset A $, 使得 $ \lim\limits_{n\to\infty}x_{n}=x $ , 则 $ (x_{n})_{n\geqslant1} $ 是 $ A $ 中的 Cauchy列. 由 $ (A, d) $完备性可知 $ (x_{n})_{n\geqslant1} $ 收敛, 从而有 $ x\in A $ , 此时 $ A=\baro{A} $, 因此 $ A $ 是 $ E $ 中闭集. 

     (2) 任取 $ (x_{n})_{n\geqslant1} $ 是 $ A $ 中的Cauchy列, 则 $ (x_{n})_{n\geqslant1} $ 也是 $ E $ 中的Cauchy列, 而 $ (E, d) $ 完备, 故 $ (x_{n})_{n\geqslant1} $ 收敛, 记 $ \lim\limits x_{n}=x $, 由 $ A $ 是闭集知 $ x\in A $, 故 $ (A, d) $ 完备. \qed
\end{Prf}
\section{连续映射与不动点定理}