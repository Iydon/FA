% !TeX root = main.tex
\chapter{拓扑空间与度量空间}
     \pagenumbering{arabic}
\section{基本概念}

     \begin{Definition}[拓扑空间]\label{def:拓扑空间}
          设 $ E $ 是一个集合, 称 $ E $ 的子集族 $ \tau $ 是一个\textbf{拓扑}\index{T!拓扑}, 若 $ \tau $ 满足: 
          \begin{enumerate}[($ \mathrm{O}_1 $), itemindent=2.5\parindent]
                \item $ E, \varnothing\in \tau $;
               \item\label{item:O2} $ \tau $ 中任意多元素的并仍是 $ \tau $ 中元素 (任意并);
               \item\label{item:O3} $ \tau $ 中有限多元素的交仍是 $ \tau $ 中的元素(有限交) . 
          \end{enumerate}
          并称 $ (E, \tau) $ 为一个\textbf{拓扑空间}\index{T!拓扑空间}, $ \tau $ 中的元素称为\textbf{开集}\index{K!开集}, 在不引起歧义的情况下, 可以简称 $ E $ 是一个拓扑空间. 
     \end{Definition}

     \begin{Remark}\label{rmk:平凡离散}
          对集合 $ E $ , 称 $ \tau=\{ E, \varnothing \} $ 为\textbf{平凡拓扑}\index{P!平凡拓扑}, 称 $ \tau=2^{E} $ 为\textbf{离散拓扑} \index{L!离散拓扑}. 
     \end{Remark}

     \begin{Example}
          $ E=\R $, 取 $ \tau=\Big\{ \bigcup\limits_{j\in\Z}(x_{j}: x_{j+1}), (x_{j})_{j\in\Z}\subset \R\cup\{ \pm\infty \} \Big\} $是一个拓扑, 这一拓扑称为 $ E $ 的\textbf{自然拓扑}\index{Z!自然拓扑}. 
     \end{Example}

     \begin{Definition}[度量空间]\label{def:度量空间}
          设 $ E $ 是非空集合, $ d: E\times E\to \R $ 满足 $ \forall x, y\in E $
          \begin{enumerate}[(1)]
               \item 非负性: $ d(x, y)\geqslant 0 $;
               \item 正定性: $ d(x, y)=0\Longleftrightarrow x=y $ ;
               \item 对称性: $ d(x, y)=d(y, x) $ ;
               \item 三角不等式: $ d(x, z)\leqslant d(x, y)+d(y, z) $ 
          \end{enumerate}
          则称 $ (E, d) $ 是一个\textbf{度量空间}\index{D!度量空间}, 并称 $ d $ 是 $ E $ 上的\textbf{度量}\index{D!度量}
     \end{Definition}

     \begin{Remark}\label{rmk:度量不唯一}
          度量并不唯一, 例如对 $ (E, d) $ 定义度量
          \begin{equation*}
               \begin{aligned}
                    r(x, y) = \min\{ d(x, y), 1 \}, \\
                    d'(x, y) = rd(x, y), r\in\R_{+},
               \end{aligned}
          \end{equation*}
          就是两个不同的度量. 
     \end{Remark}
     
     度量空间也是拓扑空间, 记 $ B(x, r)=\{ y: d(y, x)<r \} $ 可将 $ \tau $ 中的元素定义为: 
     \[
          U\in E, U\in \tau\Leftrightarrow \forall x\in U\exists r>0(B(x, r)\subset U)
     \] 
     该拓扑 $ \tau $ 称为由度量 $ d $ \textbf{诱导}的拓扑. 

     \begin{Example} 
          在实 Euclid 空间 $ \R^{n} $ 上赋以度量
          \begin{equation}
               d(x, y)=\bigg( \sum_{i=1}^{n}(x_{i}-y_{i})^{2} \bigg)^{1/2}\tag{Euclid距离}
          \end{equation}
          以 $ C[a, b] $ 以在 $ [a, b] $ 上的连续函数全体, 在其上赋以度量
          \begin{equation}
               d(x, y) = \max_{t\in [a, b]} |x(t)-y(t)|\tag{一致距离}
          \end{equation}
     \end{Example}

     \begin{Definition}[闭集]\label{def:闭集}
          设 $ (E, d) $ 是一个拓扑空间, $ A\subset E $ , 若 $ A^{c} $ 是开集, 则称 $ A $ 是 $ E $ 上(关于 $ \tau $ )的\textbf{闭集}.\index{B!闭集}
     \end{Definition}
     \begin{Example}
          设 $ (E, d) $ 是一个度量空间, 则闭球 $ \baro{B}(x, r)=\{ y: d(y, x)\leqslant r \} $ 是闭集. 
     \end{Example} 
     \begin{Proposition}\label{prop:闭集的性质}
          闭集具有以下性质:
          \begin{enumerate}[(1)]
               \item 全空间 $ E $ 和空集 $ \varnothing $ 是闭集;
               \item 任意多个闭集的交仍是闭集(任意交);
               \item 有限多个闭集的并仍是闭集(有限并).
          \end{enumerate}
     \end{Proposition}

     \subsection{邻域和邻域基}

     \begin{Definition}[邻域基]\label{def:邻域基}
          设 $ (E, \tau) $ 是一个拓扑空间, $ \alpha\in E $, 
          \begin{enumerate}[(1)]
               \item 若子集 $ V\subset E $ 满足
               \[
                    \exists U\in \tau(x\in U\land U\subset V)
               \]
               则称 $ V $ 是 $ x $ 的一个\textbf{邻域}\index{L!邻域}, 并记 $ \CN(x) $ 为全体 $ x $ 的邻域的集合, 称为 $ x $ 的\textbf{邻域系};
               \item 若 $ \CN(x) $ 的子集族 $ \CB(x) $ 满足
               \[
                    \forall V\in \CN(x)\exists U\in \CB(x)(U\subset V)
               \]  
               则称 $ \CB(x) $ 是 $ x $的\textbf{邻域基}\index{L!邻域基}. 
          \end{enumerate}
     \end{Definition}
     \begin{Remark}
          $ \CN(x) $是 $ x $  的邻域基; 包含 $ x $ 的所有开集的集合是 $ x $ 的邻域基. 
     \end{Remark}
     \begin{Example}
          设 $ (E, d) $ 是一个度量空间, $ x\in E $, 则 
          \[
               \CB(x)=\left\{ B\left(x,\frac{1}{n}\right): n\geqslant 1 \right\} 
          \]
          是 $ x $ 的邻域基. (可用来描述极限)
     \end{Example}
     \begin{Definition}[粘着集]
          设 $ (E, \tau) $ 是一个拓扑空间, $ A\subset E $ 
          \begin{enumerate}[(1)]
               \item 设 $ x\in E $ , 若 $ \forall V\in \CN(x)((V\cap A)\sm \{ x \}\neq\varnothing) $, 则称 $ x $ 是 $ A $ 的\textbf{凝聚点}\index{N!凝聚点};
               \item 若 $ x\in A $ 或 $ A $  的凝聚点, 则称 $ x $ 是 $ A $ 的\textbf{粘着点}, 记 $ A $ 的粘着点的全体为 $ \baro{A} $ , 称为 $ A $ 的\textbf{粘着集}\index{N!粘着集}, 也即
               \[
                    x\in \baro{A}\Leftrightarrow \forall V\in \CN(x)(V\cap A\neq\varnothing).    
               \]
          \end{enumerate}
     \end{Definition}
     \begin{Remark}
          $ E $ 中包含 $ A $ 的最小闭集称为 $ A $ 的\textbf{闭包}\index{B!闭包}, 可知 $ A $ 的闭包与粘着集是相同的, 它们从不同的角度描述了相同的集合. 
     \end{Remark}
     \begin{Definition}[稠密]\label{def:稠密}
          设 $ A\subset E $ , 若 $ \baro{A}=E $ , 则称 $ A $ 在 $ E $ 中\textbf{稠密}\index{C!稠密}, 或称 $ A $ 是 $ E $ 的一个\textbf{稠子集}
     \end{Definition}
     \begin{Example}
          有利数集 $ \Q $ 与无理数集 $ \mathbb{J} $ 都在 \R 中稠密. 
     \end{Example}
     \begin{Definition}[内部]\label{def:内部}
          设 $ A\subset E $, 若 $ A\in\CN(x) $ , 则称 $ x $ 是 $ A $ 的\textbf{内点},  $ A $ 的内点的全体称为 $ A $ 的\textbf{内部}\index{N!内部}, 记作 $ \mathring{A} $ 或 $ \mathrm{Int}(A) $.  
     \end{Definition}
     \begin{Definition}[边界]\label{def:边界}
          设 $ A\subset E $, 称 $ \partial A = \baro{A}\sm\mathring{A} $ 是 $ A $ 的\textbf{边界}\index{B!边界}.
     \end{Definition}
     \begin{Example}
          设 $ E=\R $ , 取 $ A = (2, 3) $, 则 $ \partial A = \{ 2, 3 \} $ ; 取 $ A = [2, 3) $ 则 $ \partial A = \{ 2, 3 \} $. 
          
          设 $ E = \R^{2} $, 取 $ A = (2, 3)\times \{0\} $ , 则 $ \partial A = [2, 3]\times \{0\} $ , 此时 $ A\subset \partial A $  ; 取 $ \mathbb{D}=B(0, 1) $ , 则 $ \partial \mathbb{D}=\mathbb{S}^{1} $ (单位圆周). 
     \end{Example}
     \begin{Theorem}\label{thm:闭包和内部的性质}
          设 $ E $ 是一个拓扑空间, $ A, B\subset E $ , 则:
          \begin{enumerate}[(1)]
               \item $ \baro{\baro{A}}=\baro{A} $, $ \mathring{\mathring{A}}=\mathring{A} $; (幂等性)
               \item $ E\sm\mathring{A}=\overline{E\sm A} $;
               \item \label{item:闭包内部}$ \overline {A\cup B}=\baro{A}\cup \baro{B} $; $ \mathring{\widehat{A\cap B}}=\mathring{A}\cap\mathring{B} $.   
          \end{enumerate}
     \end{Theorem}
     \begin{Remark}
          考虑定理~\ref{thm:闭包和内部的性质}~的~\ref{item:闭包内部}~,注意 $ \overline {A\cap B}=\baro{A}\cap \baro{B} $ 和 $ \mathring{\widehat{A\cup B}}=\mathring{A}\cup\mathring{B} $ 不一定成立. 比如取 $ A=\Q, B=\mathbb{J} $, 则此时 $ \overline{A\cap B}=\baro{\varnothing}=\varnothing $  , 而 $ \baro{A}\cap \baro{B}=\R\cap\R=\R\neq\overline {A\cup B} $ ; 同时 $ \mathring{\widehat{A\cap B}}=\mathring{\R}=\R $, 但是 $ \mathring{A}\cup\mathring{B}=\varnothing\cup\varnothing=\varnothing\neq \mathring{\widehat{A\cap B}} $  
     \end{Remark}
     \begin{Definition}[拓扑比较]\label{def:拓扑比较}
          设 $ \tau, \tau' $ 是 $ E $ 的拓扑, 若 $ \tau'\subset\tau $, 则称 $ \tau $ 是 $ \tau' $ 的\textbf{强拓扑}\index{Q!强拓扑}, 也即: $ \tau' $--开集一定是 $ \tau $--开集.   
     \end{Definition}
     \begin{Example}
          $ E=\R $ 时, 平凡拓扑 $ \subset $ 自然拓扑 $ \subset $ 离散拓扑. 
     \end{Example}
     \begin{Definition}[拓扑子空间]\label{def:拓扑子空间}
          设 $ (E, \tau) $ 是一个拓扑空间, $ F\subset E $ , 在 $ F $ 上定义
          \[
               \tau_{F}=\{ U\cap F: U\in\tau \} , 
          \]
          则 $ \tau_{F} $ 是一个拓扑, 称为由 $ \tau $  诱导的拓扑, 并称 $ (F, \tau_{F}) $ 是 $ (E, \tau) $ 的\textbf{拓扑子空间}\index{T!拓扑子空间}. 
     \end{Definition}
     \begin{Example}
          设 $ E=\R $ , 取 $ F=(2, 3) $ , 则 $ \tau_{F} $ 中元素是 \R 中开集; 
          
          取 $ F=[2, 3) $ , 则 $ \forall2<x<3, [2,x) $ 是 $ \tau_{F} $ 中元素, 但不是 \R 中开集. 
     \end{Example}
     \begin{Example}
          设 $ (E,d) $ 是度量空间, $ F\subset E $, 取 $ \delta=d|_{F\times F} $ 则 $ (F, \delta) $ 是度量空间, 且当 $ E,  F $ 分别赋以 $ d, \delta $ 诱导的拓扑时,  $ F $ 是 $ E $ 的拓扑子空间. 
     \end{Example}
     \begin{Proposition}
          设 $ F\subset E, A\subset F $ 则 $ A $ 是 $ F $ 中的闭集 $ \Longleftrightarrow $ 存在闭集 $ B\subset E $ 使得 $ A=B\cap F $ . 
     \end{Proposition}
     \begin{Proof}
          由闭集性质可以知道
          \[
               \begin{aligned}
                    A \text{\,是\,} F \text{\,中闭集\,} & \Longleftrightarrow F\sm A \text{\,是\,} F \text{\,中开集\,} ;\\
                    & \Longleftrightarrow \exists D\in \tau(F\sm A=D\cap F);\\
                    & \Longleftrightarrow A=D^{c}\cap F
               \end{aligned}     
          \]
          其中$ D^{c} $是闭集. 	 \qed
     \end{Proof}

     \subsection{分离空间}

     \begin{Definition}[Hausdorff空间]\label{def:Hausdorff空间}
          设 $ (E, \tau) $ 是拓扑空间, 若 
          \[
               \forall x, y\in E(x\neq y)\exists U\in \CN(x)\exists V\in\CN(x) (U\cap V)=\varnothing
          \]
          则称 $ (E, \tau) $ 是\textbf{Hausdorff空间}\index{H!Hausdorff空间}或\textbf{分离空间}. 
     \end{Definition}
     \begin{Example}
          所有离散拓扑空间都是 Hausdorff空间, 所有度量空间都是 Hausdorff空间, 元素个数多于一个的平凡拓扑空间不是 Hausdorff空间(因为所有元素都只有一个邻域, 即 $ E $ ). 
     \end{Example}
     \begin{Proposition}\label{prop:Hausdorff空间的相关命题1}
          由 Hausdorff空间的定义可以知道
          \begin{enumerate}[(1)]
               \item  $ E $ 是 Hausdorff空间 $ \Longleftrightarrow $ $ \forall x\in E $, 其所有闭邻域的交为 $\{ x \}$
               \item $ E $ 是 Hausdorff空间 $ \Longrightarrow $ $\bigcap \CN(x)=\{x\}$;
               \item $ E $ 是 Hausdorff空间 $ \Longrightarrow $ $ \forall F\subset E $, $ F $ 也是Hausdorff空间. 
          \end{enumerate}
     \end{Proposition}
     
\section{完备性}
     \subsection{序列的极限}
     \begin{Definition}[极限]\label{def:极限}
          设 $ (E, \tau) $ 是拓扑空间, $ (x_{n})_{n\geqslant 1} $ 是 $ E $ 中的序列,  $ x\in E $, 若
          \[
               \forall V\in\CN(x)\exists n_{0}\in\N(n\geqslant n_{0}\Rightarrow x_{n}\in V)
          \]
          则称 $ (x_{n})_{n\geqslant1} $ \textbf{收敛}于 $ x $, 并称 $ x $ 是序列 $ (x_{n})_{n\geqslant1} $ 的极限, 记作
          \[
               \tau-\lim_{n\to \infty}x_{n}=x. 
          \] 
          当不引起歧义的时候, 前面的 $ \tau- $ 可以省略. 
     \end{Definition}
     \begin{Remark}
          对定义\ref{def:极限}进行几点说明:
          \begin{enumerate}[(1)]
               \item 上述定义中的 $ \CN(x) $ 可换成 $ \CB(x) $ ;
               \item \R 中序列的极限至多只有一个, 例如 $\lim\limits_{n\to\infty}\frac{1}{n}=0 $, $ \lim\limits_{n\to\infty} n=\infty$(无极限), 取 $ (x_{n})_{n\geqslant1}=\left\{ 1, -1, 1,  -1\ldots \right\} $ 也无极限;
               \item\label{item:Hausdorff空间上极限唯一性} 一般地, Hausdorff空间中的序列至多只有一个极限, 而平凡拓扑空间中的任意序列收敛到任意元素;
               \item 对度量空间 $ (E, d) $ , 有
               \[
                    \lim_{n\to \infty}x_{n}=x \Leftrightarrow \lim_{n\to\infty}d(x_{n}, x)=0, 
               \]
               取邻域基 $ \CB(x)=\left\{ B(x, \frac{1}{n}): n\geqslant1 \right\} $即可.  
          \end{enumerate}
     \end{Remark}
     \begin{Proposition}
          设 $ E $ 是度量空间, 则 $ A\subset E $是闭集的充要条件是对任意 $ A $中序列  $ (x_{n})_{n\geqslant1} $ 当 $ \lim\limits_{n\to\infty}x_{n}=x_{0} $ 时, 都有$ x_{0}\in A $.  
     \end{Proposition}
     \begin{Proof}
          \textsl{必要性}. 若 $ x_{0}\notin A $, 则 $ \exists r>0 (B(x, r)\cap A=\varnothing) $, 而由极限的定义知 
          \[
               \exists n_{0}\in\N(n\geqslant n_{0}\Rightarrow x_{n}\in B(x, r))\Rightarrow x_{n}\notin A. 
          \]
          与 $ (x_{n})_{n\geqslant1}\subset A $ 矛盾. 

          \textsl{充分性}. 只需证 $ A^{c} $ 是开集, 用反证法, 若 $ A^{c} $ 不是开集, 则有
          \[
               \exists x\in A^{c}\,\forall r>0(B(x, r)\cap A\neq\varnothing),
          \]
          取 $ r=1/n, n=1, 2\ldots $ , 则有 $ x_{n}\in B(x, 1/n)\cap A $ . 由构造可知 $ (x_{n})_{n\geqslant1}\subset A $且 $ x_{n}\to x\,(n\to\infty) $  , 又由已知条件知 $ x\in A $, 矛盾. 故 $ A $ 为闭集. \qed 
     \end{Proof}
     \begin{Definition}[序列的粘着值]\label{def:序列的粘着值}
          设 $ (x_{n})_{n\geqslant1} $ 是 $ E $ 中的序列,  $ x\in E $, 若
          \[
               \forall V\in\CN(x)\,\forall n_{0}\in\N\,\exists n\geqslant n_{0}(x_{n}\in V),
          \]
          则称 $ x $ 是序列 $ (x_{n})_{n\geqslant1} $ 的\textbf{粘着值}\index{N!粘着值}.  
     \end{Definition}
     \begin{Example}
          \R 中序列 $ (1, -1, 1, -1\ldots) $ 的粘着值为 $ 1, -1 $.  
     \end{Example}
     \begin{Remark}
          序列的粘着值与集合的粘着点没有关系, 如 \R 上的序列 $ (x_{n})_{n\geqslant1}=(1, 1-\sqrt{2}, {1}/{2}, {1}/{2}-\sqrt{2}, {1}/{3}, {1}/{3}-\sqrt{2}, \ldots, {1}/{n}, {1}/{n}-\sqrt{2}, \ldots) $, 该序列的粘着值 $ 0, \sqrt{2} $ 均不在序列中, 而集合 $ A=\{ x_{n} \}_{n\geqslant1} $ 的粘着集为 $ A\cup\{ 0,-\sqrt{2} \} $. 

          再取 $ (x_{n})_{n\geqslant1}=(1, 1, \ldots) $ 则该序列的粘着值为$1$,  但 $ 1 $ 不是 $ A=\{ 1 \} $ 的凝聚点.  
     \end{Remark}
     \subsection{Cauchy列与完备性}
     \begin{Definition}[Cauchy列]\label{def:Cauchy列}
          设 $ (E, d) $是度量空间, $ (x_{n})_{n\geqslant1} $ 是 $ E $  中序列, 若
          \[
               \forall \varepsilon>0\,\exists n_{0}\in\N\,(n,m\geqslant n_{0}\Rightarrow d(x_{m},x_{n}<\varepsilon),
          \]
          则称 $ (x_{n})_{n\geqslant1} $ 是一个\textbf{Cauchy列}\index{C!Cauchy列}
     \end{Definition}
     \begin{Example}
          在 \R 中,  $ (1/n)_{n\geqslant1} $ 是Cauchy列, 收敛到0; $ ((1+{1}/{n})^{n})_{n\geqslant1} $ 是Cauchy列, 收敛到\me;

          而在 \Q 中, $ (1/n)_{n\geqslant1} $ 是Cauchy列, 收敛到0; $ ((1+{1}/{n})^{n})_{n\geqslant1} $ 是Cauchy列, 不收敛.
     \end{Example}
     \begin{Proposition}\label{prop:Cauchy列的性质}
          Cauchy列有以下性质:
          \begin{enumerate}[(1)]
               \item 若 $ (x_{n})_{n\geqslant1} $ 是收敛列, 则 $ (x_{n})_{n\geqslant1} $ 是Cauchy列;
               \item 若 $ (x_{n})_{n\geqslant1} $ 是Cauchy列且有收敛子列, 则 $ (x_{n})_{n\geqslant1} $ 是收敛列;
               \item 若 $ (x_{n})_{n\geqslant1} $ 是Cauchy列, 则 $ (x_{n})_{n\geqslant1} $ 有界. (即 $ \exists x\in E\,\exists r>0\,((x_{n})_{n\geqslant1} \subset B(x, r)) $ )
          \end{enumerate}
     \end{Proposition}
     \begin{Definition}[完备]\label{def:完备}
          设 $ (E, d) $ 是度量空间, 若 $ (E, d) $ 中任一 Cauchy列收敛, 则称 $ E $ 是\textbf{完备度量空间}或 $ E $ 是\textbf{完备}\index{W!完备}的, 此时也说 $ d $ 是完备的. 
     \end{Definition}
     \begin{Example}
          实 Euclid 空间 $ \R^{n} $, 复 Euclid 空间 $ \C^{n} $ 依 Euclid度量是完备的; \Q, \J 依Euclid度量不完备. 
     \end{Example}
     \begin{Example}
          在空间 $ C[a, b] $ 上赋以度量
          \[
               d(x, y)=\max_{0\leqslant t\leqslant1}\abs{x(t)-y(t)}
          \]
          时, 度量空间 $ (C[a, b], d) $ 是完备的, 其中 $ C[a, b] $ 是定义在 $ [0, 1] $ 上的连续函数全体
     \end{Example}
     \begin{Proof}
          取  $ (x_{n})_{n\geqslant1} $ 是Cauchy列, 即
          \[
               \forall\varepsilon>0\,\exists n_{0}\in\N\,(n,m\geqslant n_{0}\Rightarrow d(x_{n},x_{m})<\varepsilon)
          \]
          即 $ \forall n, m\geqslant n_{0}\,\forall t\in[a, b]\,(\abs{x_{n}(t)-x_{m}(t)}<\varepsilon) $ , 则有 $ (x_{n}(t))_{n\geqslant1} $ 是 $ \K $ 上的 Cauchy列\,(其中$\K=\R \text{\,或\,}\C$), 从而是收敛列, 并记
          \[
               x(t)=\lim_{n\to\infty}x_{n}(t), 
          \]
          则 $ \max\limits_{a\leqslant x\leqslant b}\abs{x(t)-x_{n}(t)}\to 0\,(n\to\infty) $, 从而 $ (x_{n})_{n\geqslant1} $  一致收敛到 $ x $ , 由  $ x_{n}\in C[a, b] $ 知 $ x\in C[a, b] $ , 即 $ (C[a, b], d) $ 完备. \qed
     \end{Proof}
     \begin{Remark}
          对 $ C[0, 1] $ , 若赋以度量 $ d(x, y)=\int_{0}^{1}\abs{x(t)-y(t)}\diff x $, 则 $ (C[0, 1], d) $ 不完备, 反例如下:

          取 $ C[0, 1] $ 中的函数列 $ (x_{n}(t))_{n\geqslant1}=(t^{n})_{x\geqslant1} $, 易证  $ (x_{n})_{n\geqslant1} $ 是 Cauchy列, 但若记其极限函数是 $ x(t) $ 则有
          \[
               x(t)=\begin{cases}
                    0, & 0\leqslant t<1; \\
                    1, & t=1. 
               \end{cases}
          \]
          从而在空间给定的情形下, 完备性依赖与距离, 因此\textbf{完备不是一个拓扑概念}
     \end{Remark}
     \begin{Theorem}
          度量空间 $ (E, d) $ 完备的充分必要条件是对任意单调递减的非空子集列 $ (A_{n})_{n\geqslant1} $ 若 $ \lim\limits_{n\to\infty}\diam(A_{n})=0 $ , 则 $ \bigcap\limits_{n\geqslant1}A_{n} $ 是单点集. 其中 $ \diam(A)=\sup\limits_{x,y\in A}d(x, y) $ 是 $ A $  的\textbf{直径}. 
     \end{Theorem}
     \begin{Proof}
          \textsl{必要性} . 因为 $ \lim\limits_{n\to\infty}\diam(A_{n})=0 $ 则
          \[
               \forall \varepsilon>0\,\exists n_{0}\in\N\,(n\geqslant N\Rightarrow \diam(A_{n})<\varepsilon). 
          \]
          当 $ m>n>N $ 时, 取 $ x_{n}\in A_{n} $ , $ x_{m}\in A_{m} $ 有
          \[
               d(x_{m}, x_{n})<\diam(A_{n})<\varepsilon. 
          \]
          可以知道如此构造出的 $ (x_{n})_{n\geqslant1} $ 为Cauchy列, 又由 $ (E, d) $ 的完备性知 $ (x_{n})_{n\geqslant1} $ 收敛, 不妨设 $\lim\limits_{n\to\infty}x_{n}=x$ , 由 $ A_{n} $ 非空可知 $ x\in A_{n} $ , 故 $ x\in\bigcap\limits_{n\geqslant1}A_{n} $ . 

          再设 $ y\in\bigcap\limits_{n\geqslant1} A_{n} $ , 因 $ d(x_{n}, y)<\diam(A_{n})<\varepsilon $ 知 $ (x_{n})_{n\geqslant1} $ 收敛于 $ y $ , 由 $ E $ 是度量空间可知 $ E $ 是 Hausdorff空间, 从而$ x=y $, 故 $ \bigcap\limits_{n\geqslant1} A_{n} $ 为单点集. 

          \textsl{充分性} . 取 $ (x_{n})_{n\geqslant1} $ 是 $ E $ 中的 Cauchy列, 取 $ A_{n}=\left\{ x_{m}:m>n \right\} $, 则 $ (A_{n})_{n\geqslant1} $ 是一个单调递减的非空闭子集列, 下面检验 $ \lim\limits_{n\to\infty}\diam(A_{n})=0 $ . 

          由 $ (x_{n})_{n\geqslant1} $ 是 Cauchy列可知
          \[
               \forall \varepsilon>0\, \exists n_{0}\in\N\,(m, n>n_{0}\Rightarrow d(x_{m}, x_{n})<\varepsilon),
          \]
          则 $ \forall x, y\in A_{n_{0}} $ , 取 $ x_{n}, x_{m}\in A_{n_{0}} $ 使得 $ d(x, x_{n}), d(y, x_{m})<\varepsilon $ 则
          \[
               d(x, y)\leqslant d(x, x_{n})+d(x_{n}, x_{m})+d(x_{m}, y) < 3\varepsilon         
          \]
          从而 $ \diam(A_{n_{0}})\leqslant3\varepsilon $, 也即 $\lim\limits_{n\to\infty} \diam(A_{n})=0 $. 

          由条件知 $ \bigcap\limits_{n\geqslant1}A_{n} $ 是单点集, 不妨记作 $ \{ x_{0} \} $ , 则 $ \lim\limits_{n\to\infty}x_{n}=x_{0} $, 从而 $ (E, d) $  是完备的. \qed
     \end{Proof}
     \begin{Theorem}
          设 $ (E, d) $ 是度量空间, $ A\subset E $. 
          \begin{enumerate}[(1)]
               \item 若 $ (A, d) $ 完备, 则 $ A $ 是 $ (E, d) $ 中的闭集;
               \item 若 $ (E, d) $ 完备且 $ A $ 是闭集, 则 $ (A, d) $ 完备. 
          \end{enumerate} 
     \end{Theorem}
     \begin{Proof}
          (1) 设 $ x $ 是 $ A $ 的凝聚点, 则存在 $ (x_{n})_{n\geqslant1}\subset A $, 使得 $ \lim\limits_{n\to\infty}x_{n}=x $ , 则 $ (x_{n})_{n\geqslant1} $ 是 $ A $ 中的 Cauchy列. 由 $ (A, d) $完备性可知 $ (x_{n})_{n\geqslant1} $ 收敛, 从而有 $ x\in A $ , 此时 $ A=\baro{A} $, 因此 $ A $ 是 $ E $ 中闭集. 

          (2) 任取 $ (x_{n})_{n\geqslant1} $ 是 $ A $ 中的Cauchy列, 则 $ (x_{n})_{n\geqslant1} $ 也是 $ E $ 中的Cauchy列, 而 $ (E, d) $ 完备, 故 $ (x_{n})_{n\geqslant1} $ 收敛, 记 $ \lim\limits x_{n}=x $, 由 $ A $ 是闭集知 $ x\in A $, 故 $ (A, d) $ 完备. \qed
     \end{Proof}
\section{连续映射与不动点定理}
     \subsection{连续与一致连续}
     \begin{Definition}[连续]\label{def:连续}
          设 $ E $, $ F $ 为拓扑空间,  $ f:E\to F, x\in E $, 若
          \[
               \forall V\in\CN(f(x))\,\exists U\in\CN(x)\,(f(U)\subset V),
          \]
          则称 $ f $ 在 $ x $ 处\textbf{连续}\index{L!连续}, 或等价地
          \[
               \forall V\in\CN(f(x))\,(f^{-1}(V)\in\CN(x))
          \]
          若 $ f $ 在 $ E $ 的每一点都连续, 则称 $ f $ 在 $ E $ 上连续. 
     \end{Definition}
     \begin{Remark}
          若 $ E $ , $ F $ 为度量空间, 若任取序列 $ (x_{n})_{n\geqslant1} $ 收敛到 $ x $ , 都有 $ \lim\limits_{n\to\infty}f(x_{n})=f(x) $, 则称 $ f $ 在 $ x $ 处连续. 
     \end{Remark}
     \begin{Proposition}\label{prop:连续映射的性质}
          连续映射具有以下性质:
          \begin{enumerate}[(1)]
               \item $ f $ 在 $ E $ 上连续 $ \Longleftrightarrow $ 对 $ F $ 上的开集 $ V $ , 有 $ f^{-1}(V) $ 是 $ E $ 中开集;
               \item $ f $ 在 $ E $ 上连续 $ \Longleftrightarrow $ 对 $ F $ 上的闭集 $ B $ , 有 $ f^{-1}(B) $ 是 $ E $ 中闭集;
               \item 连续映射的复合是连续映射;
               \item $ \tau $, $ \tau' $ 是 $ E $ 中拓扑,  $ \tau $ 是 $ \tau' $ 的强拓扑 $ \Longleftrightarrow $ $ \mathrm{id}_{E}:(E, \tau)\to(E, \tau') $ 连续. 
          \end{enumerate}
     \end{Proposition}
     \begin{Definition}[开映射, 同胚]\label{def:开映射, 同胚}
          设 $ E $, $ F $ 是拓扑空间,  $ f:E\to F $ 
          \begin{enumerate}[(1)]
               \item 若任取开集 $ U\subset E $ , 有 $ f(U) $ 是 $ F $ 中开集, 则称 $ F $ 是一个\textbf{开映射}\index{K!开映射};
               \item 若 $ f $ 是双射, 且 $ f\text{\,与\,}f^{-1} $ 连续, 则称 $ f $ 是\textbf{同胚映射}\index{T!同胚}, 若 $ E $, $ F $ 间有同胚映射, 则称 $ E\text{\,与\,}F $ \textbf{同胚}.  
          \end{enumerate}
     \end{Definition}
     \begin{Definition}[一致连续]\label{def:一致连续}
          设 $ (E, d) $ 与 $ (F, \delta) $ 是度量空间, 若 $ f:E\to F $ 满足
          \[
               \forall\varepsilon>0\,\exists\eta>0\,(d(x, y)<\eta\Rightarrow\delta(f(x), f(y))<\varepsilon),
          \]
          则称 $ f $ 是\textbf{一致连续映射}\index{Y!一致连续映射}. 
     \end{Definition}

     \begin{Remark}
          一致连续映射一定是连续映射, 它将 Cauchy列映成 Cauchy列.

          $ f(x)=x $ 在\R 上一致连续,  $ f(x)=x^{2} $ 在任何\R 的有限区间上一致连续, 而在无限区间上不一致连续.  
     \end{Remark}

     \begin{Theorem}[一致连续映射的扩展]\label{thm:一致连续映射的扩展}
          设 $ (E, d)\text{\,与\,}(F, \delta) $ 是度量空间,  $ (F, \delta) $ 完备,  $ E_{0}\text{\,是\,}E $ 的稠子集, 若 $ f:E_{0}\to F $ 一致连续, 则 $ f $ 可唯一地扩展成 $ (E, d)\text{\,到\,}(F, \delta) $ 的一致连续映射 $ \tilde{f}:E\to F $.  
     \end{Theorem}
     \begin{Proof}
          \textbf{(Step 1)} 构造 $ \tilde{f} $. 
          
          由 $ E_{0}\text{\,在\,}E $ 中稠密可知 
          \[
               \forall x\in E\,\exists (x_{n})_{n\geqslant1}\subset E_{0}(\lim_{n\to\infty}x_{n}=x), 
          \] 
          定义 $ \tilde{f}(x)=\lim\limits_{n\to\infty}f(x_{n}) $ , 需证明 $ (f(x_{n}))_{n\geqslant1} $ 也是 Cauchy列, 且 $ \tilde{f}(x) $ 不依赖 $ (x_{n})_{n\geqslant1} $ 的选取. 
          \begin{enumerate}[1\degree]
               \item 因为 $ (x_{n})_{n\geqslant1} $ 是 Cauchy列, $ f $ 一致连续, 知 $ (f(x_{n}))_{n\geqslant1} $ 也是 Cauchy列. 因为 $ (F, \delta) $ 完备, 故存在 $ \tilde{f}(x) $ 使得 $ \lim\limits_{n\to\infty}f(x_{n})=\tilde{f}(x) $.
               \item 设 $ (x'_{n})_{n\geqslant1}\subset E_{0} $, 且 $ \lim\limits_{n\to\infty}x'_{n}=x $ , 记 $ y'=\lim\limits_{n\to\infty}f(x'_{n}) $ , 令 $ y=\tilde{f}(x) $ 则
               \[
                    \begin{aligned}
                         \delta(y, y') & \leqslant \delta(y, f(x_{n}))+\delta(f(x_{n}), f(x'_{n}))+\delta(f(x'_{n}), y')\\
                         & \to 0\,(n\to\infty)
                    \end{aligned}
               \]
               即 $ y=y' $ , 从而 $ \tilde{f} $ 在 $ E $ 上是良定的. 
          \end{enumerate} 

          \textbf{(Step 2)} 证明 $ \tilde{f} $ 是一致连续映射. 

          对任意 $ \varepsilon>0 $ , 因为 $ f $ 一致连续, 则 $ d(x, y)<\eta $ 时, $ \delta(f(x), f(y))<\varepsilon $, 而 
          \[
               \forall x', y'\in E\,\exists  (x_{n})_{n\geqslant1}\subset E_{0} \,\exists  (y_{n})_{n\geqslant1}\subset E_{0} \,(\lim_{n\to\infty}x_{n}=x'\land \lim_{n\to\infty}y_{n}=y' ), 
          \]  
          也即 
          \[
               \exists n_{0}\in\N\,\forall n\geqslant n_{0}\,\left(d(x_{n}, x')<\frac{\eta}{3}\land d(y_{n}, y')<\frac{\eta}{3}\right), 
          \] 
          则当 $ d(x', y')<\eta/3 $ 时, 有
          \[
               \begin{aligned}
                    d(x_{n}, y_{n}) & \leqslant d(x_{n}, x')+d(x', y')+d(y', y_{n})\\
                    & < \frac{\eta}{3}+\frac{\eta}{3}+\frac{\eta}{3}=\eta.
               \end{aligned}
          \]
          又因为 $ f $ 一致连续, 故 $ \delta(f(x_{n}), f(y_{n}))<\varepsilon $, 而又有
          \[
               \tilde{f}(x')=\lim_{n\to\infty}f(x_{n}), \tilde{f}(y')=\lim_{n\to\infty}f(y_{n})
          \]
          即
          \[
               \exists n_{1}\in\N\,\forall n\geqslant n_{1}\,(\delta(\tilde{f}(x_{n}), \tilde{f}(x'))<\varepsilon\land \delta(\tilde{f}(y_{n}), \tilde{f}(y'))<\varepsilon)
          \]
          则
          \[
               \begin{aligned}
                    \delta(\tilde{f}(x'), \tilde{f}(y')) & \leqslant \delta(\tilde{f}(x_{n}), \tilde{f}(x'))+\delta(\tilde{f}(y_{n}), \tilde{f}(x_{n}))+\delta(\tilde{f}(y_{n}), \tilde{f}(y'))\\
                    & <\varepsilon+\varepsilon+\varepsilon=3\varepsilon. 
               \end{aligned}
          \]
          从而 $ \tilde{f} $ 是一致连续的. 

          \textbf{(Step 3)} 证明 $ \tilde{f} $ 的唯一性. 

          
          设 $ {f}' $ 是 $ f $ 的另一个扩张, 由 $ E_{0} $ 稠密, 知
          \[
               \forall x\in E\,\exists (x_{n})_{n\geqslant1} \subset E_{0}\,(\lim_{n\to\infty}x_{n}=x),
          \]
          则有
          \[
               f'(x)=\lim_{n\to\infty}f'(x_{n})=\lim_{n\to\infty}f(x_{n})=\tilde{f}(x), 
          \]
          从而 $ f'=\tilde{f} $. \qed
     \end{Proof}

     \subsection{压缩映照原理}

     \begin{Definition}[H\"older映射]\label{def:H\"older映射}
          设 $ (E, d) $ 与 $ (F, \delta) $ 是度量空间,  $ f:E\to F $, $ 0<\alpha<1 $ , 
          \begin{enumerate}[(1)]
               \item 若
               \[
                    \exists \lambda>0\,\forall x, y\in E\,(\delta(f(x), f(y))\leqslant\lambda d(x, y)^{\alpha}),
               \]
               则称 $ f $ 是阶数为 $ \alpha $ 的\textbf{H\"older映射}\index{H!H\"older映射};
               \item 若 $ f $ 为阶数为 $ 1 $ 的 H\"older映射, 则称 $ f $ 是\textbf{Lipschitz映射}\index{L!Lipschitz映射}, 并称使得不等式成立的最小常数 $ \lambda $是 $ f $ 的\textbf{Lipschitz常数}.
               \item 若 $ f $ 的Lipschitz常数 $ \lambda<1 $ , 则称 $ f $ 是\textbf{压缩映射.}
          \end{enumerate} 
     \end{Definition}

     \begin{Remark}
          由定义知压缩映射 $ \subset $ Lipschitz映射 $ \subset $ H\"older映射 $ \subset $ 一致连续映射.
     \end{Remark}

     \begin{Example}
          $ f(x)=\sin x $ 是 Lipschitz映射, 且其 Lipschitz常数 $ \lambda=1 $ ;

          $ f(x)=\abs{x} $ 也是 Lipschitz映射, 其 Lipschitz常数 $ \lambda=1 $ ;
          
          $ f(x)=\sqrt{x} $ 不是 Lipschitz映射, 但它是阶为 $ 1/2 $ 的 H\"older映射, 因为 $ f'(x)=\frac{1}{2\sqrt{x}}\to\infty\,(x\to 0) $ , 但
          \[
               \abs{\sqrt{x}-\sqrt{y}}=\frac{\abs{x-y}}{\abs{\sqrt{x}+\sqrt{y}}}=\frac{\sqrt{x-y}}{\sqrt{x}+\sqrt{y}}\cdot\abs{x-y}^{1/2}\leqslant\abs{x-y}^{1/2}.
          \]
     \end{Example}

     \begin{Theorem}[压缩映照原理]\label{thm:压缩映照原理}
          设 $ (E, d) $ 是完备度量空间,  $ f:E\to E $ 是压缩映射, 则 $ f $ 有唯一不动点, 即 $ \exists!x\,(f(x)=x) $ .
     \end{Theorem}

     \begin{Proof}
          任取 $ x_{1}\in E $ , 并归纳地定义 $ x_{n+1}=f(x_{n}) $ , 则 $ (x_{n})_{n\geqslant1} $ 是 $ E $ 中序列, 往证 $ (x_{n})_{n\geqslant1} $ 是 Cauchy列, 令 $ \lambda $ 是 $ f $ 的Lipschitz常数, 则 $ \lambda<1 $ , 此时
          \[
               \begin{aligned}
                    d(x_{n+1}, x_{n}) & = d(f(x_{n}), f(x_{n-1})) \leqslant \lambda d(x_{n}, x_{n-1}) \\
                    & =\lambda d(f(x_{n-1}), f(x_{n-2}))\leqslant \lambda^{2} d(x_{n-1}, x_{n-2}) \\
                    & \leqslant \lambda^{n} d(x_{2}, x_{1}).
               \end{aligned}
          \]
          则 $ \forall p\in \N $ , 有
          \[
               d(x_{n+p}, x_{n})\leqslant\sum_{i=n+1}^{n+p}d(x_{i}, x_{i-1})\leqslant\sum_{i=n+1}^{n+p}\lambda^{i-2}d(x_{2}, x_{1})=\frac{\lambda^{n-1}}{1-\lambda}d(x_{2}, x_{1})\to 0\,(n\to\infty)
          \]
          从而 $ (x_{n})_{n\geqslant1} $ 是 Cauchy列, 因为 $ E $ 完备, 则存在 $ x\in E $ , 使得 $ \lim\limits_{n\to \infty}x_{n}=x $ , 此时
          \[
               f(x)=f(\lim_{n\to\infty}x_{n})=\lim_{n\to\infty}f(x_{n})=\lim_{n\to\infty}x_{n+1}=x,
          \]
          即 $ x $ 是 $ f $ 的不动点.

          若 $ y\in E $ 也是 $ f $ 的不动点, 则由
          \[
               d(x, y)=d(f(x), f(y))\leqslant\lambda d(x, y)\,(\lambda<1)
          \]
          知 $ d(x, y)=0 $ , 即 $ x=y $, 故不动点唯一.\qed 
     \end{Proof}



\section{度量空间的完备化}
	\begin{Definition}[等距同构]\label{def:等距同构}
           设 $ (E, d) $ 与 $ (F, \delta) $ 是度量空间, 设 $ f:(E, d)\to(F, \delta) $, 若
           \[
                \forall x, y\in E\,\big(d(x, y)=\delta(f(x), f(y))\big),
           \] 
           则称 $ f $ 是 $ (E, d) $ 到 $ (F, \delta) $ 的\textbf{等距映射}; 若 $ f $ 是双射, 则称 $ f $ 是 $ (E, d) $ 到 $ (F, \delta) $ 的\textbf{等距同构映射}\index{D!等距同构映射}.
     \end{Definition}

     \begin{Example}
          设 $ E:[0,\infty), d(x, y)=\abs{x-y} $, $ F=[0, \infty), \delta(x, y)=\abs{x^{2}-y^{2}} $, 则 $ f:E\to F, x\mapsto \sqrt{x} $ 是等距同构. 
     \end{Example}

     下面证明本节的主要定理

     \begin{Theorem}[度量空间的完备化]
           设 $ (E, d) $ 是度量空间, 则存在等距意义下唯一的完备度量空间 $ (\widehat{E}, \hat{d}) $ 满足
           
           (i) $ E\subset \widehat{E} $;\hspace{6em}(ii) $ \hat{d}|_{E}=d $;\hspace{6em}  (iii)$ E $ 在 $ \widehat{E} $ 中稠密. 

           \noindent 并称 $ (\widehat{E}, \hat{d}) $ 是 $ (E, d) $ 的\textbf{完备化空间}.
     \end{Theorem}
     \begin{Proof}

          \textbf{(Step 1)} 构造空间 $ (\widehat{E}, \hat{d}) $ .

          取 $ \widetilde{E} $ 是 $ E $ 中 Cauchy列的全体, 并在 $ \widetilde{E} $ 上取等价关系
          \[
               (x_{n})_{n\geqslant1}\sim (y_{n})_{n\geqslant1}\Longleftrightarrow \lim_{n\to\infty}d(x_{n},y_{n})=0,
          \]
          并取 $ \widehat{E}=\widetilde{E}/\sim $ . 因为 $ \forall a\in E $ , 存在常数列  $ (a)_{n\geqslant1} $, 使得 $ \lim\limits_{n\to\infty}a=a $, 记其等价类为 $ \hat{a} $ , 则存在映射 $ \iota :E\to\widehat{E}, a\mapsto \hat{a} $ 是一个单射, 由此知 $ E\subset\widehat{E} $. 

          定义 $ \widehat{E} $ 上的度量 $ \hat{d} $ , 设 $ (x_{n})_{n\geqslant1} $ 与 $ (y_{n})_{n\geqslant1} $ 分别是 $ \hat{x} $, $ \hat{y} $ 的代表元, 则
          \[
               \hat{d}(\hat{x}, \hat{y})=\lim_{n\to\infty}d(x_{n},y_{n}).
          \]
          因 $ \hat{d} $ 的非负性, 正定性, 对称性与三角不等式由 $ d $ 的相应性质可证, 故只需证明 $ \lim\limits_{n\to\infty}d(x_{n}, y_{n}) $ 存在且不依赖代表元的选取.

          \begin{enumerate}[(1)]

               \item 由 $ (x_{n})_{n\geqslant1} $ , $ (y_{n})_{n\geqslant1} $ 都是 Cauchy列, 当 $ m, n\to\infty $ 时
               \[
                    \begin{aligned}
                         \abs{d(x_{n}, y_{n})-d(x_{m}, y_{m})} & \leqslant \abs{d(x_{n}, y_{n})-d(x_{m}, y_{n})+ d(x_{m}, y_{n})-d(x_{m}, y_{m})}\\
                         & \leqslant \abs{d(x_{n}, y_{n})-d(x_{m}, y_{n})}+\abs{d(x_{m}, y_{n})-d(x_{m}, y_{m})}\\
                         & \leqslant d(x_{n}, x_{m})+d(y_{n}, y_{m})\to 0
                    \end{aligned}
               \]
               则 $ d(x_{n}, y_{n})_{n\geqslant1} $ 是实的Cauchy列,  故 $ \lim\limits_{n\to\infty}d(x_{n}, y_{n}) $ 存在;

               \item 再设 $ (x'_{n})_{n\geqslant1} $ , $ (y'_{n})_{n\geqslant1} $ 也是 $ \hat{x} $ , $ \hat{y} $ 的代表元, 则 $ n\to\infty $ 时
               \[
                    \abs{d(x'_{n}, y'_{n})-d(x_{n}, y_{n})}\leqslant d(x'_{n}, x_{n})+d(y'_{n}, y_{n})\to0.
               \]
               从而 $ \hat{d} $ 是一个度量, 且 $ \forall a, b\in E $ , 有 $ \hat{d}(a, b)=d(a, b) $ , 从而 $ \hat{d}|_{E}=d $ .
          \end{enumerate}
          接下来证明 $ E $ 在 $ \widehat{E} $ 中稠密. 任取 $ \hat{x}\in\widehat{E} $ , 设 $ (x_{n})_{n\geqslant1} $ 是它的一个代表元, 对每一个 $ n\in\N $ , 定义 $ \baro{x}_{n} $ 表示常数列 $ (x_{n})_{m\geqslant1} $(可看成与 $ x_{n}\in E $ 为同一元素) , 那么
           \[
               \lim_{n\to\infty}\hat{d}(\hat{x}, \baro{x}_{n})=\lim_{n\to\infty}\lim_{m\to\infty}d (x_{m}, d_{n})\to 0
          \]
          从而 $ E $ 在 $ \widehat{E} $ 中稠密. 

          \textbf{(Step 2)} 证明 $ (\widehat{E}, \hat{d}) $ 完备.

          取 $ (\hat{x}_{(n)})_{n\geqslant1} $ 为 $ \widehat{E} $ 中的Cauchy列, 由 $ E $ 在 $ \widehat{E} $ 中稠密可知
          \[
               \forall n\in \N\,\exists x_{n}\in E\,\left(\hat{d}(\hat{x}_{(n)}, \baro{x}_{n})<\frac{1}{n}\right)
          \]
          于是可得
          \[
               \begin{aligned}
                    d(x_{n}, x_{m}) & =\hat{d}(\baro{x}_{n}, \baro{x}_{m}) \leqslant \hat{d}(\baro{x}_{n}, \hat{x}_{(n)})+\hat{d}(\hat{x}_{(m)}, \hat{x}_{(n)})+\hat{d}(\baro{x}_{m}, \hat{x}_{(m)})\\
                    & < \frac{1}{n}+\hat{d}(\hat{x}_{(m)}, \hat{x}_{(n)})+\frac{1}{m}\to0\,(m, n\to\infty)
               \end{aligned}
          \]
          即可知 $ (x_{n})_{n\geqslant1} $ 是 Cauchy列, 则有 $ (x_{n})_{n\geqslant1}\in\widetilde{E} $ , 设 $ (x_{n})_{n\geqslant1} $ 在 $ \widehat{E} $ 中的代表元为 $ \hat{x} $ , 则有
          \[
               \hat{d}(\hat{x}_{(n)}, \hat{x})\leqslant\hat{d}(\hat{x}_{(n)}, \baro{x}_{n})+\hat{d}(\hat{x}, \baro{x}_{n})<\frac{1}{n}+\hat{d}(\hat{x}, \baro{x}_{n})\to 0\,(n\to\infty),
          \]
          因此, $ \lim\limits_{n\to\infty}\hat{x}_{(n)}=\hat{x} $ , 所以 $ (\widehat{E}, \hat{d}) $ 是一个完备的度量空间.

          \textbf{(Step 3)} 最后证明这种完备化在等距意义下一唯一的.

          我们可以把 $ (\widehat{E}, \hat{d}) $ 满足的条件归纳为
          \begin{enumerate}[(1)]
               \item $ \iota:E\to\widehat{E} $ 是单射, 可记为 $ E\hookrightarrow \widehat{E} $;
               \item $ \hat{d}|_{\iota(E)}=d $ : 任取 $ a, b\in E $ , 有 $ \hat{d}(\iota(a), \iota(b))=d(a, b) $ ;
               \item $ \iota(E) $ 在 $ \widehat{E} $ 中稠密 .
          \end{enumerate}
          也就是说 $ E $ 与 $ \widehat{E} $ 上的稠密子集 $ \iota(E) $ 等距同构.

          设 $ (E', d') $ 是另一个满足如上条件的度量空间, 并记 $ E $ 到 $ E' $ 的等距映射为 $ \iota' $ , 定义映射
          \[
                f: \iota(E)  \to E', \hat{a} \mapsto \iota'(a)
          \]
          由 (1), (2) 可知 $ f $ 是等距的, 则它是一致连续的. 又由 (3) 知 $ \iota(E) $ 在 $ \widehat{E} $ 中稠密, 以及 $ (E', d') $ 完备, 根据定理\ref{thm:一致连续映射的扩展}可知 $ f $ 可唯一地扩展成 $ \widehat{E} $ 到 $ E' $ 的一致连续映射 $ \tilde{f} $. 对任意 $ \hat{x}, \hat{y}\in \widehat{E} $ , 其代表元分别为 $ (x_{n})_{n\geqslant1} $与 $ (y_{n})_{n\geqslant1} $, 则有
          \[
               \begin{aligned}
                    d'(\tilde{f}(\hat{x}), \tilde{f}(\hat{y})) & = d'(\lim_{n\to\infty}f(x_{n}), \lim_{n\to\infty}f(y_{n}))\\
                    & = \lim_{n\to\infty} d'(f(x_{n}), f(y_{n}))\\
                    & = \lim_{n\to\infty} d(x_{n}, y_{n})=\hat{d}(\hat{x}, \hat{y}).
               \end{aligned}
          \]
          故 $ \tilde{f}:\widehat{E}\to E' $ 也是等距映射, 因此也是单射. 由 $ \iota'(E) $ 在 $ E' $ 中稠密可知 $ \tilde{f} $ 是满射, 故 $ \tilde{f} $ 是双射. 所以在等距同构意义下, $ E $ 的完备化度量空间是唯一的. \qed
     \end{Proof}

     \begin{Remark}
          若 $ (E, d) $ 完备, 则 $ \widehat{E}=E $ , 这说明完备化只需做一次, 且完备化空间是包含 $ E $ 的最小完备化空间.
     \end{Remark}

\section{紧性}
     \subsection{紧性, 紧空间}
     \begin{Definition}[紧性]\label{def:紧性}
           设 $ E $ 是拓扑空间,
           \begin{enumerate}[(1)]
                \item 若 $ E $ 上的开集族 $ (O_{i})_{i\in \alpha} $ 满足 $ E=\bigcup_{i\in\alpha}O_{i} $ , 则称 $ (O_{i})_{i\in\alpha} $ 是 $ E $ 的一个\textbf{开覆盖}.
                \item 若对 $ E $ 的任意开覆盖都有有限子覆盖, 则称 $ E $ 是\textbf{紧}的\index{J!紧}.
           \end{enumerate}
     \end{Definition}

     \begin{Proposition}\label{prop:紧性的另一刻画}
           $ E $ 紧的充分必要条件是任意闭集族 $ (F_{i})_{i\in\alpha}\subset E $, 若 $ \forall \beta\in\fin\alpha $ 都有 $ \bigcap_{i\in\beta}F_{i}\neq \varnothing $ , 则有 $ \bigcap_{i\in\alpha}F_{i}\neq\varnothing $. 
     \end{Proposition}

     \begin{Proof}
          \textsl{必要性}. 用反证法, 假设 $ \bigcap_{i\in\alpha}F_{i}=\varnothing $ , 即 $ \bigcup_{i\in\alpha}F_{i}^{c}=E $ , 即 $ (F_{i}^{c})_{i\in\alpha} $ 是 $ E $ 的一个开覆盖, 因为 $ E $ 是紧的,  所以存在有限集 $ \beta\in\fin\alpha $ ,  使得 $ \bigcup_{i\in\beta}F_{i}^{c}=E $ , 即 $ \bigcap_{i\in\beta}F_{i}^{c}=\varnothing $ , 矛盾.

          \textsl{充分性}. 任取 $ (O_{i})_{i\in\alpha} $ 是 $ E $ 的开覆盖, 即 $ \bigcup_{i\in\alpha}O_{i}=E $ , 则 $ \bigcap_{i\in\alpha}O_{i}^{c}=\varnothing $ , 则存在有限集 $ \beta\in\fin \alpha $, 使得 $ \bigcap_{i\in\beta}O_{i}^{c}=\varnothing $ , 即 $ \bigcup_{i\in\beta}O_{i}=E $ , 从而 $ E $ 是紧的.\qed
     \end{Proof}

     \begin{Proposition}
          设 $ E $ 是紧的拓扑空间, $ (F_{n})_{n\geqslant1} $ 是 $ E $ 中单降的非空闭集族, 则 $ \bigcap_{n\geqslant1}F_{n}\neq\varnothing $. 
     \end{Proposition}

     \begin{Proof}
          对任意 $ j_{i}\in\N\,(i=1, 2, 3,\dots k) $ , 都有 $ \bigcap_{i=1}^{k}F_{i}=F_{\max\{ j_{i} \}}\neq\varnothing $ , 故 $ \bigcap_{n\geqslant1}F_{n}\neq\varnothing $ .\qed
     \end{Proof}

     \begin{Theorem}
           设 $ E $ 是拓扑空间,  $ F\subset E $ 是子空间, 则 $ F $ 紧的充分必要条件是若 $ E $ 中的开集族 $ (O_{i})_{i\in\alpha} $ 满足 $ F\subset \bigcup_{i\in\alpha}O_{i} $ , 则存在有限集 $ \beta\in\fin\alpha $ , 使得 $ F\subset\bigcup_{i\in\beta}O_{i} $ (也就是说, $ F $ 任一在 $ E $ 中的开覆盖 $ (O_{i})_{i\in\alpha} $ 都存在有限的子覆盖).
     \end{Theorem}
     
     \begin{Proof}
          设 $ F\subset E $ , $ (O_{i})_{i\in\alpha} $ 是 $ E $ 中的开集族, 且满足 $ F\subset \bigcup_{i\in\alpha}O_{i} $ . 令
          \[
               U_{i}=F\cap O_{i}, i\in\alpha,
          \]
          则 $ (U_{i})_{i\in\alpha} $ 是空间 $ F $ 上的开覆盖, 反之,  $ F $ 的任意开覆盖都可以表示成以上形式.

          \textsl{必要性}. 若 $ F $ 是紧子空间, 那么存在有限集 $ \beta\in\fin\alpha $ , 使得 $ F=\bigcup_{i\in\beta}U_{i} $ , 所以有 $ F\subset\bigcup_{i\in\beta}O_{i} $.

          \textsl{充分性}. 若存在有限集 $ \beta\in\fin\alpha $, 使得 $ F=\bigcup_{i\in\beta}O_{i} $ , 则
          \[
               F=F\cap\bigg( \bigcup_{i\in\beta}O_{i} \bigg)=\bigcup_{i\in\beta}U_{i},
          \]
          所以 $ F $ 是紧子空间.\qed
     \end{Proof}

     \begin{Proposition}\label{prop:紧->闭}
          设 $ E $ 是 Hausdorff空间, $ F\subset E $ 是子空间 , 若 $ F $是紧集 , 则 $ F $ 是闭集. 
     \end{Proposition}

     \begin{Proof}
          令 $ x\in F^{c} $ , 由 $ E $ 是 Hausdorff空间, 则对任意 $  y\in F $ , 存在 $ U_{y}\in\CN(x) $, $ V_{y}\in\CN(y) $ , 其中 $ U_{y} $, $ V_{y} $ 是开集, 使得 $ U_{y}\cap V_{y}=\varnothing $ , 则 $\{ V_{y}:y\in F \}$ 是 $ F $ 的开覆盖, 因为 $ F $ 是紧集, 所以存在 $ y_{1}, y_{2}, \dots, y_{n} $, 使得 $ \{ V_{y_{i}}: i=1, 2, \dots, n \} $ 是 $ F $ 的开覆盖, 令 $ U=\bigcap_{i=1}^{n}U_{y_{i}} $ 是开集 (同时也是 $ x $ 的开邻域), 而 $ U\cap V_{y_{i}}=\varnothing\,(i=1, 2, \dots, n) $ , 从而 $ U\cap F=\varnothing $. 因此 $ F^{c} $ 是开集, 即 $ F $ 是闭集.\qed
     \end{Proof}

     \begin{Proposition}
          设 $ E $ 是紧 Hausdorff空间, $ F\subset E $ 是子空间, 则 $ F $ 是紧集的充分必要条件是 $ F $ 是闭集. 
     \end{Proposition}
     
     \begin{Proof}
          \textsl{必要性}. 由命题\ref{prop:紧->闭}可得.

          \textsl{充分性}. 设 $ E $ 是紧的, $ F\subset E $ 为闭集, 设 $ (F_{i})_{i\geqslant1} $ 是 $ F $ 的一族闭子集, 且具有有限交性质 (即 $ \forall \beta\in\fin\alpha $, $ \bigcap_{i\in\beta}F_{i}\neq\varnothing $), 则由 $ F $ 是闭集,  $ F_{i} $ 也是闭集. 由命题\ref{prop:紧性的另一刻画}知 $ \bigcap_{i\in\alpha}F_{i}\neq\varnothing $ , 从而 $ F $ 是紧的.\qed
     \end{Proof}

     \begin{Proposition}
          设 $ E $ 是 Hausdorff空间,  $ (K_{i})_{i\in\alpha} $ 是 $ E $ 中紧集族. 若 $ \bigcap_{i\in\alpha}K_{i}=\varnothing $, 则存在有限集 $ \beta\in\fin\alpha $, 使得 $ \bigcap_{i\in\beta}K_{i}=\varnothing $.   
     \end{Proposition}
     
     \begin{Proof}
          在集族 $ (K_{i})_{i\in\alpha} $ 中任取一元素, 记为 $ K_{0} $, 因为 $ \bigcap_{i\in\alpha}K_{i}=\varnothing $, 
          知 $ \bigcup_{i\in\alpha}K_{i}^{c}=E $, 由 $ E $ 是 Hausdorff的, 故 $ K_{i}^{c} $ 是开集, 从而 $ \bigcup_{i\in\alpha}K_{i}^{c} $ 是 $ K_{0} $ 的开覆盖, 由于 $ K_{0} $ 是紧的, 故有有限集 $ \beta\in\fin\alpha $, 使得 $ \bigcup_{i\in\beta}K_{i}^{c}\subset K_{0} $ 则
          \[
               K_{0}\cap \bigg( \bigcup_{i\in\beta}K_{i} \bigg)^{c}=\varnothing \Rightarrow K_{0}\cap\bigcap_{i\in\beta}K_{i}=\varnothing.
          \]
          即存在有限个子集的交为空集.\qed
     \end{Proof}

     \begin{Proposition}
          设 $ E $ 是紧的 Hausdorff空间, 则 $ E $ 中每一点都有紧邻域基.
     \end{Proposition}

     \begin{Proof}
          任取 $ x\in E $, 取 $ V $ 是 $ x $ 的开邻域, 则 $ \baro{V} $ 闭, 因为 $ E $ 是紧的 Hausdorff空间, 从而 $ \baro{V} $ 是 $ x $ 的一个紧邻域. 往证 $ \forall U\in\CN(x) $ 是开邻域, $ \exists V\in\CN(x) $ 是开邻域且 $ \baro{V}\subset U $. 
          
          注意到 $ U $ 是开集, 则 $ U^{c} $ 为闭集, 同时可知 $ U^{c} $ 为紧集. $ \forall y\in U^{c} $, 存在  $ U_{y}, V_{y} $分别为 $ y, x $ 的开邻域, 使得 $ U_{y}\cap V_{y}=\varnothing $. 因为 $ U^{c} $ 紧, 则存在 $ y_{1}, y_{2}, \dots, y_{n} $, 使得 $ U^{c}\subset \bigcup_{i=1}^{n}U_{y_{i}} $, 再令 $ V=\bigcap_{i=1}^{n}V_{y_{j}} $, 则有
          \[
               V\cap\left( \bigcup_{i=1}^{n}U_{y_{i}} \right)=\varnothing\Rightarrow \baro{V}\cap\left( \bigcup_{i=1}^{n}U_{y_{i}} \right)\Rightarrow \baro{V}\subset U.
          \]
          从而 $ x $ 有紧邻域基.\qed
     \end{Proof}

     \begin{Proposition}
          设 $ E $ 是紧集,  $ F $ 是拓扑空间,  $ f:E\to F $ 连续, 则 $ f(E) $ 是 $ F $ 中的紧集(即连续映射将紧集映成紧集)
     \end{Proposition}

     \begin{Proof}
          取 $ (O_{i})_{i\in\alpha} $ 是 $ f(E) $ 的开覆盖, 则 $ (f^{-1}(O_{i}))_{i\in\alpha} $ 是 $ E $ 的开覆盖, 因为 $ E $ 紧, 则存在有限集 $ \beta\in\fin\alpha $, 使得 $ \bigcup_{i\in\beta}f^{-1}(O_{i})=E $, 故 $ \bigcup_{O_{i}}=f(E) $. 即 $ f(E) $ 紧.\qed
     \end{Proof}

     \begin{Corollary}
          设 $ E $ 是紧集, $ f:E\to\R $ 是连续映射, 则 $ f $ 在 $ E $ 上有界且取得上确界与下确界.
     \end{Corollary}

     \begin{Proposition}
          设 $ E $ 是紧集,  $ F $ 是Hausdorff空间,  $ f:E\to F $ 是连续的单射, 则有 $ f:E\to f(E) $ 是同胚映射.
     \end{Proposition}

     \begin{Proof}
          只需证明 $ f^{-1}:F\to E $ 是连续映射, 即 $ f $ 把闭集映成闭集. 设 $ A $ 是 $ E $ 中的闭集, 因为 $ E $ 紧, 所以 $ A $ 紧, 因为 $ f $ 是连续映射, 则 $ f(A) $ 紧. 又因为 $ F $ 是 Hausdorff的, 则 $ f(A) $ 是 $ f(E) $ 中闭集, 故 $ f^{-1} $ 是连续映射, 故 $ f:E\to f(E) $ 是同胚映射.\qed
     \end{Proof}

     \begin{Definition}[局部紧]\label{def:局部紧}\index{J!局部紧}
           设 $ E $ 是拓扑空间, 若 $ E $ 中每一点都有一个紧邻域, 则称 $ E $ 是\textbf{局部紧}空间.
     \end{Definition}

     \begin{Example}
          \R 与 $ (0, 1] $ 是局部紧的.
     \end{Example}

     \begin{Remark}
          若 $ E $ 是局部紧的 Hausdorff空间, 则 $ E $ 中每一点都有一个紧邻域基.
     \end{Remark}

     \begin{Theorem}[Urysohn引理]
           设 $ E $ 是局部紧的 Hausdorff空间. 设 $ A $ 和 $ B $ 是$ E $ 中两个不相交的非空闭子集, 并设其中的一个是紧集, 那么存在一个连续的函数 $ f:E\to[0,1] $ 使得 $ f|_{A}=0 $, 且 $ f|_{B}=1 $.
     \end{Theorem}

     \begin{Theorem}
           设 $ (E, d), (F, \delta) $ 是度量空间, 且 $ E $ 紧. 若 $ f:E\to F $ 连续, 则 $ f $ 一致连续.
     \end{Theorem}

     \begin{Proof}
          因为 $ f $ 连续, 则 
          \[
               \forall x\in E\,\forall \varepsilon>0\,\exists\eta_{x}>0\,(y\in B(x, \eta_{x})\Rightarrow \delta(f(x), f(y))<\varepsilon).
          \]
          设 $\left\{ B(x, \eta_{x}/2) : x\in E \right\}$ 是 $ E $ 的一个开覆盖, 因为 $ E $ 是紧集, 则存在 $ x_{1}, x_{2},\dots,x_{n} $ 使得
          \[
               \bigcup_{i=1}^{n} B\left(x_{i}, \frac{\eta_{x_{i}}}{2}\right)=E.
          \]
          令 $ \eta=\min\limits_{i=1, 2, \dots, n}\{ x_{i} \} $, 对 $ \forall y\in E $, 当 $ d(x', y)<\eta $ 时, 存在 $ j\in\N, 1\leqslant j\leqslant n $, 使得 $ x'\in B(x_{j}, \eta_{x_{j}}) $, 则
          \[
               d(x_{j}, y) \leqslant d(y, x')+d(x', x_{j})\leqslant\eta_{x_{j}}/2+\eta<\eta_{x_{j}}.
          \]
          故 $ y\in B(x_{j}, \eta_{x_{j}}) $, 所以有 $ \delta(f(y), f(x_{j}))<\varepsilon $, 则
          \[
               \delta(f(x'), f(y))\leqslant \delta(f(x'), f(x_{j}))+\delta(f(x_{j}), f(y))<2\varepsilon.
          \]
          因此 $ f $ 一致连续.\qed
     \end{Proof}

     \subsection{紧度量空间的刻画}
    
     \begin{Example}
     在实直线$ \mathbb{R} $上, 紧致与有界闭等价, 且有以下的等价表述:
     \begin{enumerate}[(1)]
     \item 无限子集必有凝聚点;
     \item 任一序列必有收敛子列;
     \item 完备且可被有限多个半径为$ \varepsilon $的开球覆盖.
     \end{enumerate}
     \end{Example}
     
     类似地, 度量空间的紧性也可以如此刻画:
     
     \begin{Theorem}[紧等价]\label{thm:紧等价}\index{L!列紧}\index{X!序列紧}\index{Y!预紧}
     下面的命题等价:
     \begin{enumerate}[(1)]
     \item $ (E,d) $是紧空间;
     \item $ (E,d) $是\textbf{列紧}的, 即$ (E,d) $中的无限子集必有凝聚点;
     \item $ (E,d) $是\textbf{序列紧}的, 即$ (E,d) $中任一序列都有收敛子列;
     \item $ (E,d) $是完备的且是\textbf{预紧}的, 即$ \forall\varepsilon>0 $, $ E $可被有限多个以$ \varepsilon $为半径的开球覆盖. (若$ F\subset E $, $ (B(x,\varepsilon))_{x\in F} $覆盖$ E $, 则称$ F $是$ E $的一个$ \varepsilon $-\textbf{网})
     \end{enumerate}
     \end{Theorem}

     \begin{Proof}
     (1) $ \Rightarrow $ (2) : 用反证法, 设存在无限集$ F $无凝聚点, 即
     \[
     \forall x\in E\,\exists\eta_x>0\,(B(x,\eta_x)\cap(F\sm\{x\})=\varnothing),
     \]
     则$ (B(x,\eta_x))_{x\in E} $是$ E $的开覆盖. 因为$ E $是紧的, 存在$ x_1,x_2, \dots,x_n $使得
     \[
     \bigcup_{j=1}^nB(x_j,\eta_{x_j})=E,
     \]
     而$ B(x_j,\eta_{x_j})\cap(F\sm\{x_j\}=\varnothing $, 从而$ B(x_j,\eta_{x_j}) $至多包含$ F $中的一个点, 则$ E=\bigcup_{j=1}^nB(x_j,\eta_{x_j}) $至多包含$ F $中的有限个点, 这与$ F $是无限集矛盾.
     
     (2) $ \Rightarrow $ (3) : 任取$ (x_n)_{n\geqslant 1} $是$ E $中的序列, 若$ (x_n)_{n\geqslant 1} $中只取有限多个不同值, 则$ (x_n)_{n\geqslant 1} $有收敛子列是显然的. 若$ (x_n)_{n\geqslant 1} $中有无限多个不同值, 则$ A=\{ x_n : n\geqslant 1 \} $是无限集, 由$ E $列紧知$ A $有凝聚点$ x $, 则存在$ (x_{n_k})_{k\geqslant 1}\subset(x_n)_{n\geqslant 1} $使得$ \lim\limits_{k\to\infty}x_{n_{k}}=x $.
     
     (3) $ \Rightarrow $ (4) : 任取$ E $中的Cauchy列$ (x_n)_{n\geqslant 1} $, 由$ E $序列紧可知$ (x_n)_{n\geqslant 1} $存在收敛子列, 从而$ (x_n)_{n\geqslant 1} $是收敛列, 这说明$ (E,d) $是完备的.
     
     用反证法, 反设$ E $不是预紧的, 即存在$ \varepsilon_0>0 $使得$ E $的开覆盖$ (B(x,\varepsilon))_{x\in E} $不存在有限子覆盖. 取$ x_1\in E $, 由$ B(x_1,\varepsilon)\ne E $, 可取$ x_2\in E\sm B(x_1,\varepsilon_0) $. 依此继续下去, 取
     \[
     x_n\in E\sm\left( \bigcup_{i=1}^{n-1}B(x_i,\varepsilon_0) \right)
     \]
     可得一$ E $中的序列$ (x_n)_{n\geqslant 1} $满足
     \[
     \forall j\ne k\,(d(x_j,x_k)\geqslant\varepsilon_0),
     \]
     即$ (x_n)_{n\geqslant 1} $不存在收敛子列, 矛盾.
     
     (4) $ \Rightarrow $ (3) : 因为$ E $是完备的, 只需证明任意序列$ (x_n)_{n\geqslant 1}\subset E $存在Cauchy子序列, 令$ \varepsilon_1=1/2 $, 由预紧性可知存在有限子集$ F\subset E $使得$ E=\bigcup_{x\in E}B(x,\varepsilon_1) $, 则$ (x_n)_{n\geqslant 1} $存在无穷子序列使得$ (x_{1,i})_{i\geqslant 1}\subset B(x,\varepsilon_1) $. 此时
     \[
     \forall j, k\,(d(x_{1,j},x_{1,k})<1).
     \]
     再令$ \varepsilon_2=1/2^2 $, 则$ (x_{1,i})_{i\geqslant 1} $存在无穷子列$ (x_{2,i})_{i\geqslant 1}\subset B(x',\varepsilon_2) $, 由此进行下去得到一串序列
     \[
     (x_n)_{n\geqslant 1}\supset(x_{1,i})_{i\geqslant 1}\supset(x_{2,i})_{i\geqslant 1}\supset\cdots\supset(x_{n,i})_{i\geqslant 1}\supset\cdots
     \]
     且$ d(x_{n,j},x_{n,k})<1/2^{n-1} $. 从而可取$ (x_{i,i})_{i\geqslant 1} $是一Cauchy列.
     
     (3) $ \Rightarrow $ (1) : 由$ E $序列紧可知$ E $预紧, 设$ (O_i)_{i\in\alpha} $是$ E $的开覆盖且不存在有限子覆盖. 由$ E $预紧可知$ \forall n\in\N $, 存在有限$ 1/n $-网$ F_n $使得$ (B(x,1/n))_{x\in F_n} $覆盖$ E $, 则$ \exists y_n\in F_n $使得$ B(y_n,1/n) $不能被有限个$ O_i $覆盖.
     
     考虑这样构造出的序列$ (y_n)_{n\geqslant 1} $, 由$ E $序列紧可知$ (y_n)_{n\geqslant 1} $存在收敛到$ y $的子序列$ (y_{n_k})_{k\geqslant 1} $. 存在某个$ i_0\in\alpha $使得$ y\in O_{i_0} $, 且存在$ \eta>0 $使得$ B(y,\eta)\subset O_{i_0} $, 则取$ n_k $使得$ \frac{1}{n_k}<\frac{\eta}{2} $且$ d(y_{n_k},y)<\frac{\eta}{2} $, 则$ \forall x\in B(y_{n_k},1/n_k) $有
     \[
     d(x,y)\leqslant d(x,y_{n_k})+d(y_{n_k},y)<\frac{\eta}{2}+\frac{1}{n_k}<\eta.
     \]
     即$ B(y_{n_k},1/n_k)\subset B(y,\eta)\subset O_{i_0} $, 这与$ B(y_{n_k},1/n_k) $不能被有限个$ O_i $覆盖矛盾.\qed
     \end{Proof}
     
     \begin{Definition}[相对紧]\label{def:相对紧}\index{X!相对紧}
     设$ (E,d) $是一个度量空间, $ A\subset E $. 若$ \baro{A} $是紧集, 则称$ A $\textbf{相对紧}.
     \end{Definition}
     
     \begin{Remark}\label{rmk:预紧性的刻画}
     有关预紧性的刻画, 由定理\ref{thm:紧等价}中的``(3) $ \Leftrightarrow $ (4)"可知
     \begin{enumerate}[(1)]
     \item $ E $是预紧的$ \Longleftrightarrow $ $ E $的任一序列存在Cauchy子列;
     
     \item $ A $相对紧$ \Longleftrightarrow $ $ A $中无穷序列存在子列收敛到$ E $中的元素;
     
     \item 若$ E $是完备的, 则$ A $相对紧$ \Longleftrightarrow $ $ A $预紧.
     \end{enumerate}
     \end{Remark}
     
\section{乘积拓扑}
	
	\begin{Definition}[乘积拓扑]\label{def:乘积拓扑}\index{C!乘积拓扑}
	设$ (E_i,\tau_i)_{i\in\alpha} $是一族拓扑空间, $ E=\prod\limits_{i\in\alpha}E_i $是$ E_i $的Descartes积, 其任一元素$ x=(x_i)_{i\in\alpha} $, $ x_i\in E_i $.
	\begin{enumerate}[(1)]
	\item 若$ \beta\in\fin\alpha $, $ U_i $是$ (E_i,\tau_i) $的开集, 则称
	\[
	O=\prod_{i\in\beta}U_i\times\prod_{i\in\alpha\sm\beta}E_i
	\]
	是$ E $的\textbf{基础开集}.
	
	\item\label{item:乘积拓扑任意并} 称由基础开集的并构成的集合是$ E $的开集, 所有这样的开集构成的集合称为$ E $上的\textbf{乘积拓扑}.
	\end{enumerate}
	\end{Definition}
	
	\begin{Remark}
	关于乘积拓扑的一些注记:
	\begin{enumerate}[(1)]
	\item 考虑乘积拓扑$ \tau $:
	
	\hspace{2em}(O$ _1 $) $ \varnothing\in\tau,\ E\in\tau $是显然的.
	
	\hspace{2em}(O$ _2 $) 由\ref{item:乘积拓扑任意并}中的任意并性质是显然的.
	
	\hspace{2em}(O$ _3 $) 设$ O=\prod\limits_{i\in\beta}U_i\times\prod\limits_{i\in\alpha\sm\beta}E_i $, $ O'=\prod\limits_{i\in\beta'}U'_i\times\prod\limits_{i\in\alpha\sm\beta'}E_i $, 其中$ \beta,\beta'\in\fin\alpha $, 那么
	\[
	O\cap O'=\prod_{i\in\beta\cap\beta'}(U_i\cap U_i')\times\prod_{i\in\beta\sm\beta'}U_i\times\prod_{i\in\beta'\sm\beta}U_i'\times\prod_{i\in\alpha\sm(\beta\cup\beta')}E_i
	\]
	是开集(这因$ \beta\cup\beta'\in\fin\alpha $), 由数学归纳法可知对有限个开集的交都成立. 从而$ \tau $ well-defined.
	
	\item 若$ \alpha $是无限集, $ O=\prod\limits_{i\in\alpha}U_i $通常不是开集. 若$ \forall i\in\alpha,\ U_i\ne E_i $, 且$ x=(x_i)_{i\in\alpha}\in O $. 假设$ O $是开集, 则存在基础开集$ O' $使得$ x\in O' $且$ O'\subset O $. 但基础开集$ O' $应形如$ \prod\limits_{i\in\beta}U_i\times\prod\limits_{i\in\alpha\sm\beta}E_i\supset O $, 矛盾.
	
	\item $ \R^n $上的自然拓扑与$ \R\times\R\times\cdots\times\R $上的乘积拓扑是一致的.
	
	\item 若$ \forall i\in\alpha $, 有$ F_i\subset E_i $是闭集, 那么$ \prod\limits_{i\in\alpha}F_i $是闭集, 这因
	\[
	\left(\prod_{i\in\alpha}F_i\right)^c=\bigcup_{i\in\alpha}\left(F_i^c\times\prod_{j\ne i}E_j\right).
	\]
	
	\item 设$ \alpha=\alpha_1\cup\alpha_2 $, $ E^{(1)}=\prod\limits_{i\in\alpha_1}E_i $, $ E^{(2)}=\prod\limits_{i\in\alpha_2}E_i $, $ E=\prod\limits_{i\in\alpha}E_i $. 则$ E $上的乘积拓扑与$ E^{(1)}\times E^{(2)} $上的乘积拓扑一致.
	\end{enumerate}
	\end{Remark}
	
	\begin{Theorem}
	设$ (E_i)_{i\in\alpha} $是一族拓扑空间, $ E=\prod_{i\in\alpha}E_i $, 令$ \tau $是$ E $上的乘积拓扑, 定义
	\[
	p_i : E\to E_i\qquad p_i(x)=x_i,
	\]
	其中$ x=(x_i)_{i\in\alpha} $, 则$ \tau $是使得各$ p_i $连续的最弱拓扑, 并且$ p_i $是开映射. 这里$ p_i $称为\textbf{正规投影}.
	\end{Theorem}
	\begin{Proof}
	对$ i\in\alpha $, 任取$ U_i\subset E_i $是开集, 由
	\[
	p_i^{-1}(U_i)=U_i\times\prod_{j\ne i}E_j
	\]
	是开集可知$ p_i $是连续的.
	
	设$ \tau' $是使得$ p_i $连续的拓扑, 则任取$ U_i\subset E_i $是开集, 应有
	\[
	p_i^{-1}(U_i)=U_i\times\prod_{j\ne i}E_j\in\tau',
	\]
	由\ref{item:O3} 可知$ \tau $的基础开集是$ \tau' $中的元素, 从而$ \tau\subset\tau' $, 即$ \tau $是使得$ p_i $连续的拓扑中最弱的一个.
	
	最后说明$ p_i $是开映射. 设$ O $是$ E $中的开集, 往证$ p_i(O) $也是开集. 因为
	\[
	\forall \tilde{y}_i\in p_i(O)\,\exists y=(y_j)_{j\in\alpha}\in O(y_i=\tilde{y}_i)
	\]
	因为$ O $是开的, 存在基础开集$ O'=\prod\limits_{j\in\beta}U_j\times\prod\limits_{i\in\alpha\sm\beta}E_j $使得$ y\in O' $且$ O'\subset O $. 从而
	\[
	p_i(O')=\begin{cases}
	U_i & ,i\in\beta\\ E_i &, i\notin\beta
	\end{cases}
	\]
	是开集, 故$ p_i $是开映射.\qed
	\end{Proof}
	
	\begin{Corollary}
	设$ (E_i)_{i\in\alpha} $是一族拓扑空间, $ E=\prod\limits_{i\in\alpha}E_i $是乘积拓扑空间.
	\begin{enumerate}[(1)]
	\item 设$ (x^{(n)})_{n\geqslant 1} $是$ E $中的序列, $ x\in E $, 则$ (x^{(n)})_{n\geqslant 1} $收敛到$ x $的充分必要条件是$\forall i\in\alpha,\  (x_i^{(n)})_{n\geqslant 1} $收敛到$ x_i $.
	
	\item 设$ F $是拓扑空间, 映射$ f : F\to E $连续的充分必要条件是$ \forall i\in\alpha,\ p_i\circ f $是连续的.
	\end{enumerate}
	\end{Corollary}
	
	\begin{Proposition}\label{prop:乘积拓扑空间的继承性质}
	设$ (E_i)_{i\in\alpha} $是一族拓扑空间 $ E=\prod\limits_{i\in\alpha}E_i $是乘积拓扑空间.
	\begin{enumerate}[(1)]
	\item 若$ \forall i\in\alpha $, $ E_i $是Hausdorff空间, 则$ E $也是Hausdorff空间.
	
	\item (Tychonoff) 若$ \forall i\in\alpha $, $ E_i $是紧空间, 则$ E $也是紧空间.
	
	\item 设$ \alpha=\N^\ast $, 若$ \forall i\in\alpha,\ E_i $是可度量化的, 则$ E $是可度量化的. (可度量化是指存在$ E $上的度量$ d $使得$ d $诱导的拓扑与$ E $上原本定义的拓扑一致)
	\end{enumerate}
	\end{Proposition}
	
	\begin{Remark}
	设$ (E,d_E) $和$ (F,d_F) $都是度量空间, 在$ E\times F $上赋以度量
	\[
	d_{E\times F}\left((x_1,y_1),(x_2,y_2)\right)=\max\{ d_E(x_1,x_2), d_F(y_1,y_2) \},
	\]
	则$ (E,d_E) \times (F,d_F) $的乘积拓扑与$ (E\times F, d_{E\times F}) $上度量$ d_{E\times F} $诱导的拓扑一致.
	\end{Remark}
	
	\begin{Proposition}
	设$ (E,d) $是度量空间, 则$ d : E\times E\to\R $是连续的.
	\end{Proposition}
	\begin{Corollary}\label{col:距离函数连续}
	设$ (E,d) $是度量空间, $ \forall A\subset E $, 映射$ x\mapsto d(x,A) $是连续的.
    \end{Corollary}
	
\section{赋范线性空间}
	
	\subsection{Banach空间}
	
	\begin{Definition}[赋范空间]\label{def:赋范空间}\index{F!赋范空间}
	设$ E $是$ \K $上的线性空间, $ \norm{\cdot} : E\to[0,\infty) $是$ E $上的实值函数, 满足$ \forall x,y\in E,\,\forall\lambda\in\K $有
	
	\begin{enumerate}[(1)]
	\item 正定性: $ \norm{x}=0\Longleftrightarrow x=0 $;
	
	\item 齐次性: $ \norm{\lambda x}=\abs{\lambda}\cdot\norm{x} $;
	
	\item 三角不等式: $ \norm{x+y}\leqslant\norm{x}+\norm{y} $.
	\end{enumerate} 
	则称$ \norm{\cdot} $是$ E $上的\textbf{范数}, 并称$ (E,\norm{\cdot}) $是\textbf{赋范(线性)空间}.
	\end{Definition}
	
	\begin{Example}
	常见空间上的范数:
	
	\begin{enumerate}[(1)]
	\item 在Euclid空间$ \K^n $上, 定义
		\[
		\norm{x}_p=\left(\sum_{i=1}^n\abs{x_i}^p\right)^{1/p}\qquad(1\leqslant p<\infty),
		\]
		当$ p=\infty $时定义为$ \norm{x}_\infty=\max\limits_{1\leqslant i\leqslant n}\abs{x_i} $, 则有$ \norm{x}_p\,(1\leqslant p\leqslant\infty) $是$ \K^n $上的范数, 并称$ p=2 $的情形称为$ \K^n $上的\textbf{Euclid范数}. 另外, 对$ n $阶正定矩阵$ A $, 若定义
		\[
		\norm{x}_A=\norm{Ax}_2,
		\]
		则它也是$ \K^n $上的范数.
	
	\item 在连续函数空间$ C[a,b] $上定义
	\[
	\norm{x}=\sup_{a\leqslant t\leqslant b}\abs{x(t)},
	\]
	则$ \norm\cdot $是$ C[a,b] $上的范数.
	\end{enumerate}
	
	\end{Example}
	
	\begin{Remark}
	赋范空间一定是度量空间, 因度量可以被范数诱导:
	\[
	d(x,y):=\norm{x-y}.
	\]
	\end{Remark}
	
	\begin{Definition}[Banach空间]\label{def:Banach空间}\index{B!Banach空间}
	设$ (E,\norm{\cdot}) $是赋范空间, $ d(x,y)=\norm{x-y} $是范数诱导的距离. 若$ (E,d) $完备, 则称$ (E,\norm{\cdot}) $是\textbf{Banach空间}.
	\end{Definition}
	
	由Banach空间的定义可知任意赋范空间$ (E,\norm{\cdot}) $的完备化$ (\widehat{E},\norm\cdot) $是Banach空间. 同时因为在赋范空间上有自然的线性运算, 可以在赋范空间上定义级数:
	
	\begin{Definition}[级数]\label{def:级数}\index{J!级数}
	设$ (E,\norm{\cdot}) $是赋范空间, $ \sum\limits_{n\geqslant 1}x_n $是空间中的\textbf{级数}, 并称$ S_n=\sum\limits_{k=1}^nx_k $是级数的\textbf{部分和}.
	
	\begin{enumerate}[(1)]
	\item 若$ (S_n)_{n\geqslant 1} $在$ (E,\norm\cdot) $中依范数收敛到$ S $, 则称级数$ \sum\limits_{n\geqslant 1}x_n $在$ (E,\norm\cdot) $上\textbf{收敛}, 并称$ S $是$ \sum\limits_{n\geqslant 1}x_n $的\textbf{和}, 记作$ S=\sum\limits_{n\geqslant 1}x_n $.
	
	\item 若$ (S_n)_{n\geqslant 1} $是Cauchy列, 则称级数$ \sum\limits_{n\geqslant 1}x_n $是\textbf{Cauchy级数}.
	
	\item 若$ \sum\limits_{n\geqslant 1}\norm{x_n} $收敛, 则称级数$ \sum\limits_{n\geqslant 1}x_n $\textbf{绝对收敛}.
	\end{enumerate}
	\end{Definition}
	
	\begin{Remark}
	考虑赋范空间$ ((0,1),\abs\cdot) $上的级数$ \sum\limits_{n\geqslant 1}2^{-n} $, 则$ \sum\limits_{n\geqslant 1}2^{-n} $绝对收敛但不收敛.
	\end{Remark}
	
	\begin{Theorem}
	赋范空间$ (E,\norm\cdot) $是Banach空间当且仅当$ (E,\norm\cdot) $上绝对收敛的级数一定收敛.
	\end{Theorem}
	\begin{Proof}
	\textsl{必要性.} 设$ \sum\limits_{n\geqslant 1}x_n $绝对收敛, 并记$ S_n $是部分和, 因为$ \forall n,p\in\mathbb{N} $, 有
	\[
	\norm{S_{n+p}-S_n}=\norm{\sum_{k=n+1}^{n+p}x_k}\leqslant\sum_{k=n+1}^{n+p}\norm{x_k}.
	\]
	由$ \sum\limits_{n\geqslant 1}x_n $收敛可知
	\[
	\forall p\in\mathbb{N}\left(\lim_{n\to\infty}\norm{S_{n+p}-S_n}=0\right),
	\]
	从而$ (S_n)_{n\geqslant 1} $是Cauchy列. 由$ (E,\norm\cdot) $是Banach空间知其完备, 从而$ (S_n)_{n\geqslant 1} $收敛, 即级数$ \sum\limits_{n\geqslant 1}x_n $收敛.
	
	\textsl{充分性.} 设$ (x_n)_{n\geqslant 1} $是Cauchy列, 则
	\[
	\exists(x_{n_k})_{k\geqslant 1}\subset(x_n)_{n\geqslant 1}\left(\norm{x_{n_{k+1}}-x_{n_k}}<2^{-k}\right),
	\]
	从而$ \sum\limits_{k\geqslant 1}\norm{x_{n_{k+1}}-x_{n_k}} $收敛, 即$ \sum\limits_{k\geqslant 1}(x_{n_{k+1}}-x_{n_k}) $绝对收敛, 故它收敛, 其部分和$ S_{n_k}=x_{n_{k+1}} $, 从而序列$ (x_{n_k})_{k\geqslant 1} $收敛. 由$ (x_n)_{n\geqslant 1} $是Cauchy列知它收敛, 于是$ (E,\norm\cdot) $完备, 即$ (E,\norm\cdot) $是Banach空间.\qed
	\end{Proof}
	
	\begin{Definition}[范数等价]\index{D!等价}
		设$ \norm{\cdot}_1 $与$ \norm{\cdot}_2 $是线性空间$ E $上的两个范数, 若
		\[
		\exists c_1,c_2>0\,\forall x\in E\,(c_1\norm{x}_1\leqslant\norm{x}_2\leqslant c_2\norm{x}_1),
		\]
		则称$ \norm{\cdot}_1 $与$ \norm{\cdot}_2 $\textbf{等价}.
	\end{Definition}
	
	\begin{Remark}
	记$ \id_E : (E,\norm{\cdot}_1)\to(E\,\norm{\cdot}_2) $是恒等映射, 则由
	\[
	\norm{x-y}_2\leqslant c_2\norm{x-y}_1
	\]
	可知$ \id_E $是Lipschitz的. 再考虑$ \id_E^{-1} : (E,\norm{\cdot}_2)\to(E\,\norm{\cdot}_1) $, 由
	\[
	\norm{x-y}_1\leqslant\frac{1}{c_1}\norm{x-y}_2
	\]
	可知$ \id_E^{-1} $也是Lipschitz的. 从而$ \id_E $与$ \id_E^{-1} $都是一致连续的. 于是$ (E,\norm\cdot_1) $完备当且仅当$ (E,\norm\cdot_2) $完备, $ A\subset E $在$ (E,\norm\cdot_1) $上紧当且仅当$ A $在$ (E,\norm\cdot_2) $上紧.
	\end{Remark}
	
	\begin{Theorem}
	设$ E $是数域$ \K $上的有限维线性空间, 则$ E $上所有范数都等价.
	\end{Theorem}
	\begin{Proof}
	设$ \dim E=n $, 取$ E $中的一组基$ e_1,e_2, \dots,e_n $, 则
	\[
	\forall x\in E\,\exists[x_1,x_2, \dots,x_n]^\mathrm{T}\in\K^n\,\left( x=\sum_{i=1}^nx_ie_i \right),
	\]
	那么可以定义
	\[
	\varPhi : \K^n\to E,\qquad \begin{bmatrix}
	x_1\\x_2\\\vdots\\x_n
	\end{bmatrix}\mapsto\sum_{i=1}^nx_ie_i.
	\]
	从而$ \varPhi $是一个线性同构.
	
	令$ \K^n $上的范数$ \norm\cdot_\infty $为
	\[
	\norm{\begin{bmatrix}x_1\\x_2\\\vdots\\x_n\end{bmatrix}}_\infty=\max_{1\leqslant i\leqslant n}\abs{x_i},
	\]
	由此可以诱导$ E $上的范数$ \norm\cdot_\infty $:
	\[
	\norm{x}_\infty=\norm{\varPhi^{-1}(x)}_\infty.
	\]
	再任取$ E $上的另一个范数$ \norm{\cdot} $, 则有
	\[
	\norm{x}=\norm{\sum_{i=1}^nx_ie_i}\leqslant\sum_{i=1}^n\abs{x_i}\norm{e_i}\leqslant\norm{x}_\infty\sum_{i=1}^n\norm{e_i}.
	\]
	令$ c_2=\sum_{i=1}^n\norm{e_i} $, 则有$ \norm{x}\leqslant c_2\norm{x}_\infty $. 再定义映射
	\[
	\varphi : \K^n\to\C,\qquad\begin{bmatrix}x_1\\x_2\\\vdots\\x_n\end{bmatrix}\mapsto\norm{\varPhi\left(\begin{bmatrix}
	x_1\\x_2\\\vdots\\x_n
	\end{bmatrix}\right)}=\norm{x}.
	\]
	易知$ \varphi $是连续的, 则$ \varphi $在$ \K^n $的单位球面$ \mathbb{S}^{n-1} $上有下确界, 记作$ c_1 $, 则$ c_1\geqslant 0 $. 而对$ \forall x\in E $, 因
	\[
	\norm{x}=\varphi\left(\begin{bmatrix}x_1\\x_2\\\vdots\\x_n\end{bmatrix}\right)=\norm{\begin{bmatrix}x_1\\x_2\\\vdots\\x_n\end{bmatrix}}_\infty\varphi\left(\begin{bmatrix}x_1\\x_2\\\vdots\\x_n\end{bmatrix}\cdot\norm{\begin{bmatrix}x_1\\x_2\\\vdots\\x_n\end{bmatrix}}_\infty^{-1}\right)\geqslant c_1\norm{\begin{bmatrix}x_1\\x_2\\\vdots\\x_n\end{bmatrix}}_\infty=c_1\norm{x}_\infty.
	\]
	可知$ \norm{x}\geqslant c_1\norm{x}_\infty $. 则有
	\[
	c_1\norm{x}_\infty\leqslant\norm{x}\leqslant c_2\norm{x}_\infty,
	\]
	从而$ E $上所有范数均等价.\qed
	\end{Proof}
	
	\begin{Remark}\label{rmk:有限维赋范空间等价K^n}
	由上述定理可知任何有限维赋范空间在范数等价的意义下均可看作$ \K^n $, 从而有限维赋范空间都是完备的, 并且有限维赋范空间上的有界闭集都是紧集.
	\end{Remark}
	
	\begin{Theorem}[\textbf{Riesz}]\label{thm:Riesz定理}
	设$ (E,\norm{\cdot}) $是赋范空间, 则$ E $是有限维的当且仅当$ E $的闭单位球是紧的.
	\end{Theorem}
	
	为了证明上面的Riesz定理, 我们先说明下面一个引理. 这一引理的证明技巧性相对较强.
	
	\begin{Lemma}\label{lem:Riesz前引理}
	设$ (E,\norm\cdot) $是赋范空间, $ F $是$ E $的闭线性子空间且$ F\ne E $, 则
	\[
	\forall\varepsilon>0\,\exists e\in E\,(\norm{e}=1\land d(e,F)\geqslant 1-\varepsilon).
	\]
	\end{Lemma}
	\begin{Proof}
	令$ x\in E\sm F $, 取$ d=d(x,F)=\inf\{ \norm{x-y} : y\in F \} $, 则
	\[
	\exists y_0\in F\left(d\leqslant\norm{x-y_0}\leqslant\frac{d}{1-\varepsilon}\right).
	\]
	再令$ e=\frac{x-y_0}{\norm{x-y_0}} $, 那么有$ \norm{e}=1 $, 且$ \forall z\in F $都有
	\[
	\norm{e-z}=\norm{\frac{x-y_0}{\norm{x-y_0}}-z}=\frac{1}{\norm{x-y_0}}\norm{x-y_0-\norm{x-y_0}z}\geqslant\frac{1}{d/(1-\varepsilon)}d=1-\varepsilon.
	\]
	其中由于$ y_0+\norm{x-y_0}z\in F $, 故$ \norm{x-y_0-\norm{x-y_0}z}\geqslant d $. 于是$ d(e,F)\geqslant 1-\varepsilon $.\qed
	\end{Proof}
	
	下面我们可以证明Riesz定理:
	
	\textbf{定理\,\,\ref{thm:Riesz定理}\,\,的证明}\ \ \textsl{必要性}. 由注\ref{rmk:有限维赋范空间等价K^n}可知.
	
	\textsl{充分性}. 用反证法, 假设$ E $是无限维的, 则可以取一单位向量$ x_1 $, 并且记$ F_1=\mathrm{span}\{x_1\}=\{ \lambda_1x_1 : \lambda_1\in\K \} $. 由$ \dim F_1=1<\infty $可知$ F_1 $完备, 从而$ F_1 $是$ E $的闭线性子空间. 故由引理\ref{lem:Riesz前引理}可知
	\[
	\exists x_2\in E\,\left(\norm{x_2}=1\land d(x_2,F_1)\geqslant\frac{1}{2}\right).
	\]
	再取$ F_2=\mathrm{span}\{x_1,x_2\} $, 同理$ F_2 $是$ E $的闭线性子空间, 从而
	\[
	\exists x_3\in E\,\left(\norm{x_3}=1\land d(x_3,F_2)\geqslant\frac{1}{2}\right).
	\]
	依此进行下去可得一序列$ (x_n)_{n\geqslant 1}\subset\{ x : \norm{x}=1 \} $使得
	\[
	\forall n\ne m\,\left(\norm{x_n-x_m}\geqslant\frac{1}{2}\right),
	\]
	即$ (x_n)_{n\geqslant 1} $没有收敛子列. 但由$ \{ x : \norm{x}=1 \} $紧可知它是序列紧的, 即$ (x_n)_{n\geqslant 1} $存在收敛子列, 矛盾.\qed
	
     \subsection{一个Banach空间的例子—— $ L_p $空间}
     
     \begin{Definition}[$ \ell_{p} $空间]\label{def:lp空间}
		设$ 0<p\leqslant\infty $, 考虑具有以下性质的序列空间:
          当 $ 0<p<\infty $ 时, 定义
          \[
               \ell_{p}=\bigg\{ x=(x_{n})_{n\geqslant1}: \norm{x}_{p}=\bigg( \sum\limits_{n\geqslant1}\abs{x_{n}}^{p} \bigg)^{1/p}<\infty, x_{n}\in\K \bigg\}
          \]
           当 $ p=\infty $ 时, 定义
           \[
                \ell_{\infty}=\{ x=(x_{n})_{n\geqslant1}:\norm{x}_{\infty}=\sup_{n\geqslant1}\abs{x_{n}}<\infty, x_{n}\in\K \}
           \]
           称其为\textbf{$ \ell_{p} $空间}\index{L!$ \ell_{p} $空间}, 再取 $ c_{0} $ 表示所有极限为 $ 0 $ 的数列构成的集合, 显然有 $ c_{0}\subset\ell_{\infty} $. 
     \end{Definition}
	
	下面的内容大致与实分析课程笔记中3.4节相同, 关于一部分定理的进一步讨论可以参见实分析的课程笔记. 将上述序列空间连续化, 就得到所谓的$ \CL_p $空间:
	
	\begin{Definition}[$ \CL_p $空间]\label{def:Lp空间}
		设$ 0<p\leqslant\infty $, $ f : X\to\K $是$ (X,\mu) $上的可测函数. 当$ p<\infty $时, 若
		\[
		\int_X\abs{f}^p\diff\mu<\infty,
		\]
		则称$ f $是$ p $-\textbf{方可积函数}\index{P!$ p $-方可积函数}, 并记$ \CL_p(X,\mu) $是$ p $-方可积函数的全体. 在$ \CL_p(X,\mu) $上可以定义函数
		\[
		\norm\cdot_p : \CL_p(X,\mu)\to\R,\qquad f\mapsto\left(\int_X\abs{f}^p\diff\mu\right)^{1/p},
		\]
		当$ p=\infty $时, 若$ \exists M\geqslant 0 $使得$ \abs{f}\leqslant M $ a.e., 也即$ \mu\{ x : \abs{f(x)}>M \} $=0, 则称$ f $是\textbf{本性有界函数}\index{B!本性有界函数}, 并记$ \CL_\infty(X,\mu) $是本性有界函数的全体. 在$ \CL_\infty $上可以定义函数
		\[
		\norm\cdot_\infty : \CL_\infty(X,\mu)\to\R,\qquad f\mapsto\esssup f=\inf\{ M : \abs{f}\leqslant M \text{\,a.e.} \},
		\]
		且称$ \norm{f}_\infty=\esssup f $是$ f $的\textbf{本性上确界}\index{B!本性上确界}(\underline{ess}ential \underline{sup}remum)
	\end{Definition}
	
	本节主要研究有关$ \CL_p $空间的这样几个问题:
	\begin{enumerate}[(1)]
	\item $ \CL_p(X,\mu) $ (或者在不致混淆的情形下记作$ \CL_p(X) $)是否是线性空间?\label{item:1.7节主要问题1}
	\item $ \norm\cdot_p $是否是$ \CL_p(X) $上的范数?\label{item:1.7节主要问题2}
	\item 若$ \norm\cdot_p $是范数, 赋范空间$ (\CL_p(X),\norm\cdot_p) $是否是Banach空间? 若不是, 能否在其上定义度量使得它成为完备的度量空间?\label{item:1.7节主要问题3}
	\end{enumerate}
	为了说明这些问题, 需要先证明几个不等式.
	
	\begin{Lemma}[Young]
	对任意的$ x, y\in[0,\infty) $且$ 0<\alpha,\beta<1 $, $ \alpha+\beta=1 $, 有
	\[
	xy\leqslant \alpha x^{1/\alpha}+\beta y^{1\beta}.
	\]
	\end{Lemma}
	\begin{Proof}
	令$ x=\exp(\alpha t) $, $ y=\exp(\beta s) $, 则所求证的不等式化为
	\[
	\exp(\alpha t+\beta s)\leqslant\alpha\exp t+\beta\exp s.
	\]
	这由$ \exp x $的凸性显然.\qed
	\end{Proof}
	
	\begin{Remark}
	Young不等式具有以下推广, 设$ a_i\in[0,\infty),\ 0<t_i<1 $且$ \sum\limits_{i=1}^nt_i=1 $时, 有
	\[
	\prod_{i=1}^na_i^{t_i}\leqslant\sum_{i=1}^nt_ia_i.
	\]
	我们将在下一定理的证明中使用这一结论.
	\end{Remark}
	
	\begin{Proposition}[H\"older]
			设$ \frac{1}{p}=\sum\limits_{i=1}^n\frac{1}{p_i} $, $ f=\prod\limits_{i=1}^nf_i $时, 有
			\[
			\lVert f\rVert_p\leqslant\prod_{i=1}^n\lVert f_i\rVert_{p_i}.
			\]
	\end{Proposition}
	\begin{Proof}
			记$ b_i=\lVert f_i\rVert_{p_i} $, 若某个$ b_i=0 $, 那么某个$ f_i=0 $ a.e., 即$ f=0 $ a.e., 此时命题显然成立. 若某个$ b_i=\infty $, 右侧为$ \infty $, 命题也显然成立. 因此不妨设所有的$ b_i $都有限且为正数.
			
			若某个$ p_i $, 不妨$ p_n=\infty $, 则有
			\[
			\lvert f\rvert\leqslant\lvert f_1\rvert\cdot\lvert f_2\rvert\cdot\cdots\cdot\lvert f_{n-1}\rvert\cdot\lvert f_n\rvert\leqslant \lvert f_1\rvert\cdot\lvert f_2\rvert\cdot\cdots\cdot\lvert f_{n-1}\rvert\cdot\lVert f_n\rVert_\infty,
			\]
			此时$ \frac{1}{p_n}=0 $, 则$ \frac{1}{p}=\frac{1}{p_1}+\frac{1}{p_2}+\cdots+\frac{1}{p_{n-1}} $, 从而
			\[
			\lVert f\rVert_p\leqslant\lVert f_1f_2\cdots f_{n-1}\rVert_p\cdot\lVert f_n\rVert_\infty.
			\]
			因此可设所有的$ p_i $都有限.
			
			现在$ 0<p_i<\infty,\ 0<b_i<\infty,\ \forall i=1,2, \dots,n $, 那么由
			\[
			\dfrac{f}{b_1b_2\cdots b_n}=\dfrac{f_1}{b_1}\cdot\dfrac{f_2}{b_2}\cdot\cdots\cdot\dfrac{f_n}{b_n}
			\]
			可知
			\begin{align*}
			\left| \dfrac{f}{b_1b_2\cdots b_n} \right|^p&=\left| \dfrac{f_1}{b_1} \right|^p\cdot\left| \dfrac{f_2}{b_2} \right|^p\cdot\cdots\cdot\left| \dfrac{f_n}{b_n} \right|^p\\
			&=\left(\left| \dfrac{f_1}{b_1} \right|^{p_1}\right)^{p/p_1}\cdot\left(\left| \dfrac{f_2}{b_2} \right|^{p_2}\right)^{p/p_2}\cdot\cdots\cdot\left(\left| \dfrac{f_n}{b_n} \right|^{p_n}\right)^{p/p_n}\leqslant\sum_{i=1}^n\dfrac{p}{p_i}\left| \dfrac{f_i}{b_i} \right|^{p_i}.
			\end{align*}
		两侧积分得
			\[
			\left(\dfrac{1}{b_1b_2\cdots b_n}\right)^p\int_X\lvert f\rvert^p\diff\mu\leqslant\sum_{i=1}^n\dfrac{p}{p_i}\cdot\dfrac{1}{b_i^{p_i}}\int_X\lvert f_i\rvert^{p_i}\diff\mu=1.
			\]\qed
	\end{Proof}
		
	\begin{Corollary}[H\"older]
			当$ p,q\geqslant 1 $且$ \frac{1}{p}+\frac{1}{q}=1 $时, 有
			\[
			\lVert gh\rVert_1\leqslant\lVert g\rVert_p\lVert h\rVert_q,
			\]
			并称这样的$ (p,q) $为一对\textbf{共轭指数}, 并约定$ (1,\infty) $也是一对共轭指数.
	\end{Corollary}
	
	\begin{Theorem}
	设$ 0<p\leqslant\infty $, 则$ \CL_p(X) $是线性空间, 且任取$ f, g\in\CL_p(X) $, 有
	\begin{enumerate}[(1)]
	\item Minkowski不等式: $ p\in[1,\infty] $时, 有$ \norm{f+g}_p\leqslant\norm f_p+\norm g_p $.
	\item $ 0<p<1 $时, $ \norm{f+g}_p^p\leqslant \norm f_p^p+\norm g_p^p $.
	\end{enumerate}
	\end{Theorem}
	\begin{Proof}
	首先说明$ \CL_p(X) $是线性空间, 当$ p<\infty $时, 对任意$ f, g\in\CL_p(X) $, 有
	\[
	\abs{f(x)+g(x)}^p\leqslant\begin{cases}
	\abs{f(x)}^p+\abs{g(x)}^p & ,0<p\leqslant 1\\
	2^{p-1}(\abs{f(x)}^p+\abs{g(x)}^p) & ,p>1
	\end{cases}
	\]
	其中第一种情形由$ t^p $的上凸性质直接得到, 第二种情形由$ t^p $的下凸性可知
	\[
	\left(\frac{\abs{f+g}}{2}\right)^p\leqslant\left(\frac{\abs f+\abs g}{2}\right)^p\leqslant\frac{\abs{f}^p}{2}+\frac{\abs{g}^p}{2}
	\]
	即为式中结论. 于是$ f+g\in\CL_p(X) $. 又由
	\[
	\forall\lambda\in\K\,\norm{\lambda f}_p=\abs{\lambda}\cdot\norm{f}_p
	\]
	可知$ \lambda f\in\CL_p(X) $. 于是$ \CL_p(X) $是线性空间.
	
	当$ 0<p<1 $时, 由
	\[
		\abs{f(x)+g(x)^p}\leqslant\abs{f(x)}^p+\abs{g(x)}^p
	\]
	可知$ \norm{f+g}_p^p\leqslant \norm f_p^p+\norm g_p^p $成立. 当$ p\geqslant 1 $时, 取$ q $是$ p $的共轭指数, 那么
	\begin{align*}
		\int_X\lvert f+g\rvert^p\diff\mu&=\int_X\lvert f+g\rvert^{p-1}\lvert f+g\rvert\diff\mu\\
		&\leqslant\int_X\lvert f+g\rvert^{p-1}(\lvert f\rvert+\lvert g\rvert)\diff\mu\\
		&=\int_X\lvert f+g\rvert^{p-1}\lvert f\rvert\diff\mu+\int_X\lvert f+g\rvert^{p-1}\lvert g\rvert\diff\mu\\
		&\leqslant\left( \int_X(\lvert f+g\rvert^{p-1})^q\diff\mu \right)^{1/q}\left(\int_X\lvert f\rvert^p\diff\mu\right)^{1/p}+\left( \int_X(\lvert f+g\rvert^{p-1})^q\diff\mu \right)^{1/q}\left(\int_X\lvert g\rvert^p\diff\mu\right)^{1/p}\\
		&=\left( \int_X\lvert f+g\rvert^p\diff\mu \right)^{1/q}\left( \left( \int_X\lvert f\rvert^p\diff\mu \right)^{1/p}+\left( \int_X\lvert g\rvert^p\diff\mu \right)^{1/p} \right).
     \end{align*}
     %% 这段东西就不要读了, 恶心.
	化简即得
	\[
		\left( \int_X\lvert f+g\rvert^p\diff\mu \right)^{1-1/q}\leqslant\left( \int_X\lvert f\rvert^p\diff\mu \right)^{1/p}+\left( \int_X\lvert g\rvert^p\diff\mu \right)^{1/p},
	\]
	即$ \lVert f+g\rVert_p\leqslant\lVert f\rVert_p+\lVert g\rVert_p $.
	
	而$ p=\infty $时, 注意到
	\[
		\forall x\in X\,(\abs{f(x)+g(x)}\leqslant\esssup f+\esssup g),
	\]
	从而有$ \norm{f+g}_\infty\leqslant\norm f_\infty+\norm g_\infty $, 即Minkowski不等式成立, 且由$ \forall\lambda\in\K,\ \norm{\lambda f}_\infty=\abs{\lambda}\cdot\norm{f}_\infty $可知$ \CL_\infty(X) $也是线性空间.\qed
	\end{Proof}
	
	至此, 问题\,\ref{item:1.7节主要问题1}\,已经解决. 但是这样定义的$ \norm{\cdot}_p $并不是$ \CL_p(X) $上的范数, 因为它不满足正定性: 即
	\[
	\norm{f}_p=0\Longleftrightarrow f=0\ \text{a.e.}
	\]
	而非$ f=0 $. 为此, 需要重构$ \CL_p(X) $空间, 使得$ \norm{\cdot}_p $能够成为其上的范数.
	
	记$ \CL_0(X) $是$ (X,\mu) $上全体可测函数构成的线性空间, 并定义$ \CL_0(X) $上的等价关系
	\[
	f\sim g\Longleftrightarrow f=g\ \text{a.e.},
	\]
	再记$ L_0(X)=\CL_0(X)/\sim $. 设$ f\in\CL_0(X) $是$ \tilde{f}\in L_0(X) $的一个代表元, 那么对任意$ \tilde{f},\ \tilde{g}\in L_0(X) $, $ \forall \lambda\in\K $, 定义
	\[
	\tilde{f}+\tilde{g}=\widetilde{f+g};\qquad \lambda\tilde{f}=\widetilde{\lambda f}
	\]
	后$ L_0(X) $成为线性空间. 类似地记$ L_p(X)=\CL_p(X)/\sim $后有$ L_p(X) $也是线性空间, 定义$ \lVert\tilde f\rVert_p=\norm{f}_p $后在$ p\geqslant 1 $时$ L_p(X) $成为赋范空间. 之后我们经常将$ \CL_p(X) $等同于$ L_p(X) $, 且将$ \tilde{f} $就看作$ f $并且记作$ f $. 当$ 0<p<1 $时$ L_p(X) $不成为赋范空间, 但定义$ d(f,g)=\norm{f-g}_p^p $后成为一个度量空间. 那么现在问题\,\ref{item:1.7节主要问题2}\,也解决完毕了, 下面主要来解决问题\,\ref{item:1.7节主要问题3}\,, 即空间的完备性.
	
	\begin{Theorem}
	对$ 0<p\leqslant\infty $, 空间$ L_p(X) $是完备的.
	\end{Theorem}
	\begin{Proof}
	设$ (f_n)_{n\geqslant 1} $是$ L_p(X) $中的Cauchy列, 可以取子列$ (f_{n_k})_{k\geqslant 1} $满足
	\[
	\norm{f_{n_{k+1}}-f_{n_k}}<2^{-k},
	\]
	令$ k\geqslant 2 $时$ g_k=f_{n_k}-f_{n_{k-1}} $且$ g_1=f_{n_1} $, 则$ \norm{g_k}_p^p<2^{-pk} $. 只需证明$ (f_{n_k})_{k\geqslant 1} $收敛即可.
	
	当$ 0<p<1 $时, 任取$ k_0\in\N $, 那么
	\[
	\int_X\abs{\sum_{k=1}^{k_0}g_k}^p\diff\mu\leqslant\sum_{k=1}^{k_0}\int_X\abs{g_k}^p\diff\mu<\sum_{k\geqslant 1}2^{-pk}<\infty,
	\]
	从而
	\[
	\int_X\bigg(\sum_{k\geqslant 1}\abs{g_k}\bigg)^p\diff\mu<\infty\Longrightarrow\sum_{k\geqslant 1}\abs{g_k}<\infty\ \text{a.e.}
	\]
	这说明$ \sum\limits_{k\geqslant 1}g_k $几乎处处绝对收敛, 去掉一个零测集后认为$ \sum\limits_{k\geqslant 1}g_{k} $收敛, 并定义
	\[
	f(x):=\sum_{k\geqslant 1}g_k(x)=\lim_{k\to\infty}f_{n_k}(x),
	\]
	则$ f $是可测函数, 且
	\[
	\int_X\abs{f}^p\diff\mu\leqslant\int_X\bigg(\sum_{k\geqslant 1}\abs{g_k}^p\bigg)\diff \mu<\infty,
	\]
	即$ f\in L_p(X) $. 且$ \forall x\in X $都有$ \abs{f(x)-f_{n_k}(x)}=\bigg|\sum\limits_{i\geqslant k+1}g_i(x)\bigg| $, 令$ k\to\infty $有
	\[
	\norm{f-f_{n_k}}_p^p=\int_X\abs{f-f_{n_k}}^p\diff\mu\leqslant\int_X\bigg|\sum_{i\geqslant k+1}g_i\bigg|^p\diff\mu\leqslant\sum_{i\geqslant k+1}2^{-pi}\to 0,
	\]
	即$ \lim\limits_{k\to\infty}f_{n_{k}}=f $.
	
	当$ 1\leqslant p<\infty $时, 由Minkowski不等式, 有
	\[
	\left( \int_X\left( \sum_{k=1}^{k_0}\abs{g_k} \right)^p\diff\mu \right)^{1/p}\leqslant\left( \sum_{k=1}^{k_0}\int_X\abs{g_k}^p\diff\mu \right)^{1/p}\leqslant\sum_{k\geqslant 1}2^{-k}<\infty,
	\]
	后续步骤同$ 0<p<1 $的情形.
	
	当$ p=\infty $时, 由于$ \forall m,n\in\N $, 去掉一个零测集之后可以认为$ \forall m,n\in\N $都有
	\[
	\abs{f_n(x)-f_m(x)}\leqslant\norm{f_n-f_m}_\infty,
	\]
	且$ (f_n)_{n\geqslant 1} $是Cauchy列, 那么$ (f_n(x))_{n\geqslant 1} $是$ \K $上的Cauchy列, 从而它收敛, 记其极限为$ f(x) $, 于是也有$ \lim\limits_{k\to\infty}f_{n_k}=f $.\qed
	\end{Proof}
	
     从而$ p\geqslant 1 $时, $ L_p(X) $成为一个Banach空间, 即使$ 0<p<1 $, 在赋以度量$ d(f,g)=\norm{f-g}_p^p $后, $ L_p(X) $也是一个完备的度量空间.
     
     对一般的测度空间$ (X,\mu) $, 考虑最简单的两种特殊情形: 一类是离散情形, 即
	\[
	X=\{ x_1,x_2, \dots,x_n, \dots \},
	\]
	并赋予$ \mu\{x_i\}=1 $, 那么此时$ L_p(X) $同构于$ \ell_p $. 另一类是连续情形, 即$ X=(0,1) $, 那么$ L_p(X) $同构于$ L_p(0,1) $. 且对于任意的$ (a,b)\subset\R $, 存在等距同构$ T : L_p(0,1)\to L_p(a,b) $.
	
	\begin{Definition}[可分性]\index{K!可分性}
	若赋范空间(或度量空间)$ E $有可数稠密子集, 则称$ E $是\textbf{可分的}.
	\end{Definition}
	
	\begin{Proposition}
	$ \ell_p $在$ 0<p<\infty $时可分, 在$ p=\infty $时不可分.
	\end{Proposition}
	\begin{Proof}
	对$ \ell_p $, 当$ 0<p<\infty $时, 可取
	\[
	A_1=\{ (x_1,x_2, \dots,x_n,0, \dots,0, \dots) : x_i\in\Q, n\geqslant 1 \}
	\]
	是$ \ell_p $的稠密子集, 且$ A_1 $是可数的(因它是可数个可数集的并). 但对$ \ell_\infty $, 取它的子集
	\[
	A_2=\{ (\varepsilon_1,\varepsilon_2, \dots,\varepsilon_n, \dots) : \varepsilon_i=\pm 1 \}\subset\ell_\infty,
	\]
	注意到$ \abs{A_2}=2^{\aleph_0}=\aleph $, 且$ \forall x,y\in A_2,\ x\ne y $, 都有$ \norm{x-y}=2 $, 故$ A_2 $不可能有可数的稠密子集, 从而$ \ell_\infty $不可分.\qed
	\end{Proof}
	
	下面是对于实分析中提到的一部分结论的复习.
	
	\begin{Definition}[简单函数]\index{J!简单函数}\index{J!阶梯函数}
	设$ (X,\mathcal S) $是可测空间, 若$ X_1,X_2, \dots,X_n\in\mathcal S $是可测集, $ a_1,a_2, \dots,a_n\in\K $, 则称$ f=\sum\limits_{i=1}^na_i1_{X_i} $是$ (X,\mathcal S) $上的\textbf{简单函数}. 特别地, 若$ X=\R $且$ X_i $均为区间, 则称其为\textbf{阶梯函数}.
	\end{Definition}
	
	\begin{Proposition}
	设$ (X,\mathcal S) $是可测空间.
	\begin{enumerate}[(1)]
	\item $ X $上的可测函数可以通过简单函数逐点逼近;
	\item $ X $上几乎处处有限的函数可以被连续函数逐点逼近.
	\end{enumerate}
	\end{Proposition}
	
	\begin{Theorem}[Lebesgue控制收敛定理]
	设$ (X,\mu) $是测度空间, $ (f_n)_{n\geqslant 1} $是可测函数列, 若$ (f_n)_{n\geqslant 1} $几乎处处收敛到$ f $, 且存在$ g\in L_1(X) $使得$ \abs{f_n(x)}\leqslant\abs{g(x)} $ a.e., 则有$ f\in L_1(X) $且
	\[
	\lim_{n\to\infty}\int_Xf_n\diff\mu=\int_Xf\diff\mu.
	\]
	\end{Theorem}
	
	有关于实分析的结论复习到此结束. 下面主要讨论$ L_p $空间中的几类重要的稠密子集.
	
	\begin{Theorem}\label{thm:简单函数族在Lp中稠密性}
	对$ 0<p<\infty $, 可积简单函数族在$ L_p(X) $中稠密.
	\end{Theorem}
	\begin{Proof}
	设$ f\geqslant 0 $, 由简单逼近可知存在单调递增的简单函数列$ (f_n)_{n\geqslant 1} $使得
	\[
	\lim_{n\to\infty}f_n=f\ \text{a.e.},
	\]
	由$ f_n\leqslant f $可知$ \int_X\abs{f_n}^p\diff\mu\leqslant\int_X\abs{f}^p\diff\mu<\infty $, 即$ f_n $是可积的. 由Lebesgue控制收敛定理可知
	\[
	\lim_{n\to\infty}\norm{f-f_n}_p^p=\lim_{n\to\infty}\int_X\abs{f-f_n}^p\diff\mu=\int_X\lim_{n\to\infty}\abs{f-f_n}^p\diff\mu=0,
	\]
	从而在$ L_p(X) $上有$ \lim\limits_{n\to\infty}f_n=f $.
	
	对$ f $带号的情形由$ f=f^+-f^- $即可, 对$ f $复值的情形由$ f=\Re f+\imag\Im f $即可.\qed
	\end{Proof}
	
	\begin{Corollary}
	设$ X=(a,b) $, $ 0<p<\infty $, 则
	\begin{enumerate}[(1)]
	\item $ (a,b) $上可积的阶梯函数族在$ L_p(a,b) $中稠密;
	\item $ (a,b) $上有紧支撑的连续函数族在$ L_p(a,b) $中稠密.
	\end{enumerate}
	\end{Corollary}
	\begin{Proof}
	(1) 由定理\,\ref{thm:简单函数族在Lp中稠密性}\,, 只需证明命题对$ \mu(A)<\infty $的$ f=1_A $成立即可. 由集合的简单逼近可知$ \forall\varepsilon>0 $, 存在开集$ U $使得$ m(A\triangle U)<\varepsilon $, 此时
	\[
	\norm{1_A-1_U}_p^p=\int_a^b\abs{1_A-1_U}^p\diff\mu=\mu(A\triangle U)<\varepsilon,
	\]
	令$ \varepsilon\to 0^+ $即可.
	
	(2) 由(1)可知只需证明对$ m(E)<\infty $的$ 1_E $可被有紧支撑的函数逼近即可. 由Lebesgue测度的内正则性与外正则性可知$ \forall\varepsilon>0 $, 存在紧集$ A $和开集$ B $使得
	\[
	(A\subset E\subset B)\land(m(B\sm A)<\varepsilon)
	\]
	由Urysohn引理可知存在连续函数$ g $使得
	\[
	g(x)=\begin{cases}
	1 & ,x\in A\\0 & ,x\notin B
	\end{cases}
	\]
	从而有
	\[
	\abs{g(x)-1_E(x)}\begin{cases}
	=0 & ,x\in A\cup B^c\\\leqslant 1 & ,x\in B\sm A
	\end{cases}
	\]
	于是
	\[
	\norm{g-1_E}_p=\left(\int_X\abs{g-1_E}^p\diff\mu\right)^{1/p}<\varepsilon^{1/p},
	\]
	令$ \varepsilon\to 0^+ $即可.\qed
	\end{Proof}
	
	\begin{Proposition}
	$ L_p(0,1) $在$ 0<p<\infty $时可分, 在$ p=\infty $时不可分.
	\end{Proposition}
	\begin{Proof}
     当$ p<\infty $时, 注意到由Weierstrass定理可知$ P(0,1) $在$ C(0,1) $上稠密, 由 Lusin定理知 $ C(0, 1) $ 在 $ B(0, 1) $ 上稠密, 则对 $ \forall f\in L_{p}(0, 1) $, 构造序列 $ (f_{n})_{n\geqslant1} $:
     \[
          f_{n}=f|_{(1/n, 1-1/n)},
     \]
     则知 $ (f_{n})_{n\geqslant1}\subset B(0, 1) $, 即 $ B(0, 1) $ 在 $ L_{p}(0, 1) $ 上稠密. 注意到稠密性是有传递性的, 从而$ P(0,1) $在$ L_p(0,1) $上稠密, 于是$ L_p(0,1) $可分.
	
	当$ p=\infty $时, 取它的一个子集
	\[
	A=\{ 1_{[x,y]} : -\infty<x<y<\infty \},
	\]
	注意到$ \abs{A}=\aleph $且任取两个不同的函数$ 1_{[x_1,y_1]} $与$ 1_{[x_2,y_2]} $, 有$ \norm{1_{[x_1,y_1]}-1_{[x_2,y_2]}}_\infty=1 $, 从而$ A $不可能有可数的稠密子集, 从而$ L_\infty(0,1) $不可分.\qed
	\end{Proof}
	
\section{连续函数空间}
	
	\subsection{等度连续性与Ascoli定理}
	
	设$ (K,d_K) $是紧度量空间, $ (E,d_E) $是度量空间. 记$ C(K,E) $是从$ K $到$ E $的连续映射的全体, 那么可以定义$ C(K,E) $上的函数如下
	\[
	\varDelta(f,g)=\sup_{x\in K}d_E(f(x),g(x)),
	\]
	则:
	\begin{enumerate}[(1)]
	\item $ \varDelta $是$ C(K,E) $上的度量;
	\item 若$ E $完备, 则$ C(K,E) $完备;
	\item 若$ E $是赋范空间, 则$ C(K,E) $也是赋范空间, 其上范数$ \norm{f}_\infty=\sup\limits_{x\in K}\norm{f(x)} $.
	\end{enumerate}
	下面考虑$ C(K,E) $的紧性: 赋范时只需$ K $和$ E $均是有限维的即可, 而一般的情况需要引入下面的概念:
	
	\begin{Definition}[等度连续]\index{D!等度连续}\index{Y!一致等度连续}
	设$ (K,d_K) $与$ (E,d_E) $都是度量空间, 且$ \CH\subset C(K,E) $是连续函数族.
	\begin{enumerate}[(1)]
	\item 设$ x_0\in K $, 若
	\[
	\forall\varepsilon>0\,\exists\eta>0\,\forall f\in\mathcal H\,(x\in B(x_0,\eta)\Rightarrow d_E(f(x),f(x_0))<\varepsilon),
	\]
	则称$ \mathcal H $在$ x_0 $处\textbf{等度连续}, 若$ \mathcal H $在$ K $中每一点都等度连续, 则称$ \mathcal H $在$ K $上等度连续.
	\item 若$ \mathcal H $满足
	\[
	\forall\varepsilon>0\,\exists\eta>0\,\forall f\in\mathcal H\,(d_K(x,y)<\eta\Rightarrow d_E(f(x),f(y))<\varepsilon)
	\]
	则称$ \mathcal H $在$ K $上\textbf{一致等度连续}.
	\end{enumerate}
	\end{Definition}
	
	\begin{Remark}
	有关等度连续的一些注记:
	\begin{enumerate}[(1)]
	\item 若$ \mathcal H=\{f\} $, 则$ \mathcal H $等度连续$ \Longleftrightarrow f $连续.
	\item 设$ \mathcal H=\{ f_1,f_2, \dots,f_n \} $, 则$ \mathcal H $等度连续$ \Longleftrightarrow f_i $连续.
	\item 给定常数$ c>0 $, 则所有从$ K $到$ E $的常数为$ c $的Lipschitz映射构成的子集一致等度连续; 进一步对$ \alpha>0 $, 所有从$ K $到$ E $的常数为$ c $, 阶数为$ \alpha $的H\"older映射构成的子集一致等度连续.
	\item 设$ (f_n)_{n\geqslant 1} $一致收敛到$ f $, 即
	\[
	\lim_{n\to\infty}\sup_{x\in K}d_E(f_n(x),f(x))=0,
	\]
	则$ (f_n)_{n\geqslant 1} $等度连续.
	\end{enumerate}
	\end{Remark}
	
	下面的定理说明了当$ K $是紧空间时, 等度连续具有一致性:
	
	\begin{Theorem}
	设$ (K,d_K) $是紧度量空间, $ (E,d_E) $是度量空间, 则$ \mathcal H $在$ K $上一致等度连续当且仅当$ \CH $在$ K $上等度连续.
	\end{Theorem}
	\begin{Proof}
	\textsl{必要性.}\ \ 显然.
	
	\textsl{充分性.}\ \ $ \forall\varepsilon>0 $, 由$ \mathcal H $在$ K $上等度连续可知
	\[
	\forall x\in K\,\exists\eta_x>0\,\forall f\in\mathcal H\,(d_K(x,y)<\eta_x\Rightarrow d_E(f(x),f(y))<\varepsilon)
	\]
	因为$ K $是紧的, 故存在$ x_1,x_2, \dots,x_n $使得
	\[
	K=\bigcup_{i=1}^nB\left(x_i,\frac{\eta_{x_i}}{2}\right),
	\]
	再取$ \eta=\min\{ \eta_{x_1}/2,\eta_{x_2}/2, \dots,\eta_{x_n}/2 \} $, 设$ d_K(x,y)<\eta $, 那么$ \forall f\in\mathcal H $, 存在$ k $使得$ x\in B(x_k,\eta_{x_k}/2) $, 且此时
	\[
	d_K(x_k,y)\leqslant d_K(x_k,x)+d_K(x,y)<\eta+\frac{\eta_{x_k}}{2}<\eta_{x_k},
	\]
	从而
	\[
	d_E(f(x),f(y))\leqslant d_E(f(x),f(x_k))+d_E(f(x_k),f(y))<\varepsilon+\varepsilon=2\varepsilon,
	\]
	即$ \mathcal H $在$ K $上一致等度连续.\qed
     \end{Proof}
     
z     \begin{Theorem} \label{thm:逐点->一致}
          设 $ (K, d_{K}) $ 是紧度量空间, $ (E, d_{E}) $ 是度量空间,  $ (f_{n})_{n\geqslant1} $ 是 $ C(K, E) $ 是中等度连续的序列, 若 $ (f_{n})_{n\geqslant1} $ 逐点收敛于 $ f $, 则 $ f\in C(K, E) $ 且 $ (f_{n})_{n\geqslant1} $ 一致收敛到 $ f $.
     \end{Theorem}
     \begin{Proof}
          先证 $ f $ 连续, 由 $ (f_{n})_{n\geqslant1} $ 等度连续, $ K $紧, 则 $ (f_{n})_{n\geqslant1} $ 一致等度连续, 则有
          \[
               \forall \varepsilon>0\,\exists\eta>0\,\forall n\geqslant1\,(y\in B(x, \eta)\Rightarrow d_{E}(f_{n}(x), f_{n}(y))<\varepsilon),
          \]
          故
          \[
               \begin{aligned}
               d_{E}(f(x), f(y)) & \leqslant d_{E}(f(x), f_{n}(x))+d_{E}(f_{n}(x), f_{n}(y))+d_{E}(f_{n}(y), f(y))\\
               & < d_{E}(f(x), f_{n}(x))+\varepsilon+d_{E}(f_{n}(y), f(y)),
               \end{aligned}
          \]
          令 $ n\to\infty $, $ d_{E}(f(x), f(y))\leqslant\varepsilon $, 即 $ f $ 在 $ x $ 处连续, 由 $ x $ 的任意性得 $ f\in C(K, E) $.

          再证 $ (f_{n})_{n\geqslant1}\rightrightarrows f $, 因为 $ K $ 是紧集, 故 $ \forall \varepsilon>0 $, $ \exists x_{1}, x_{2}, \dots, x_{n}\in E $, $ \eta_{x_{1}}, \eta_{x_{2}}, \dots , \eta_{x_{n}}>0 $ 使得 
          \[
               \forall n\geqslant1\,(x\in B(x_{k}, \eta_{x_{k}})\Rightarrow d_{E}(f_{n}(x), f_{n}(x_{k}))<\varepsilon), 
          \]
          且 $ K=\bigcup_{k=1}^{n}B(x_{k}, \eta_{x_{k}}) $, 则
          \[
               \forall x\in K\,\exists k\,(x\in B(x_{k}, \eta_{x_{k}})).
          \]
          从而
          \[
               \begin{aligned}
                   d_{E}(f_{m}(x), f(x)) & \leqslant d_{E}(f_{m}(x), f_{m}(x_{k}))+d_{E}(f_{m}(x_{k}), f(x_{k}))+d_{E}(f(x_{k}), f(x))\\
                   & <\varepsilon+\varepsilon+\varepsilon.
               \end{aligned}
          \]
          其中第一个 $ \varepsilon $ 因为 $ (f_{n})_{n\geqslant1} $ 等度连续, 第二个 $ \varepsilon $ 因为 $ f_{n} $ 逐点收敛于 $ f $, 第三个 $ \varepsilon $ 因为 $ f $ 的连续性. 故有 $ (f_{n})_{n\geqslant1}\rightrightarrows f $.\qed
     \end{Proof}

     下面证明本节的中心定理.

     \begin{Theorem}[Arzel\`a-Ascoli]\label{thm:Ascoli}
          设 $ (K, d_{K}) $ 是紧度量空间, $ (E, d_{E}) $ 是度量空间, $ \CH\subset C(K, E) $, 则 $ \CH $ 在 $ C(K, E) $ 中相对紧的充分必要条件为
          \begin{enumerate}[(1)]
               \item $ \CH $ 等度连续;
               \item  $ \forall x\in K $,  $ \CH $ 的轨迹 $ \CH(x)=\{ f(x):f\in\CH \} $ 在 $ E $ 中相对紧.
          \end{enumerate}
     \end{Theorem}

     \begin{Proof}
          \textsl{必要性}. 因为 $ \CH $ 相对紧 $ \Rightarrow $ $ \baro{\CH} $ 紧 $ \Rightarrow $ $ \baro{\CH} $ 预紧$ \Rightarrow $ $ \CH $ 预紧. 则 $ \forall \varepsilon>0, \exists f_{1}, f_{2}, \dots f_{n}\in\CH $, 使得 $ \forall f\in\CH\,\exists f_{k}\,(\varDelta(f, f_{k})<\varepsilon) $, i.e.
          \[
               \sup_{x\in K}d_{E}(f(x), f_{k}(x))<\varepsilon.   
          \]
          由 $ f_{1}, f_{2}, \dots, f_{n} $ 连续知它们一致连续(Cantor 定理), 从而 
          \[
               \forall \varepsilon>0\,\forall x\in K\,\exists\eta>0\forall k\,(y\in B(x, \eta)\Rightarrow d_{E}(f_{k}(x), f_{k}(y))<\varepsilon),
          \]
          则对 $ \forall f\in\baro{\CH} $, 有
          \[
               \begin{aligned}
                    d_{E}(f(x), f(y)) & \leqslant d_{E}(f(x), f_{k}(x))+d_{E}(f_{k}(x)+f_{k}(y))+d_{E}(f_{k}(y), f(y))\\
                    & <\varepsilon+\varepsilon+\varepsilon
               \end{aligned}   
          \]
          其中第一个和第三个 $ \varepsilon $ 因为 $ \CH $ 的预紧性, 第二个 $ \varepsilon $ 因为 $ f_{k} $ 一致连续. 从而可知 $ \baro{\CH} $ 一致等度连续, 故 $ \CH $ 等度连续.
          
          定义映射:
          \[
               \rho_{x}: C(K, E)\to E, f \mapsto f(x)
          \]
          则 $ \rho_{x} $ 连续, 而由 $ \baro{\CH} $ 的紧性及连续函数映紧集为紧集可知
          \[
               \rho_{x}(\baro{\CH})=\baro{\CH}(x)=\baro{\CH(x)}    
          \]
          紧, 从而 $ \CH(x) $ 相对紧.

          \textsl{充分性}. 要说明 $ \baro{\CH} $ 紧, 只需说明 $ \baro{\CH} $ 完备且预紧.

          先说明 $ \baro{\CH} $ 完备. 任取 $ \baro{\CH} $ 中的 Cauchy列 $ (f_{n})_{n\geqslant1} $, 则有 $ E $ 中的 Cauchy列 $ (f_{n}(x))_{n\geqslant1} $, 由 $ \baro{\CH}(x) $ 紧且完备可知 $ (f_{n}(x))_{n\geqslant1} $ 有收敛点 $ f(x) $. 即 $ (f_{n})_{n\geqslant1} $ 逐点收敛于 $ f $. 又由 $ \CH $ 等度连续可知 $ \baro{\CH} $ 也是等度连续的, 从而 $ (f_{n})_{n\geqslant1} $ 等度连续, 于是 $ f $ 连续, 且 $ (f_{n})_{n\geqslant1}\rightrightarrows f $. 从而有 $ f\in\baro{\CH} $, 即 $ \baro{\CH} $ 完备.

          再说明 $ \baro{\CH} $ 预紧, 由 $ \baro{\CH} $ 等度连续, 有
          \begin{equation}\label{eq:H等度连续}
               \forall \varepsilon>0\,\exists\eta>0\,\forall f\in\baro{\CH}\,\left(d_{K}(x, y)<\eta\Rightarrow d_{E}(f(x), f(y))<\frac{\varepsilon}{3}\right).
          \end{equation}
          由 $ K $ 紧知其预紧, 故存在 $ K $ 的有限 $ \eta $-网 $ \{ x_{1}, x_{2}, \dots, x_{n} \} $ 使
          \[
               \forall f\in\CH\,\left(x\in B(x_{k}, \eta)\Rightarrow d_{E}(f(x), f_{k}(x))<\frac{\varepsilon}{3}\right),
          \]
          且 $ K=\bigcup_{k=1}^{n}B(x_{k}, \eta) $, 定义映射
          \[
               \varPhi: C(K, E)\to E^{n}, f\mapsto (f(x_{1}), f(x_{2}), \dots, f(x_{n}))     
          \]
          并在 $ E^{n} $ 上定义度量 $ \delta((x_{k})_{k=1}^{n}, (y_{k})_{k=1}^{n})=\sup\limits_{k} d_{E}(x_{k}, y_{k}) $, 因为 $ \baro{\CH}(x_{k}), k=1, 2, \dots, n $ 预紧, 则有 $ \varPhi(\baro{\CH})\subset\prod\limits_{k=1}^{n}\baro{\CH}(x_{k}) $ 是预紧的. 故存在 $ \varPhi(\baro{\CH}) $ 的有限 $ \varepsilon/3 $-网 $ \{ \varPhi(f_{1}),\varPhi(f_{2}),\dots \varPhi(f_{m}) \} $, 即
          \[
               \forall f\in\baro{\CH}\,\exists f_{k}\,\left(\delta(\varPhi(f), \varPhi(f_{k}))<\frac{\varepsilon}{3}\right)
          \] 
          由此只需验证 $ f_{1}, f_{2},\dots,f_{m}  $ 是 $ \baro{\CH} $ 的 $ \varepsilon $-网. 则对任意 $ x $ , 存在 $ x_{r} $, 使得 $ d_{K}(x, x_{r})<\eta $, 所以有
          \[
               \begin{aligned}
                    d_{E}(f(x), f_{k}(x)) & \leqslant d_{E}(f(x), f(x_{r}))+ d_{E}(f(x_{r}), f_{k}(x_{r}))+d_{E}(f_{k}(x_{r}), f_{k}(x))\\
                    & \leqslant \varepsilon/3+\varepsilon/3+\varepsilon/3=\varepsilon
               \end{aligned}
          \]
          其中第一个和第三个 $ \varepsilon/3 $ 因为式\,\eqref{eq:H等度连续}, 第二个 $ \varepsilon $ 因为 $ \varPhi(\baro{\CH}) $ 预紧. 所以 $ \varDelta(f, f_{k})<\varepsilon $, 从而 $ \{f_{1}, f_{2}, \dots , f_{m}\} $ 是 $ \baro{\CH} $ 的 $ \varepsilon $-网.\qed
     \end{Proof}

     \begin{Corollary}
          设 $ (K, d_{K}) $ 是紧度量空间, $ \CH\subset C(K, \K^{n}) $, 则 $ \CH $ 相对紧当且仅当 $ \CH $ 等度连续且 $ \forall x\in K $ 都有 $ \CH(x) $ 有界.
     \end{Corollary}

     \begin{Corollary}
          设开集 $ \varOmega\subset\K^{n} $,  $ (f_{k})_{k\geqslant1}\subset C(\varOmega, \K^{n}) $, 若
          \begin{enumerate}[(1)]
               \item $ (f_{k})_{k\geqslant1} $ 在 $ \varOmega $ 的任一紧子集 $ K $ 上等度连续;
               \item $ \forall x\in\varOmega, \{ f_{k}(x):k\geqslant1 \} $ 有界.
          \end{enumerate} 
          则 $ (f_{k})_{k\geqslant1} $ 有一子列在 $ \varOmega $ 上的任一紧子集 $ K $ 上一致连续.
     \end{Corollary}
     
     \begin{Proof}
          对 $ \forall p\in\N^{*} $, 取
          \[
               K_{p}=\left\{ x\in\varOmega: d(x, \varOmega^{c})\geqslant\frac{1}{p} \right\}\cap \baro{B(0, p)}\subset\varOmega  
          \]
          是一个有界闭集, 因为 $ K_{p}\subset\K^{n} $ 知 $ K_{p} $ 紧. 且由定义知 $ \bigcup_{p\geqslant1}K_{p}=\varOmega $. 由 Ascoli定理知 $ (f_{k})_{k\geqslant1} $ 在 $ K_{1} $ 上有一致收敛子列 $ (f_{1, k})_{k\geqslant1} $; 而 $ (f_{1, k})_{k\geqslant1} $ 在 $ K_{2} $ 上又有一致收敛子列 $ (f_{2, k})_{k\geqslant1}\dots $

          由对角线法则, 取 $ (f_{k, k})_{k\geqslant1} $, 则它在 $ K_{p} $ 上一致收敛 ($ \forall p\geqslant1 $), 因为任意紧子集$ K $必含于某个集 $ K_{p} $, 故 $ (f_{k, k})_{k\geqslant1} $ 在任一紧子集上一致收敛.\qed
     \end{Proof}

     下面是 Ascoli定理的一个重要推论, 它在复分析中已经介绍过, 即 Montel 正规定则.

     \begin{Theorem}[Montel]
          设 $ \varOmega $是 \C 中的开集, $ (f_{n})_{n\geqslant1} $ 是一族在 $ \varOmega $ 上定义的全纯函数, 若对 $ \varOmega $ 的紧子集 $ K $, 有 $ (f_{n})_{n\geqslant1} $ 在 $ K $ 上一致有界, 则 $ (f_{n})_{n\geqslant1} $ 有一个子列在 $ \varOmega $ 上任一紧子集上收敛到 $ \varOmega $ 上的全纯函数 $ f $.
     \end{Theorem}

     \begin{Proof}
          由 Cauchy 积分公式, 有
          \[
               f_{n}=\frac{1}{2\pi \imag}\int_{\partial D(z_{0}, r)} \frac{f_{n}(\xi)}{\xi-z}\diff \xi,\qquad \forall z\in D(z_{0}, r),
          \]
          两侧对 $ z $ 求导有
          \[
               f'_{n}=\frac{1}{2\pi \imag}\int_{\partial D(z_{0}, r)} \frac{f_{n}(\xi)}{(\xi-z)^{2}}\diff \xi,\qquad \forall z\in D(z_{0}, r),
          \]
          对 $ z\in D(z_{0}, r/2) $, 有
          \[
               \begin{aligned}
                    \abs{f'_{n}(x)} & \leqslant \frac{1}{2\pi}\sup_{x\in D(z_{0}, r/2)}\abs{f_{n}(z)}\cdot\int_{\partial D(z_{0}, r)}\frac{\abs{\diff \xi}}{\abs{\xi-z}^{2}}\\
                    & \leqslant \frac{M}{2\pi}\cdot 2\pi r\cdot \frac{4}{r^{2}}=\frac{4M}{r},
               \end{aligned}
          \]
          其中 $ M $是 $ (f_{n})_{n\geqslant1} $ 在 $ K $ 上的界, 则 $ (f_{n}|_{\baro{D}(z_{0}, r/2)})_{n\geqslant1} $ 是以 $ 4M/r $ 为常数的  Lipschitz函数列, 从而等度连续, 有因为对任意紧子集 $ K\subset\varOmega $, 
          $ K $ 可被有限个 $ D(z_{k}, r/2) $ 覆盖, 故 $ (f_{n}|_{K})_{n\geqslant1} $ 等度连续, 从而它有子列在 $ K $ 上一致收敛到 $ f $, 且由 $ f_{n_{k}} $ 全纯知 $ f $ 全纯. \qed
     \end{Proof}
     
	\subsection{Stone-Weierstrass定理}
	
	\begin{Definition}[子代数, 可分点的]\index{Z!子代数}\index{K!可分点的}
	设 $ \CA\subset C(K,\K) $, 其中$ K $是紧度量空间.
	\begin{enumerate}[(1)]
	\item 若$ \CA $是$ C(K,\K) $的线性子空间且关于乘法封闭, 则称$ \CA $是$ C(K,\K) $的\textbf{子代数};
	\item 若$ \CA $满足
	\[
	\forall x,y\in K\,,x\ne y\,\exists f\in\CA\,(f(x)\ne f(y))
	\]
	则称$ \CA $在$ K $上是\textbf{可分点的}.
	\end{enumerate}
	\end{Definition}
	
	\begin{Theorem}[Stone-Weierstrass]
	设$ K $是紧度量空间, $ \CA $是$ C(K,\K) $的子代数, 若
	\begin{enumerate}[(1)]
	\item $ \CA $在$ K $上是可分点的;
	\item $ \forall x\in K\,\exists f\in\CA\,(f(x)\ne 0) $;
	\item $ f\in\CA\Longrightarrow \baro{f}\in\CA $.(也称$ \CA $是\textbf{自伴}的),
	\end{enumerate}
	则$ \CA $在$ C(K,\K) $中稠密.
	\end{Theorem}
	
	这一定理的证明非常复杂, 此处略去.
	
	\begin{Example}
	下面是利用Stone-Weierstrass定理得到的一些结论:
	\begin{enumerate}[(1)]
	\item $ [0,1] $上偶多项式全体在$ C([0,1],\K) $上稠密, 但$ [-1,1] $上的偶多项式全体不在$ C([-1,1],\K) $上稠密.
	\item $ [0,\pi] $上全体三角多项式构成的线性空间在$ C[0,\pi] $上稠密, 即
	\[
	T=\mathrm{span} \{ \cos nx,\ \sin nx : n\in\N \}=\left\{ \sum_{k=1}^na_k\cos kx+b_k\sin kx : n\in\N \right\}
	\]
	在$ C[0,\pi] $上稠密. 由
	\[
	\cos^2x=\frac{1+\cos 2x}{2},\qquad \sin^2x=\frac{1-\cos 2x}{2},\qquad \cos x\sin x=\frac{\sin 2x}{2}
	\]
	可知$ T $是一个代数.
	\end{enumerate}
	\end{Example}
	
	\section*{本章习题}
	\addcontentsline{toc}{section}{本章习题}
	
	习题后面括号中的序号表示对应书中习题的编号.
	
	\begin{enumerate}[label=\textbf{\arabic*.}, ref=\arabic*]
	\item 证明度量空间$ (E,d) $是完备的充分必要条件是: 对$ E $中任一序列$ (x_n)_{n\geqslant 1} $, 若对$ \forall n\geqslant 1 $, 有$ d(x_n,x_{n+1})<2^{-n} $, 则序列$ (x_n)_{n\geqslant 1} $收敛. (2.2)
	\item 设$ (E,d) $是度量空间, $ (x_n)_{n\geqslant 1} $是$ E $中的Cauchy列, 并有$ A\subset E $. 假设$ A $的闭包$ \baro{A} $在$ E $中完备且有$ \lim\limits_{n\to\infty}d(x_n,A)=0 $. 证明: $ (x_n)_{n\geqslant 1} $在$ E $中收敛. (2.3)
	\item 设$ (E,d) $是度量空间, $ \alpha>0 $. 设$ A\subset E $满足$ \forall x,y\in A $且$ x\ne y $必有$ d(x,y)\geqslant\alpha $. 证明: $ A $是完备的. (2.4)
	\item 设$ (E,d) $是度量空间, $ (x_n)_{n\geqslant 1} $是$ E $中发散的Cauchy列. 证明: 
		\begin{enumerate}[(1)]
		\item 任取$ x\in E $, 序列$ (d(x,x_n))_{n\geqslant 1} $收敛到一个正数, 记作$ g(x) $;
		\item 函数$ x\mapsto1/g(x) $是一个从$ E $到$ \R $的连续函数;
		\item 上面定义的函数无界.(2.6)
		\end{enumerate}
	\item 设 $ f: \R^{n}\to\R $ 是一致连续函数, 证明存在两个非负常数 $ a $ 和 $ b $, 使得
		\[
			\abs{f(x)}\leqslant a\norm{x}+b.
		\]
		这里 $ \norm{x} $ 是 $ x $ 的 Euclid 范数. (2.8)
	\item 构造一个反例说明, 在压缩映照原理\,\ref{thm:压缩映照原理}\,中, 如果我们把映射 $ f $ 满足的条件减弱为
		\[
			d(f(x), f(y))<d(x, y)\qquad \forall x, y\in E\wedge x\neq y,
		\]
		则结论不成立. (2.10)
	\item 设$ (E,d) $是紧的度量空间, 在压缩映照原理中若将映射$ f $满足的条件减弱到
		\[
		\forall x,y\in E\,,x\ne y\,(d(f(x),f(y))<d(x,y))
		\]
		则$ f $仍然存在唯一不动点. (2.10$ ' $)
	\item 设$ (E,d) $是一个完备的度量空间, $ f $是其上的映射, 且满足$ f^n $是压缩映射(这里$ f^n $表示$ f $的$ n $次复合). 证明: $ f $有唯一的不动点, 并给出例子说明$ f $可以不连续. (2.11)
	\item 记区间$ I=(0,\infty) $上的自然拓扑为$ \tau $.
	\begin{enumerate}[(1)]
		\item 证明$ \tau $可以被以下完备的距离$ d $诱导:
			\[
			d(x,y)=\abs{\log x-\log y};
			\]
		\item 设函数$ f : I\to I $一次连续可微, 且满足对某个$ \lambda<1 $, 任取$ x\in I $都有$ x\abs{f'(x)}\leqslant\lambda f(x) $. 证明$ f $在$ I $上存在唯一的不动点. (2.12)
	\end{enumerate}
	\item 设$ (E,d) $是完备度量空间, $ f $和$ g $是$ E $上两个可交换的压缩映射(即$ fg=gf $). 证明$ f $和$ g $有唯一的共同不动点. 并举出反例说明当可交换条件不满足时结论不成立. (2.15)
	\item 设$ E=\{ x=(x_n)_{n\geqslant 1} : \forall n\geqslant 1, (x_n=0)\lor(x_n=1) \}=\{ 0,1 \}^\N $. 在$ E $上定义函数
		\[
		\varphi(x)=\sum_{n\geqslant 1}\frac{2x_n}{3^{n  }}
		\]
		并在$ \{ 0,1 \} $上赋予离散拓扑(即$ d(0,1)=1 $的度量诱导的拓扑), 则在$ E $上有相应的乘积拓扑. 证明: $ \varphi $是$ E $到$ \R $的紧子集$ \varphi(E) $上的同胚. (1.11)
	\item 设 $ E $ 是 $ \R $ 上所有的实系数多项式构成的线性空间, 对任一 $ p\in E $, 定义
		\[
			\norm{p}_{\infty}=\max_{x\in[0, 1]}\abs{p(x)}.
		\]
		\begin{enumerate}[(1)]
			\item 证明 $ \norm{\cdot}_{\infty} $ 是 $ E $ 上的范数.
			\item 任取一个 $ a\in\R $, 定义线性映射 $ L_{a}:E\to \R $ 满足 $ L_{a}(p)=p(a) $. 证明 $ L_{a} $ 连续的充分必要条件是 $ a\in[0, 1] $, 并且给出该连续线性映射的范数.
			\item 设 $ a<b $ 并定义 $ L_{a, b}:E\to \R $ 满足
			\[
				L_{a, b}(p)=\int_{a}^{b}p(x)\diff x,
			\]
			给出 $ a, b $ 的取值范围, 使其成为 $ L_{a, b} $ 连续的充分必要条件, 然后确定 $ L_{a, b} $ 的范数. (3.2)
		\end{enumerate}
	\item 设$ (E,\norm{\cdot}_\infty) $是习题3.2中定义的赋范空间, 设$ E_0 $是$ E $中常数项为0的多项式构成的线性子空间(即$ p\in E_0\Longleftrightarrow p(0)=0 $).
		\begin{enumerate}[(1)]
		\item 证明$ N(p)=\norm{p'}_\infty $定义了$ E_0 $上的一个范数, 并且对任意$ p\in E_0 $, 有$ \norm{p}_\infty\leqslant N(p) $;
		\item 证明$ L(p)=\int_0^1\frac{p(x)}{x}\diff x $定义了$ E_0 $上关于$ N $的连续线性泛函, 并求出它的范数;
		\item 上面定义的$ L $是否关于$ \norm{\cdot}_\infty $连续?
		\item 范数$ \norm{\cdot}_\infty $和$ N $在$ E_0 $上是否等价? (3.3)
		\end{enumerate}
	\item 设$ E $是由$ [0,1] $上所有连续函数构成的线性空间, 定义$ E $的两个范数分别为$ \norm{f}_1=\int_0^1\abs{f(x)}\diff x $和$ N(f)=\int_0^1x\abs{f(x)}\diff x $.
		\begin{enumerate}[(1)]
		\item 验证$ N $的确是$ E $上的范数, 且$ N\leqslant\norm{\cdot}_1 $.
		\item 设函数
		\[
		f_n(x)=\begin{cases}
		n-n^2x & ,x\leqslant 1/n\\
		0 & ,\text{其他}
		\end{cases}
		\]
		证明函数列$ (f_n)_{n\geqslant 1} $在$ (E,N) $中收敛到0, 它在$ (E,\norm{\cdot}_1) $中是否收敛? 由这两个范数在$ E $上诱导的拓扑是否相同?
		\item 设$ a\in(0,1] $, 并令$ B=\{ f\in E : f(x)=0, \forall x\in[0,a] \} $. 证明这两个范数在$ B $上诱导相同的拓扑. (3.4)
		\end{enumerate}
	\item 设 $ \varphi:[0, 1]\to [0, 1] $ 连续函数并且不恒等于 1. 设 $ \alpha\in\R $, 定义 $ C([0, 1],\R) $ \footnote{这里 $ C([0, 1],\R) $ 表示从 $ [0, 1] $ 到 \R 上的连续函数的全体 }上的映射 $ T $ 为
		\[
			T(f)(x)=\alpha+\int_{0}^{x}f(\varphi(t))\diff t.
		\]
		证明 $ T^2 $ 是压缩映射. 再根据以上结论证明下面的方程存在唯一解: (3.5)
		\begin{equation}
			f(0)=\alpha, f'(x)=f(\varphi(x))\qquad x\in[0, 1].
		\end{equation}
	\item 设$ \alpha\in\R,\ a>0,\ b>1 $, 考察如下微分方程
		\begin{equation}
		f(0)=\alpha,\qquad f'(x)=af(x^b),\qquad x\in[0,1]
		\end{equation}
		\begin{enumerate}[(1)]
		\item 令$ M>0 $, 验证$ E=C([0,1],\R) $上赋予范数
		\[
		\norm{f}=\sup_{0\leqslant x\leqslant 1}\abs{f(x)}\exp(-Mx)
		\]
		后成为一个Banach空间.
		\item 定义映射
		\[
		T : E\to E,\qquad f(x)\mapsto\alpha+\int_0^x af(t^b)\diff t,
		\]
		证明: 选取合适的$ M $后可以使得$ T $是压缩映射.
		\item 证明本题中的微分方程有唯一解. (3.6)
		\end{enumerate}
	\item (\textbf{Hamel基})设$ E $是域$ \K $上的无限维线性空间, 设$ (e_i)_{i\in\alpha} $是$ E $中的一组向量. 若$ E $中任意向量可以用$ (e_i)_{i\in\alpha} $中的有限个向量唯一线性表示, 即对任意$ x\in E $, 存在唯一一组$ (x_i)_{i\in\alpha}\subset\K $使得仅有有限多个$ x_i\ne 0 $且$ x=\sum\limits_{i\in\alpha}x_ie_i $成立, 则称$ (e_i)_{i\in\alpha} $是$ E $中的Hamel基.
		\begin{enumerate}[(1)]
		\item 由Zorn引理证明$ E $有一组Hamel基.
		\item 假设$ E $还是一个赋范空间, 证明$ E $上必存在不连续的线性泛函.
		\item 证明: 在任意无限维赋范空间上, 一定存在一个比原来的范数严格强的范数(即新范数诱导的拓扑一定比原来的范数诱导的拓扑强且不相同), 由此说明若线性空间$ E $上任意两个范数都诱导相同的拓扑, 则$ E $有限维. (3.7)
		\end{enumerate}
	\item 设$ f\in L_2(\R) $, $ g(x)=\frac{1}{x}1_{[1,\infty)}(x) $, 证明: $ fg\in L_1(\R) $. 并举出反例说明$ f_1,f_2\in L_1(\R) $时$ f_1f_2\notin L_1(\R) $. (3.10)
	\item 设$ (X,\CA, \mu) $是有限测度空间.
		\begin{enumerate}[(1)]
		\item 证明: 若$ 0<p<q\leqslant\infty $, 那么$ L_q(X)\subset L_p(\varOmega) $, 并举反例说明结论在$ \mu(X)=\infty $的测度空间上不成立.
		\item 证明: 若$ f\in L_\infty(X) $, 则$ f\in\bigcap_{p<\infty}L_p(X) $且$ \lim\limits_{p\to\infty}\norm{f}_p=\norm{f}_\infty $.
		\item 设$ f\in\bigcap_{p<\infty}L_p(X) $且满足$ \limsup\limits_{p\to\infty}\norm{f}_p<\infty $, 证明$ f\in L_\infty(X) $. (3.11)
		\end{enumerate}
	\item (\textbf{插值不等式})设$ 0<p<q\leqslant\infty $, $ 0\leqslant\theta\leqslant 1 $, 并令
		\[
		\frac{1}{s}=\frac{\theta}{p}+\frac{1-\theta}{q},
		\]
		证明$ f\in L_p(X)\cap L_q(X) $, 则$ f\in L_s(X) $且
		\[
		\norm{f}_s\leqslant\norm{f}_p^{\theta}\norm{f}_q^{1-\theta}.
		\]
		(3.12)
	\item (\textbf{卷积})在实数集$ \R $上取Lebesgue $ \sigma $-代数与Lebesgue测度, 并设$ f, g\in L_1(\R) $.
		\begin{enumerate}[(1)]
		\item 证明:
		\[
		\int_{\R\times\R}f(u)g(v)\diff u\diff v=\left(\int_\R f(u)\diff u\right)\left(\int_\R g(v)\diff v\right)=\int_\R\left(\int_\R f(x-y)g(y)\diff y \right)\diff x,
		\]
		并由此导出函数$ x\mapsto\int_\R f(x-y)g(y)\diff y $在$ \R $上几乎处处有定义.
		\item 定义$ f $与$ g $的\textbf{卷积}$ f\ast g $为
		\[
		f\ast g(x)=\begin{cases}
		\int_\R f(x-y)g(y)\diff y & ,\text{当积分存在时}\\
		0 & ,\text{其他情形}
		\end{cases}
		\]
		证明$ f\ast g\in L_1(\R) $且$ \norm{f\ast g}_1\leqslant\norm{f}_1\norm{g}_1 $.
		\item 取$ f=1_{[0,1]} $, 求$ f\ast f $. (3.16)
		\end{enumerate}
	\item 对任意$ x\in[0,1] $, 设$ f_n(x)=x^n $. 在$ [0,1] $的哪些点处, $ (f_n)_{n\geqslant 1} $等度连续? (5.1)
	\item 设$ K $是度量空间而$ E $是赋范空间, $ (f_n)_{n\geqslant 1}\subset C(K,E) $.
		\begin{enumerate}[(1)]
		\item 证明: 若$ (f_n)_{n\geqslant 1} $在某一点$ x $等度连续, 那么对任意收敛到$ x $的点列$ (x_n)_{n\geqslant 1} $, 都有$ (f_n(x)-f_n(x_n))_{n\geqslant 1} $收敛到0.
		\item 证明: 若$ (f_n(x))_{n\geqslant 1} $在$ E $中收敛到$ y $, 那么对任意收敛到$ x $的点列$ (x_n)_{n\geqslant 1} $, 都有$ (f_n(x_n))_{n\geqslant 1} $也收敛到$ y $.
		\item 取$ f_n=\sin nx $, 证明$ (f_n)_{n\geqslant 1} $在$ \R $上无处等度连续. (5.2)
		\end{enumerate}
	\item 考虑函数序列$ (f_n)_{n\geqslant 1} $, 这里$ f_n(t)=\sin\sqrt{t+4( n\pi )^2} $, 其中$ t\in[0,\infty) $.
		\begin{enumerate}[(1)]
		\item 证明$ (f_n)_{n\geqslant 1} $等度连续并且逐点收敛到0.
		\item 用$ C_b([0,\infty),\R) $表示$ [0,\infty) $上所有有界连续实函数构成的空间, 并赋予范数
		\[
		\norm{f}_\infty=\sup_{t\geqslant 0}\abs{f(t)},
		\]
		那么$ (f_n)_{n\geqslant 1} $在$ C_b([0,\infty),\R) $中是否相对紧? (5.5)
		\end{enumerate}
	\item 设$ (K,d) $是紧度量空间, 证明所有从$ K $到$ \R $的Lipschitz函数构成的集合在$ C(K,\R) $中稠密. (5.10)
	\end{enumerate}
	
	
	\section*{本章注记}
	\addcontentsline{toc}{section}{本章注记}
	\begin{enumerate}
	\item 本章中以下命题/结论/注记未在正文中进行证明, 其证明参见附录A的相关内容:
		
		\hspace{4em}命题\,\ref{prop:Hausdorff空间的相关命题1}\,, 注1.2.2, 命题\,\ref{prop:Cauchy列的性质}\,, 命题\,\ref{prop:连续映射的性质}\,, 注\,\ref{rmk:预紧性的刻画}\,, 命题\,\ref{prop:乘积拓扑空间的继承性质}\,, 推论\,\ref{col:距离函数连续}.
	\item 有关定向集, 网和网的极限的进一步讨论参见附录A.
	\item 有关压缩映照原理的进一步讨论参见附录B中习题2.10, 习题2.10$ ' $和习题2.11.
	\item 有关一般空间的Hamel基的进一步讨论参见附录B中习题3.7.
	\item 有关$ L_p $空间上的插值不等式参见附录B中习题3.12.
	\item 关于Lebesgue可测函数的卷积的讨论参见附录B中习题3.16.
	\end{enumerate}