% !TeX root = main.tex
% This is the Appendix B.

\chapter{一些习题课上讲过的题目}

\section{第1、2章习题}

	\textbf{习题1.11}\ [作业]\ \ 设$ E=\{ x=(x_n)_{n\geqslant 1} : \forall n\geqslant 1, (x_n=0)\lor(x_n=1) \}=\{ 0,1 \}^\N $. 在$ E $上定义函数
	\[
	\varphi(x)=\sum_{n\geqslant 1}\frac{2x_n}{3}
	\]
	并在$ \{ 0,1 \} $上赋予离散拓扑(即$ d(0,1)=1 $的度量诱导的拓扑), 则在$ E $上有相应的乘积拓扑. 证明: $ \varphi $是$ E $到$ \R $的紧子集$ \varphi(E) $上的同胚.
	\begin{Proof}
	因为$ \{ 0,1 \} $是紧空间, 由Tychonoff定理可知$ E=\prod\limits_{n\geqslant 1}\{ 0,1 \} $也是紧空间, 而$ \R $是Hausdorff空间, 故只需证明$ \varphi $是连续单射即可. 在$ E $上取
	\[
	\rho(x,y)=\sup_{n\geqslant 1}\frac{d(x_n,y_n)}{n}
	\]
	那么$ \rho $是$ E $的度量且$ \rho $诱导的拓扑与乘积拓扑一致(这一结论的证明见定理\,\ref{prop:Hausdorff空间的相关命题1}\,中(3)的证明过程). 对$ 0<\delta<1 $, $ \rho(x,y)<\delta $意味着$ \forall n\geqslant 1\,(d(x_n,y_n)/n<\delta) $, 也即
	\[
	\forall n\geqslant 1\,(d(x_n,y_n)<n\delta).
	\]
	由实数的Archimedes性, 存在$ N\in\N $使得$ N\delta<1<(N+1)\delta $, 从而只需$ x_n=y_n $对$ n=1,2,\cdots,N $成立即可, 也即
	\[
	B_\rho(x,\delta)\subset\{ y : x_n=y_n, 1\leqslant n\leqslant N \}.
	\]
	那么$ \forall\varepsilon>0 $, 存在$ n_0\in\N $使得$ \sum\limits_{n\geqslant n_0+1}\frac{2}{3^n}<\varepsilon $, 而注意到
	\[
	\diam(\varphi(B_\rho(x,\delta)))\leqslant\diam(\varphi\{ t : x_n=y_n, 1\leqslant n\leqslant N \})<\sum_{n\geqslant N+1}\frac{2}{3^n},
	\]
	取足够小的$ \delta $使得$ N\geqslant n_0 $, 那么此时上式右侧$ <\varepsilon $, 于是$ \varphi $连续.
	
	而$ \forall x,y\in E $且$ x\ne y $, 取最小的使得$ x_k\ne y_k $的正数$ k $, 那么
	\[
	\abs{\varphi(y)-\varphi(x)}\geqslant\frac{2\abs{x_k-y_k}}{3^k}-\sum_{n\geqslant k+1}\frac{2}{3^n}=\frac{2}{3^k}-\frac{1}{3^k}=\frac{1}{3^k}>0,
	\]
	即$ \varphi(x)\ne\varphi(y) $, 故$ \varphi $是单射.\qed
	\end{Proof}

	\textbf{习题2.2}\ [习题课]\ \ 证明度量空间$ (E,d) $是完备的充分必要条件是: 对$ E $中任一序列$ (x_n)_{n\geqslant 1} $, 若对$ \forall n\geqslant 1 $, 有$ d(x_n,x_{n+1})<2^{-n} $, 则序列$ (x_n)_{n\geqslant 1} $收敛.
	\begin{Proof}
	\textsl{必要性}. 由题设可知对$ \forall n,p\geqslant 1 $, 有
	\[
	d(x_n,x_{n+p})\leqslant\sum_{k=n}^{n+p-1}d(x_k,x_{k+1})<\sum_{k=n}^{n+p-1}2^{-k}<\sum_{k=n}^\infty 2^{-k}=2^{1-n}
	\]
	可知$ n, p\to\infty $时上式趋于0, 从而$ (x_n)_{n\geqslant 1} $是Cauchy列. 由度量空间完备可知$ (x_n)_{n\geqslant 1} $收敛.
	
	\textsl{充分性}. 设$ (y_n)_{n\geqslant 1} $是$ (E,d) $上的Cauchy列, 则可取子列$ (y_{n_k})_{k\geqslant 1} $使得$ \forall k\geqslant 1 $有$ d(y_{n_k},y_{n_{k+1}})<2^{-k} $. 由题设可知$ (y_{n_k})_{k\geqslant 1} $收敛, 那么$ (y_n)_{n\geqslant 1} $也收敛.\qed
	\end{Proof}
	
	\textbf{习题2.3}\ [作业]\ \ 设$ (E,d) $是度量空间, $ (x_n)_{n\geqslant 1} $是$ E $中的Cauchy列, 并有$ A\subset E $. 假设$ A $的闭包$ \bar{A} $在$ E $中完备且有$ \lim\limits_{n\to\infty}d(x_n,A)=0 $. 证明: $ (x_n)_{n\geqslant 1} $在$ E $中收敛.
	\begin{Proof}
	由$ d(x,A) $的定义与$ \lim\limits_{n\to\infty}d(x_n,A)=0 $可知
	\[
	\forall n\geqslant 1\,\exists y_n\in A\,\left(d(x_n,y_n)<d(x_n,A)+\frac{1}{n}\right).
	\]
	则$ \forall n,m\geqslant 1 $, 有
	\begin{align*}
	d(y_n,y_m)&\leqslant d(y_m,x_n)+d(x_n,x_m)+d(x_m,y_m)\\
	&d(x_n,A)+\frac{1}{n}+d(x_n,x_m)+d(x_m,A)+\frac{1}{m}\to 0\qquad (n,m\to\infty)
	\end{align*}
	于是$ (y_n)_{n\geqslant 1}\subset A\subset\bar{A} $是Cauchy列. 由$ \bar{A} $完备可知$ (y_n)_{n\geqslant 1} $收敛于某个$ y_0\in\bar{A} $. 从而
	\[
	d(x_n,y_0)\leqslant d(x_n,y_n)+d(y_n,y_0)\leqslant\frac{1}{n}+d(x_n,A)+d(y_n,y_0)\to 0\qquad(n\to\infty)
	\]
	于是$ (x_n)_{n\geqslant 1} $在$ E $中收敛.\qed
	\end{Proof}
	
	\textbf{习题2.4}\ [作业]\ \ 设$ (E,d) $是度量空间, $ \alpha>0 $. 设$ A\subset E $满足$ \forall x,y\in A $且$ x\ne y $必有$ d(x,y)\geqslant\alpha $. 证明: $ A $是完备的.
	\begin{Proof}
	设$ (x_n)_{n\geqslant 1}\subset A $是Cauchy列, 则
	\[
	\forall 0<\varepsilon<\alpha\,\exists n_0\in\N\,(n,m\geqslant n_0\Rightarrow d(x_n,x_m)<\varepsilon<\alpha),
	\]
	从而由题设可知只能$ x_n=x_m $. 即$ \forall n\geqslant n_0\,(x_n=x_{n_0}) $. 从而$ (x_n)_{n\geqslant 1} $收敛且极限为$ x_{n_0} $, 故$ A $完备.\qed
	\end{Proof}
	
	\textbf{习题2.6}\ [习题课]\ \ 设$ (E,d) $是度量空间, $ (x_n)_{n\geqslant 1} $是$ E $中发散的Cauchy列. 证明:
	
	(1) 任取$ x\in E $, 序列$ (d(x,x_n))_{n\geqslant 1} $收敛到一个正数, 记作$ g(x) $;
	
	(2) 函数$ x\mapsto1/g(x) $是一个从$ E $到$ \R $的连续函数;
	
	(3) 上面定义的函数无界.
	\begin{Proof}
	(1) $ \forall n,m\geqslant 1 $, 有
	\[
	\abs{d(x,x_n)-d(x,x_m)}\leqslant d(x_n,x_m)\to 0\qquad (n,m\to\infty)
	\]
	从而$ (d(x,x_n))_{n\geqslant 1} $是$ \R $上的Cauchy列. 于是它收敛, 并设其极限是$ \lambda $, 则有$ \lambda\geqslant 0 $. 若$ \lambda=0 $, 则有$ x_n\to x $, 这与$ (x_n)_{n\geqslant 1} $发散矛盾, 从而只能$ \lambda>0 $.
	
	(2) 注意到函数$ t\mapsto 1/t $是连续的, 只需证明$ g $是连续函数. 对$ \forall x,y\in E $, 由
	\[
	\abs{g(x)-g(y)}=\abs{\lim\limits_{n\to\infty}d(x,x_n)-\lim\limits_{n\to\infty}d(y,x_n)}=\lim\limits_{n\to\infty}\abs{d(x,x_n)-d(y,x_n)}\leqslant d(x,y)
	\]
	可知$ g $是一致连续的, 从而$ g $连续.
	
	(3) 因为$ (x_n)_{n\geqslant 1} $是Cauchy列, 则
	\[
	\forall\varepsilon>0\,\exists n_0\in\N\,(n\geqslant n_0\Rightarrow d(x_m,x_{n_0})<\varepsilon)
	\]
	则$ \lim\limits_{n\to\infty}d(x_n,x_{n_0})\leqslant\varepsilon $. 从而$ g(x_{n_0})\leqslant\varepsilon $, 于是$ 1/g(x_{n_0})\geqslant 1/\varepsilon $. 由$ \varepsilon $的任意性可知$ x\mapsto 1/g(x) $无界.\qed
	\end{Proof}
	
	\textbf{习题2.8}\ [作业]\ \ 设 $ f: \R^{n}\to\R $ 是一致连续函数, 证明存在两个非负常数 $ a $ 和 $ b $, 使得
	\[
		\abs{f(x)}\leqslant a\norm{x}+b.
	\]
	这里 $ \norm{x} $ 是 $ x $ 的 Euclid 范数.

	\begin{Proof}
		由于 $ f $ 是一致连续的, 取 $ \varepsilon=1 $, 则
		\[
			\exists \delta>0\,(\norm{x-y}<\delta\Rightarrow\abs{f(x)-f(y)}<1),
		\]
		由实数的 Archimedes 性知 $ \exists n\in\N\,(n\delta\leqslant\norm{x}<(n+1)\delta) $, 则此时取 $ 0 $ 到 $ x $ 连线上划分
		\[
			x_{0}=0,\quad x_{1}=\delta\cdot\frac{x}{\norm{x}},\quad x_{2}=2\delta\cdot\frac{x}{\norm{x}}\dots x_{n}=n\delta\cdot\frac{x}{\norm{x}},\quad x_{n+1}=x.
		\]
		则 
		\[
			\begin{aligned}
				\abs{f(x)-f(0)} & \leqslant\sum_{k=0}^{n}\abs{f(x_{k+1})-f(x_{k})}\\
				& \leqslant n+1\\
				& = n\cdot\frac{1}{\delta}\cdot\delta+1\\
				& \leqslant\frac{1}{\delta}\cdot\norm{x}+1.
			\end{aligned}
		\]
		故 $ \abs{f(x)}\leqslant\abs{f(0)}+\norm{x}/\delta+1 $. 取 $ a=1/\delta, b=\abs{f(0)}+1 $ 即可.\qed
	\end{Proof}
	
	\begin{Remark}
		本题结论在1维情形下, 即是一致连续函数在无穷远处是线性增长的, 这是一个较强的结论, 它说明一个一直连续函数在充分远处可以使用线性函数控制.
	\end{Remark}
	
	\textbf{习题2.10}\ [作业]\ \ 构造一个反例说明, 在压缩映照原理\ref{thm:压缩映照原理}中, 如果我们把映射 $ f $ 满足的条件减弱为
	\[
		d(f(x), f(y))<d(x, y)\qquad \forall x, y\in E\wedge x\neq y,
	\]
	则结论不成立.

	\begin{Solution}
		取 $ E=\R $, $ f(x)=\sqrt{x^{2}+1} $, 则
		\[
			\begin{aligned}
				d(f(x), f(y)) & =\abs{\sqrt{x^{2}+1}-\sqrt{y^{2}+1}}\\
				& \leqslant\abs{\frac{\xi}{\sqrt{\xi^{2}+1}}}\cdot\abs{x-y}<\abs{x-y}.
			\end{aligned}
		\]
		但 $ f(x)=x $ 显然无解.
	\end{Solution}
	\begin{Remark}
		本题也可以举出另外的反例. 考虑无解的方程 $ \arctan x=\pi/2 $, 那么可以构造映射
		\[
			f : \R\to\R,\qquad x\mapsto x+\frac\pi 2-\arctan x,
		\]
		注意到
		\[
			f'=1-\frac{1}{\xi^2+1}=\frac{\xi^2}{\xi^2+1}<1,
		\]
		可知如此构造的 $ f $ 满足题设条件, 但方程 $ f(x)=x $ 等价于 $ \arctan x=\pi/2 $ 无解, 从而不存在不动点.
	\end{Remark}
	
	\textbf{习题2.10'}\ [习题课]\ \ 设$ (E,d) $是紧的度量空间, 在压缩映照原理中若将映射$ f $满足的条件减弱到
	\[
	\forall x,y\in E\,,x\ne y\,(d(f(x),f(y))<d(x,y))
	\]
	则$ f $仍然存在唯一不动点.
	\begin{Proof}
	记$ g(x)=d(x,f(x)) $, 由$ f $连续且$ d $连续可知$ g $也是连续的. 因为$ E $是紧的, 故$ g(E) $也是紧的, 从而$ g(E) $能取到最小值$ \lambda $. 反设$ \lambda>0 $, 则$ \exists x_0\in E\,(d(x_0,f(x_0)))=\lambda $, 那么
	\[
	d(f(x_0),f^2(x_0))<d(x_0,f(x_0))=\lambda.
	\]
	这与$ \lambda $是$ g $的最小值矛盾. 从而只能$ \lambda=0 $, 此时$ x_0 $即为$ f $的不动点.
	
	再说明唯一性. 若$ x_0\ne y_0 $都是$ f $的不动点, 由
	\[
	d(f(x_0),f(y_0))=d(x_0,y_0)
	\]
	矛盾.\qed
	\end{Proof}
	
	\textbf{习题2.11}\ [习题课]\ \ 设$ (E,d) $是一个完备的度量空间, $ f $是其上的映射, 且满足$ f^n $是压缩映射(这里$ f^n $表示$ f $的$ n $次复合). 证明: $ f $有唯一的不动点, 并给出例子说明$ f $可以不连续.
	\begin{Proof}
	因$ f^n $是压缩映射, 由压缩映照原理可知$ f^n $有唯一的不动点$ x_0 $, 也即$ f^n(x_0)=x_0 $, 那么
	\[
	f^n(f(x_0))=f^{n+1}(x_0)=f(f^n(x_0))=f(x_0),
	\]
	从而$ f(x_0) $也是$ f^n $的不动点. 由$ f^n $不动点的唯一性可知只能$ x_0=f(x_0) $, 即$ x_0 $是$ f $的不动点.
	
	再说明唯一性. 若$ x_0\ne y_0 $都是$ f $的不动点, 则$ f^n(y_0)=y_0 $, 即$ y_0 $也是$ f^n $的不动点, 从而$ x_0=y_0 $.
	
	取$ f=1_\Q $, 那么注意到$ f^2\equiv 1 $是压缩映射, 但$ f $并不连续.\qed
	\end{Proof}
	
	\textbf{习题2.12}\ [习题课]\ \ 记区间$ I=(0,\infty) $上的自然拓扑为$ \tau $.
	
	(1) 证明$ \tau $可以被以下完备的距离$ d $诱导:
	\[
	d(x,y)=\abs{\log x-\log y};
	\]
	
	(2) 设函数$ f : I\to I $一次连续可微, 且满足对某个$ \lambda<1 $, 任取$ x\in I $都有$ x\abs{f'(x)}\leqslant\lambda $. 证明$ f $在$ I $上存在唯一的不动点.
	\begin{Proof}
	(1) $ d $是距离是显然的, 下证它完备: 任取$ (I,d) $中的Cauchy列$ (x_n)_{n\geqslant 1} $, 那么
	\[
	\forall\varepsilon>0\,\exists n_0\in\N\,(n,m\geqslant n_0\Rightarrow d(x_n,x_m)=\abs{\log{x_n}-\log{x_m}<\varepsilon})
	\]
	则$ (\log x_n)_{n\geqslant 1} $是$ \R $中的Cauchy列, 从而存在$ z\in\R $使得
	\[
	\lim_{n\to\infty}\abs{\log x_n-z}=0,
	\]
	也即$ \lim\limits_{n\to\infty}\abs{x_n-\exp z}=0 $, 从而$ (x_n)_{n\geqslant 1} $收敛到$ \exp z $, 于是$ (I,d) $完备.
	
	再说明$ \tau $可以被$ d $诱导. 记$ d $诱导的拓扑为$ \tau_d $, 则$ \forall r>0,x\in I $, 考虑$ (I,d) $中的球$ B_0(x,r)=\{ y : d(y,x)<r \} $. 而
	\[
	d(y,x)<r\Longleftrightarrow\abs{\log y-\log x}<r\Longleftrightarrow y\in(x\exp (-r),x\exp r),
	\]
	于是$ \tau_d\subset \tau $. 而对$ I $中任意开区间$ (a,b) $, 设$ x=\sqrt{ab} $且$ r=\frac{1}{2}\log\frac{b}{a} $, 那么$ (a,b)=(x\exp(-r),x\exp r) $. 于是$ \tau\subset\tau_d $. 这说明$ \tau=\tau_d $.
	
	(2) 由Cauchy中值定理可知$ \forall x,y\in I $, 不妨$ x<y $, 存在$ \xi\in(x,y) $使得
	\[
	\abs{\frac{\log f(x)-\log f(y)}{x-y}}=\abs{\frac{f'(\xi)/f(\xi)}{1/\xi}}\leqslant\lambda.
	\]
	故$ f $是压缩映射, 故存在唯一不动点.\qed
	\end{Proof}
	\begin{Remark}
	(2) 有另证: 由题设可知
	\[
	\pm\frac{f'(t)}{f(t)}\leqslant\frac{\lambda}{t},\qquad t\in I,
	\]
	两侧同时在$ [x,y] $上积分得
	\[
	\pm(\log f(y)-\log f(x))\leqslant\lambda(\log y-\log x),
	\]
	从而$ \abs{\log f(y)-\log f(x)}\leqslant\lambda\abs{\log y-\log x} $. 因此$ f : I\to I $是完备度量空间$ (I,d) $上的压缩映射. 因此由压缩映照原理可知$ f $在$ I $上存在唯一不动点.
	
	并且注意到$ I $上的Euclid距离是不完备的, 尽管$ d $与Euclid距离诱导出的拓扑是相同的, 但$ d $却是完备的. 这说明完备性并不是一个拓扑概念, 它跟空间上赋予的度量有关.
	\end{Remark}
	
	\textbf{习题2.15}\ [习题课]\ \ 设$ (E,d) $是完备度量空间, $ f $和$ g $是$ E $上两个可交换的压缩映射(即$ fg=gf $). 证明$ f $和$ g $由唯一的共同不动点. 并举出反例说明当可交换条件不满足时结论不成立.
	\begin{Proof}
	设$ f $的不动点是$ x_0 $, 即$ x_0=f(x_0) $. 那么
	\[
	g(x_0)=g(f(x_0))=f(g(x_0)),
	\]
	即$ g(x_0) $也是$ f $的不动点, 从而$ g(x_0)=x_0 $, $ x_0 $是$ g $的不动点. 由对称性可证另一侧.
	
	若去掉可交换的条件, 取$ f\equiv\frac{1}{4} $而$ g\equiv\frac{3}{4} $即可. 此时注意到
	\[
	f(g(x))=\frac{1}{4},\qquad g(f(x))=\frac{3}{4},
	\]
	也即$ f $与$ g $的不动点分别是$ \frac{1}{4} $与$ \frac{3}{4} $.\qed
	\end{Proof}
	
\section{第3章习题(不完全)}

	\textbf{习题3.2}\ [习题课]\ \ 设 $ E $ 是 $ \R $ 上所有的实系数多项式构成的线性空间, 对任一 $ p\in E $, 定义
	\[
		\norm{p}_{\infty}=\max_{x\in[0, 1]}\abs{p(x)}.
	\]
	\begin{enumerate}[(1)]
		\item 证明 $ \norm{\cdot}_{\infty} $ 是 $ E $ 上的范数.
		\item 任取一个 $ a\in\R $, 定义线性映射 $ L_{a}:E\to \R $ 满足 $ L_{a}(p)=p(a) $. 证明 $ L_{a} $ 连续的充分必要条件是 $ a\in[0, 1] $, 并且给出该连续线性映射的范数.
		\item 设 $ a<b $ 并定义 $ L_{a, b}:E\to \R $ 满足
		\[
			L_{a, b}(p)=\int_{a}^{b}p(x)\diff x,
		\]
		给出 $ a, b $ 的取值范围, 使其成为 $ L_{a, b} $ 连续的充分必要条件, 然后确定 $ L_{a, b} $ 的范数.
	\end{enumerate}

	\begin{Proof}
		(1) 验证范数的4条性质:
		\begin{itemize}
			\item $ \norm{p}_{\infty}\geqslant 0 $ 显然成立;
			\item $ \norm{p}_{\infty}\Rightarrow \forall x\in [0, 1]\,(\abs{p(x)}=0) $, 由代数基本定理可知 $ \forall x\in\R\,(p(x)=0) $, 即 $ p=0 $;
			\item $ \norm{\lambda p}_{\infty}=\abs{\lambda}\norm{p}_{\infty} $ 显然成立;
			\item $ \norm{p+q}_{\infty}\leqslant\norm{p}_{\infty}+\norm{q}_{\infty} $ 显然成立.
		\end{itemize}

		(2) \textsl{必要性}. 若 $ L_{a} $ 是连续的, 则
		\[
			\abs{L_{a}(p)}=\abs{p(a)}\leqslant \norm{L_{a}}\norm{p}_{\infty}\qquad\norm{L_{a}}<\infty.
		\]
		取 $ p_{n}=x^{n} $, 则 $ (p_{n})_{n\geqslant 1} $ 是 $ E $ 的 Hamel 基, 则上式说明
		\[
			\norm{L_{a}(p_{n})}=\abs{a^{n}}\leqslant\abs{a}^{n}\leqslant \norm{L_{a}}\norm{x^{n}}_{\infty},
		\]
		而由 $ \norm{p_{n}}_{\infty}=1 $, 知
		\[
			\abs{a}^{n}\leqslant\norm{L_{a}},
		\]
		令 $ n\to\infty $, 则有 $ \abs{a}\leqslant1 $. 

		同理取 $ p_{n}=(1-x)^{n} $ 也是 $ E $ 的 Hamel 基, 类似可得 $ \abs{1-a}\leqslant1 $, 从而 $ a\in[0, 1] $.

		\textsl{充分性}. 由 $ a\in[0, 1] $ 可知
		\[
			\forall p\in E\,\big(\abs{L_{a}(p)}=\abs{p(a)}\leqslant\norm{p}_{\infty}\big).
		\]
		则 $ L_{a} $ 连续.

		下面计算 $ \norm{L_{a}} $. 由充分性证明的过程与注\ref{rmk:范数性质}的\ref{rmk:范数性质最小C}可知 $ \norm{L_{a}}\leqslant1 $, 取 $ p(x)\equiv 1 $, 则 $ \norm{p}_{\infty}=1 $, 且 $ \abs{L_{a}(p)}=1 $, 故 $ \norm{L_{a}}=1 $.

		(3) $ a, b\in[0, 1] $, 且 $ \norm{L_{a, b}}=b-a $. 下面给出证明.

		\textsl{必要性}. 若 $ L_{a, b} $ 连续, 则
		\[
			\forall p\in E\,\bigg(\abs{L_{a, b}(p)}=\abs{\dint_{a}^{b}p(x)\diff x}\leqslant \norm{L_{a, b}}\norm{p}_{\infty}\bigg)\qquad \norm{L_{a, b}}<\infty.
		\]
		取 $ p_{n}=x^{n} $, $ \norm{p_{n}}_{\infty}=1 $, 则
		\[
			\abs{\dint_{a}^{b}x^{n}\diff x}=\abs{\frac{b^{n+1}-a^{n+1}}{n+1}}\leqslant\norm{L_{a, b}}<\infty.
		\]
		若 $ \abs{a}<\abs{b} $, 则
		\[
			\frac{\abs{b}^{n+1}}{n+1}\left( 1-\abs{\frac{a}{b}}^{n+1} \right) \leqslant \norm{L_{a, b}}<\infty,
		\]
		令 $ n\to\infty $ 知 $ \abs{b}\leqslant1 $, 因此 $ \abs{a}\leqslant1, \abs{b}\leqslant1 $. 同理 $ \abs{b}<\abs{a} $ 时, 亦有 $ \abs{a}\leqslant1, \abs{b}\leqslant1 $; 当 $ \abs{a}=\abs{b} $ 时, 因为 $ a<b $, 所以 $ a=-b $, 也可以得出 $ \abs{a}\leqslant1, \abs{b}\leqslant1 $

		与 (2) 一样, 再取 $ p_{n}=(1-x)^{n} $, 可以得到 $ \abs{a-1}\leqslant1, \abs{b-1}\leqslant1 $.

		\textsl{充分性}. 若 $ a, b\in[0, 1] $, 则有
		\[
			\abs{L_{a, b}(p)}=\abs{\int_{a}^{b}p(x)\diff x}\leqslant \abs{b-a}\cdot \norm{p}_{\infty},
		\]
		则 $ L_{a, b} $ 连续. 

		因为 $ \norm{L_{a, b}}<b-a $, 而当 $ p(x)\equiv 1 $ 时 $ \abs{L_{a, b}(p)}=b-a $, 故 $ \norm{L_{a, b}}=b-a $.\qed
	\end{Proof}
	
	\textbf{习题3.4}\ [习题课]\ \ 设$ E $是由$ [0,1] $上所有连续函数构成的向量空间, 定义$ E $的两个范数分别为$ \norm{f}_1=\dint_0^1\abs{f(x)}\diff x $和$ N(f)=\dint_0^1x\abs{f(x)}\diff x $.
	\begin{enumerate}[(1)]
	\item 验证$ N $的确是$ E $上的范数, 且$ N\leqslant\norm{\cdot}_1 $.
	\item 设函数
	\[
	f_n(x)=\begin{cases}
	n-n^2x & ,x\leqslant 1/n\\
	0 & ,\text{其他}
	\end{cases}
	\]
	证明函数列$ (f_n)_{n\geqslant 1} $在$ (E,N) $中收敛到0, 它在$ (E,\norm{\cdot}_1) $中是否收敛? 由这两个范数在$ E $上诱导的拓扑是否相同?
	\item 设$ a\in(0,1] $, 并令$ B=\{ f\in E : f(x)=0, \forall x\in[0,a] \} $. 证明这两个范数在$ B $上诱导相同的拓扑.
	\end{enumerate}
	
	\begin{Proof}
	(1) 首先我们需要说明$ N $确实是一个范数. 其中正定性由
	\[
	N(f)=0\Longleftrightarrow xf(x)=0\Longleftrightarrow f=0
	\]
	可知, 而齐次性与三角不等式是显然的, 从而$ N $是$ E $上的一个范数. 所求证不等式由
	\[
	N(f)=\int_0^1x\abs{f(x)}\diff x\leqslant\int_0^1\abs{f(x)}\diff x=\abs{f}_1
	\]
	对任意$ f\in C[0,1] $成立可知.
	
	(2) 注意到
	\[
	N(f_n)=\int_0^{1/n}x(n-n^2x)\diff x=\int_0^{1/n}nx(1-nx)\diff x=\frac{1}{n}\int_0^1t(1-t)\diff t=\frac{1}{6n},
	\]
	从而由$ \lim\limits_{n\to\infty}N(f_n)=0 $可知$ (f_n)_{n\geqslant 1} $依范数$ N $收敛到0. 不妨设$ (f_n)_{n\geqslant 1} $依范数$ \norm{\cdot}_1 $收敛到$ f\in E $, 那么
	\[
	\norm{f-f_n}_1=\int_0^{1/n}\abs{n(1-nx)-f(x)}\diff x+\int_{1/n}^1\abs{f(x)}\diff x\geqslant\int_{1/n}^1\abs{f(x)}\diff x,
	\]
	上式中令$ n\to\infty $可得
	\[
	\int_0^1\abs{f(x)}\diff x=0,
	\]
	也即$ f=0 $, 但此时
	\[
	\norm{f_n-f}_1=\norm{f_n}=\int_0^{1/n}\abs{n(1-nx)}\diff x=\frac{1}{2},
	\]
	这与它依范数$ \norm{\cdot}_1 $收敛到$ f $矛盾. 从而$ (f_n)_{n\geqslant 1} $不依范数$ \norm{\cdot}_1 $收敛. 这也说明了这两个范数在$ E $上诱导的拓扑不同.
	
	(3) 由于等价的范数诱导相同的拓扑, 只需证明$ N $与$ \norm{\cdot}_1 $在空间$ B $上等价即可. 由(1)已知$ N\leqslant\norm\cdot_1 $成立, 而
	\[
	N(f)=\int_0^1x\abs{f(x)}\diff x=\int_a^1x\abs{f(x)}\diff x\geqslant a\int_a^1\abs{f(x)}\diff x=a\int_0^1\abs{f(x)}\diff x=a\norm{f}_1.
	\]
	也即$ \norm{\cdot}_1\leqslant\frac{1}{a}N $, 从而$ \norm{\cdot}_1 $与$ N $在$ B $上是等价的.\qed
	\end{Proof}

	\textbf{习题3.5}\ [习题课]\ \ 设 $ \varphi:[0, 1]\to [0, 1] $ 连续函数并且不恒等于 1. 设 $ \alpha\in\R $, 定义 $ C([0, 1],\R) $ \footnote{这里 $ C([0, 1],\R) $ 表示从 $ [0, 1] $ 到 \R 上的连续函数的全体 }上的映射 $ T $ 为
	\[
		T(f)(x)=\alpha+\dint_{0}^{x}f(\varphi(t))\diff t.
	\]
	证明 $ T $ 是压缩映射. 再根据以上结论证明下面的方程存在唯一解:
	\begin{equation}\label{eq:3.5题公式}
		f(0)=\alpha, f'(x)=f(\varphi(x))\qquad x\in[0, 1].
	\end{equation}

	\begin{Proof}
		对$\forall f, g\in C[0, 1]$, $ \forall x\in[0, 1] $ 都有
		\[
			\begin{aligned}
				\norm{Tf-Tg} & =\max_{x}\dint_{0}^{x}(f(\varphi(t))-g(\varphi(t)))\diff t \\
				& \leqslant x\cdot \max_{x\in[0, 1]}\abs{f(x)-g(x)} \\
				& =x\cdot \norm{f-g}.
			\end{aligned}
		\]
		进一步 
		\[
			\begin{aligned}
				\abs{T^{2}f-T^{2}g} & =\abs{\dint_{0}^{x}(Tf)(\varphi(t))-(Tg)(\varphi(t))\diff t} \\
				& \leqslant\dint_{0}^{x}\varphi (t)\norm{f-g}\diff t \\
				& =\norm{f-g}\dint_{0}^{x}\varphi(t)\diff t \\
				& =\norm{f-g}\dint_{0}^{1}\varphi(t)\diff t \\
				& \leqslant \sup \abs{f-g}
			\end{aligned}
		\]
		因为 $ \norm{Tf-Tg}\leqslant\lambda\norm{f-g} $, 从而
		\[
			\sup\abs{\dint_{0}^{x}f(\varphi(t))-g(\varphi(t))\diff t}\leqslant\lambda\sup\abs{f-g}.
		\]
		则
		\[
			\norm{T^{2}f-T^{2}g}\leqslant\dint_{0}^{1}\varphi (t)\diff t\cdot \norm{f-g},
		\]
		取 $ \lambda=\dint_{0}^{1}\varphi(t)\diff t<1 $, 即 $ T^{2} $ 是压缩映射, 则 $ T^{2} $ 有唯一不动点.

		而方程 \eqref{eq:3.5题公式} 成立 $ \Longleftrightarrow $ $ f(x)=\alpha +\dint_{0}^{x} f(\varphi(t))\diff t $. 即 $ Tf=f $, 故方程 \eqref{eq:3.5题公式} 有唯一解. 
	\end{Proof}
	
	\textbf{习题3.9}\ [作业]\ \ 设$ E $是Banach空间.
	\begin{enumerate}[(1)]
	\item 设$ u\in\CB(E) $且$ \norm{u}<1 $, 证明$ \id_E-u $在$ \CB(E) $中可逆;
	\item 设$ GL(E) $是$ \CB(E) $中可逆元构成的集合, 证明: $ GL(E) $关于复合运算构成一个群, 且它是$ \CB(E) $中的开集;
	\item 证明$ u\mapsto u^{-1} $是$ GL(E) $上的同胚.
	\end{enumerate}
	
	\begin{Proof}
	(1) 由$ \norm{u}<1 $和
	\[
	\sum_{n\geqslant 0}\norm{u^n}\leqslant\sum_{n\geqslant 0}\norm{u}^n<\infty
	\]
	可知级数$ \sum\limits_{n\geqslant 0}u^n $绝对收敛, 而$ \CB(E) $完备, 从而$ \sum\limits_{n\geqslant 0}u^n $收敛, 那么由
	\[
	(\id_E-u)\sum_{n\geqslant 0}u^n=\sum_{n\geqslant 0}u^n-\sum_{n\geqslant 1}u^n=\id_E
	\]
	可知$ \id_E-u $是右可逆的, 其右逆为$ \sum\limits_{n\geqslant 0}u^n $. 同理可证它是左可逆的, 且左逆也是$ \sum\limits_{n\geqslant 0}u^n $, 故$ \id_E-u $可逆, 其逆就是$ \sum\limits_{n\geqslant 0}u^n $.
	
	(2) 首先$ GL(E) $关于复合运算封闭, 且满足结合律, $ \id_E $是其中的单位元且由定义可知其中每个元素都有逆元. 于是$ (GL(E),\circ) $是一个群. 下面说明$ GL(E) $是$ \CB(E) $中的开集, 只需要说明对$ u\in GL(E) $, 对任意$ v $使得$ \norm{v-u}<\delta $, 都有$ v\in GL(E) $即可. 注意到
	\[
	v=u-(u-v)=u(\id_E-u^{-1}(u-v)),
	\]
	且$ \norm{v-u}<\delta<1/\norm{u^{-1}} $时, 有
	\[
	\norm{u^{-1}(u-v)}\leqslant \norm{u^{-1}}\norm{u-v}<1,
	\]
	由(1)的结论可知$ \id_E-u^{-1}(u-v) $可逆, 从而$ v\in GL(E) $, 也即$ GL(E) $是开集.
	
	(3) 容易证明$ u\mapsto u^{-1} $是到自身的双射, 只需要证明它连续. 注意到当$ \norm{u}<1 $时, 有
	\[
	\norm{(\id_E-u)^{-1}}=\norm{\sum_{n\geqslant 0}u^n}\leqslant\sum_{n\geqslant 0}\norm{u}^n\leqslant(1-\norm{u})^{-1}.
	\]
	于是对使得$ \norm{u-v} $足够小的$ v\in Gl(E) $, 由
	\begin{align*}
	\norm{v^{-1}-u^{-1}}&=\norm{v^{-1}(u-v)u^{-1}}\\
	&\leqslant\norm{v^{-1}}\norm{u-v}\norm{u^{-1}}\\
	&=\norm{(\id_E-u^{-1}(u-v))^{-1}u^{-1}}\norm{u-v}\norm{u^{-1}}\\
	&\leqslant(1-\norm{u^{-1}(u-v)})^{-1}\norm{u-v}\norm{u^{-1}}^2\\
	&\leqslant(1-\norm{u^{-1}\norm{u-v}})^{-1}\norm{u-v}\norm{u^{-1}}^2\\
	&\leqslant 2\norm{u-v}\norm{u^{-1}}^2
	\end{align*}
	可知映射$ u\mapsto u^{-1} $是连续的. 同理可证其逆连续, 从而它是同胚.\qed
	\end{Proof}
	
	\textbf{补充题3.1$ ^* $(张恭庆1.4.2)}\ [习题课]\ \ 设$ C(0,1] $表示$ (0,1] $上连续有界的函数全体, 在$ C(0,1] $上定义$ \norm{x}=\sup\limits_{0<t\leqslant 1}\abs{x(t)} $, 证明:
	\begin{enumerate}[(1)]
	\item $ \norm{\cdot} $是$ C(0,1] $上的范数;
	\item $ \ell_\infty $与$ C(0,1] $的一个子空间等距同构.
	\end{enumerate}
	
	\begin{Proof}
	(1) 由
	\[
	\norm{x}=0\Longleftrightarrow\sup_{0<t\leqslant 1}\abs{x(t)}=0\Longleftrightarrow x=0
	\]
	可知正定性成立, 由
	\[
	\norm{\lambda x}=\sup_{0<t\leqslant 1}\abs{\lambda x(t)}=\abs{\lambda}\sup_{0<t\leqslant 1}\abs{x(t)}=\abs{\lambda}\norm{x}
	\]
	可知齐次性成立, 再由
	\[
	\norm{x+y}=\sup_{0<t\leqslant 1}\abs{x(t)+y(t)}\leqslant\sup_{0<t\leqslant 1}\abs{x(t)}+\sup_{0<t\leqslant 1}\abs{y(t)}=\norm{x}+\norm{y}
	\]
	可知三角不等式成立. 于是$ \norm{\cdot} $的确是$ C(0,1] $上的范数.
	
	(2) 考虑映射
	\[
	T : \ell_\infty\to C(0,1],\qquad (x_n)_{n\geqslant 1}\mapsto\sum_{n\geqslant 1}\left(\frac{x_{n+1}-x_n}{\frac{1}{n+1}-\frac{1}{n}}\left(t-\frac{1}{n}\right)+x_n\right)1_{\left( \frac{1}{n+1},\frac{1}{n} \right]},
	\]
	其中右侧的函数是以$ (1/n,x_n)_{n\geqslant 1} $为节点的分段线性函数. 容易证明$ T $是一个线性映射, 又因为
	\[
	f=0\equiv \forall n\geqslant 1, x_n=f\left(\frac{1}{n}\right)=0
	\]
	即$ \ker T=\{ (0)_{n\geqslant 1} \} $从而$ T $是单射, 且$ \norm{f}=\sup\limits_{n\geqslant 1}\abs{x_n} $, 于是$ T $是一个等距同构映射. 从而$ \ell_\infty $与$ C(0,1] $的一个子空间等距同构.\qed
	\end{Proof}
	
	\section{第7周习题课(2019年10月14日)}
	
	\textbf{习题3.3}\ [作业]\ \ 设$ (E,\norm{\cdot}_\infty) $是习题3.2中定义的赋范空间, 设$ E_0 $是$ E $中常数项为0的多项式构成的线性子空间(即$ p\in E_0\Longleftrightarrow p(0)=0 $).
	\begin{enumerate}[(1)]
	\item 证明$ N(p)=\norm{p'}_\infty $定义了$ E_0 $上的一个范数, 并且对任意$ p\in E_0 $, 有$ \norm{p}_\infty\leqslant N(p) $;
	\item 证明$ L(p)=\int_0^1\frac{p(x)}{x}\diff x $定义了$ E_0 $上关于$ N $的连续线性泛函, 并求出它的范数;
	\item 上面定义的$ L $是否关于$ \norm{\cdot}_\infty $连续?
	\item 范数$ \norm{\cdot}_\infty $和$ N $在$ E_0 $上是否等价?
	\end{enumerate}
	\begin{Proof}
	(1) 先证明$ N(p) $是$ E_0 $上的范数, 注意到
	\[
	N(p)=0\Longleftrightarrow p'=0\Longleftrightarrow p=c,
	\]
	其中$ c $是一个常数, 由$ p(0)=0 $可知$ c=0 $, 从而$ p=0 $, 正定性成立. 再由
	\[
	N(\lambda p)=\norm{\lambda p'}_\infty=\abs{\lambda}\cdot\norm{p'}_\infty=\abs{\lambda}N(p)
	\]
	与
	\[
	N(p+q)=\norm{p'+q'}_\infty\leqslant\norm{p'}_\infty+\norm{q'}_\infty\leqslant N(p)+N(q)
	\]
	可知齐次性与三角不等式也成立, 从而$ N(p) $是$ E_0 $上的范数.
	
	再设$ p\in E_0 $, 任取$ x\in[0,1] $, 由Lagrange中值定理可知
	\[
	\abs{\frac{p(x)-p(0)}{x-0}}=\abs{p'(\xi)}\leqslant\norm{p'}_\infty=N(p),
	\]
	也即$ \abs{p(x)}\leqslant \norm{p'}_\infty\abs{x}\leqslant \norm{p'}_\infty $这说明$ \norm{p}_\infty\leqslant\norm{p'}_\infty=N(p) $.
	
	(2) 由积分算子的线性性可知$ L(p) $是线性泛函, 且由Lagrange中值定理可知
	\[
	\abs{\int_0^1\frac{p(x)}{x}\diff x}\leqslant\sup_{0\leqslant x\leqslant 1}\abs{\frac{p(x)}{x}}\leqslant\sup_{0\leqslant x\leqslant 1}\abs{p'(x)}=N(p),
	\]
	从而$ L(p)\leqslant N(p) $, 这说明$ L $关于范数$ N $是连续的, 且有$ \norm{L}\leqslant 1 $. 再取$ p(x)=x\in E_0 $, 此时$ L(p)=1=N(p) $, 从而$ \norm{L}=1 $.
	
	(3) $ L $关于$ \norm{\cdot}_\infty $不连续. 用反证法证明, 假设$ L $关于$ \norm{\cdot}_\infty $连续, 设
	\[
	 F_0=\{ f\in C[0,1] : f(0)=0 \},
	\]
	容易验证$ F_0 $是一个Banach空间. 由Weierstrass逼近定理可知$ \forall f\in C[0,1] $, 可以使用多项式一致逼近, 从而$ E_0 $在$ F_0 $中稠密, 于是$ E_0 $上的连续线性泛函可以唯一地扩张成$ F_0 $上的连续线性泛函, 将扩张后的连续线性泛函记作$ \tilde{L} $, 那么取$ F_0 $中的元素$ f(x)=x^{1/n} $后注意到$ \norm{f}_\infty=1 $且$ \tilde{L}(f)=n $可知$ \tilde{L} $关于$ \norm{\cdot}_\infty $不连续, 矛盾.
	
	(4) 若两范数等价, 那么它们诱导出的拓扑应当是相同的, 而(3)已经说明了它们诱导的拓扑不相同, 于是两范数不等价.\qed
	\end{Proof}
	
	\textbf{习题3.8}\ [作业]\ \ 设$ E $是数域$ \K $上的有限维向量空间, 其维数$ \dim E=n $, $ \{ e_1,e_2,\cdots,e_n \} $是$ E $的一组基. 任取$ u\in\CL(E) $, 令$ [u] $表示$ u $在这组基下对应的矩阵.
	\begin{enumerate}[(1)]
	\item 证明映射$ u\mapsto[u] $建立了$ \CL(E) $到$ \mathbb{M}_n(\K) $之间的同构映射.
	\item 假设$ E=\K^n $且$ \{ e_1,e_2,\cdots,e_n \} $是经典基(即$ e_i=[\delta_{i,1}],\delta_{i,2},\cdots,\delta{i,n}] $, 这里$ \delta $是Kronecker符号), 并约定$ E=\K^n $上赋予Euclid范数. 证明: 若$ [u] $可被正交相似对角化, 则$ \norm{u}=\max\{ \abs{\lambda_1},\abs{\lambda_2},\cdots,\abs{\lambda_n} \} $, 这里$ \lambda_1,\lambda_2,\cdots,\lambda_n $是$ u $的特征值.
	\item 取$ \{ e_1,e_2,\cdots,e_n \} $如上, 试由$ [u] $中的元素分别确定$ p=1 $与$ p=\infty $时$ u : (\K^n,\norm{\cdot}_p)\to(\K^n,\norm{\cdot}_p) $的范数.
	\end{enumerate}
	\begin{Proof}
	(1) 由高等代数知识可知$ u\mapsto[u] $是一个线性的双射, 而任取$ \mathbb{M}_n(\K) $上的一个范数$ \norm{\cdot} $, 取
	\[
	\norm{u}=\norm{[u]}
	\]
	即使$ \CL(E) $上的一个范数, 这说明这一映射连续, 于是$ \CL(E) $与$ \mathbb{M}_n(\K) $同构.
	
	(2) 由$ u $可被正交相似对角化可知存在酉矩阵$ p $与对角阵$ \lambda=\mathrm{diag}\,\{ \lambda_1,\lambda_2,\cdots,\lambda_n \} $使得
	\[
	u=p^{-1}\lambda p.
	\]
	那么对任意单位向量$ x $, 记$ y=p^\dagger x $, 那么$ \norm{y}=\sqrt{x^\dagger pp^\dagger x}=\sqrt{x^\dagger x}=1 $. 从而
	\begin{align*}
	\norm{ux}=\sqrt{x^\dagger uu^\dagger x}=\sqrt{x^\dagger p\lambda^\dagger\lambda p^\dagger x}&=\sqrt{\sum_{k=1}^n\abs{\lambda_k}^2\abs{y_k}^2}\\
	&\leqslant\max_{k}\abs{\lambda_k}\sqrt{\abs{y_k}^2}=\max_{k}\abs{\lambda_k}.
	\end{align*}
	于是$ \norm{u}\leqslant\max\limits_{k}\abs{\lambda_k} $. 而取$ k_0 $是使得特征值模最大的下标, 并设$ \lambda_{k_0} $对应的特征向量是$ x_{k_0} $, 那么由
	\[
	\norm{ux_{k_0}}=\abs{\lambda_{k_0}}\norm{x_{k_0}}=\max_{k}\abs{\lambda_k}\norm{x_{k_0}},
	\]
	从而$ \norm{u}=\max\limits_{k}\abs{\lambda_k} $.
	
	(3) 当$ p=1 $时, 对$ \forall x\in\K^n $, 有$ x=\sum\limits_{k=1}^n x_ke_k $成立, 那么由定义有
	\begin{align*}
	\norm{ux}_1=\norm{\begin{bmatrix}
	\Sigma u_{1,k}x_k\\\vdots\\\Sigma u_{n,k}x_k
	\end{bmatrix}}=\sum_{i=1}^n\abs{\sum_{k=1}^nu_{i,k}x_k}&\leqslant\sum_{i=1}^n\sum_{k=1}^n\abs{u_{i,k}}\abs{x_k}\\
	&\leqslant\max_{k}\left(\sum_{i=1}^n\abs{u_{i,k}}\right)\sum_{k=1}^n\abs{x_k}=\max_{k}\left(\sum_{i=1}^n\abs{u_{i,k}}\right)\norm{x}_1,
	\end{align*}
	从而$ \norm{u}_1\leqslant\max\limits_{k}\left(\sum\limits_{i=1}^n\abs{u_{i,k}}\right) $. 再取$ k_0 $是使得右侧取最大的列指标, 那么
	\[
	\norm{ue_{k_0}}_1=\max_{k}\left(\sum_{i=1}^n\abs{u_{i,k}}\right)\norm{e_{k_0}}_1,
	\]
	从而有$ \norm{u}_1=\max\limits_{k}\left(\sum\limits_{i=1}^n\abs{u_{i,k}}\right) $.
	
	而当$ p=\infty $时, 有
	\begin{align*}
	\norm{ux}_\infty=\max_i\abs{\sum_{k=1}^nu_{i,k}x_k}&\leqslant\max_i\sum_{k=1}^n\abs{u_{i,k}}\abs{x_k}\\
	&\leqslant\max_i\max_k\abs{x_k}\sum_{k=1}^n\abs{u_{i,k}}=\max_i\sum_{k=1}^n\abs{u_{i,k}}\norm{x}_\infty.
	\end{align*}
	从而$ \norm{u}_\infty\leqslant\max\limits_i\sum\limits_{k=1}^n\abs{u_{i,k}} $. 再取$ i_0 $是使得右侧最大的行指标, 那么
	\[
	\norm{u(\mathrm{sgn}\,u_{i_0,k}e_{i_0})}_\infty=\max_i\sum_{k=1}^n\abs{u_{i,k}}\norm{\mathrm{sgn}\,u_{i_0,k}e_{i_0}}_\infty,
	\]
	故$ \norm{u}_\infty=\max\limits_i\sum\limits_{k=1}^n\abs{u_{i,k}} $.\qed
	\end{Proof}