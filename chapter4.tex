% !TeX root = main.tex

\chapter{紧算子}

\section{有限秩算子与紧算子}

	\begin{Definition}[有限秩算子, 紧算子]\index{Y!有限秩算子}\index{J!紧算子}
	设$ E, F $均为Banach空间, $ T : E\to F $是线性算子(在本章中以后直接称为算子).
	\begin{enumerate}[(1)]
	\item 若$ T : E\to F $连续, 且$ \dim T(E)<\infty $, 则称$ T $是\textbf{有限秩算子}, 其全体记作$ \Fr(E,F) $.
	\item 若$ T(B_E) $相对紧, 则称$ T $是\textbf{紧算子}, 其全体记作$ \CK(E,F) $.
	\end{enumerate}
	特别地, 若$ E=F $, 记$ \Fr(E)=\Fr(E,F) $, $ \CK(E)=\CK(E,F) $.
	\end{Definition}
	
	\begin{Example}~
	\begin{enumerate}[(1)]
	\item 若$ E $有限维, 则$ T : E\to F $是有限秩算子.
	\item 设$ H $是Hilbert空间, $ F\subset H $是有限维子空间, 且$ \{ \seq{f} \} $是$ F $的规范正交基, 令
	\[
	T : H\to F,\qquad x\mapsto\sum_{k=1}^n\lrangle{x,f_k}f_k
	\]
	则$ T $是有限秩的.
	\item 有限秩算子一定是紧算子, 这因$ \baro{T(B_E)}\subset\norm{T}\bar{B}_{T(E)} $是紧的.
	\item 设$ H $是Hilbert空间, $ (e_n)_{n\geqslant 1} $是$ H $的规范正交基, 令
	\[
	T : H\to H,\qquad x\mapsto\sum_{n\geqslant 1}\frac{1}{n}\lrangle{x,e_n}e_n
	\]
	则$ T $是紧的.
	\item 由(3)知$ \Fr(E,F)\subset\CK(E,F) $. 又因为相对紧集是有界集, 可知$ \CK(E,F)\subset\CB(E,F) $, 则
	\[
	\Fr(E,F)\subset\CK(E,F)\subset\CB(E,F).
	\]
	\end{enumerate}
	\end{Example}
	
	\begin{Definition}[理想]\index{L!理想}
	设$ \CB $是一个代数, $ \CA\subset\CB $, 若
	\[
	\forall x\in\CB\,\forall a\in\CA\,(ax\in\CA\land xa\in\CA)
	\]
	则称$ \CA $是$ \CB $的一个\textbf{理想}.
	\end{Definition}

	\begin{Theorem}
		$ \Fr(E, F) $ 与 $ \CK(E, F) $ 都是 $ \CB(E, F) $ 的线性子空间, 且 $ \Fr(E) $ 与 $ \CK(E) $ 都是 $ \CB(E) $ 的理想.
	\end{Theorem}
	\begin{Proof}
		(1) 任取 $ T_{1}, T_{2}\in\Fr(E, F) $, 则 $ \dim T_{1}(E)<\infty, \dim T_{2}(E)<\infty $, 由
		\[
			(T_{1} + T_{2})(E)\subset T_{1}(E)+T_{2}(E)
		\]
		可知 $ \dim(T_{1}+T_{2})(E)<\infty $, 故 $ T_{1}+T_{2}\in\Fr(E, F) $, 再任取 $ \lambda\in\K $, 由
		\[
			(\lambda T_{1})(E)=T_{1}(E)
		\]
		知 $ \dim(\lambda T_{1})(E)<\infty $, 故 $ \lambda T_{1}\in\Fr(E, F) $, 于是 $ \Fr(E, f) $ 是 $ \CB(E, F) $ 的线性子空间.
		
		再取 $ T_{1}, T_{2}\in \CK(E, F) $, 并定义
		\[
			\varphi: F\times F\to F\qquad (x, y)\mapsto x+y.
		\]
		则由 $ \norm{\varphi(x, y)}=\norm{x+y}\leqslant 2\max\{ \norm{x}, \norm{y} \} $ 知 $ \varphi $ 连续. 从而
		\[
			\varphi(\baro{T_{1}(B_{E})}\times \baro{T_{2}(B_{E}}))=\baro{T_{1}(B_{E})}+\baro{T_{2}(B_{E})}
		\]
		是紧集, 故 $ T_{1}(B_{E})+T_{2}(B_{E})\subset\baro{T_{1}(B_{E})}+\baro{T_{2}(B_{E})} $ 相对紧, 也即 $ T_{1}+T_{2}\in\CK(E, F) $. 又由 $ (\lambda T_{1})(B_{E})=\lambda T_{1}(B_{E}) $ 相对紧, 故 $ \lambda T_{1}\in\CK(E, F) $, 于是 $ \CK(E, F) $ 也是 $ \CB(E, F) $ 的线性子空间.

		(2) 任取 $ T\in\Fr(E), S\in\CB(E) $, 由
		\[
			(TS)(E) = T(S(E))\subset T(E)
		\]
		知 $ \dim(TS)(E)<\infty $, 即 $ TS\in\Fr(E) $. 再由
		\[
			(ST)(E)=S(T(E))
		\]
		与 $ S $ 的有界性知 $ \dim(ST)(E)<\infty $, 即 $ ST\in\Fr(E) $. 于是 $ \Fr(E) $ 是 $ \CB(E) $ 的理想.

		再任取 $ T\in\CK(E), S\in\CB(E) $. 则由
		\[
			(TS)(B_{E}) = T(S(B_{E}))\subset T(\norm{S}B_{E}) = \norm{S}T(B_{E})
		\]
		知 $ TS(B_{E}) $ 相对紧, 故 $ TS\in\CK(E) $. 再由
		\[
			ST(B_{E})=S(T(B_{E}))
		\] 
		与 $ S $ 的连续性知 $ (ST)(B_{E}) $ 相对紧. 故 $ ST\in\CK(E) $, 于是 $ \CK(E) $ 也是 $ \CB(E) $ 的理想.\qed
	\end{Proof}

	\begin{Theorem}
		设 $ T\in\CB(E, F) $, 则 $ T $ 紧的充分必要条件是 $ \Star{T} $ 紧.
	\end{Theorem}
	\begin{Proof}
		\textsl{必要性}. 设 $ T $ 是紧算子, 则 $ T(B_{E}) $ 相对紧. 需证 $ \Star{T}(B_{\Star{F}}) $ 相对紧. 只需证明对任意 $ (e_{n})_{n\geqslant1}\subset \Star{T}(B_{\Star{F}}) $ 都有收敛子列. 令 $ (f_{n})_{n\geqslant1}\subset B_{\Star{F}} $ 使得 $ \Star{T}(f_{n})=e_{n} $, 即 $ f_{n}\circ T=e_{n} $. 则由 $ \Star{T} $ 的连续性, 只需说明 $ (f_{n})_{n\geqslant1} $ 有收敛子列即可. 注意到 $ f_{n} $ 的定义域为 $ \baro{T(B_{E})} $, 这是一个紧集.

		使用 Arzel\`a-Ascoli 引理, 要说明 $ (f_{n})_{n\geqslant1} $ 等度连续, 且对任意 $ x\in\baro{T(B_{E})} $, 都有 $ (f_{n}(x))_{n\geqslant1} $ 相对紧. 由 $ (f_{n})_{n\geqslant1}\subset B_{\Star{F}} $ 知 $ (f_{n})_{n\geqslant1} $ 有界, 从而 $ (f_{n}(x))_{n\geqslant1} $ 有界. 于是 $ (f_{n}(x))_{n\geqslant1} $ 相对紧, 则 $ \forall\varepsilon>0, \forall x, y\in\baro{T(B_{E})} $, 若 $ \norm{x-y}<\varepsilon $, 则
		\[
			\abs{f_{n}(x)-f_{n}(y)} = \abs{f_{n}(x-y)}\leqslant\norm{x-y}<\varepsilon.
		\] 
		故 $ (f_{n})_{n\geqslant1} $ 等度连续, 从而 $ (f_{n})_{n\geqslant1} $ 相对紧, 则存在 Cauchy 子列,
		记作 $ (f_{n_{k}})_{k\geqslant1} $, 由
		\[
			\begin{aligned}
				\norm{e_{n_{k}}-e_{n_{j}}} & = \sup_{y\in B_{E}}\norm{\lrangle{e_{n_{k}}-e_{n_{j}}, y}} \\
				& = \sup_{y\in B_{E}}\norm{\lrangle{\Star{T}(f_{n_{k}}), y}-\lrangle{\Star{T}(f_{n_{j}}), y}}\\
				& = \sup_{y\in B_{E}}\norm{\lrangle{f_{n_{k}}, Ty}-\lrangle{f_{n_{j}}, Ty}}\\
				& =  \sup_{x\in T(B_{E})}\norm{\lrangle{f_{n_{k}}, x}-\lrangle{f_{n_{j}}, x}}\\
				& \leqslant \norm{f_{n_{k}}-f_{n_{j}}}_{\baro{T(B_{E})}}\to 0.
			\end{aligned}
		\]
		故 $ (e_{n_{k}})_{k\geqslant1} $ 也是 Cauchy 列. 由 $ \Star{E} $ 的完备性可知 $ (e_{n_{k}})_{k\geqslant1} $ 收敛, 故 $ \Star{T}(B_{\Star{F}}) $ 相对紧, 即 $ \Star{T} $ 是紧算子.

		\textsl{充分性}. 若 $ \Star{T} $ 是紧算子, 由必要性可知 $ T^{**} $ 是紧算子, 由 $ T = T^{**}|_{E} $ 可知 $ T(B_{E})\subset T^{**}(B_{E}) $ 相对紧, 故 $ T $ 是紧算子.\qed
	\end{Proof}

	\begin{Proposition}
		设 $ E $ 是 Banach 空间, 则 $ \baro{\Fr(E)}=\CB(E) $ 当且仅当 $ \dim E<\infty $.
	\end{Proposition}
	\begin{Proof}
		\textsl{充分性}. 因为 $ \dim E<\infty $, 故 $ \CB(E)=\Fr(E) $.

		\textsl{必要性}. 由 $ \baro{\Fr(E)}=\CB(E) $, 考虑恒同算子 $ \id_{E} $, 则存在 $ T\in\Fr(E) $ 使得 $ \norm{\id_{E}-T}<1 $, 由习题 3.9 结论知 $ \id_{E}-(\id_{E}-T)=T $ 可逆, 从而
		\[
			E = T^{-1}(T(E))
		\]
		是有限维的.\qed
	\end{Proof}

	\begin{Definition}[逼近性质]\index{B!逼近性质}\label{def:逼近性质}
		设 $ E $ 是 Banach 空间, 若 $ \baro{\Fr(E)}=\CK(E) $, 则称 $ E $ 具有\textbf{逼近性质}.
	\end{Definition}

	\begin{Theorem}
		Hilbert 空间具有逼近性质.
	\end{Theorem}
	\begin{Proof}
		只考虑 $ H $ 可分的情形, 即 $ H $ 的规范正交基是可数的, 记为 $ (e_{n})_{n\geqslant1} $. 设 $ P_{n} $ 是 $ H $ 到 $ \Span\{ \seq{e} \} $ 的投影, 即
		\[
			P_{n}: H\to\Span\{ \seq{e} \}\qquad x\mapsto \sum_{k=1}^{n}\lrangle{x, e_{k}}e_{k},
		\]
		则 $ \norm{P_{n}}=1 $. 由 $ \dim P_{n}(H)=n<\infty $ 可知 $ P_{n} $ 有限秩, 由 $ \Fr(H) $ 是 $ \CB(H) $ 的理想可知
		\[
			\forall T\in\CK(H)\,(TP_{n}\in\Fr(H)\land P_{n}T\in\Fr(H)).
		\]
		令 $ T_{n}=P_{n}T $, 需证明 $ (T_{n})_{n\geqslant1} $ 依范数收敛到 $ T $, 对 $ \forall\varepsilon>0 $, 由 $ \baro{T(B_{H})} $ 紧可知存在 $ \seq[m]{x} $ 使得
		\[
			\baro{T(B_{H})}\subset\bigcup_{j=1}^{m}B\left(x_{j}, \frac{\varepsilon}{2}\right),
		\]
		由此可知
		\[
			\norm{(\id -P_{n})(x_{j})}=\Big( \sum_{k\geqslant n+1} \abs{\lrangle{x_{j}, e_{k}}}^{2} \Big)^{1/2}.
		\]
		于是 $ \exists n_{0}\in\N $ 使得 $ n>n_{0} $ 时, 有 $ \forall j= 1, 2, \dots, m\,(\norm{(\id-P_{n})(x_{j})}<\varepsilon/2) $, 故对任意的 $ y\in\baro{T(B_{H})} $ 总成立
		\[
			\norm{(\id-P_{n})(y)}\leqslant\norm{(\id-P_{n})(y-x_{j})}+\norm{(\id-P_{n})(x_{j})}<\frac{\varepsilon}{2}+\frac{\varepsilon}{2}=\varepsilon
		\]
		从而
		\[
			\norm{T-T_{n}}=\sup_{x\in B_{H}}\norm{(T-T_{n})(x)}=\sup_{x\in B_{H}}\norm{(\id-P_{n})(Tx)}<\varepsilon.
		\]
		从而 $ (T_{n})_{n\geqslant1} $ 依范数收敛到 $ T $. 于是 $ \CK(H)\subset\baro{\Fr(H)} $.

		下面说明 $ \CK(H) $ 是闭集. 设 $ (T_{n})_{n\geqslant1} $ 是 $ \CK(H) $ 中的序列, 且收敛于 $ T\in\CB(H) $, 往证 $ T\in\CK(H) $. 取 $ (Tx_{n})_{n\geqslant1} $ 是 $ T(B_{E}) $ 中有收敛子列的序列, 由 $ T_{1} $ 紧性知存在子序列使得 $ \left(T_{1}x_{n_{(1, k)}}\right)_{k\geqslant1} $ 收敛. 再由 $ T_{2} $ 紧知存在子序列 $ \left(T_{2}x_{n_{(2, k)}}\right)_{k\geqslant1} $ 收敛, 依此进行下去, 由对角线方法可取序列 $ \left(x_{n_{(k, k)}}\right) $ 满足对 $ \forall j $, 有 $ \left(T_{j}x_{n_{(k, k)}}\right)_{k\geqslant1} $ 收敛, 则
		\[
			\begin{aligned}
				\norm{Tx_{n_{(k, k)}}-Tx_{n_{(k', k')}}} & \leqslant \norm{(T-T_{j})\left(x_{n_{(k, k)}}-x_{n_{(k', k')}}\right)+T_{j}\left(x_{n_{(k, k)}}-x_{x_{(k', k')}}\right)}\\
				& \leqslant 2\norm{T-T_{j}}+\norm{T_{j}\left(x_{n_{(k, k)}}-x_{n_{(k', k')}}\right)}\to 0.
			\end{aligned}
		\] 
		从而 $ \left( Tx_{n_{(k, k)}} \right)_{k\geqslant1} $ 是 Cauchy 列, 从而 $ T(B_{E}) $ 相对紧, 故 $ T $ 紧. 由 $ \Fr(H)\subset\CK(H) $ 可知 $ \baro{\Fr{H}}\subset \CK(H) $. 再由上证明, $ \CK(H)\subset\baro{\Fr(H)} $, 故 $ \baro{\Fr(H)}=\CK(H) $.\qed
	\end{Proof}

\section{紧算子的谱性质}
	\begin{Definition}[谱]\label{def:谱}
		设 $ T\in\CB(E) $ 
		\begin{enumerate}[(1)]
			\item 令集合 
			\[
				\sigma(T) = \{ \lambda\in\K: \lambda\id-T\ \text{不可逆} \}
			\]
			称 $ \sigma(T) $ 为 $ T $ 的\textbf{谱集}\index{P!谱集}, 并称 $ \rho(T)=\K\sm\sigma(T) $ 为 $ T $ 的\textbf{预解集}\index{Y!预解集}.
			\item 记 $ \lambda\id-T $ 为 $ \lambda -T $. 若 $ \lambda -T $ 不是单射, 则 $ \exists x\in E, x\ne0 $ 使得
			\[
				(\lambda-T)(x)=0.
			\]
			也即 $ \lambda x=T x $, 则称 $ \lambda $ 为 $ T $ 的\textbf{特征值}\index{T!特征值}, 称 $ \ker(\lambda -T) $ 为 $ T $ 关于 $ \lambda $ 的\textbf{特征子空间}\index{T!特征子空间}, 并称非零向量 $ x\in\ker(\lambda-T) $ 为 $ T $ 相应于 $ \lambda $ 的\textbf{特征向量}\index{T!特征向量}.
			\item 对任意 $ \lambda\in\rho(T) $, 称 $ R(\lambda, T)=(\lambda-T)^{-1} $ 为 $ T $ 的\textbf{预解式}\index{Y!预解式}.
		\end{enumerate}
	\end{Definition}

	\begin{Example}
		几个谱集的例子
		\begin{enumerate}[(1)]
			\item 设 $ T = \left[\begin{smallmatrix}
				1 & 0 \\ 0 & 2
			\end{smallmatrix}\right] $, 则 $ \sigma(T)=\{ 1, 2 \} $;
			\item 对 $ f\in C[0, 1] $, 定义 $ M_{f}g =fg $, 其中 $ g\in C[0, 1] $, 则 $ M_{f}\in\CB(C[0, 1]) $, 且 $ \sigma(M_{f})=\{ f(t):t\in [0, 1] \} $;
			\item 设 $ T $ 满足 $ Tx=\sum\limits_{n\geqslant1}\frac{1}{n}\lrangle{x, e_{n}}e_{n} $, 则 $ \sigma(T)=\{ 0, 1, \frac{1}{2}, \frac{1}{3}, \dots \} $.
		\end{enumerate}
	\end{Example}

	\begin{Proposition}
		$ \forall \lambda, \mu\in\rho(T) $, 有
		\[
			R(\lambda, T)-R(\mu, T)=(\mu-\lambda)R(\lambda, T)R(\mu, T) = (\mu-\lambda)R(\mu, T)R(\lambda, T),
		\]
		这被称为\textbf{预解方程}\index{Y!预解方程}.
	\end{Proposition}
	\begin{Proof}
		由
		\[
			\begin{aligned}
				(\mu-\lambda)R(\mu, T)R(\lambda, T) & = R(\lambda, T)(\mu-\lambda)R(\mu, T)\\
				& = (\lambda-T)^{-1}((\mu-T)-(\lambda-T))(\mu-T)^{-1}\\
				& = (\lambda-T)^{-1}-(\mu-T)^{-1}\\
				& = R(\lambda, T)-R(\mu, T)
			\end{aligned}
		\]
		即证.\qed
	\end{Proof}

	\begin{Theorem}[谱半径]\index{P!谱半径}\label{thm:谱半径}
		设 $ T\in\CB(E) $
		\begin{enumerate}[(1)]
			\item 极限 $ \lim\limits_{n\to\infty} \norm{T^{n}}^{1/n} $ 存在, 且
			\[
				\lim_{n\to\infty}\norm{T^{n}}^{1/n}=\inf_{n\geqslant1}\norm{T^{n}}^{1/n},
			\]
			并将其记作 $ r(T) $, 称为算子 $ T $ 的\textbf{谱半径}. 
			\item $ \sigma(T) $ 是 \K 中的紧集, 且 $ \sigma(T)\subset\{ \lambda\in\K:\abs{\lambda}\leqslant r(T) \} $. 
		\end{enumerate}
	\end{Theorem}
	\begin{Proof}
		(1) 易知$ \liminf\limits_{n\to\infty}\norm{T^n}^{1/n}\geqslant\inf\limits_{n\geqslant 1}\norm{T^n}^{1/n} $, 记右侧为$ a $, 往证$ \forall\varepsilon>0 $, 都有$ \limsup\limits_{n\to\infty}\norm{T^n}^{1/n}\leqslant a+\varepsilon $. 由下确界的定义, 有
		\[
		\forall\varepsilon>0\,\exists n_0\in\N\,(\norm{T^{n_0}}^{1/n_0}\leqslant a+\varepsilon)
		\]
		且由带余除法定理可知$ \forall n\in\N $, $ n\geqslant n_0 $, 都有
		\[
		n=q(n)\cdot n_0+r(n),
		\]
		其中$ q(n)\in\N,\ r(n)\in\N $且$ 0\leqslant r(n)<n_0 $. 于是
		\[
		\norm{T^n}^{1/n}=\norm{T^{q(n)\cdot n_0+r(n)}}^{1/n}=\left(\norm{T^{n_0}}^{1/n_0}\right)^{q(n)\cdot n_0/n}\cdot\norm{T}^{r(n)/n}
		\]
		在上式中令$ n\to\infty $, 注意到$ q(n)\cdot n_0/n\to 1 $而$ r(n)/n\to 0 $, 有
		\[
		\limsup_{n\to\infty}\norm{T^n}^{1/n}\leqslant\limsup_{n\to\infty}\left(\norm{T^{n_0}}^{1/n_0}\right)^{q(n)\cdot n_0/n}\cdot\norm{T}^{r(n)/n}\leqslant a+\varepsilon,
		\]
		令$ \varepsilon\to 0^+ $即可.
		
		(2) 反证法. 若存在$ \lambda\in\sigma(T) $使得$ \abs{\lambda}>r(T) $, 由$ r(T)=\lim\limits_{n\to\infty}\norm{T^n}^{1/n} $可知
		\[
		\exists 0<c<1\,\exists n_0\in\N\,(n\geqslant n_0\Rightarrow\norm{T^n}^{1/n}\leqslant c\abs{\lambda})
		\]
		从而$ \norm{\left(\frac{T}{\lambda}\right)^n}\leqslant c^n $, 这说明$ \sum\limits_{n\geqslant 0}\left(\frac{T}{\lambda}\right)^n $在$ \CB(E) $中收敛, 且$ (\lambda-T)^{-1}=\frac{1}{\lambda}\sum\limits_{n\geqslant 0}\left(\frac{T}{\lambda}\right)^n $, 矛盾. 从而$ \lambda\notin\sigma(T) $. 作映射
		\[
		f : \K\to\CB(E),\qquad \lambda\mapsto\lambda-T,
		\]
		易证$ f $连续, 再由习题3.9的结论可知$ GL(E) $是开集, 从而$ \rho(T)=f^{-1}(GL(E)) $是开集, 从而$ \sigma(T)=\K\sm\rho(T) $是闭集. 由上一结论可知$ \sigma(T) $是有界集, 从而$ \sigma(T) $是紧的.\qed
	\end{Proof}

	\begin{Theorem}[谱半径定理]\index{P!谱半径定理}\label{thm:谱半径定理}
		设 $ \K=\C, T\in\CB(E) $, 则 $ \sigma(T) $ 非空, 且
		\[
			r(T)=\sup_{\lambda\in\sigma(T)}\abs{\lambda}.
		\]
	\end{Theorem}
	\begin{Proof}
		\textsl{这是一个不正式的证明.}
		
		任取$ T\in\CB(E) $, 作映射
		\[
		R : \rho(T)\to\CB(E),\qquad \lambda\mapsto(\lambda-T)^{-1}
		\]
		由习题3.9的结论可知$ R $是连续的, 再取$ \xi\in\Star{\CB(E)} $, 定义
		\[
		\varphi : \rho(T)\to\C,\qquad \lambda\mapsto\xi(R(\lambda))
		\]
		那么$ \varphi $在$ \rho(T) $上连续. 任取$ \lambda_0\in\rho(T) $, 当$ \abs{\lambda-\lambda_0}<1/\norm{R(\lambda_0)} $时, 由
		\[
		\begin{aligned}
		(\lambda-T)^{-1}&=(\lambda-\lambda_0+\lambda_0-T)^{-1}\\
		&=R(\lambda_0)\sum_{n\geqslant 0}(-1)^nR(\lambda_0)^n(\lambda-\lambda_0)^n\\
		&=\sum_{n\geqslant 0}(-1)^nR(\lambda_0)^{n+1}(\lambda-\lambda_0)^n
		\end{aligned}
		\]
		可知$ \norm{(\lambda-\lambda_0)(\lambda_0-T)^{-1}}\leqslant 1 $, 从而上面的级数在$ \CB(E) $中绝对收敛, 继而收敛, 于是$ \varphi $在$ \rho(T) $上是全纯的.
		
		而当$ \abs{\lambda}\geqslant\norm{T} $时, 有$ R(\lambda)=(\lambda-T)^{-1}=\sum\limits_{n\geqslant 0}\frac{T^n}{\lambda^{n+1}} $, 它在$ \CB(E) $中绝对收敛, 从而$ \varphi(\lambda)=\sum\limits_{n\geqslant 0}\frac{\xi(T^n)}{\lambda^{n+1}} $, 于是
		\[
		\abs{\varphi(\lambda)}\leqslant\sum_{n\geqslant 0}\frac{\abs{\xi(T^n)}}{\lambda^{n+1}}\leqslant\sum_{n\geqslant 0}\frac{\norm{\xi}\norm{T}^n}{\lambda^{n+1}}=\frac{\norm{\xi}}{\lambda}\cdot\frac{1}{1-\norm{T}/\abs{\lambda}},
		\]
		令$ \lambda\to\infty $, 就有$ \varphi(\lambda)\to 0 $. 若$ \sigma(T)=\varnothing $, 那么$ \rho(T)=\C $, 此时由Liouville定理可知$ \varphi=0 $. 这意味着$ (\lambda-T)^{-1} $总是存在的, 从而对任意$ \xi\in\Star{\CB(E)} $, 都有$ \xi((\lambda-T)^{-1})=0 $. 但由Hahn-Banach定理可知存在$ \xi\in\Star{\CB(E)} $使得$ \xi((\lambda-T)^{-1})=0 $, 矛盾. 于是$ \sigma(T)\ne\varnothing $.
		
		由定理~\ref{thm:谱半径}~的(2)可知$ r(T)\geqslant\sup\limits_{\lambda\in\sigma(T)}\abs{\lambda} $, 将右侧记作$ \alpha $, 往证$ r(T)\leqslant\alpha $. 任取$ \lambda $使得$ \abs{\lambda}>\alpha $, 则$ \lambda\in\rho(T) $, 由$ \varphi(\lambda) $在$ \C\sm\bar{B}(0,\alpha) $全纯可知$ \varphi(\lambda) $绝对收敛. 记$ S_n=\frac{T^n}{\lambda^{n+1}} $后由一致有界原理可知$ (\norm{S_n})_{n\geqslant 1} $有界, 记$ M=\sup\limits_{n\geqslant 1}\norm{S_n} $, 那么
		\[
		\norm{T^n}^{1/n}\leqslant M^{1/n}\abs{\lambda}^{1+1/n},
		\]
		在上式中令$ n\to\infty $即可.\qed
	\end{Proof}

	\begin{Example}
		下面是一些算子和其谱半径的例子.
		\begin{enumerate}[(1)]
			\item 考虑对角矩阵 $ T=\diag\{ \seq{\lambda} \} $, 则 $ r(T)=\max\limits_{1\leqslant k\leqslant n}\abs{\lambda_{k}} $. 或使用定义, $ \forall m\geqslant1 $ 成立
			\[
				\norm{T^{m}}^{1/m}=\norm{\diag\{ \seq{\lambda^{m}} \}}^{1/m}=\max_{1\leqslant k\leqslant n}\abs{\lambda_{k}^{m}}^{1/m}=\max_{1\leqslant k\leqslant n}\abs{\lambda_{k}}.
			\]
			\item 考虑 Jordan 块
			\[
				T = J(0, 3)=\begin{bmatrix}
					0 & 1 & 0\\
					0 & 0 & 1\\
					0 & 0 & 0
				\end{bmatrix}
			\]
			则 $ r(T)=0 $, 此因 $ T^{3}=0 $. 进一步, 幂零矩阵的谱半径都是 0.
			\item 考虑算子 $ M_{f} $, 则
			\[
				r(M_{f}) = \lim_{n\to\infty}\norm{M_{f}^{n}}^{1/n}=\lim_{n\to\infty}\norm{M_{f^{n}}}^{1/n}=\lim_{n\to\infty}\norm{f^{n}}^{1/n}=\norm{f}.
			\]
			\item 考虑算子 $ T: x\mapsto\sum\limits_{n\geqslant1}\frac{1}{n}\lrangle{x, e_{n}}e_{n} $, 则 $ r(T)=1 $.
			\item 设 $ (e_{n})_{n\geqslant1} $ 是可分 Hilbert 空间 $ H $ 上的规范正交基, 考虑右移算子
			\[
				s: H\to H\qquad e_{n}\mapsto e_{n+1}\qquad (\forall n\geqslant1).
			\]
			由
			\[
				\Bnorm{s^{m}\sum_{n\geqslant1}\lambda_{n}e_{n}}=\Bnorm{\sum_{n\geqslant1}\lambda_{n}e_{n+m}}=\Big( \sum_{n\geqslant1}\abs{\lambda_{n}}^{2} \Big)^{1/2}
			\]
			知 $ \norm{s^{m}}\leqslant1 $. 又由 $ \norm{s^{m}e_{n}}=\norm{e_{n+m}} $ 知 $ \norm{s^{m}}=1 $, 故 $ r(s)=1 $. 
		\end{enumerate}
	\end{Example}

	\begin{Corollary}
		设 $ T\in\CB(E) $, 则 $ r(T)\leqslant\norm{T} $.
	\end{Corollary}
	\begin{Proof}
		由定义
		\[
			r(T)=\lim_{n\to\infty}\norm{T^{n}}^{1/n}\leqslant\lim_{n\to\infty}(\norm{T}^{n})^{1/n}=\norm{T}
		\]
		即证.\qed
	\end{Proof}
	
	\begin{Theorem}\label{thm:lambda-T的性质}
		设 $ T\in\CK(E), \lambda\in\K $ 且 $ \lambda\ne0 $, 有
		\begin{enumerate}[(1)]
			\item $ \forall n\in\N $, 有 $ \dim\ker(\lambda-T)^{n}<\infty $.
			\item $ \forall n\in\N $, $ (\lambda-T)^{n}(E) $ 是闭集.
			\item $ \exists n\in\N $, 使得 $ \ker(\lambda-T)^{n+1}=\ker(\lambda-T)^{n} $.
			\item $ \exists n\in\N $, 使得 $ (\lambda-T)^{n+1}(E)=(\lambda-T)^{n}(E) $.
		\end{enumerate}
	\end{Theorem}

	\begin{Proof}
		若 $ E $ 有限维, 命题显然成立, 不妨设 $ E $ 是无限维的.

		(1) 先考虑 $ n=1 $ 时的情形, 若 $ \ker(\lambda-T) $ 无限维, 记 $ K_{1}=\ker(\lambda-T) $ 后有 $ (\lambda-T)B_{K_{1}}=0 $, 即 $ \lambda B_{K_{1}}=T(B_{K_{1}}) $. 因为 $ T\in\CK(E) $, 故 $ T(B_{K_{1}}) $ 相对紧, 从而有 $ \lambda B_{K_{1}} $ 也相对紧, 而这说明 $ K_{1} $ 有限维, 矛盾.
		
		注意到 
		\[
			(\lambda-T)^{n}=\sum_{k=0}^{n}(-1)^{k}\binom{n}{k}\lambda^{n-k}T^{k} = \lambda^{n}-\sum_{k=1}^{n}(-1)^{k-1}\binom{n}{k}\lambda^{n-k}T^{k},
		\]
		故类似 $ n=1 $ 的情形可证.

		(2) 令 $ H_{n}= (\lambda-T)^{n}(E) $. 先考虑 $ n=1 $ 的情形, 作映射
		\[
			\tilde{T}: E/K_{1}\to H_{1}\qquad x+K_{1}\mapsto (\lambda-T)(x),
		\]
		只需说明 $ \tilde{T}^{-1} $ 有界即可. 用反证法, 假设 $ \tilde{T}^{-1} $ 无界, 则 $ \exists (x_{n}+K_{1})_{n\geqslant1} $ 使得 $ \norm{x_{n}+x_{1}}_{E/K_{1}}=1 $ 且 $ \tilde{T}(x_{n}+K_{1})\to 0 $. 由商模的定义可知 $ \forall n\geqslant1, \exists x_{n}' $ 使得  $ x_{n}' $ 是 $ x_{n}+K_{1} $ 的代表元: 满足 $ 1\leqslant\norm{x_{n}'}<2 $ 且 $ (\lambda-T)x_{n}'\to 0 $. 因为 $ T $ 紧, 故存在 $ (x_{n_{k}}')_{k\geqslant1}\subset(x_{n}')_{n\geqslant1} $ 使得 $ (Tx_{n_{k}}')_{k\leqslant1} $ 收敛, 记其极限为 $ z $, 则
		\[
			\lambda x_{n_{k}}'=Tx_{n_{k}}'+(\lambda-T)x-{n_{k}}'\to z+0=z
		\]
		即 $ x_{n_{k}}'\to z/\lambda $, 于是 $ (\lambda-T)x_{n_{k}}'\to (\lambda-T)z/\lambda=0 $, 由此可知 $ z\in\K_{1} $, 则
		\[
			\norm{x_{n_{k}}'+K_{1}}_{E/K_{1}}\leqslant\norm{x_{n_{k}}'-\frac{z}{\lambda}}\to 0,
		\]
		但这与 $ \norm{x_{n}+K_{1}}_{E/K_{1}}=1 $ 矛盾, 故 $ \tilde{T}^{-1} $ 有界, 从而 $ H_{1} $ 是闭集.

		下面使用归纳法, 设 $ \seq{H} $ 都是闭集, 则它们都是 Banach 空间. 因为
		\[
			H_{n+1}=(\lambda-T)^{n+1}(E)=(\lambda-T)(\lambda-T)^{n}(E)=(\lambda-T)(H_{n})
		\]
		则 $ \forall x\in E $, 由
		\[
			T(\lambda-T)^{n}(x)=(\lambda-T)^{n}(Tx)
		\]
		知 $ T(H_{n})\subset H_{n} $. 由 $ T|_{H_{n}} $ 紧知 $ \dim K_{n}<\infty $, 类似 $ n=1 $ 的情形可证 $ H_{n+1} $ 闭.

		(3) 注意到
		\[
			K_{1}\subset K_{2}\subset\cdots\subset K_{n}\subset\cdots
		\]
		假设所有包含关系都是严格的, 则 $ \exists x_{n}\in K_{n+1} $ 使得 $ \norm{x_{n}}=1 $ 且 $ d(x_{n}, K_{n})\geqslant1/2 $, 则对 $ m>n $, 有
		\[
			\begin{aligned}
				Tx_{m}-Tx_{n} & = Tx_{m}-\lambda x_{m}+\lambda x_{m}-\lambda x_{n}+\lambda x_{n}-Tx_{n}\\
				& = \lambda x_{m}+((T-\lambda)x_{m}-\lambda x_{n}+(\lambda-T)x_{n})
			\end{aligned}
		\]
		记 $ y=(T-\lambda)x_{m}-\lambda x_{n}+(\lambda-T)x_{n}\in K_{m} $, 则有
		\[
			\norm{Tx_{{m}}-Tx_{n}}=\abs{\lambda}\norm{x_{m}+\frac{y}{\lambda}}\geqslant\abs{\lambda}\cdot\frac{1}{2}
		\]
		即 $ (Tx_{n})_{n\geqslant1} $ 无收敛子列, 这与 $ T $ 是紧算子矛盾.

		(4) 因为 $ T $ 紧, 故 $ \Star{T} $ 紧. 由 (3) 知 $ \exists n\in\N $ 使
		\[
			\ker(\lambda-\Star{T})^{n+1}=\ker(\lambda-\Star{T})^{n}
		\]
		由双极定理, 有 $ \baro{(\lambda-T)^{n+1}(E)}=\baro{(\lambda-T)^{n}(E)} $, 再由 (2) 知 $ (\lambda-T)^{n}(E) $ 是闭集, 故命题得证.\qed
	\end{Proof}
	
	\begin{Remark}
		上一定理说明了紧算子具有某种意义上的有限性, 且其限制到闭集上几乎就是可逆的(模掉零空间后可逆).
	\end{Remark}

	\begin{Theorem}\label{thm:紧算子的点谱集性质}
		设 $ T\in\CB(E), \lambda\in\K $, 有
		\begin{enumerate}[(1)]
			\item 任取 $ \lambda\in\sigma(T)\sm \{ 0 \} $, 必有 $ \lambda\in\sigma_{p}(T) $, 其中 $ \sigma_{p}(T) $ 是 $ T $ 的所有特征值的集合, 称为 $ T $ 的\textbf{点谱集}.
			\item $ T $ 的非零特征值至多可数, 从而 $ \sigma_{p}(T) $ 至多可数, 特别地,  $ \sigma_{p}(T) $ 是紧集. 若将其所有的特征值排成一列 $ (\lambda_{n}) $, 当 $ (\lambda_{n}) $ 无限时, 必有 $ \lim\limits_{n\to\infty}\lambda_{n}=0 $.
		\end{enumerate}
	\end{Theorem}
	\begin{Proof}
		(1) 设 $ \lambda\in\sigma(T)\sm\{ 0 \} $, 若 $ \lambda-T $ 是单射, 则由定理~\ref{thm:lambda-T的性质}~知
		\[
			\exists n\in\N\,((\lambda-T)^{n+1}(E)=(\lambda-T)^{n}(E)),
		\]
		则
		\[
			\forall x\in E\exists y\in E\,((\lambda-T)^{n+1}(y)=(\lambda-T)^{n}(x)),
		\]
		由 $ \lambda-T $ 是单射可知 $ x=(\lambda-T)y $, 由 $ x $ 的任意性可知 $ \lambda-T $ 是满射, 由开映射定理知 $ \lambda-T $ 是同构, 即 $ \lambda-T $ 可逆. 这与 $ \lambda\in\sigma(T) $ 矛盾. 从而 $ \lambda-T $ 不是单射. 即 $ \ker(\lambda-T)\ne\{ 0 \} $. 这意味着 $ \exists x\ne 0 $ 使得 $ \lambda x=Tx $, 也即 $ \lambda\in\sigma_{p}(T)\sm\{ 0 \} $. 

		(2) 只需说明 $ \forall \eta>0 $, 只存在有限多个不同的 $ \lambda\in\sigma_{p}(T)\sm\{ 0 \} $ 使得 $ \lambda\leqslant\eta $ 即可.

		用反证法. 设 $ \exists\eta>0 $ 使得有无穷多个不同的 $ \lambda_{n}\in\sigma_{p}(T) $ 使得 $ \abs{\lambda_{n}}\geqslant\eta $, 则设 $ e_{n} $ 是对应的特征向量, 有 $ Te_{n}=\lambda_{n}e_{n} $, 并记 $ E_{n}=\Span\{ \seq{e} \} $.

		先证明 $ \seq{e} $ 线性无关. 否则存在不全为 0 的 $ \seq{a} $ 使得
		\[
			\begin{bmatrix}
				1 & 1 & \cdots & 1\\
				\lambda_{1} & \lambda_{2} & \cdots & \lambda_{n}\\
				\vdots & \vdots & \ddots & \vdots \\
				\lambda_{1}^{n} & \lambda_{2}^{n} & \cdots & \lambda_{n}^{n}
			\end{bmatrix}\begin{bmatrix}
				a_{1}e_{1}\\
				a_{2}e_{2}\\
				\vdots\\
				a_{n}e_{n}
			\end{bmatrix}=0
		\]
		注意到系数矩阵是 Vandermonde 矩阵, 记作 $ V_{n} $, 由 $ \seq{\lambda} $ 互不相同知 $ \det V_{n}\ne0 $, 即方程只有零解, 矛盾.

		再说明存在某序列 $ (Ty_{n})_{n\geqslant1} $ 没有收敛子列, 其中 $ y_{n}\in\bar{B}_{E} $. 注意到
		\[
			E_{1}\subset E_{2}\subset\cdots\subset E_{n}\subset\cdots
		\]
		且所有的包含关系都是严格的, 从而 $ \exists y_{n}\in E_{n+1} $ 使得 $ \norm{y_{n}}=1 $ 且 $ d(y_{n}, E_{n})\geqslant1/2 $, 对 $ m>n $ 有
		\[
			\norm{Ty_{m}-Ty_{n}}=\norm{-\lambda_{m+1}y_{m}+Ty_{m}-Ty_{n}+\lambda_{m+1}y_{m}}
		\]
		并注意到 $ -\lambda_{m+1}y_{m}+Ty_{m}-Ty_{n}\in E_{m} $, 从而有
		\[
			\norm{Ty_{m}-Ty_{n}}\geqslant\abs{\lambda_{m+1}}\cdot\frac{1}{2}\geqslant\frac{\eta}{2}
		\]
		矛盾.\qed
	\end{Proof}
	
\section{Hilbert空间上的自伴紧算子}

	\subsection{自伴算子与正算子}
	
	\begin{Definition}[自伴算子, 正算子]\index{Z!自伴算子}\index{Z!正算子}
	若$ T\in\CB(H) $且$ \Star{T}=T $, 则称$ T $是\textbf{自伴算子}. 若$ T $是自伴的且
	\[
	\forall x\in H\,(\lrangle{Tx,x}\geqslant 0)
	\]
	则称$ T $是\textbf{正算子}.
	\end{Definition}
	
	\begin{Example}
	考虑最简单的两种算子: 即$ n $阶复矩阵和乘法算子:
	\begin{enumerate}[(1)]
	\item 设$ A\in\mathbb{M}_n(\C) $, 则$ A $是自伴的当且仅当$ A $是Hermite的, 也即$ A^\dagger=A $; $ A $是正的当且仅当$ A $是半正定的.
	\item 设$ f\in C[0,1] $, $ M_f\in\CB(L_2(0,1)) $, 那么$ M_f $是自伴的当且仅当$ \bar{f}=f $; $ M_f $是正的当且晋档$ f\geqslant 0 $.
	\end{enumerate}
	\end{Example}
	
	\begin{Remark}
	(1) 若$ T $是自伴的, 则由
	\[
	\baro{\lrangle{Tx,x}}=\lrangle{x,Tx}=\lrangle{\Star{T}x,x}=\lrangle{Tx,x}
	\]
	可知$ \lrangle{Tx,x}\in\R $, 但反之未必成立.
	
	(2) 若$ T $是正算子, 则映射$ \lrangle{x,y}\mapsto\lrangle{Tx,y} $具有除正定性以外的所有内积的性质, 故有Cauchy-Schwarz不等式成立:
	\[
	\abs{\lrangle{Tx,y}}^2\leqslant\lrangle{Tx,x}\lrangle{Ty,y},\qquad\forall x,y\in H
	\]
	从而
	\[
	\norm{T}=\sup_{\norm{x}=1}\norm{Tx}=\sup_{\norm{x}=\norm{y}=1}\abs{\lrangle{Tx,x}}^{1/2}\abs{\lrangle{Ty,y}}^{1/2}=\sup_{\norm{x}=1}\abs{\lrangle{Tx,x}},
	\]
	故$ T $是正算子时, 有$ \norm{T}=\sup\limits_{\norm{x}=1}\abs{\lrangle{Tx,x}} $.
	\end{Remark}
	
	\begin{Theorem}\label{thm:自伴算子的谱性质}
	设$ T $是$ H $上的自伴算子, 则:
	\begin{enumerate}[(1)]
	\item $ r(T)=\norm{T}=\sup\limits_{\norm{x}=1}\abs{\lrangle{Tx,x}} $;
	\item 令$ m=\inf\limits_{\norm{x}=1}\lrangle{Tx,x} $, $ M=\sup\limits_{\norm{x}=1}\lrangle{Tx,x} $, 那么$ \sigma(T)\subset[m,M] $, 且$ m\in\sigma(T) $, $ M\in\sigma(T) $, 并且有
	\[
	r(T)=\norm{T}=\max\{\abs{m},\abs{M}\}=\max_{\lambda\in\sigma_p(T)}\abs{\lambda}.
	\]
	\end{enumerate}
	\end{Theorem}
	\begin{Proof}
	(1) 先证$ r(T)=\norm{T} $. 由$ T $自伴可知$ \Star{T}=T $, 即$ \forall x\in H,\norm{x}=1 $都有
	\[
	\norm{Tx}^2=\lrangle{Tx,Tx}=\lrangle{x,\Star{T}Tx}\leqslant\norm{T^2}\norm{x}^2,
	\]
	故$ \norm{T^2}\leqslant\norm{T}^2\leqslant\norm{T^2} $, 也即$ \norm{T^2}=\norm{T}^2 $成立. 重复以上过程, 有$ \norm{T}^{2^n}=\tnorm{T^{2^n}} $, 故
	\[
	r(T)=\lim_{n\to\infty}\norm{T^n}^{1/n}=\lim_{n\to \infty}\tnorm{T^{2^n}}^{2^{-n}}=\norm{T}.
	\]
	
	下面说明$ \sup\limits_{\norm{x}=1}\abs{\lrangle{Tx,x}}=\norm{T} $, 记左侧为$ \alpha $, 则$ \alpha\leqslant\norm{T} $是显然的. 往证反向不等式成立, 则$ \forall x\in H $, 有
	\[
	\abs{\lrangle{Tx,x}}=\norm{x}^2\abs{\lrangle{T\sgn x,\sgn x}}\leqslant\alpha\norm{x}^2,
	\]
	在上式中将$ x $替换为$ x+\lambda y $和$ x-\lambda y $, 有
	\[
	\begin{aligned}
	\abs{\lrangle{T(x+\lambda y),x+\lambda y}}&\leqslant\alpha\norm{x+\lambda y}^2\\
	\abs{\lrangle{T(x-\lambda y),x-\lambda y}}&\leqslant\alpha\norm{x-\lambda y}^2
	\end{aligned}
	\]
	由平行四边形法则
	\[
	\abs{\lrangle{T(x+\lambda y),x+\lambda y}-\lrangle{T(x-\lambda y),x-\lambda y}}=4\abs{\Re\bar{\lambda}\lrangle{Tx,y}}
	\]
	另一方面
	\[
	\begin{aligned}
	\abs{\lrangle{T(x+\lambda y),x+\lambda y}-\lrangle{T(x-\lambda y),x-\lambda y}}&\leqslant\alpha(\norm{x+\lambda y}^2+\norm{x-\lambda y}^2)\\
	&=2\alpha(\norm{x}^2+\abs{\lambda}^2\norm{y}^2)
	\end{aligned}
	\]
	故
	\[
	2\abs{\Re\bar{\lambda}\lrangle{Tx,y}}\leqslant\alpha(\norm{x}^2+\abs{\lambda}^2\norm{y}^2).
	\]
	再由$ \lambda $的任意性, 取$ \lambda=\sgn\baro{\lrangle{Tx,y}} $, 有
	\[
	\abs{\lrangle{Tx,y}}\leqslant\frac{\alpha}{2}(\norm{x}^2+\norm{y}^2)
	\]
	再令$ y=x $, 就有$ \abs{\lrangle{Tx,x}}\leqslant\alpha $, 故$ \abs{\lrangle{Tx,x}}=\alpha $.
	
	(2) 设$ \lambda\in\K $, 且$ \lambda\notin[m,M] $, 记$ d(\lambda)=d(\lambda,[m,M])>0 $. 对$ \norm{x}=1 $, 有
	\[
	\abs{\lrangle{(\lambda-T)x,x}}=\abs{\lambda-\lrangle{Tx,x}}\geqslant d(\lambda)>0,
	\]
	故
	\[
	d(\lambda)\norm{x}^2\leqslant\abs{\lrangle{(\lambda-T)x,x}}\leqslant\norm{(\lambda-T)x}\norm{x},
	\]
	即$ d(\lambda)\norm{x}\leqslant\norm{(\lambda-T)x} $, 令$ (\lambda-T)x=0 $知$ x=0 $, 再取$ ((\lambda-T)x_n)_{n\geqslant 1} $是Cauchy列, 可知$ (x_n)_{n\geqslant 1} $也是Cauchy列, 从而$ (\lambda-T)(H) $是闭的. 故由
	\[
	(\lambda-T)(H)=\baro{(\lambda-T)(H)}=\ker (\bar{\lambda}-\Star{T})^\bot
	\]
	与$ \bar{\lambda}-\Star{T}=\bar{\lambda}-T $可知$ \lambda\notin[m,M]\Rightarrow\bar{\lambda}\notin[m,M] $, 也可验证$ \bar{\lambda}-T $是单射, 故$ \ker(\bar{\lambda}-T)=\{0\} $, 于是$ (\lambda-T)(H) $在$ H $中稠密, 即$ (\lambda-T)(H)=H $, 故$ \lambda\in\rho(T) $. 这说明$ \sigma(T)\subset[m,M] $.
	
	再证明$ m\in\sigma(T) $, 由$ m $的定义可知存在$ (x_n)_{n\geqslant}\subset H $且$ \norm{x_n}=1 $使得$ m=\lim\limits_{n\to\infty}\lrangle{Tx_n,x_n} $, 则有
	\[
	\lim_{n\to \infty}\lrangle{(T-m)x_n,x_n}=\lim_{n\to\infty}\lrangle{Tx_n,x_n}-m=0,
	\]
	由$ m $的定义可知$ T-m $是正的, 由Cauchy-Schwarz不等式可知
	\[
	\abs{\lrangle{(T-m)x_n,y}}\leqslant\lrangle{(T-m)x_n,x_n}^{1/2}\lrangle{(T-m)y,y}^{1/2}\leqslant\lrangle{(T-m)x_n,x_n}^{1/2}\norm{T-m}^{1/2}\norm{y},
	\]
	令$ y=(T-m)x_n $, 则有
	\[
	\norm{(T-m)x_n}^2\leqslant\lrangle{(T-m)x_n,x_n}^{1/2}\norm{T-m}^{1/2}\norm{(T-m)x_n}
	\]
	即
	\[
	\norm{(T-m)x_n}\leqslant\lrangle{(T-m)x_n,x_n}^{1/2}\norm{T-m}^{1/2}\to 0,
	\]
	于是$ (T-m)x_n\to 0 $, 这说明$ T-m $不可逆, 于是$ m\in\sigma(T) $. 类似可证$ M-T $是正算子, 不可逆, 从而$ M\in\sigma(T) $. 由谱半径定理可知$ r(T)=\max\{ \abs{m},\abs{M} \} $.\qed
	\end{Proof}
	
	\begin{Corollary}
	设$ T $是自伴算子, 则$ T $是正算子当且仅当$ \sigma(T)\subset[0,\infty) $, 且若$ T $是正算子, 则$ \norm{T}\in\sigma(T) $.
	\end{Corollary}
	\begin{Proof}
	若$ T $是正算子, 由$ \forall x\in H\,(\lrangle{Tx,x}\geqslant 0) $可知$ m\geqslant 0,\ M\geqslant 0 $, 从而
	\[
	\sigma(T)\subset[m,M]\subset[0,\infty).
	\]
	反之, 若$ \sigma(T)\subset[0,\infty) $, 那么$ m\geqslant 0,\ M\geqslant 0 $, 故
	\[
	\forall x\in H(\lrangle{Tx,x}\geqslant 0)
	\]
	即$ T $是正的.
	
	若$ T $是正算子, 由$ M=r(T)=\norm{T} $可知$ M\in\sigma(T) $.\qed
	\end{Proof}
	
	\begin{Corollary}
	设$ T $是$ H $上的自伴紧算子, 则存在$ T $的特征值$ \lambda $使得$ \abs{\lambda}=\norm{T} $.
	\end{Corollary}
	\begin{Proof}
	由定理~\ref{thm:自伴算子的谱性质}~的(2)可知$ \norm{T}=\max\limits_{\lambda\in\sigma(T)}\abs{\lambda} $, 再由定理~\ref{thm:紧算子的点谱集性质}~可知$ \forall\lambda\in\sigma(T)\sm\{0\} $, 都有$ \lambda\in\sigma_p(T) $, 故命题得证.\qed
	\end{Proof}
	
	\subsection{自伴紧算子的谱分解}
	
	\begin{Definition}[Hilbert空间的直和]\index{Z!直和}
	设$ \{H_i\}_{i\in\alpha} $是一组Hilbert空间, 令$ \bigoplus\limits_{i\in\alpha}H_i $是$ \prod\limits_{i\in\alpha}H_i $的子集, 其中$ (x_i)_{i\in\alpha}\in\bigoplus\limits_{i\in\alpha}H_i $使得$ \sum\limits_{i\in\alpha}\norm{x_i}^2<\infty $, 即
	\[
	\bigoplus_{i\in\alpha}H_i=\left\{ (x_i)_{i\in\alpha} : \sum_{i\in\alpha}\norm{x_i}^2<\infty \right\}
	\]
	定义其上的加法和数乘
	\[
	(x_i)_{i\in\alpha}+(y_i)_{i\in\alpha}:=(x_i+y_i)_{i\in\alpha},\qquad\lambda(x_i)_{i\in\alpha}:=(\lambda x_i)_{i\in\alpha},
	\]
	并赋予范数
	\[
	\norm{(x_i)_{i\in\alpha}}=\Big( \sum_{i\in\alpha}\norm{x_i}^2 \Big)^{1/2},
	\]
	它可以被以下的内积诱导
	\[
	\lrangle{(x_i)_{i\in\alpha},(y_i)_{i\in\alpha}}=\sum_{i\in\alpha}\lrangle{x_i,y_i},
	\]
	则由各$ H_i $是Hilbert空间可知$ \bigoplus\limits_{i\in\alpha}H_i $也是Hilbert空间, 并称其为$ \{ H_i \}_{i\in\alpha} $的\textbf{直和}.
	\end{Definition}
	
	\begin{Theorem}[自伴紧算子的谱分解]\label{thm:自伴紧算子的谱分解}
	设$ T $是$ H $上的自伴紧算子, $ V_\lambda $表示对应于$ \lambda $的特征子空间, 则
	\begin{enumerate}[(1)]
	\item $ H=\bigoplus\limits_{\lambda\in\sigma_p(T)}V_\lambda $, 即$ H $有由$ T $的特征向量构成的正交基.
	\item 空间$ \baro{T(H)} $有一个由特征向量$ (e_n)_{n\geqslant 1} $构成的正交基, 此处$ (e_n)_{n\geqslant 1} $是分别对应于特征值$ (\lambda_n)_{n\geqslant 1} $的特征向量(这里序列$ (\lambda_n)_{n\geqslant 1} $也可以是有限的)使得
	\[
	\forall x\in H\,\Big(Tx=\sum_{n\geqslant 1}\lambda_n\lrangle{x,e_n}e_n\Big)
	\]
	在范数意义下收敛, 且若$ (\lambda_n)_{n\geqslant 1} $无限, 则成立$ \lim\limits_{n\to\infty}\lambda_n=0 $.
	\end{enumerate}
	\end{Theorem}
	\begin{Proof}
	(1) 由$ T $是紧算子可知$ \sigma(T) $之多可数, 记作
	\[
	\sigma(T)=\{ \seq{\lambda},\dots \}
	\]
	且不妨$ \abs{\lambda_1}\geqslant\abs{\lambda_2}\geqslant\cdots\geqslant\abs{\lambda_n}\geqslant\cdots $, 其中序列中的特征值可以重复, 重复次数即为其代数重数. 设$ \lambda $和$ \mu $是$ T $的不同特征值, 则存在非零的$ x,y $使得$ Tx=\lambda x $且$ Ty=\mu y $. 不妨设$ \lambda\ne 0 $, 则
	\[
	\lambda\lrangle{x,y}=\lrangle{Tx,y}=\lrangle{x,Ty}=\mu\lrangle{x,y}
	\]
	可知$ (\lambda-\mu)\lrangle{x,y}=0 $, 由$ \lambda=\mu\ne0 $可知只能$ \lrangle{x,y}=0 $, 故$ \bigoplus\limits_{\lambda\in\sigma_p(T)}V_\lambda\subset H $.
	
	再令$ \tilde{V}=\Big(\bigoplus\limits_{\lambda\in\sigma_p(T)}V_\lambda\Big)^\bot $, 则需证明$ \forall z\in\tilde{V} $都有$ Tz\in\tilde{V} $. 若$ \lambda\ne 0 $, 则有
	\[
	\forall y\in V_\lambda\,(\lambda\lrangle{z,y}=0)\Rightarrow\lrangle{z,Ty}=0\Rightarrow\lrangle{Tz,y}=0\Rightarrow T_z\bot V_\lambda.
	\]
	若$ \lambda=0 $, 则
	\[
	\forall y\in V_0\,(\lrangle{Tz,y}=\lrangle{z,Ty}=0)\Rightarrow Tz\bot V_0,
	\]
	从而$ Tz\in\tilde{V} $. 故$ T|_{\tilde V} $有意义且为紧算子, 从而$ \sigma(T|_{\tilde V}) $非空, 若有$ \lambda\in\sigma(T|_{\tilde V}) $非零, 则存在非零向量$ x\in\tilde{V} $使得$ Tx=\lambda x $, 故$ \lambda\in\sigma_p(T) $, 这与$ \tilde{V} $的定义矛盾. 故只能$ \lambda=0 $, 即$ \sigma(T|_{\tilde{V}})=\{0\} $, 由谱半径定理可知$ \norm{T}=r(T)=0 $, 从而$ T|_{\tilde V}=0 $, 即$ \tilde{V}=\{0\} $, 也即
	\[
	\bigoplus_{\lambda\in\sigma_p(T)}V_\lambda=H.
	\]
	
	(2) 因$ \baro{T(H)}=\bigoplus\limits_{\lambda\in\sigma_p(T)\sm\{0\}}V_\lambda $, 于是有一个由特征向量$ (e_n)_{n\geqslant 1} $构成的正交基, 此时特征向量$ (e_n)_{n\geqslant 1} $对应于特征值$ (\lambda_n)_{n\geqslant 1} $, 使得
	\[
	\forall x\in H\,\bigg(Tx=\sum_{n\geqslant 1}\lambda_n\lrangle{x,e_n}e_n\bigg)
	\]
	并且若$ (\lambda_n)_{n\geqslant 1} $无限, 有$ \lim\limits_{n\to\infty}\lambda_n=0 $.\qed
	\end{Proof}
	
\section*{本章习题}
	\addcontentsline{toc}{section}{本章习题}
	
	习题后面括号中的序号表示对应书中习题的编号.
	
	\begin{enumerate}[label=\textbf{\arabic*.}, ref=\arabic*]
	\item (11.1) 设 $ E $ 是 Banach 空间, $ T\in\CB(E) $. 并设 $ (\lambda_{n})_{n\geqslant1} $ 是 $ \rho(T) $ 中收敛到 $ \lambda\in\K $ 的数列. 证明: 若 $ (R(\lambda_{n}, T)) $ 在 $ \CB(E) $ 中有界, 则 $ \lambda\in\rho(T) $.
	\item (11.3) 设 $ 1\leqslant p\leqslant\infty $ 定义 $ \ell_{p} $ 上的算子 $ S $ (前移算子) 为 $ S(x)(n)=x(n+1) $, 这里 $ x=(x(n))_{n\geqslant1}\in\ell_{p} $.
		\begin{enumerate}[(1)]
			\item 证明: 当 $ p<\infty $ 时, $ \sigma_{p}(S)=\{ \lambda\in\K:\abs{\lambda}<1 \} $; 当 $ p=\infty $ 时, $ \sigma_{p}(S)=\{ \lambda\in\K:\abs{\lambda}\leqslant1 \} $.
			\item 由此导出 $ \sigma(S)=\{ \lambda\in\K:\abs{\lambda}\leqslant1 \} $.
		\end{enumerate}
	\item (11.8) 设 $ E $ 和 $ F $ 是赋范空间, 证明下面的命题成立:
		\begin{enumerate}[(1)]
			\item 若 $ (x_{n})_{n\geqslant1} $ 是 $ E $ 中弱收敛的序列, 则 $ (x_{n})_{n\geqslant1} $ 有界.
			\item 若 $ T\in\CB(E, F) $ 且 $ x_{n} $ 弱收敛到 $ x $, 则 $ T(x_{n}) $ 弱收敛到 $ T(x) $.
			\item 若 $ T\in\CB(E, F) $ 是紧算子且 $ x_{n} $ 弱收敛到 $ x $, 则 $ T(x_{n}) $ 依范数收敛到 $ T(x) $.
			\item 若$ E $自反, $ T\in\CB(E,F) $且当$ x_n $弱收敛到$ x $时, 有$ T(x_n) $依范数收敛到$ T(x) $, 则$ T $是紧算子. 
			\item 若 $ E $ 自反, $ T\in\CB(E, \ell_{1}) $ 或 $ T\in\CB(c_{0}, E) $, 则 $ T $ 是紧算子.
		\end{enumerate}
	\end{enumerate}