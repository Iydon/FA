% !TeX root = main.tex

\chapter{紧算子}

\section{有限秩算子与紧算子}

	\begin{Definition}[有限秩算子, 紧算子]\index{Y!有限秩算子}\index{J!紧算子}
	设$ E, F $均为Banach空间, $ T : E\to F $是线性算子(在本章中以后直接称为算子).
	\begin{enumerate}[(1)]
	\item 若$ T : E\to F $连续, 且$ \dim T(E)<\infty $, 则称$ T $是\textbf{有限秩算子}, 其全体记作$ \Fr(E,F) $.
	\item 若$ T(B_E) $相对紧, 则称$ T $是\textbf{紧算子}, 其全体记作$ \CK(E,F) $.
	\end{enumerate}
	特别地, 若$ E=F $, 记$ \Fr(E)=\Fr(E,F) $, $ \CK(E)=\CK(E,F) $.
	\end{Definition}
	
	\begin{Example}~
	\begin{enumerate}[(1)]
	\item 若$ E $有限维, 则$ T : E\to F $是有限秩算子.
	\item 设$ H $是Hilbert空间, $ F\subset H $是有限维子空间, 且$ \{ \seq{f} \} $是$ F $的规范正交基, 令
	\[
	T : H\to F,\qquad x\mapsto\sum_{k=1}^n\lrangle{x,f_k}f_k
	\]
	则$ T $是有限秩的.
	\item 有限秩算子一定是紧算子, 这因$ \baro{T(B_E)}\subset\norm{T}\bar{B}_{T(E)} $是紧的.
	\item 设$ H $是Hilbert空间, $ (e_n)_{n\geqslant 1} $是$ H $的规范正交基, 令
	\[
	T : H\to H,\qquad x\mapsto\sum_{n\geqslant 1}\frac{1}{n}\lrangle{x,e_n}e_n
	\]
	则$ T $是紧的.
	\item 由(3)知$ \Fr(E,F)\subset\CK(E,F) $. 又因为相对紧集是有界集, 可知$ \CK(E,F)\subset\CB(E,F) $, 则
	\[
	\Fr(E,F)\subset\CK(E,F)\subset\CB(E,F).
	\]
	\end{enumerate}
	\end{Example}
	
	\begin{Definition}[理想]\index{L!理想}
	设$ \CB $是一个代数, $ \CA\subset\CB $, 若
	\[
	\forall x\in\CB\,\forall a\in\CA\,(ax\in\CA\land xa\in\CA)
	\]
	则称$ \CA $是$ \CB $的一个\textbf{理想}.
	\end{Definition}

	\begin{Theorem}
		$ \Fr(E, F) $ 与 $ \CK(E, F) $ 都是 $ \CB(E, F) $ 的线性子空间, 且 $ \Fr(E) $ 与 $ \CK(E) $ 都是 $ \CB(E) $ 的理想.
	\end{Theorem}
	\begin{Proof}
		(1) 任取 $ T_{1}, T_{2}\in\Fr(E, F) $, 则 $ \dim T_{1}(E)<\infty, \dim T_{2}(E)<\infty $, 由
		\[
			(T_{1} + T_{2})(E)\subset T_{1}(E)+T_{2}(E)
		\]
		可知 $ \dim(T_{1}+T_{2})(E)<\infty $, 故 $ T_{1}+T_{2}\in\Fr(E, F) $, 再任取 $ \lambda\in\K $, 由
		\[
			(\lambda T_{1})(E)=T_{1}(E)
		\]
		知 $ \dim(\lambda T_{1})(E)<\infty $, 故 $ \lambda T_{1}\in\Fr(E, F) $, 于是 $ \Fr(E, f) $ 是 $ \CB(E, F) $ 的线性子空间.
		
		再取 $ T_{1}, T_{2}\in \CK(E, F) $, 并定义
		\[
			\varphi: F\times F\to F\qquad (x, y)\mapsto x+y.
		\]
		则由 $ \norm{\varphi(x, y)}=\norm{x+y}\leqslant 2\max\{ \norm{x}, \norm{y} \} $ 知 $ \varphi $ 连续. 从而
		\[
			\varphi(\baro{T_{1}(B_{E})}\times \baro{T_{2}(B_{E}}))=\baro{T_{1}(B_{E})}+\baro{T_{2}(B_{E})}
		\]
		是紧集, 故 $ T_{1}(B_{E})+T_{2}(B_{E})\subset\baro{T_{1}(B_{E})}+\baro{T_{2}(B_{E})} $ 相对紧, 也即 $ T_{1}+T_{2}\in\CK(E, F) $. 又由 $ (\lambda T_{1})(B_{E})=\lambda T_{1}(B_{E}) $ 相对紧, 故 $ \lambda T_{1}\in\CK(E, F) $, 于是 $ \CK(E, F) $ 也是 $ \CB(E, F) $ 的线性子空间.

		(2) 任取 $ T\in\Fr(E), S\in\CB(E) $, 由
		\[
			(TS)(E) = T(S(E))\subset T(E)
		\]
		知 $ \dim(TS)(E)<\infty $, 即 $ TS\in\Fr(E) $. 再由
		\[
			(ST)(E)=S(T(E))
		\]
		与 $ S $ 的有界性知 $ \dim(ST)(E)<\infty $, 即 $ ST\in\Fr(E) $. 于是 $ \Fr(E) $ 是 $ \CB(E) $ 的理想.

		再任取 $ T\in\CK(E), S\in\CB(E) $. 则由
		\[
			(TS)(B_{E}) = T(S(B_{E}))\subset T(\norm{S}B_{E}) = \norm{S}T(B_{E})
		\]
		知 $ TS(B_{E}) $ 相对紧, 故 $ TS\in\CK(E) $. 再由
		\[
			ST(B_{E})=S(T(B_{E}))
		\] 
		与 $ S $ 的连续性知 $ (ST)(B_{E}) $ 相对紧. 故 $ ST\in\CK(E) $, 于是 $ \CK(E) $ 也是 $ \CB(E) $ 的理想.\qed
	\end{Proof}

	\begin{Theorem}
		设 $ T\in\CB(E, F) $, 则 $ T $ 紧的充分必要条件是 $ \Star{T} $ 紧.
	\end{Theorem}
	\begin{Proof}
		\textsl{必要性}. 设 $ T $ 是紧算子, 则 $ T(B_{E}) $ 相对紧. 需证 $ \Star{T}(B_{\Star{F}}) $ 相对紧. 只需证明对任意 $ (e_{n})_{n\geqslant1}\subset \Star{T}(B_{\Star{F}}) $ 都有收敛子列. 令 $ (f_{n})_{n\geqslant1}\subset B_{\Star{F}} $ 使得 $ \Star{T}f_{n}=e_{n} $, 即 $ f_{n}\circ T=e_{n} $. 则由 $ \Star{T} $ 的连续性, 只需说明 $ (f_{n})_{n\geqslant1} $ 有收敛子列即可. 注意到 $ f_{n} $ 的定义域为 $ \baro{T(B_{E})} $, 这是一个紧集.

		使用 Arzel\`a-Ascoli 引理, 要说明 $ (f_{n})_{n\geqslant1} $ 等度连续, 只需说用 $ \forall x\in\baro{T(B_{E})} $, 都有 $ (F_{n}(x))_{n\geqslant1} $ 相对紧. 由 $ (f_{n})_{n\geqslant1}\subset B_{\Star{F}} $ 知 $ (f_{n})_{n\geqslant1} $ 有界, 从而 $ (f_{n}(x))_{n\geqslant1} $ 有界. 于是 $ (f_{n}(x))_{n\geqslant1} $ 相对紧, 则 $ \forall\varepsilon>0, \forall x, y\in\baro{T(B_{E})} $, 若 $ \norm{x-y}<\varepsilon $, 则
		\[
			\abs{f_{n}(x)-f_{n}(y)} = \abs{f_{n}(x-y)}\leqslant\norm{x-y}<\varepsilon.
		\] 
		故 $ (f_{n})_{n\geqslant1} $ 等度连续, 从而 $ (f_{n}){n\geqslant1} $ 相对紧, 则存在 Cauchy 子列,
		记作 $ (f_{n_{k}})_{k\geqslant1} $, 由
		\[
			\begin{aligned}
				\norm{e_{n_{k}}-e_{n_{j}}} & = \sup_{y\in B_{E}}\norm{\lrangle{e_{n_{k}}-e_{n_{j}}, y}} \\
				& = \sup_{y\in B_{E}}\norm{\lrangle{\Star{T}f_{n_{k}}, y}-\lrangle{\Star{T}f_{n_{j}}, y}}\\
				& = \sup_{y\in B_{E}}\norm{\lrangle{f_{n_{k}}, Ty}-\lrangle{f_{n_{j}}, Ty}}\\
				& =  \sup_{x\in T(B_{E})}\norm{\lrangle{f_{n_{k}}, x}-\lrangle{f_{n_{j}}, x}}\\
				& \leqslant \norm{f_{n_{k}}-f_{n_{j}}}_{\baro{T(B_{E})}}\to 0.
			\end{aligned}
		\]
		故 $ (e_{n_{k}})_{k\geqslant1} $ 也是 Cauchy 列. 由 $ \Star{E} $ 的完备性可知 $ (e_{n_{k}})_{k\geqslant1} $ 收敛, 故 $ \Star{T}(B_{\Star{F}}) $ 相对紧, 即 $ \Star{T} $ 是紧算子.

		\textsl{充分性}. 若 $ \Star{T} $ 是紧算子, 由必要性可知 $ T^{**} $ 是紧算子, 由 $ T = T^{**}|E $ 可知 $ T(B_{E})\subset T^{**}(B_{E}) $ 相对紧, 故 $ T $ 是紧算子.\qed
	\end{Proof}

	\begin{Proposition}
		设 $ E $ 是 Banach 空间, 则 $ \baro{\Fr(E)}=\CB(E) $ 当且仅当 $ \dim E<\infty $.
	\end{Proposition}
	\begin{Proof}
		\textsl{充分性}. 因为 $ \dim E<\infty $, 故 $ \CB(E)=\Fr(E) $.

		\textsl{必要性}. 由 $ \baro{\Fr(E)}=\CB(E) $, 考虑恒同算子 $ \id_{E} $, 则存在 $ T\in\Fr(E) $ 使得 $ \norm{\id_{E}-T}<1 $, 由习题 3.9 结论知 $ \id_{E}-(\id_{E}-T)=T $ 可逆, 从而
		\[
			E = T^{-1}(T(E))
		\]
		是有限维的.\qed
	\end{Proof}

	\begin{Definition}[逼近性质]\index{B!逼近性质}\label{def:逼近性质}
		设 $ E $ 是 Banach 空间, 若 $ \baro{\Fr(E)}=\CK(E) $, 则称 $ E $ 具有\textbf{逼近性质}.
	\end{Definition}

	\begin{Theorem}
		Hilbert 空间具有逼近性质.
	\end{Theorem}
	\begin{Proof}
		只考虑 $ H $ 可分的情形, 即 $ H $ 的规范正交基是可数的, 记为 $ (e_{n})_{n\geqslant1} $. 设 $ P_{n} $ 是 $ H $ 到 $ \Span\{ \seq{e} \} $ 的投影, 即
		\[
			P_{n}: H\to\Span\{ \seq{e} \}\qquad x\mapsto \sum_{k=1}^{n}\lrangle{x, e_{k}}e_{k},
		\]
		则 $ \norm{P_{n}}=1 $. 由 $ \dim P_{n}(H)=n<\infty $ 可知 $ P_{n} $ 有限秩, 由 $ \Fr(H) $ 是理想可知
		\[
			\forall T\in\CK(H)\,(TP_{n}\in\Fr(H)\land P_{n}T\in\Fr(H)).
		\]
		令 $ T_{n}=P_{n}T $, 需证明 $ (T_{n})_{n\geqslant1} $ 依范数收敛到 $ T $, 对 $ \forall\varepsilon>0 $, 由 $ \baro{T(B_{H})} $ 紧可知存在 $ \seq[m]{x} $ 使得
		\[
			\baro{T(B_{H})}\subset\bigcup_{j=1}^{m}B(x_{j}, \varepsilon),
		\]
		由此可知
		\[
			\norm{(\id -P_{n})(x_{j})}=\Big( \sum_{k\geqslant n+1} \abs{\lrangle{x_{j}, e_{k}}}^{2} \Big)^{1/2}.
		\]
		于是 $ \exists n_{0}\in\N $ 使得 $ n>n_{0} $ 时, 有 $ \forall j= 1, 2, \dots, m\,(\norm{(\id-P_{n})(x_{j})}<\varepsilon/2) $, 故对任意的 $ y\in\baro{T(B_{H})} $ 总成立
		\[
			\norm{(\id-P_{n})(y)}\leqslant\norm{(\id-P_{n})(y-x_{j})}+\norm{(\id-P_{n})(x_{j})}<\frac{\varepsilon}{2}+\frac{\varepsilon}{2}=\varepsilon
		\]
		从而
		\[
			\norm{T-T_{n}}=\sup_{x\in B_{H}}\norm{(T-T_{n})(x)}=\sup_{x\in B_{H}}\norm{(\id-P_{n})(Tx)}<\varepsilon.
		\]
		从而 $ (T_{n})_{n\geqslant1} $ 依范数收敛到 $ T $. 于是 $ \CK(H)\subset\baro{\Fr(H)} $.

		下面说明 $ \CK(H) $ 是闭集. 设 $ (T_{n})_{n\geqslant1} $ 是 $ \CK(H) $ 中的序列, 且收敛于 $ T\in\CB(H) $, 往证 $ T\in\CK(H) $. 取 $ (Tx_{n})_{n\geqslant1} $ 是 $ T(B_{E}) $ 中有收敛子列的序列, 由 $ T_{1} $ 紧性知存在子序列使得 $ \left(T_{1}x_{n_{(1, k)}}\right)_{k\geqslant1} $ 收敛. 再由 $ T_{2} $ 紧知存在子序列 $ \left(T_{2}x_{n_{(2, k)}}\right)_{k\geqslant1} $ 收敛, 依此进行下去, 由对角线方法可取序列 $ \left(x_{n_{(k, k)}}\right) $ 满足对 $ \forall j $, 有 $ \left(T_{j}x_{n_{(k, k)}}\right)_{k\geqslant1} $ 收敛, 则
		\[
			\begin{aligned}
				\norm{Tx_{n_{(k, k)}}-Tx_{n_{(k', k')}}} & \leqslant \norm{(T-T_{j})\left(x_{n_{(k, k)}}-x_{n_{(k', k')}}\right)+T_{j}\left(x_{n_{(k, k)}}-x_{x_{(k', k')}}\right)}\\
				& \leqslant 2\norm{T-T_{j}}+\norm{T_{j}\left(x_{n_{(k, k)}}-x_{n_{(k', k')}}\right)}\to 0.
			\end{aligned}
		\] 
		从而 $ \left( Tx_{n_{(k, k)}} \right)_{k\geqslant1} $ 是 Cauchy 列, 从而 $ T(B_{E}) $ 相对紧, 故 $ T $ 紧. 由 $ \Fr(H)\subset\CK(H) $ 可知 $ \baro{\Fr}\subset \CK(H) $. 再由上证明, $ \CK(H)\subset\baro{\Fr(H)} $, 故 $ \baro{Fr(H)}=\CK(H) $.\qed
	\end{Proof}

\section{紧算子的谱性质}
	\begin{Definition}[谱]\label{def:谱}
		设 $ T\in\CB(E) $ 
		\begin{enumerate}[(1)]
			\item 令集合 
			\[
				\sigma(T) = \{ \lambda\in\K: \lambda\id-T\ \text{不可逆} \}
			\]
			称 $ \sigma(T) $ 为 $ T $ 的\textbf{谱集}\index{P!谱集}, 并称 $ \rho(T)=\K\sm\sigma(T) $ 为 $ T $ 的\textbf{预解集}\index{Y!预解集}.
			\item 记 $ \lambda\id-T $ 为 $ \lambda -T $. 若 $ \lambda -T $ 不是单射, 则 $ \exists x\in E, x\ne0 $ 使得
			\[
				(\lambda-T)(x)=0.
			\]
			也即 $ \lambda x=T x $, 则称 $ \lambda $ 为 $ T $ 的\textbf{特征值}\index{T!特征值}, 称 $ \ker(\lambda -T) $ 为 $ T $ 关于 $ \lambda $ 的\textbf{特征子空间}\index{T!特征性空间}, 并称非零向量 $ x\in\ker(\lambda-T) $ 为 $ T $ 相应于 $ \lambda $ 的\textbf{特征向量}\index{T!特征向量}.
			\item 对任意 $ \lambda\in\rho(T) $, 称 $ R(\lambda, T)=(\lambda-T)^{-1} $ 为 $ T $ 的\textbf{预解式}\index{Y!预解式}.
		\end{enumerate}
	\end{Definition}

	\begin{Example}
		几个谱集的例子
		\begin{enumerate}[(1)]
			\item 设 $ T = \left[\begin{smallmatrix}
				1 & 0 \\ 0 & 2
			\end{smallmatrix}\right] $, 则 $ \sigma(T)=\{ 1, 2 \} $;
			\item 对 $ f\in C[0, 1] $, 定义 $ M_{f}g =fg $, 其中 $ g\in C[0, 1] $, 则 $ M_{f}\in\CB(C[0, 1]) $, 且 $ \sigma(M_{f})=\{ f(t):t\in [0, 1] \} $;
			\item 设 $ T $ 满足 $ Tx=\sum\limits_{n\geqslant1}\frac{1}{n}\lrangle{x, e_{n}}e_{n} $, 则 $ \sigma(T)=\{ 0, 1, \frac{1}{2}, \frac{1}{3}, \dots \} $.
		\end{enumerate}
	\end{Example}

	\begin{Proposition}
		$ \forall \lambda, \mu\in\rho(T) $, 有
		\[
			R(\lambda, T)-R(\mu, T)=(\mu-\lambda)R(\lambda, T)R(\mu, T) = (\mu-\lambda)R(\mu, T)R(\lambda, T),
		\]
		这被称为\textbf{预解方程}\index{Y!预解方程}.
	\end{Proposition}
	\begin{Proof}
		由
		\[
			\begin{aligned}
				(\mu-\lambda)R(\mu, T)R(\lambda, T) & = R(\lambda, T)(\mu-\lambda)R(\mu, T)\\
				& = (\lambda-T)^{-1}((\mu-T)-(\lambda-T))(\mu-T)^{-1}\\
				& = (\lambda-T)^{-1}-(\mu-T)^{-1}\\
				& = R(\lambda, T)-R(\mu, T)
			\end{aligned}
		\]
		即证.\qed
	\end{Proof}

	\begin{Theorem}[谱半径]\index{P!谱半径}\label{thm:谱半径}
		设 $ T\in\CB(E) $
		\begin{enumerate}[(1)]
			\item 极限 $ \lim\limits_{n\to\infty} \norm{T^{n}}^{1/n} $ 存在, 且
			\[
				\lim_{n\to\infty}\norm{T^{n}}^{1/n}=\inf_{n\geqslant1}\norm{T^{n}}^{1/n},
			\]
			并将其记作 $ r(T) $, 称为算子 $ T $ 的\textbf{谱半径}. 
			\item $ \sigma(T) $ 是 \K 中的紧集, 且 $ \sigma(T)\subset\{ \lambda\in\K:\abs{\lambda}\leqslant r(T) \} $. 
		\end{enumerate}
	\end{Theorem}

	\begin{Theorem}[谱半径定理]\index{P!谱半径定理}\label{thm:谱半径定理}
		设 $ \K=\C, T\in\CB(E) $, 则 $ \sigma(T) $ 非空, 且
		\[
			r(T)=\sup_{\lambda\in\sigma(T)}\abs{\lambda}.
		\]
	\end{Theorem}

	\begin{Example}
		~
		\begin{enumerate}[(1)]
			\item 考虑对角矩阵 $ T=\diag\{ \seq{\lambda} \} $, 则 $ r(T)=\max\limits_{1\leqslant k\leqslant n}\abs{\lambda_{k}} $. 或使用定义, $ \forall m\geqslant1 $ 成立
			\[
				\norm{T^{m}}^{1/m}=\norm{\diag\{ \seq{\lambda^{m}} \}}^{1/m}=\max_{1\leqslant k\leqslant n}\abs{\lambda_{k}^{m}}^{1/m}=\max_{1\leqslant k\leqslant n}\abs{\lambda_{k}}.
			\]
			\item 考虑 Jordan 块
			\[
				T = J(0, 3)=\begin{bmatrix}
					0 & 1 & 0\\
					0 & 0 & 1\\
					0 & 0 & 0
				\end{bmatrix}
			\]
			则 $ r(T)=0 $, 此因 $ T^{3}=0 $. 进一步, 幂零矩阵的谱半径都是 0.
			\item 考虑算子 $ M_{f} $, 则
			\[
				r(M_{f}) = \lim_{n\to\infty}\norm{M_{f}^{n}}^{1/n}=\lim_{n\to\infty}\norm{M_{f^{n}}}^{1/n}=\lim_{n\to\infty}\norm{f^{n}}^{1/n}=\norm{f}.
			\]
			\item 考虑算子 $ T: x\mapsto\sum\limits_{n\geqslant1}\frac{1}{n}\lrangle{x, e_{n}}e_{n} $, 则 $ r(T)=1 $.
			\item 设 $ (e_{n})_{n\geqslant1} $ 是可分 Hilbert 空间 $ H $ 上的规范正交基, 考虑右移算子
			\[
				s: H\to H\qquad e_{n}\mapsto e_{n+1}\qquad (\forall n\geqslant1).
			\]
			由
			\[
				\Bnorm{s^{m}\sum_{n\geqslant1}\lambda_{n}e_{n}}=\Bnorm{\sum_{n\geqslant1}\lambda_{n}e_{n+m}}=\Big( \sum_{n\geqslant1}\abs{\lambda_{n}}^{2} \Big)^{1/2}
			\]
			知 $ \norm{s^{m}}\leqslant1 $. 又由 $ \norm{s^{m}e_{n}}=\norm{e_{n+m}} $ 知 $ \norm{s^{m}}=1 $, 故 $ r(s)=1 $. 
		\end{enumerate}
	\end{Example}

	\begin{Corollary}
		设 $ T\in\CB(E) $, 则 $ r(T)\leqslant\norm{T} $.
	\end{Corollary}
	\begin{Proof}
		由定义
		\[
			r(T)=\lim_{n\to\infty}\norm{T^{n}}^{1/n}\leqslant\lim_{n\to\infty}(\norm{T}^{n})^{1/n}=\norm{T}
		\]
		即证.\qed
	\end{Proof}
	
	\begin{Theorem}\label{thm:lambda-T的性质}
		设 $ T\in\CK(E), \lambda\in\K $ 且 $ \lambda\ne0 $, 有
		\begin{enumerate}[(1)]
			\item $ \forall n\in\N $, 有 $ \dim\ker(\lambda-T)^{n}<\infty $.
			\item $ \forall n\in\N $, $ (\lambda-T)^{n}(E) $ 是闭集.
			\item $ \exists n\in\N $, 使得 $ \ker(\lambda-T)^{n+1}=\ker(\lambda-T)^{n} $.
			\item $ \exists n\in\N $, 使得 $ (\lambda-T)^{n+1}(E)=(\lambda-T)^{n}(E) $.
		\end{enumerate}
	\end{Theorem}

	\begin{Proof}
		若 $ E $ 有限维, 命题显然成立, 不妨设 $ E $ 是无限维的.

		(1) 先考虑 $ n=1 $ 时的情形, 若 $ \ker(\lambda-T) $ 无限维, 记 $ K_{1}=\ker(\lambda-T) $ 后有 $ (\lambda-T)B_{K_{1}}=0 $, 即 $ \lambda B_{K_{1}}=T(B_{K_{1}}) $. 因为 $ T\in\CK(E) $, 故 $ T(B_{K_{1}}) $ 相对紧, 从而有 $ \lambda B_{K_{1}} $ 也相对紧, 而这说明 $ K_{1} $ 有限维, 矛盾.
		
		注意到 
		\[
			(\lambda-T)^{n}=\sum_{k=0}^{n}(-1)^{k}\binom{n}{k}\lambda^{n-k}T^{k} = \lambda^{n}-\sum_{k=1}^{n}(-1)^{k-1}\binom{n}{k}\lambda^{n-k}T^{k},
		\]
		故类似 $ n=1 $ 的情形可证.

		(2) 令 $ H_{n}= (\lambda-T)^{n}(E) $. 先考虑 $ n=1 $ 的情形, 作映射
		\[
			\tilde{T}: E/K_{1}\to H_{1}\qquad x+K_{1}\mapsto (\lambda-T)(x),
		\]
		只需说明 $ \tilde{T}^{-1} $ 有界即可. 用反证法, 假设 $ \tilde{T}^{-1} $ 无界, 则 $ \exists (x_{n}+K_{1})_{n\geqslant1} $ 使得 $ \norm{x_{n}+x_{1}}_{E/K_{1}}=1 $ 且 $ \tilde{T}(x_{n}+K_{1})\to 0 $. 由商模的定义可知 $ \forall n\geqslant1, \exists x_{n}' $ 使得  $ x_{n}' $ 是 $ x_{n}+K_{1} $ 的代表元: 满足 $ 1\leqslant\norm{x_{n}'}<2 $ 且 $ (\lambda-T)x_{n}'\to 0 $. 因为 $ T $ 紧, 故存在 $ (x_{n_{k}}')_{k\geqslant1}\subset(x_{n}')_{n\geqslant1} $ 使得 $ (Tx_{n_{k}}')_{k\leqslant1} $ 收敛, 记其极限为 $ z $, 则
		\[
			\lambda x_{n_{k}}'=Tx_{n_{k}}'+(\lambda-T)x-{n_{k}}'\to z+0=z
		\]
		即 $ x_{n_{k}}'\to z/\lambda $, 于是 $ (\lambda-T)x_{n_{k}}'\to (\lambda-T)z/\lambda=0 $, 由此可知 $ z\in\K_{1} $, 则
		\[
			\norm{x_{n_{k}}'+K_{1}}_{E/K_{1}}\leqslant\norm{x_{n_{k}}'-\frac{z}{\lambda}}\to 0,
		\]
		但这与 $ \norm{x_{n}+K_{1}}_{E/K_{1}}=1 $ 矛盾, 故 $ \tilde{T}^{-1} $ 有界, 从而 $ H_{1} $ 是闭集.

		下面使用归纳法, 设 $ \seq{H} $ 都是闭集, 则它们都是 Banach 空间. 因为
		\[
			H_{n+1}=(\lambda-T)^{n+1}(E)=(\lambda-T)(\lambda-T)^{n}(E)=(\lambda-T)(H_{n})
		\]
		则 $ \forall x\in E $, 由
		\[
			T(\lambda-T)^{n}(x)=(\lambda-T)^{n}(Tx)
		\]
		知 $ T(H_{n})\subset H_{n} $. 由 $ T|_{H_{n}} $ 紧知 $ \dim K_{n}<\infty $, 类似 $ n=1 $ 的情形可证 $ H_{n+1} $ 闭.

		(3) 注意到
		\[
			K_{1}\subset K_{2}\subset\cdots\subset K_{n}\subset\cdots
		\]
		假设所有包含关系都是严格的, 则 $ \exists x_{n}\in K_{n+1} $ 使得 $ \norm{x_{n}}=1 $ 且 $ d(x_{n}, K_{n})\geqslant1/2 $, 则对 $ m>n $, 有
		\[
			\begin{aligned}
				Tx_{m}-Tx_{n} & = Tx_{m}-\lambda x_{m}+\lambda x_{m}-\lambda x_{n}+\lambda x_{n}-Tx_{n}\\
				& = \lambda x_{m}+((T-\lambda)x_{m}-\lambda x_{n}+(\lambda-T)x_{n})
			\end{aligned}
		\]
		记 $ y=(T-\lambda)x_{m}-\lambda x_{n}+(\lambda-T)x_{n}\in K_{m} $, 则有
		\[
			\norm{Tx_{{m}}-Tx_{n}}=\abs{\lambda}\norm{x_{m}+\frac{y}{\lambda}}\geqslant\abs{\lambda}\cdot\frac{1}{2}
		\]
		即 $ (Tx_{n})_{n\geqslant1} $ 无收敛子列, 这与 $ T $ 是紧算子矛盾.

		(4) 因为 $ T $ 紧, 故 $ \Star{T} $ 紧. 由 (3) 知 $ \exists n\in\N $ 使
		\[
			\ker(\lambda-\Star{T})^{n+1}=\ker(\lambda-\Star{T})^{n}
		\]
		由双极定理, 有 $ \baro{(\lambda-T)^{n+1}(E)}=\baro{(\lambda-T)^{n}(E)} $, 再由 (2) 知 $ (\lambda-T)^{n}(E) $ 是闭集, 故命题得证.\qed
	\end{Proof}
	
	\begin{Remark}
		上一定理说明了紧算子具有某种意义上的有限性, 且其限制到闭集上几乎就是可逆的 ( 模掉零空间后可逆).
	\end{Remark}

	\begin{Theorem}
		设 $ T\in\CB(E), \lambda\in\K $, 有
		\begin{enumerate}[(1)]
			\item 任取 $ \lambda\in\sigma(T)\sm \{ 0 \} $, 必有 $ \lambda\in\sigma_{p}(T) $, 其中 $ \sigma_{p}(T) $ 是 $ T $ 的所有特征值的集合, 称为 $ T $ 的\textbf{点谱集}.
			\item $ T $ 的非零特征值至多可数, 从而 $ \sigma_{p}(T) $ 至多可数, 特别地,  $ \sigma_{p}(T) $ 是紧集. 若将其所有的特征值排成一列 $ (\lambda_{n}) $, 当 $ (\lambda_{n}) $ 无限时, 必有 $ \lim\limits_{n\to\infty}\lambda_{n}=0 $.
		\end{enumerate}
	\end{Theorem}
	\begin{Proof}
		(1) 设 $ \lambda\in\sigma(T)\sm\{ 0 \} $, 若 $ \lambda-T $ 是单射, 则由定理~\ref{thm:lambda-T的性质}~知
		\[
			\exists n\in\N\,((\lambda-T)^{n+1}(E)=(\lambda-T)^{n}(E)),
		\]
		则
		\[
			\forall x\in E\exists y\in E\,((\lambda-T)^{n+1}(y)=(\lambda-T)^{n}(x)),
		\]
		由 $ \lambda-T $ 是单射可知 $ x=(\lambda-T)y $, 由 $ x $ 的任意性可知 $ \lambda-T $ 是满射, 由开映射定理知 $ \lambda-T $ 是同构, 即 $ \lambda-T $ 可逆. 这与 $ \lambda\in\sigma(T) $ 矛盾. 从而 $ \lambda-T $ 不是单射. 即 $ \ker(\lambda-T)\ne\{ 0 \} $. 这意味着 $ \exists x\ne 0 $ 使得 $ \lambda x=Tx $, 也即 $ \lambda\in\sigma_{p}(T)\sm\{ 0 \} $. 

		(2) 只需说明 $ \forall \eta>0 $, 只存在有限多个不同的 $ \lambda\in\sigma_{p}(T)\sm\{ 0 \} $ 使得 $ \lambda\leqslant\eta $ 即可.

		用反证法. 设 $ \exists\eta>0 $ 使得有无穷多个不同的 $ \lambda_{n}\in\sigma_{p}(T) $ 使得 $ \abs{\lambda_{n}}\geqslant\eta $, 则设 $ e_{n} $ 是对应的特征向量, 有 $ Te_{n}=\lambda_{n}e_{n} $, 并记 $ E_{n}=\Span\{ \seq{e} \} $.

		先证明 $ \seq{e} $ 线性无关. 否则存在不全为 0 的 $ \seq{a} $ 使得
		\[
			\begin{bmatrix}
				1 & 1 & \cdots & 1\\
				\lambda_{1} & \lambda_{2} & \cdots & \lambda_{n}\\
				\vdots & \vdots & \ddots & \vdots \\
				\lambda_{1}^{n} & \lambda_{2}^{n} & \cdots & \lambda_{n}^{n}
			\end{bmatrix}\begin{bmatrix}
				a_{1}e_{1}\\
				a_{2}e_{2}\\
				\vdots\\
				a_{n}e_{n}
			\end{bmatrix}=0
		\]
		注意到系数矩阵是 Vandermonde 矩阵, 记作 $ V_{n} $, 由 $ \seq{\lambda} $ 互不相同知 $ \det V_{n}\ne0 $, 即方程只有零解, 矛盾.

		再说明存在某序列 $ (Ty_{n})_{n\geqslant1} $ 没有收敛子列, 其中 $ y_{n}\in\bar{B}_{E} $. 注意到
		\[
			E_{1}\subset E_{2}\subset\cdots\subset E_{n}\subset\cdots
		\]
		且所有的包含关系都是严格的, 从而 $ \exists y_{n}\in E_{n+1} $ 使得 $ \norm{y_{n}}=1 $ 且 $ d(y_{n}, E_{n})\geqslant1/2 $, 对 $ m>n $ 有
		\[
			\norm{Ty_{m}-Ty_{n}}=\norm{-\lambda_{m+1}y_{m}+Ty_{m}-Ty_{n}+\lambda_{m+1}y_{m}}
		\]
		并注意到 $ -\lambda_{m+1}y_{m}+Ty_{m}-Ty_{n}\in E_{m} $, 从而有
		\[
			\norm{Ty_{m}-Ty_{n}}\geqslant\abs{\lambda_{m+1}}\cdot\frac{1}{2}\geqslant\frac{\eta}{2}
		\]
		矛盾.\qed
	\end{Proof}